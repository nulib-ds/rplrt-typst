\documentclass[12pt]{book}
\usepackage{xcolor}
\usepackage{tocloft}
\usepackage[most]{tcolorbox}
\usepackage{afterpage}
\usepackage{changepage} % Add this package
\usepackage{lastpage}
\usepackage{enumitem}% http://ctan.org/pkg/enumitem
\usepackage[russian, greek, english]{babel}
\usepackage{emptypage}
\usepackage{comment}
\usepackage{textcase}
\usepackage{microtype}
\usepackage{ifthen}
\usepackage{fontspec}
\usepackage{graphicx}
\usepackage{float}
\usepackage{pdfpages}
\usepackage[hidelinks, hyperfootnotes=false]{hyperref}
\usepackage{xurl}
\usepackage[a4paper, margin=1in, top=0.75in]{geometry}
\usepackage{fancyhdr}
\usepackage{lipsum}
\usepackage{ragged2e}
\usepackage{titlesec}
\usepackage[flushmargin, bottom]{footmisc}
\usepackage{tikz}
\usepackage{wrapfig}
\usepackage[framemethod=TikZ]{mdframed}
\usepackage[absolute,overlay]{textpos}
\usepackage{etoolbox}
\usepackage{quoting}
\usepackage{setspace}
\hypersetup{breaklinks=true}
\quotingsetup{vskip=1em}
% Page 280 Randall article after Second the political failure of Russian Liberalism
% Fix first Randalls footnote
% REmove page numbers for parts
% Set the main font to IBM Plex Sans and IBM Plex Serif
% Decrease Spacing around block quotes in saul's article
% Footnote spacing issue on 181 with "See Viacheslav Ivanov, selected Essays trans. Rober Bird" Line starts with To be sure, Bulgakov was no naif
\interfootnotelinepenalty=1000
\widowpenalty=10000
\clubpenalty=10000
\usetikzlibrary{calc}
\definecolor{custombg}{RGB}{227, 224, 224}

\renewcommand{\section}[1]{}
\newfontfamily\titlefont{IBM Plex Sans}[Scale=1.2]
\newfontfamily\chaptertitlefont{IBM Plex Sans SemiBold}
\newfontfamily\tocfont{IBM Plex Sans Condensed Medium}
\newfontfamily\tocnfont{IBM Plex Sans Condensed}
\newfontfamily\seriffont{Adobe Text Pro}
\newfontfamily\monofont{IBM Plex Mono}
\newfontfamily\publisherfont{IBM Plex Sans Condensed Medium}
\newfontfamily\datefont{IBM Plex Sans Condensed}
\newfontfamily\footnotetextfont{Adobe Text Pro}
\newfontfamily\plexcondensed{IBM Plex Sans Condensed Medium}
\newfontfamily\authorheadfont{IBM Plex Sans SemiBold}
\newfontfamily\leftmarkfont{IBM Plex Sans}
\newfontfamily\IBMPlexSansSemiBold{IBM Plex Sans}[Weight=SemiBold]

\babelhyphenation[english]{Ber-di-a-ev}
\babelhyphenation[english]{the-o-dic-ean}
\babelhyphenation[english]{th-o-u-gh}
\babelhyphenation[english]{Dos-to-ev-sky}
\babelhyphenation[english]{Le-ont}
\babelhyphenation[english]{apo-pha-tic}
\babelhyphenation[english]{in-sti-tu-ti-on}
\hyphenation{Da-ni-lev-sko-go}

\setlength{\headheight}{20pt} % Adjust as needed
\setlength{\headsep}{15pt}
\setlength{\topskip}{11pt}
\newcommand{\biobox}[1]{
  \vspace{2em}
  \begin{mdframed}[linewidth=0, backgroundcolor=custombg, innerleftmargin=10pt, innerrightmargin=10pt]
    \begin{minipage}{\linewidth}
      \seriffont\fontsize{12}{18}\selectfont
      #1
    \end{minipage}
  \end{mdframed}
  \vspace*{-\fill} % Ensures the box is at the bottom of the main content
}

\newcommand{\abstractbox}[5]{
  \clearpage % Force a page break before the abstract box
  \newpage
  \fancyhf{}
  \begin{tikzpicture}[remember picture, overlay]
    % Define the grey box
    \node[
        anchor=north west,
        fill={rgb,255:red,227; green,224; blue,224},  % Light grey background with RGB values (243, 245, 245)
        minimum width=\dimexpr\paperwidth-1.5in, % Box width (0.5in margin on left and right)
        minimum height=\dimexpr\paperheight-1in % Box height (1in margin on top and bottom)
    ] at ($(current page.north west) + (0.75in,-.5in)$) {}; % Offset by 0.75in from left and 1in from top
  \end{tikzpicture}

  % Title and content inside the box
  \vspace*{-.5in} % Adjust to align content correctly with the box
  \noindent
  \begin{adjustwidth}{.5in}{.5in} % Adjust the left and right margins
  \begin{minipage}[t][\dimexpr\paperheight-2in-2em][t]{\textwidth-1in} % Adjusted height of the box minus margins and extra space
      \begin{center}
          \thispagestyle{empty}
          \vspace{2em}
          {\leftmarkfont\fontsize{12pt}{14pt}\fontseries{sb}\selectfont{#1}} \\
          \ifthenelse{\equal{#2}{}}{}{
              {\leftmarkfont\fontsize{12pt}{14pt}\fontseries{sb}\selectfont{#2}} \\
          }
          \vspace{1em}
          \plexcondensed\textit{\fontsize{9pt}{12pt}\selectfont by} \fontsize{12pt}{14pt}\selectfont #3 \\[2em]
          \vspace{1em}
          \leftmarkfont\fontsize{17.5pt}{21pt}\fontseries{sb}\selectfont{Abstract} \\[1em]
      \end{center}
      \noindent
      {\seriffont\fontsize{12}{18}\selectfont #4} % Placeholder for abstract text

    \vspace{2em}
    \begin{center}
    \includegraphics[width=0.75cm]{articlend.png}
    \end{center}
      % Spacer to push the keywords to the bottom
      \vspace*{\fill}
      {\seriffont\fontsize{12}{14}\seriffont\textbf{Keywords:} #5}
  \end{minipage}
  \end{adjustwidth}
}

\newcommand{\epigraph}[2]{
  {\begin{flushright}
  #2
\end{flushright}}
}

\makeatletter
% Customize the footnote rule
\renewcommand{\footnoterule}{%
  \hrule width 1\textwidth height 0.4pt % The rule itself
}
\makeatother

\makeatletter
\setlength{\skip\footins}{8pt}
\renewcommand{\@makefntext}[1]{%
    \noindent% Set footnote text font
    {\footnotetextfont
    \arabic{footnote}.~#1} % Format footnote number and text
}
\makeatother
%\renewcommand{\UrlFont}{\normalfont\itshape} % Adjust this to match the style you want for URLs
% Redefine the subsection title format for unnumbered subsections
\titleformat{\subsection}
  {\centering\fontseries{sb}\titlefont\fontsize{12}{12}\selectfont} % Added \itshape for italics
  {}{}{\noindent} % Adjust spacing after subsection
\titlespacing*{\subsection}{0pt}{2em}{1em}
\setlength{\footnotesep}{1.25em} % Adjust global footnote spacing
\setlength{\parskip}{0.25em}  % Adjust the space between paragraphs
% Set DOI and publisher as new commands with alignment adjustment
\newcommand{\doi}[1]{\fontsize{9}{11}\selectfont\monofont{\href{https://doi.org/#1}{DOI: #1}}}
\newcommand{\publisher}{
  {\publisherfont\fontsize{9}{11}\selectfont
  NORTHWESTERN UNIVERSITY STUDIES IN RUSSIAN \\
  PHILOSOPHY, LITERATURE, AND RELIGIOUS THOUGHT}
}

% Custom page style for chapter title pages (with publisher and DOI)
\fancypagestyle{chaptertitlepage}{
  \fancyhf{} % Clear all header and footer fields
  \fancyhead[L]{\publisher}
  \renewcommand{\headrulewidth}{0pt} % No header rule on title pages
  \fancyfoot[LE, RO]{\thepage} % Left on even pages, right on odd pages
}


% Define a custom command for chapter titles with the image
\newcommand{\chaptertitle}[3]{
  \clearpage % Start each chapter on a new page
  \phantomsection % Create an anchor for the hyperlink
  \thispagestyle{chaptertitlepage}
  \markboth{#1}{} % Set chapter title for header
  \begin{center}
    \vspace*{.5in} % Space below header (increase this value for more padding)
    \includegraphics[width=0.5in]{./articlestart.png} \\[1em]
    \chaptertitlefont\fontsize{28pt}{30pt}\selectfont{#1} \\ % Chapter title
    \vspace{0.5em}
    \ifthenelse{\equal{#2}{}}{}{\fontsize{21}{26}\fontseries{m}\selectfont #2 \\} % Subtitle
    \vspace{0.5em}
    \authorheadfont\fontsize{16}{22}\selectfont #3 % Author
  \end{center}
  \vspace{1em} % Adjust this to control spacing between title and content
  \fancyhf{}
  % Set the page style for the rest of the chapter to chaptercontentpage
  \pagestyle{chaptercontentpage}
}
\titleformat{\part}[display]
{\tocfont\fontsize{20}{20}\selectfont\centering} % Apply the custom font and center the title
  {}
  {0pt}
  {}
\renewcommand{\thepart}{} % This makes the part number empty
\newcommand{\blankpart}[1]{
  \cleardoublepage
  \thispagestyle{empty}
  \part{\tocfont\selectfont #1}
  \thispagestyle{empty}
  \cleardoublepage
}

% Set chapter fonts to \tocfont

\begin{document}
\includepdf[pages=-, fitpaper=true]{./cover.pdf}

\newpage
\thispagestyle{empty} % Clear page style for copyright page
\vspace*{\fill}
\begin{minipage}{\textwidth}
\raggedright
Copyright \copyright 2024\ The Authors.\par
Permission is granted to copy and/or distribute
this document under the terms of the
\href{https://creativecommons.org/licenses/by/4.0/}{CC-BY 4.0 License.}
  \vspace{2em}

You are free to share and adapt this work for any purpose, even commercially, so long as credit is given to the authors.
\vspace{2em}

Proudly published for the web and print by Northwestern University Libraries in Evanston Illinois.
  \vspace{2em}
\\ Typeset in IBM Plex, Adobe Text Pro, and Good Pro using \LaTeX.
\vfill
\vspace{2em}
Library of Congress ISSN 3065\textendash 0755
\end{minipage}
\pagestyle{empty}
\vspace*{\fill}
\newpage
% Title page
\begin{spacing}{0.1}
\end{spacing}
\includepdf[pages=-, fitpaper=true]{./toc.pdf}
%Get rid of the headers on all subsequent pages

\begingroup
\pagestyle{empty} % No headers on TOC pages
\thispagestyle{empty} % No page numbers on TOC page
\endgroup
\clearpage
\tolerance=1000 \emergencystretch=2em \hyphenpenalty=10000
\spaceskip=.15em plus 0.1em
\section{from the editors}

\setcounter{page}{5}

\newpage

\fancypagestyle{chaptertitlepage}{
  \fancyhf{} % Clear all header and footer fields
  \fancyhead[L]{\begin{minipage}[t]{0.7\textwidth}\publisher\end{minipage}}
  \fancyhead[R]{\begin{minipage}[t]{\textwidth}\raggedleft \datefont\fontsize{10}{11}\selectfont Volume 1 (2024): \thepage\textendash \pageref{sec:editors} \\ \doi{10.71521/2pzk-s566} \end{minipage}}
  \renewcommand{\headrulewidth}{0pt} % No header rule on title pages
  \fancyfoot[RE]{\thepage}
  \fancyfoot[LO]{\thepage}
}
\fancypagestyle{chaptercontentpage}{
  \fancyhf{} % Clear all header and footer fields
\fancyhead[CE]{%
  \fontsize{11}{11}\leftmarkfont%
  \addfontfeature{LetterSpace=10.0}%
  \textit{\MakeUppercase{\leftmark}}%
}
  \fancyhead[CO]{\authorheadfont\addfontfeature{LetterSpace=10.0}\fontsize{11}{11}\selectfont\textbf{{\uppercase{Editorial Staff}}}}
  \renewcommand{\headrulewidth}{0pt} % No header rule on content pages
  \fancyfoot[RE]{\thepage}
  \fancyfoot[LO]{\thepage}
}
\chaptertitle{From the Editors}{}{}

\addcontentsline{toc}{chapter}{{\tocnfont From the Editors}}
\seriffont\fontsize{12}{18}\selectfont
\vspace{-3em}
\noindent We are pleased to present the inaugural issue of \emph{Northwestern University Studies in Russian Philosophy, Literature, and Religious Thought}\textemdash the annual, online, peer-reviewed journal of the Northwestern~University~Research Initiative in Russian Philosophy, Literature, and Religious Thought.

Since its founding in 2022, the NU RPLRT Research Initiative has flourished. It has attracted approximately 160 scholars from around the world, ranging from doctoral students to the most eminent figures in our fields. They and their work are profiled on our comprehensive website.\footnote{\url{https://rprt.northwestern.edu/index.html}}

\emph{Northwestern University Studies in Russian Philosophy, Literature, and Religious Thought} builds on the success of our online research forum, \emph{Northwestern University Forum in Russian Philosophy, Literature, and Religious Thought},\footnote{\url{https://sites.northwestern.edu/nurprt/}} as well as on our conference series.

In April 2023 we held our inaugural conference on Northwestern's campus in Evanston, Illinois: \emph{What's New about the New Atheism? The Enduring Relevance of Russian Philosophy}. A year later our second annual conference honored an esteemed Northwestern professor: \emph{Celebrating Gary Saul Morson: Humanistic Traditions in Russian Thought and Literature}. In addition, we have co-sponsored conferences at Universidade Federal de Juiz de Fora, Brazil (November 2022), Universidade Católica Portuguesa (UCP) in Lisbon (March 2023), the Pontifical University of John Paul II in Krakow, Poland (June 2024), and the Hamilton Center for Classical and Civic Education at the University of Florida (November 2024). Our third annual conference, \emph{Evil in Russian Thought and Literature}, co-sponsored by Trinity College, University of Cambridge, will be held at St.~John's College, University of Cambridge, July 31\textendash August 2, 2025.

One of the main purposes of our conference series is to generate innovative scholarship for publication on our research forum and in our journal. The journal's inaugural issue features several articles that were presented as papers at our first annual conference: those by Amy Singleton Adams, Jillian Pignataro, Julia Berest, Daniela Steila, and Daniel Adam Lightsey. The journal's second volume (2025) will feature articles from the conference we co-sponsored with the Hamilton Center: \emph{Religion, Human Dignity, and Human Rights: New Paradigms for Russia and the West}. The third issue of the journal will, we hope, feature articles from our summer 2025 conference at Cambridge.

Normally a preface "From the Editors" includes a few words about the articles which follow. Instead, we refer the reader to the "Afterword" at the end of this issue, written by Caryl Emerson, a distinguished member of our Advisory Board who has been integral to the NU RPLRT Research Initiative from the very beginning.

As this inaugural issue demonstrates, \emph{Northwestern University Studies in Russian Philosophy, Literature, and Religious Thought} showcases outstanding scholarship in each of these three titular areas and, especially, at the intersections among them. We invite submissions for research articles and book reviews for publication in the 2025 issue. Please contact Susan McReynolds and Randall Poole.

{\seriffont
\fontsize{12}{12}
\selectfont
\vspace{2em}

\noindent Susan McReynolds 

\noindent Northwestern University 

\noindent Founding Co-Director and Editor-in-Chief
\vspace{1em}

\noindent Randall A. Poole 

\noindent College of St. Scholastica

\noindent Co-Director and Editor

\vspace{1em}

\noindent Bradley Underwood 

\noindent Northwestern University

\noindent Associate Director and Associate Editor

\vspace{1em}

\noindent Octavian Gabor

\noindent Methodist College 

\noindent Editor

\vspace{1em}

\noindent Peter Gregory Winsky

\noindent University of Southern California

\noindent Editor
\vspace{1em}

\noindent Aerith Netzer

\noindent Northwestern University Libraries

\noindent Digital Publishing and Repository Librarian

}
\label{sec:editors}

\newpage

\part{Distinguished Contributions}

\section{Morson - Varieties of Belief in \emph{The Brothers Karamazov}}
\clearpage

\newpage

\fancypagestyle{chaptertitlepage}{
  \fancyhf{} % Clear all header and footer fields
  \fancyhead[L]{\begin{minipage}[t]{0.7\textwidth}\publisher\end{minipage}}
  \fancyhead[R]{\begin{minipage}[t]{\textwidth}\raggedleft \datefont\fontsize{10}{11}\selectfont Volume 1 (2024): \thepage\textendash\pageref{sec:morson}  \\ \doi{10.71521/2pzk-s566} \end{minipage}}
  \renewcommand{\headrulewidth}{0pt} % No header rule on title pages
  \fancyfoot[RE]{\thepage}
  \fancyfoot[LO]{\thepage}
}
\thispagestyle{chaptertitlepage} % Apply custom page style for chapter title

\fancypagestyle{chaptercontentpage}{
  \fancyhf{} % Clear all header and footer fields
\fancyhead[CE]{%
  \fontsize{11}{11}\leftmarkfont%
  \addfontfeature{LetterSpace=10.0}%
  \textit{\MakeUppercase{Varieties of Belief in the Brothers Karamazov}}%
}
  \fancyhead[CO]{\authorheadfont\fontsize{11}{11}\selectfont{{\uppercase{Gary Saul Morson}}}}
  \renewcommand{\headrulewidth}{0pt} % No header rule on content pages
  \fancyfoot[RE]{\thepage}
  \fancyfoot[LO]{\thepage}
}
\fancyhf
\clearpage


\abstractbox{Varieties of Belief in \emph{The Brothers Karamazov}}{}{Gary Saul Morson}{What does it mean to believe something? Could it be that "belief" is one word for many different states of mind? This article considers how Dostoevsky examined situations in which one believes and disbelieves something at the same time. That is evidently the case with self-deception, but Dostoevsky also considers how one can work oneself up to a feeling or cultivate the sense that one is insulted even when one is not. In \textit{The Brothers Karamazov}, Fyodor Pavlovich and each of the three brothers show something important about faith and belief. Dostoevsky believed that faith can be, and was for him, not a state but a process: just as happiness lies in the quest for happiness, faith can lie in the quest for faith.}{Fyodor Dostoevsky, \textit{Brothers Karamazov}, play acting, belief, faith, doubt, proof, miracles}

\chaptertitle{Varieties of Belief in \\ \emph{The Brothers Karamazov}}{}{
Gary Saul Morson}

\addcontentsline{toc}{chapter}{Varieties of Belief in \textit{The Brothers Karamazov}\\\textit{by} Gary Saul Morson}

\epigraph{}{Lord, I believe; help Thou my unbelief. \textbf{Mark 9:24}}

\noindent Starting with its first chapter, \emph{The Brothers Karamazov} introduces a theme to be developed throughout the novel: what does it mean to believe something? Could "belief" (or faith\textemdash also \emph{vera}) be a single term for several different states of mind? And what about second-order beliefs: can one be mistaken in one's belief about what one believes? Can one believe and disbelieve something at the same time?

Describing Fyodor Pavlovich's marriage to Adelaida Ivanovna, "who belonged to a fairly rich and distinguished noble family," the novel's narrator explains that her choice was perplexing: "How it came to pass that an heiress, who was also a beauty , and moreover one of those lively, intelligent girls not at all uncommon in this generation, but somehow also to be found in the last, could have married such an insignificant 'puny fellow, as everyone called him, I won't attempt to explain."\footnote{Fyodor Dostoevsky, \emph{The Brothers Karamazov}, 2\textsuperscript{nd} edition, ed.~Susan McReynolds, the Garnett translation revised by Ralph E. Matlaw and Susan McReynolds (New York: Norton, 2011), 11. Further references are to \emph{BK}. I have occasionally modified the translation.} And then, of course, he does, by citing another strange courtship:

\begin{quote}
But I knew a young lady, still of the 'romantic' generation before the last, who after some years of enigmatic love for a gentleman, whom she might easily have married at any moment, invented insuperable obstacles to their union, and ended by throwing herself one stormy night into a rather deep and rapid river, and so perished, entirely to satisfy her own caprice, and to be like Shakespeare's Ophelia, and indeed, if this precipice, a chosen and favorite spot of hers, had been less picturesque, if there had been a prosaic flat bank in its place, most likely the suicide would not have taken place (\emph{BK}, 11\textendash 12).
\end{quote}

There have been many such cases in Russian life, the narrator continues. A person adopts a role and acts it out, like a stage actor who madly decides he really is the character he plays. In the same way, the person subscribes to a whole set of her generation's prescribed beliefs. Does she really believe them?

Elsewhere Dostoevsky describes such thinking as "wearing a uniform." One adopts a prefabricated philosophy wholesale. Dostoevsky loved to satirize people like this: in \emph{Crime and Punishment}, Lebeziatnikov professes one absurd progressive belief after another, and yet it is plain he has never actually assessed the arguments for any one of them. In Dostoevsky's view, people who think this way\textemdash especially the educated who pride themselves on their critical thinking\textemdash buy a whole package at a time. No arguments or evidence could ever shake any of these beliefs because no argument or evidence led to its adoption. As Jonathan Swift once remarked, no one was ever talked out of a belief he had not first been talked into.

Adelaida Ivanovna resembled this would-be Ophelia, except that the beliefs she adopted were those of the next, radical, rather than romantic, generation. Her marriage to Fyodor Pavlovich, the narrator surmises, resulted from "an echo of alien influences" (or "foreign currents of opinion" {[}\emph{otgoloskom chuzhikh veyanij}{]}). Perhaps she "wanted to assert feminine impendence," or "go against social conventions" or "against the despotism of her family and birth"\textemdash all canned phrases for hackneyed fashionable ideas. Because she perceived the world through these lenses, "an obliging imagination persuaded her" that the sponger Fyodor Pav\-lo\-vich was "one of the boldest and most ironical people of that epoch that was transitional to everything better"\textemdash the hackneyed ideas accumulate here\textemdash even though he was obviously, to anyone not blinded by assumed ways of seeing, "just an evil buffoon and nothing more" (\emph{BK}, 12).

We may ask: did Adelaida Ivanovna actually believe these ideas? She evidently did in the sense that she acted on them. But the element of play acting here\textemdash the pleasure of the heroic role and the work of her "obliging imagination"\textemdash strongly suggest she didn't. It seems one can sincerely profess what one at some level knows to be untrue. In that case, which does one really believe? This question will recur in Dostoevsky's descriptions of other theatrical behavior, most obviously, that of Fyodor Pavlovich.

Belief (and faith) are examined time and again. Dostoevsky interrupts his description of "The Unfortunate Gathering" with two chapters about faith: "Believing Peasant Women" (or "Peasant Women who Have Faith"\textemdash {[}\emph{Veruyushchie baby}{]}) and "A Lady of Little Faith" (\emph{Malovernaya dama}). The repetition of the root \emph{vera} (belief or faith) in these chapter titles suggests the importance and complexity of this theme.

\subsection*{Doubting Thomas}

In different ways, Fyodor Pavlovich and his three legitimate sons illustrate important insights about the nature of belief. When the narrator introduces Alyosha, he mentions the boy's belief in miracles. Readers will probably suspect Alyosha of "mysticism and fanaticism," he anticipates, but in this case that judgment would be hasty. Alyosha, he says, "was more of a realist than anyone." To explain this apparent paradox, the narrator expatiates on the nature of belief.

\begin{quoting}
\noindent Oh, no doubt in the monastery he fully believed in miracles, but, to my thinking, miracles are never a stumbling block to the realist. It is not miracles that dispose realists to belief. The genuine realist, if he is an unbeliever, will always find strength and ability to disbelieve in the miraculous, and if he is confronted with a miracle as an irrefutable fact, he would rather disbelieve his own senses than admit the fact. Even if he admits it, he admits it as a fact of nature till then unrecognized by him (\emph{BK}, 28).
\end{quoting}

\noindent Religions usually offer miracles as proof. Jesus raised Lazarus from the dead, so how can you doubt His divinity? The narrator regards this argument as naïve because it presumes that people are bound to accept clear and indubitable evidence.

David Hume famously refuted miracles by asking, which is more likely: that the laws of nature were suspended or that people in a credulous age, who wanted to believe in the supernatural, found a way to do so? He asked us to weigh evidence rationally, but, according to the narrator of \emph{Karamazov}, people already committed to one side of the argument will not do so. Even if the miracles were performed right in front of the "realist's" eyes\textemdash "if he were {[}directly{]} confronted with a miracle as an irrefutable fact"\textemdash the demonstration would make no difference to his belief or disbelief. "He would rather disbelieve his own senses."

If that is so, is he truly a realist? Scientific realists supposedly respect the evidence. One does not just deduce or assert a truth, one has to test it by experiment. The experiment can disprove the hypothesis, or it is not a real experiment. As Karl Popper would later say, the hypothesis must be falsifiable. If a scientist is so wedded to a theory that he ignores an experiment that disconfirms it, is he really a scientist at all? Or, to use the narrator's term, a "realist"? As the narrator points out, many who act like real scientists in their day-to-day work find themselves unable to do so when the ideology they associate with science itself is questioned. In \emph{A Writer's Diary} Dostoevsky imagines the chemist Mendeleev, who vigorously denied the existence of supernatural beings, being lifted into the air by demons who perform tricks that preclude scientific explanation. As a respecter of the evidence, Mendeleev would be obliged to acknowledge the supernatural, but would he?

According to the narrator, the scientist has one last resource: to admit the disconfirming evidence he has witnessed, but maintain that this apparently supernatural occurrence is really "a fact of nature till then unrecognized." The problem with this move is that it makes the argument against the supernatural into a tautology: if anything manifestly supernatural is simply called another form of the natural, then naturalism is true by definition.

The passage from \emph{Karamazov} continues:

\begin{quoting}
\noindent Faith does not, in the realist, spring from the miracle but the miracle from faith. If the realist once believes, then he is bound by his very realism to admit the miraculous also. The Apostle Thomas said that he would not believe until he saw, but when he did see he said, "My Lord and my God!" Was it the miracle that forced him to believe? Most likely not, but he believed solely because he desired to believe and possibly he fully believed in his secret heart even when he said, "I shall not believe except I see" (\emph{BK}, 28).
\end{quoting}

\noindent Evidently it is possible to believe and disbelieve at the same time. One may sincerely profess a conviction and yet, like Doubting Thomas, not accept it in one's "secret heart." The narrator supposes that, even though Thomas thought the miracle convinced him, it in fact merely revealed to him what he already believed. If so, it is possible not only to believe and disbelieve simultaneously, but also simultaneously to change one's mind and not change it.

Thomas believed "solely because he desired to believe." Is it really possible to believe something "solely" because one wants to? I cannot decide to believe in Ptolemaic astronomy just because I choose to. And yet one can get oneself to believe what one either knows or easily could know to be untrue. It happens whenever one deceives oneself.

According to the narrator, Alyosha believed in miracles because he had faith. Far from deceiving himself, he was a "realist" believing without reservation in the world as he experienced it. But what if that world should disconfirm his belief? What if he had to choose between his realism and his belief in miracles? That is precisely what happens.

\subsection*{Fyodor Pavlovich: The Liar}

Early in the novel the narrator describes how Fyodor Pavlovich loves to play a part "even to his own direct disadvantage," as he does when Miusov proposes to adopt little Mitya (\emph{BK}, 15). Fyodor Pavlovich's treatment of his infant son has gone far beyond mere neglect: "he wholly and utterly abandoned his child by Adelaida Ivanovna, not from malice towards him or because of any wounded matrimonial feelings, but simply because he completely forgot him" (BK, 14). Describing Fyodor Pavlovich, Dostoevsky sometimes stretches the boundaries of realism: is it possible actually to forget one's child "completely"?

To be sure, Fyodor Pavlovich can be reminded of his son, even if he might pretend otherwise. "Long afterward he {[}Miusov{]} used to recount, as a characteristic trait of the man, that when he began to speak to Fyodor Pavlovich about Mitya, for some time he looked completely uncomprehending, as though he seemed surprised that he had a young son somewhere in the house. If there may have been some exaggeration in Pytor Aleksandrovich's {[}Miusov's{]} story, still there must have been something resembling the truth" (\emph{BK}, 15). One may suppose that Fyodor Pavlovich, when reminded of what he usually forgot, pretended to be still forgetting\textemdash that is, he pretended to be what he really was.

Fyodor Pavlovich takes the greatest pleasure in parodying or mocking anything that anyone regards as sacred. "Great elder, speak!," he proclaims in Father Zosima's cell. "What must I do to gain eternal life?," aping the lawyer's appeal to Jesus (Luke 11:27). One reason it is sometimes "difficult to decide" whether Fyodor Pavlovich is pretending or sincere is that he himself cannot always tell the difference. "With old liars who have been acting all their lives," the narrator explains, "there are moments when they enter so completely into their part that they tremble or shed tears of emotion in earnest, although at that very moment, or a second later they are able to whisper to themselves, 'You know you are lying, you shameless old sinner! You're acting now, in spite of your 'holy' moment of wrath.'"\footnote{\emph{BK}, 68. My thanks to Ryan Serrano, who is writing a splendid dissertation on role-playing, for calling my attention to this passage.}

One can shed tears "in earnest" while knowing "at that very moment" that one is lying. That isn't true with ordinary self-deception. Neither does it resemble Doubting Thomas's mistaken belief in his change of mind. Fyodor Pavlovich "really" feels what he knows is pretense. But not completely, because the sense of having pretended does not altogether disappear.

But it can, as sometimes happens when people choose to feel insulted. "You know it is sometimes very pleasant to take offense," Father Zosima observes to Fyodor Pavlovich. "A man may know that nobody has insulted him, but that he has invented the insult for himself, has lied and exaggerated to make it picturesque, has caught at a word and made a mountain out of a molehill\textemdash he knows that himself, yet he will be the first to take offense, and will revel in his resentment till he feels great pleasure in it, and so pass on to genuine vindictiveness" (\emph{BK}, 43). \emph{And so pass on to genuine vindictiveness}: What begins as pretense becomes genuine feeling\textemdash genuine in the sense that one really feels vindictive and loses the awareness of pretense. And yet the feeling is at the same time fake. We lack a name for this condition.

How much of the world's evil is due to it? As Dostoevsky often shows, people sometimes crave to be insulted and injured so as to feel morally superior. As Dmitri says of his fiancé Katerina Ivanovna, she loves her own virtue, not him. She tries to get him to betray her so she can show her nobility. But is it nobility to rise above an offense one has deliberately provoked? How aware is she of what she is doing? When Ivan hints at her pretense, she falls into hysterics, which suggests she both knows and does not know at the same time.

Katerina Ivanovna lives a lie, but she does so sincerely in the sense that she does not know she lives a lie. What make Fyodor Pavlovich unique\textemdash not only in this book but, I think, in world literature\textemdash is that he knows that his "genuine" vindictiveness, which he really feels, is not at all genuine, and yet is able to continue believing in it without losing the consciousness of falsity. Katerina Ivanovna lies, and conceals the lie from herself: Fyodor Pavlovich is fully aware of lying to himself and yet can indulge in it with all the fervor that she does. Indeed, he does everything he can to exaggerate the contradictions, as a kind of sport.

\subsection*{Dmitri}

Dmitri tells Alyosha that, in the depths of degradation, he experiences ugliness as beautiful. "Beauty is a terrible and awful thing," he confides. "It is terrible because it has not been defined and is undefinable, for God sets us nothing but riddles. Here the two shores of the rivers meet and all contradictions stand side by side. \ldots{} We must solve them as we can, and try to keep a dry skin in the water"(\emph{BK}, 98). One cannot keep a dry skin in the water, or water would not be wet; if shores meet, then they are no longer shores; but some contradictions do seem to stand side by side. How, Dmitri wonders, can he regard something as ugly and beautiful at the same time? What's more, a man with the ideal of the Madonna in his heart\textemdash who believes in it sincerely\textemdash can at the very same moment adhere to "the ideal of Sodom. What's still more awful is that a man with the ideal of Sodom in his soul does not renounce the ideal of the Madonna, and his heart may be on fire with that ideal, genuinely on fire, just as in the days of his youth and innocence" (\emph{BK}, 98). In youth, one might well believe in goodness and beauty without being aware of one's own evil and ugly impulses. Lack of self-knowledge would explain that. But what about simultaneously, and with full awareness, adhering to both with equal sincerity? Isn't that like believing in two mutually exclusive propositions? If I believe it is raining, how can I at that very moment believe it is not raining? If one can believe both, what does it mean "to believe"?

Dmitri simultaneously and knowingly believes in Sodom and the Madonna. That is not supposed to be possible, and if it somehow is, it shouldn't be! "Man is broad, too broad, indeed," Dmitri concludes. "I'd have him narrower." That, of course, is what Ivan's Grand Inquisitor proposes to do.

\subsection*{Ivan's Belief and Disbelief}

At once an extreme moralist and an amoral denier that objective morality exists, Ivan is even more disturbed than Dmitri by his adherence to contradictory beliefs. As an amoralist, he believes that the laws of nature explain everything. They have no moral content: asking whether Newton's law of universal gravitation is moral or amoral makes no sense. Since people are just complex natural objects subject to the same natural laws as everything else, morality can be nothing more than social convention\textemdash necessary for social stability, perhaps, but no less groundless for that. It is one thing on this side of the Pyrenees, something else on the other, as Pascal observed. "Three degrees of latitude upset the whole of jurisprudence and one meridian determines what is true."\footnote{Blaise Pascal, \emph{Pascal Pensées}, trans. A. J. Krailsheimer (Harmondsworth, England: Penguin, 1987), 46.} From this perspective, questions about life's meaning make no more sense than those about morality.

The devil who visits Ivan paraphrases his poem "The Geological Cataclysm," which argues that it will be some time before everyone grasps the amoral truth, but "everyone who recognizes the truth even now may legitimately order his life as he pleases. \ldots{} In that sense, 'all things are permitted' for him." "That's all very charming," the devil retorts, "but if you want to swindle why do you want a moral sanction for doing it? But that's our modern Russian all over. He can't bring himself to swindle without a moral sanction. He's so in love with truth \ldots{}" (\emph{BK}, 546).

But Ivan also believes just the opposite, that good and evil (at least, evil) are absolute. In reciting his horrifying stories of child abuse, Ivan means to show that such evil cannot be argued away or justified by either religious theodicy, rationalist nihilism, or any other theory. Believing firmly in both evil and moral nihilism, Ivan is torn apart.

A keen psychologist, Zosima diagnoses Ivan's problem. The conflict is "unresolved in your heart and torments it," Zosima tells him. "That question is not solved in you, and it is your grief, for it demands an answer" (\emph{BK}, 65\textendash 6). When Ivan asks if his question can ever be resolved "in the affirmative"\textemdash whether he will come to believe in meaning and good and evil\textemdash Zosima replies that "it will never be decided in the negative." That is, Ivan will never become a complacent nihilist, like his father, but will either find meaning or go on searching. "You yourself know that that is the peculiarity of your heart; and all its suffering is due to it. But thank the Creator who gave you a lofty heart, capable of such suffering, of thinking and seeking higher things, for our dwelling is in the heavens" (\emph{BK}, 66).

The devil who haunts Ivan mocks his hesitation between opposite truths by making Ivan also hesitate between affirming and denying the devil's existence. Ivan tells the devil he is only a hallucination\textemdash "It's I myself, speaking, not you"\textemdash but does one address a hallucination? "You don't exist" is surely a paradox. The devil doesn't mind this accusation of nonbeing a bit. As if he had overheard the narrator's discussion of Doubting Thomas, he remarks:

\begin{quote}
"Don't believe it then," said the gentleman, smiling amicably, "what's the good of believing against your will? Besides, proofs are of no help in believing. Especially material proofs. Thomas believed, not because he saw Christ risen, but because he wanted to believe, before he saw" (\emph{BK}, 535).
\end{quote}

The devil taunts Ivan with proofs that he is and at the same time isn't a hallucination. He offers evidence for both possibilities. He also points out that if one has to convince oneself that something is not true, one must believe it at least to some degree. "From the vehemence with which you deny my existence," the devil laughs, "I am convinced you believe in me," if only a tiny bit, and, after all, "homeopathic doses perhaps are the strongest" (\emph{BK}, 542). The devil explains that he leads Ivan "to belief and disbelief by turns" because "hesitation, suspense, conflict between belief and disbelief, is sometimes such torture to a conscientious man, such as you are, that it's better to hang oneself at once. \ldots{} It\textquotesingle s the new method" (\emph{BK}, 542). Ivan's "belief and disbelief" in morality is no less tormenting.

The devil himself claims to reside somewhere between existence and nonexistence. Much as believing while disbelieving is a distinct state of mind, "perhaps-being" is itself a way of being. If only he could either be or not be, the devil pleads, how much better it would be! "What I dream of is becoming incarnate once for all and irrevocably in the form of some merchant's wife weighing two hundred and fifty pounds and of believing all she believes. My ideal is to go to church and offer a candle in simple-hearted faith, upon my word it is" (\emph{BK}, 537). Appropriately enough, this devil is also an agnostic.

Faith, the devil leads Ivan to think, is firm and "simple-hearted," not hesitant and contradictory, but if Ivan really understood Zosima's diagnosis he would realize that sometimes the search for faith can be a kind of faith in itself. In \emph{The Idiot}, Ippolit explains that "Columbus was not happy when he had discovered America but while he was discovering it. \ldots{} It's life that matters, nothing but life, the everlasting and perpetual process, and not the goal at all."\footnote{Fyodor Dostoevsky, \emph{The Idiot}, trans. Constance Garnett (New York: Modern Library, 1962), 375.} In \emph{A Writer's Diary}, Dostoevsky explains that "happiness lies not in happiness but only in the attempt to achieve it."\footnote{Fyodor Dostoevsky, \emph{A Writer's Diary: Volume One, 1873\textendash 1876}, trans. Kenneth H. Lantz (Evanston: Northwestern UP, 1993), 335.} As happiness can be processual, so can faith.

If Ivan would only realize the meaningfulness of questing, he would resemble his creator. In his famous letter to Natalya Fonvizina, who was suffering from depression, Dostoevsky consoled her that "at such moments one thirsts for faith like 'parched grass' \ldots{} I will tell you that I am a child of the {[}materialist nineteenth{]} century. A child of disbelief and doubt. I am that today and (I know it) will remain so until the grave. How much torture this thirst for faith has cost me and costs me now, which is all the stronger in the soul for all the arguments against it."\footnote{I modify the translation of this letter as it is cited in Joseph Frank, \emph{Dostoevsky: The Years of Ordeal, 1850\textendash 1859} (Princeton: Princeton UP, 1983), 160.} Increase in disbelief fuels increase in belief, as the search for faith grows ever more intense.

Dostoevsky next affirms that, in this state of mind, he has formulated his "credo": nothing more beautiful than the image of Christ exists or ever could exist. Of course, belief in the beauty of the image of Christ does not necessarily entail belief in His existence. So Dostoevsky next formulates his oft-quoted paradox: "Even more, if someone proved to me that Christ is outside the truth, and that \emph{in reality} the truth were outside Christ, then I should prefer to remain with Christ rather than the truth." But to believe in something is to believe it is true; how can he believe in Christ yet believe that Christ lies outside the truth?

The answer is that faith as a striving for faith, faith as a process, involves just such a paradox. Faith lies in the attempt to achieve it.

\subsection*{Alyosha's Crisis}

If Ivan suffers from extreme doubt, Alyosha is all too certain. The incidents following the death of Father Zosima "exerted a very strong influence on the heart and soul of \ldots{} Alyosha, forming a crisis and turning point in his spiritual development, giving a shock to his intellect, which finally strengthened it for the rest of his life" (\emph{BK},285). To strengthen his faith, Alyosha must first question it.

Alyosha firmly expects the traditional sign that his mentor Father Zosima was a saint, the incorruptibility of his body. He is more than disappointed: not only does the elder's body not diffuse a sweet smell, it emits the most pungent "odor of corruption" even more rapidly than usual. All those who felt "jealousy of the dead man's saintliness" and who out of jealousy had even succumbed "to an intense hatred of him," rejoice at his downfall. It seems to Alyosha that either there was no miracle or, because the smell was "in excess of nature," a reverse miracle. Alyosha wonders resentfully, "Why did providence hide its face 'at the most critical moment \ldots{} as though voluntarily submitting to the blind, dumb, pitiless laws of nature?" (\emph{BK}, 293).

Rakitin reports that Madame Khokhlakova wrote him a note saying "that she would never have expected \emph{such conduct} from a man of such a revered character as Father Zosima. 'That was the very word she used: conduct.'\," The joke, of course, is that she treats the rapid decay of Zosima's body as something he was choosing to do. But in a sense it is: this is the action that Zosima would have chosen in order to teach Alyosha not to place his faith in miracles.

The narrator comments: "If the question is asked: 'Could all his grief and disturbance have been due only to the fact that his elder's body had shown signs of premature decomposition instead of at once performing miracles?' I must answer without beating about the bush, 'Yes, it certainly was'\," (\emph{BK}, 292). Alyosha demands a miracle: that is precisely what Christ refuses to do in his second temptation. As the Inquisitor paraphrases it: "If Thou wouldst know whether Thou art the Son of God then cast thyself down, for it is written that the angels shall hold him up lest he fall and bruise himself." As fully human, Jesus cannot be absolutely sure that he is who he thinks he is, but he refuses to resolve doubt by demanding a miracle.

A miracle would eliminate uncertainty, but then people would simply be acknowledging power, not demonstrating faith. As the Inquisitor explains, "Thou wouldst not enslave man by a miracle, and did crave faith freely given, not based on miracle. Thou didst crave for free love and not the base raptures of the slave before the might that has overawed him forever" (\emph{BK}, 222). Faith based on power that precludes doubt is not faith at all because it is not "freely given," not a matter of choice; and choice can be real only when there is uncertainty.

From the Inquisitor's point of view, certainty would make people happier. Jesus therefore should have accepted the temptation and performed a miracle, but He didn't. "Instead of giving a firm foundation for serving the conscience of man at rest forever, Thou didst choose all that is exceptional, vague, and enigmatic" (\emph{BK}, 223). And instead of following clear and infallible rules, "man must hereafter with free heart decide for himself what is good and evil, having only Thy image before him as his guide." An image inspires, but it does not offer infallible prescriptions.

In demanding miracles, then, Alyosha misunderstands what faith is. He experiences his "critical moment," but in the next chapter, he recovers faith by consoling Grushenka in \emph{her} "critical moment." When Alyosha returns to the monastery and listens to Father Paissy reading the Gospel over Zosima's corpse, he dreams that Father Zosima has praised this very act as truly Christian. The story Father Paissy reads, the marriage at Cana, recounts Jesus's first miracle, which no one but Jesus, Mary, and their servants detect. It is therefore a secret miracle, one that does not call attention to itself. As Alyosha now understands, that is also true of our loving kindness and compassion to each other, which no rationalist theory could explain except as a sort of bargain. Active, inconspicuous love is the true miracle.

When Madame Khokhlakova demands proof of God and immortality, Zosima tells her that "there is no proving it" but that, if one lives the right sort of life of active love, one can be convinced of it. The real miracle is the one always available to us. One is convinced without proof, and since love always flickers, doubt inevitably abides with faith. The narrator refers to Alyosha as his "future hero" because he plans to describe the many tests and temptations, doubts and renewals, that Alyosha will undergo. His faith will always be a search for faith.

"Denying, believing and doubting are to men what running is to horses," observed Pascal, who, like Dostoevsky, considered why God did not give indubitable proof of his existence (Pascal, 208). "If there were no obscurity man would not feel his corruption: If there were no light man could not hope for a cure," Pascal explained. "Thus it is not only right but useful for us that God should be partly concealed and partly revealed" (Pascal, 167). That is why the atheist objection that "there is nothing in the world which proves" God's existence misses the point. God is, and must be, \emph{deus absconditus} {[}the hidden God\textemdash Isaiah 45:15{]}.

Meditating on the death of his first wife, Dostoevsky recorded his thoughts about life and faith as constant striving. The atheists object that if Christianity is true, why does brotherhood not reign on earth? Dostoevsky answers: because so long as man is on earth he lives in a world of "struggle and development" and that life itself is always "developing." Humanity is always incomplete, always becoming itself, and so "on earth man is in a transitory state."\footnote{Fyodor Dostoevsky, \emph{The Unpublished Dostoevsky: Diaries and Notebooks (1860\textendash 1881)}, vol.~1, trans T. S. Berczynski, Barbara Heldt Monter, Arline Boyer, and Ellendea Proffer (Ann Arbor: Ardis, 1973), 39. Further references are to \emph{UD}.} People cannot completely fulfill the ideal of loving others because it conflicts with "the law of \ldots{} personal identity. \ldots{} And thus on earth mankind strives for an ideal opposed to his nature" (\emph{UD}, 41).

Abiding in uncertainty and blending affirmation with doubt, faith reflects our "transitory" nature. It is necessary "to develop, to attain, to struggle, to glimpse the ideal through all one's falls and eternally strive towards it": that is what life on earth is (\emph{UD}, 39). What Ivan never comprehends, but Alyosha learns, is that one would not so ardently seek faith if one had not already found it.

\vspace{1em}
\begin{center}
  \includegraphics[width=0.75cm]{articlend.png}
\end{center}


\biobox{\textbf{Gary Saul Morson}, the Lawrence B. Dumas Distinguished Professor of the Arts and Humanities and Professor of Slavic Languages and Literatures at Northwestern University, is an author or editor of twenty-one books—three with Morton Schapiro, two with Caryl Emerson, and one with Elizabeth Allen—as well as some 300 shorter publications on Russian and comparative literature, the philosophy of time, the relation of the humanities to economics, utopia and anti-utopia, the role of quotations in culture, and the aphorism as a literary/philosophical genre. His most popular class has enrolled over 500 students, when it was the largest course at Northwestern and the best enrolled Russian literature course in North America. A member of the American Academy of Arts and Sciences since 1995, he has won numerous awards for teaching and scholarship. His most recent book is \textit{Wonder Confronts Certainty: Russian Writers on the Timeless Questions.}}

\label{sec:morson}

\fancypagestyle{chaptercontentpage}{
  \fancyhf{} % Clear all header and footer fields
\fancyhead[CE]{%
  \fontsize{11}{11}\leftmarkfont%
  \addfontfeature{LetterSpace=10.0}%
  \textit{\MakeUppercase{\leftmark}}%
}
  \fancyhead[CO]{\authorheadfont\addfontfeature{LetterSpace=20.0}\textbf{{\vspace{-0.2em}\uppercase{Mikhail Epstein}}}}
  \renewcommand{\headrulewidth}{0pt} % No header rule on content pages
  \fancyfoot[RE]{\thepage}
  \fancyfoot[LO]{\thepage}
}
\setcounter{footnote}{0}
\section{Momentous Intersections}

\abstractbox{Momentous Intersections}{A Comparative View on Russian and Jewish Spiritual Traditions}{Mikhail Epstein}{This article is a philosophical and admittedly subjective attempt (drawing on the author's life experience) to discern general features of the Russian and Jewish national character and their interaction in the twentieth century. Several common traits of Jewish and Russian spirituality (or religious psychology) and national identity are discussed. These include messianic strivings and the belief in being God's chosen people, which simultaneously unify and divide both nations; a tendency to utopian thinking; adherence to the ideals of social equality and justice; a feeling for the national element with which personality becomes merged and thereby finds itself; a tendency to wandering, to rootlessness, and to nomadism; universalism and the easy assimilation of cultural customs; and centuries-long experience of suffering and persecution. The author's reflections on Jewish and Russian national and spiritual identity draw on a broad range of writers, including Nikolai Gogol, Fyodor Dostoyevsky, Vladimir Solovyov, Dmitry Merezhkovsky, Nikolai Berdyaev, Vasily Rozanov, Osip Mandelstam, Boris Pasternak, and Gershom Sholem.}{Russian spirituality, Jewish spirituality, Russian national character, Jewish national character, Russian messianism, Jewish messianism, notions of the "chosen people"}
\newpage
\chaptertitle{Momentous Intersections}{A Comparative View on Russian and Jewish Spiritual Traditions}{Mikhail Epstein}

\fancypagestyle{chaptertitlepage}{
  \fancyhf{} % Clear all header and footer fields
  \fancyhead[L]{\begin{minipage}[t]{0.7\textwidth}\publisher\end{minipage}}
  \fancyhead[R]{\begin{minipage}[t]{\textwidth}\raggedleft \datefont\fontsize{10}{11}\selectfont Volume 1 (2024): \thepage\textendash\pageref{sec:epstein}  \\ \doi{10.71521/96cd-fy63} \end{minipage}}
  \renewcommand{\headrulewidth}{0pt} % No header rule on title pages
  \fancyfoot[LE, RO]{\thepage} % Left on even pages, right on odd pages
}
\thispagestyle{chaptertitlepage} % Apply custom page style for chapter title

\addcontentsline{toc}{chapter}{Momentous Intersections: \\ A Comparative View on Russian and Jewish Spiritual Traditions \\ \textit{by} Mikhail Epstein}

\seriffont\fontsize{12}{18}\selectfont

\epigraph{}{
\emph{Tell me, a draughtsman of the desert,\\
a geometer of shifting sands:\\
is the unrestrained freedom of lines\\
more powerful than the blowing wind?\\
\textemdash I am not affected by the tremor\\
of his Judean concerns:\\
he molds his experience from murmur\\
  and drinks his murmur from experience.\\
  \textbf{Osip Mandelstam}
}}
\vspace{2em}


\subsection*{A Draft of a Human}\label{a-draft-of-a-human}

\noindent How can one write about nations from the lofty spiritual vantage point required to understand them without succumbing to the worst vices of nationalism or to intellectual self-indulgence? These jottings are written neither from the perspective of the scholar nor of the historian, but are rather in the nature of philosophical essay, being a subjective attempt to discern general features of the Russian and Jewish characters and of their interaction in the twentieth century. To paraphrase Mandelstam, I should define the genre of the present contribution as "the murmur of my experience."

If I could sum up my sense of Jewishness in a single word, it would be "vibration." As Mandelstam wrote about Solomon Mikhoels: "The entire strength of Judaism, the entire rhythm of dancing abstract thought, all the pride of the dance, the sole stimulus for which, in the last analysis, is the feeling of compassion for the Earth\textemdash all of this turns into quivering of the hands, into the vibration of thinking fingers, which are imbued with spirit, like articulate speech."\footnote{Osip Mandelstam, "Mikhoels," \emph{Collected Works}, 3 vols. (New York: Mezhdunarodnoe literaturnoe sodruzhestvo, 1969), 108.}

Mandelstam's observation is not merely physiognomic, but is also metaphysical, and is well in accord with the central image of his octave (see the epigraph): "the tremor of Judaic concerns." Judaism's "concerns" (in all the manifold meanings of that word pertaining to relations with the Divine and one's fellow human beings) are "tremors," both in the literal sense of "tremblings," but also "quiverings," the "vacillations" between constantly reappraised possibilities, between flight into the future and return to the present, the constant need to revise the life's course that has been roughly mapped out, to redraw it and re-plan it. This metaphor of Judaism's "concerns" arises in the context of shifting sands, the quaking, crumbling flesh of the earth, which imparts the same "tremor" of lines in headlong flight\textemdash for the draughtsman of the Jewish desert is the wind, probably the most "tremulous" of all draughtsmen. If Mandelstam's sketch of Mikhoels offers us a \emph{portrait} of vibration, then his octave portrays its \emph{landscape}. The quiver of hands and the "thinking fingers" correspond to the "shifting sands." The same image recurs in the line, "He moulds his experience from murmur." For a "murmur" is a quivering of the lips, of halting speech, a preliminary attempt at articulating sound and sense. Indeed, the word "experience" is etymologically related to "experiment," a meaning also relevant here, in the sense of an "essay"/"assay," a "trial," a vacillatingly hypothetical orientation towards the world. In the original Russian, all of these words\textemdash \emph{za\textbf{bot}a} {[}concern{]}, \emph{o\textbf{pyt}} {[}experiment/experience{]}, \emph{tre\textbf{pet}} {[}tremor/quiver{]}, \emph{le\textbf{pet}} {[}murmur{]}\textemdash not only have acoustic correspondences, but also evince correspondences in meaning, in that they convey the oscillations of the soul and the body, of thoughts and speech. \ldots{} Mandelstam discovered the unique images and words to represent Jewishness as vibration, both phonetically and semantically.

The fundament of Jewish spiritual life trembles, quakes, as if animated by continuing volcanic activity under the surface of the earth. The constant quaking, quivering, or rather, tremulousness of the Jewish sense of life finds its most forceful expression, in my view, in the poetry of Afanasy Fet and Boris Pasternak. It is all the more authentic for being unconscious, as neither acknowledged their Jewishness\textemdash indeed, they shunned it. "Trembling" and "quivering" were the epithets that Fet most frequently applied to natural phenomena. "The moon shines tremulously" ("\emph{Tikhaia, zvezdnaia noch} \ldots{}"), the "ardent light" of sun "plays quiveringly on the leaves" ("\emph{Ia prishel k tebe s privetom} \ldots{}"), "the chorus of stars quivers" ("\emph{Na stoge sena} \ldots{}"), "the leaves and stars quiver" ("\emph{Solovei i roza}"), "everything quivers and sings in spite of itself" ("\emph{Chto za vecher} \ldots{}"). In this general tremor of life, the lyrical persona joins, every fibre of his being a-quiver: "I hear the beating of my heart and the quiver of my hands and feet" ("\emph{Ia zhdu \ldots{} Solov\textquotesingle inoe ekho} \ldots{}")

The most characteristic state of the Pasternakian hero is precisely the same kind of quivering, as if his soul is becoming a single tremulous spark: "I shuddered. I blazed up and was extinguished. \ldots{}" ("Marburg"). "I would break open verse, like a garden, with all the quiver of veins \ldots{}" ("\emph{Vo vsem mne khochetsia doiti} \ldots{}"), "Nature, the world, a universal hiding place / I will remain steadfast in your long service / held fast in a secret trembling / In tears of happiness" ("\emph{Kogda razguliaetsia}"). The state of being imbued with the utmost plenitude of life extends even to nature: "The very nightingales would roll their eyes with a shudder. \ldots{}" ("\emph{Osen\textquotesingle{}}"). This quivering is the sparking of the spirit through every particle of the universe, its flight and return as the incessant flickering of life, of its possibilities that flare up and die out.

This quivering betrays the Jew's dependence on his Creator, his Jewish "fear and trembling." Kierkegaard's \emph{Fear and Trembling}, the \emph{fons et origo} of European existentialism, is in essence a depiction of Abraham, a proto-Jew. He converses, with trembling hands, and thinks, his thoughts a-tremble. Europeans and Americans keep face, they do not crumble under pressure, the contours of their physiognomy do not fragment, but are drawn clearly, as if with a ruler. It seems to me that Jews are set apart by the frequency and amplitude of their waverings and of their psychic impulses. A Jew constantly thinks and re-thinks his thoughts, feels and re-feels his feelings. On this account, he can make an impression of psychomotoric "tremulousness," bustling and pottering about, or in Saltykov-Shchedrin's phrase, "of hare-like hastiness, which prevents a Jew from sitting still for a moment."\footnote{Saltykov-Shchedrin, "July Breeze" ("\emph{Iul\textquotesingle skoe veianie}"), 1882. This satiric essay is one of the strongest statements by Russian writers of the 19th century in defense of Jews. \url{https://lechaim.ru/ARHIV/87/salt.htm}} But this is only on the surface. This "hastiness" uproots a Jew from the soil of reality and bears him off into the realm of vacillating contours, where reality consists entirely of glinting possibilities. This accounts for the success of Jews in trade, in financial dealings, on the stock-market, where one passes from the natural relations of the tangible world into the realm of possibilities, of vaulting magnitudes, of symbolic relations, where interpretations remain in flux. The Jews are the people of the Book, and not of nature; and the book, moreover, in contrast to nature, is a world of possibilities\textemdash unrestricted and volatile, which can never be entirely realised. For the Jew, reality is only still in the process of creation, remaining a thrilling possibility, constantly being corrected and remade, now being underlined, now being crossed out. The Jew is an unfinished sketch, a \emph{draft of a man}; he is nowhere to be found, his place is unoccupied\textemdash he is merely a possibility.

I was once told about a French art collector who attended an exhibition of work by Chaim Sutin and other Jewish artists and summarised his impressions thus: "not a single smooth line." It is known that Walter Benjamin, a major twentieth-century German-Jewish thinker, "did not like round numbers and straight lines. When observing the world around him, he found nothing straight in it: straight lines only existed in philosophy, whose predisposition not to notice crookedness and fragmentariness, to ignore discontinuities and conceal voids he tried to resist" (Denis Sobolev).\footnote{Denis Sobolev, \emph{Evrei i Evropa} (2008), ch.19: "Walter Benjamin: mezhdu iazykom i istoriei." \url{https://www.e-reading.co.uk/chapter.php/1004520/47/denis-sobolev-evrei-i-evropa.html}.} Finally, the most famous Jewish person of the twentieth century, Albert Einstein, is renowned for having discovered the curvature of time and space.

The word "Jew" has the unusual quality of combining with the names of other nations. Jews can be Russian, German, Polish, American, even Chinese and Egyptian, without ceasing to be Jews. Jews are always a "dual" nation: its designation requires another epithet, and sometimes even two, as in "American-Russian Jew." Only in the last sixty years has there been the Jew-as-such, although they are also called "Israeli Jews," as if there were a suspicion that "Jewish" and "Israeli" could not be one and the same thing but were rather a variety of dual national allegiances. To be a Jew always means to be someone else. The Jew is always more himself as a Jew: it is as if his essence acquires a "mark-up," the price for which is estrangement from his own roots.

Thus, the Jew is easily assimilated into other nations and can be transplanted into the soil of other cultures. This is not the grafting of one reality onto another, but the grafting of possibilities onto reality. This assimilation is simultaneously a problematisation. The Jews introduce into other cultures the realm of other possibilities, like a question and a hypothesis introduced into the circle of established realities. I am reminded of the little linguistic anecdote: "Why does a Jew always answer a question with another question?\textemdash But why not?" The point is not the separately posed "Jewish question," but rather the fact that Jewishness is itself a question, posed to itself and to others. The question is not only about Jewishness as such, but also about the foundations of the life of other nations, their beliefs, persuasions, institutions, and their social and cultural establishments. Thus, in his marvellous article, Vladimir Solovyov substituted for the stereotypical phrase "the Jewish question" a much better one\textemdash "Jewry and the Christian Question"; and Dmitry Merezhkovsky subsequently entitled one of his articles, "The Jewish Question as a Russian Question."\footnote{See Solovyov, "Jewry and the Christian Question" (1884), in Vladimir Solovyov, \emph{The Burning Bush: Writings on Jews and Judaism}, ed., trans., and with commentary by Gregory Yuri Glazov (Notre Dame, IN: University of Notre Dame Press, 2016), 277\textendash 329; and Dmitry Merezhkovsky, "The Jewish Question as a Russian Question" (1916), in \emph{A Revolution of the Spirit: Crisis of Value in Russia, 1890\textendash 1924}, ed.~Bernice Glatzre Rosenthal and Martha Bohachevsky-Chomiak, trans. Marion Schwartz (New York: Fordham University Press, 1990), 222\textendash 224.} In their relations with other nations, Jewry simultaneously casts doubt on both them and itself, so that everything shifts and a new day of creation dawns. The Jewish question arises everywhere simultaneously with Russian, German, Arab questions\textemdash and also simultaneously with a host of other questions in science, literature, music, politics, philosophy, religion. \ldots{} The Jews are like inquisitive children still at the age when they persistently raise questions about "hows" and "whys," spontaneously asking about matters that seem self-evident and settled for "adults." It is no accident that the very word "Jew" in its etymological derivation from Hebrew means "from another or from the opposing side," "not of these parts." This affords the possibility of seeing the world as strange, of being surprised by the most generally accepted truths and deeply rooted customs, and of problematising the existence of other nations and of mankind as a whole.

" \ldots{} The Jew is the spirit of negation, a protest against the dogmas of creeds \ldots{},"\textemdash remarked the American rabbi Isaac Wise.\footnote{Isaac M. Wise, \emph{Selected Writings of Isaac M. Wise}. With a Biography by the editors David Philipson and Louis Grossmann (Cincinnati, OH: The Robert Clarke Company, 1900), 182. \url{https://collections.americanjewisharchives.org/wise/attachment/5307/IMWise_selected_writings.pdf}} Eric Fromm and Martin Buber ascribed to a similar view: the Jews are ostensibly staunch iconoclasts but are not as renowned for creation. But a question only signifies negation and revolutionary overthrow in its most primitive forms; in its essence, a question is a summons to create. In the words of Ernest Renan, "the true Israelite is a man who is tormented by dissatisfaction, in the grip of an unquenchable thirst for the future."\footnote{Quoted in Nikolai Berdyaev, \emph{Smysl istorii. Opyt filosofii chelovecheskoi sud′by}, 2\textsuperscript{nd} ed.~(Paris: YMCA Press, 1969), 119} From the Jews emanate waves of probabilities from which the world is made, accomplishing one thing, discarding another\textemdash they have not yet been fashioned into a smooth line of things in existence. The Jew is himself a process of taking decisions about the world, a process in which there is also a place of for indecisiveness and vacillation. The Jew senses within himself the constant pulsation of God's free will, poised between the multiplicity of possible world-orders rather than choosing one alone. The Jew is like a fontanelle in the cranium of humanity, the incompleteness of all contours of personality and history. The Jew is a lump of primordial clay: man is still being moulded from it, it is still being squeezed and kneaded in the hands of the living God. In the Jew, reality is still fermenting and being raised by the yeast of the possible.

In a famous speech, the Renaissance thinker Giovanni Pico della Mirandola spoke thus of the dignity of man: "God took man as the creation of an indeterminate image, and having placed him at the centre of the world, said: 'I give thee, Adam, neither an appointed place nor thine own image, nor even an appointed duty, so that thy place, thy face and thy duty shall be according unto thine own desire, thine own will and thine own choice. The image of other created things is determined within the limits of the laws that I have made. Thou alone art not bound by any limits\textemdash thou shalt fashion thine own image according to thy wish.'\,"\footnote{Pico della Mirandola, "On Human Dignity," \emph{Istoriia estetiki}, 5 vols. (Moscow: Izd. Akademii khudozhestv, 1962), vol.~1: 507\textendash 508.} If the freedom has been given to man by God to fashion himself according to the likeness of other beings, from the beasts to the angels, then the Jew has been given the same freedom in relation to humanity as a whole, and can choose amongst other nations. He can fashion himself in the likeness of a Spaniard or a Frenchman, a German or a Greek, a Pole or a Russian, taking on different national traits and remaining all the while himself, a Jew, as a man remains a man precisely on account of the freedom of self-determination amongst other beings. This is the origin of the diasporic tendency of the Jews, that transnational excess of humanity which was revealed in them through persecution and expulsion: the capability to soak up, like a sponge, the emanations of other cultures, to become impregnated with them and to create them anew, imparting to them the quality of universal humanity.

Precisely the same traits of swift responsiveness, imitation, and universal sympathy are also widely discernible amongst Russians. Russia is close to the Jew, because she too is still in an embryonic state\textemdash the first day of creation has not yet dawned for her, but everything is in preparation for it. As Berdyaev observed: "There is that in the Russian soul which corresponds to the immensity, the vagueness, the infinitude of the Russian land, spiritual geography corresponds with physical. \ldots{} For this reason the Russian people have found difficulty in achieving mastery over these vast expanses and in reducing them to orderly shape."\footnote{Nicolas Berdyaev, \emph{The Russian Idea}, trans. R. M. French (Boston: Beacon Press, 1962), 2.} Russian spirituality has been and remains the spirituality of the first day of creation. "The Earth was without form and void, and darkness ruled upon the face of the deep. And the Spirit of God moved upon the face of the waters" (Genesis 1:2). This accounts for the characteristic Russian proclivity to "hover over the abyss," poised between boldness and spleen, flaring up and dying out, weariness induced by the surrounding void and the immense task of world creation. Prior to fashioning the stuff of which the world will be made, this Spirit is in a state of agitation and feverish ferment, procumbent on the stormy face of the waters in the dark maw of the abyss. Here the Spirit, as yet unresolved into differentiated elements, is still at its most whole, powerful, and menacing; it does not have a "form," like the most endless of plains, where, in Gogol's words, "everything is even, like an open wilderness \ldots{} there is nothing to beguile and charm the gaze." And further on: "But what kind of inscrutable and secret strength draws one to you? Why can one hear resounding incessantly in one's ears your melancholy song, which is borne across your entire length and breadth and from sea to sea? What is crying out and wailing, and tugs at the heart?" (\emph{Dead Souls}). This Spirit is imbued with an inscrutable and secret strength because it has as yet given form to nothing and is itself unformed, but like a bird-troika or a word-song without origin, hovers over the depths, and gives no answer to the query whither the troika is rushing or why the song is resounding. Incidentally, the Biblical phrase "the Spirit of God moved" is rendered in the original Hebrew by a verb signifying the soaring of a bird; so it is not without reason that Gogol, contemplating "poor, scattered and disconsolate" Rus′ from his strange and remote vantage point, sees "a bird-troika" in flight over it.

Russian religious thinkers and prophet-philosophers similarly proclaim that Russia needs another religion, not Orthodox Christianity or even Christianity itself, which discloses the abyss of the spirit, be it prehuman, superhuman or extra-human. Merezhkovsky observed in Dostoyevsky's work "a contradiction between this (religious) outlook, which wishes to be Orthodox at all costs, and unconscious experiences which cannot be accommodated within Orthodox Christianity. \ldots{} But the true nature of his religion, if he is not yet conscious of it and it consists of experiences at a profoundly unconscious level, is absolutely not Orthodox Christianity, not historical Christianity, even not Christianity at all; rather, it goes beyond Christianity and the New Testament\textemdash it is the Apocalypse, the Future Third Testament, the revelation of the Third Person of the Holy Trinity: the religion of the Holy Spirit."\footnote{Dmitry Merezhkovsky, \emph{V tikhom omute. Stat′i i issledovaniia raznykh let } (Moscow: Sovetskii pisatel', 1991), 321, 345.}

Thus, Russian art is fundamentally a "spiritual" art\textemdash and not merely the art of Gogol, Dostoyevsky, and Tolstoy, but also that of Gorky, Platonov, Mandelstam, Pasternak, Chagall, Brodsky, Il′ya Kabakov, Mikhail Shvartsman. And not because art is placed in the service of extrinsic religious aims and ideals, but rather that art assumes the role of religious revelation which cannot be accommodated within the dogmas of established religious creeds. This understanding of art does not even view it as a process of form-creation, that is, the skilful creation of perfect and complete forms, but rather as "artisticity," the dissolution of the created world by means of the artist's vision, the outbursts of the spirit that assail the object out of a vague urge to transform it. Russian "artisticity" (as distinct from Western art), like Russian "philosophicity" (as opposed to Western philosophy) are expanded and indeterminate mediums for the exploration of those areas of human experience that remain unexplored by religion\textemdash spiritual languor and vexation of spirit, enquiry into the ultimate meanings of existence for which no religious doctrine provides answers.

This accounts for "holiness" being the most important and indivisible category of Russian spirituality, and also its concepts of "integrity" and "totality." "Totality" precedes the division into good and evil, into beauty and ugliness, into truth and falsehood. "In its polarities and contradictions, the Russian people can only be compared to the Jewish people. And it is no accident that these peoples have a strong messianic outlook."\footnote{Nicolas Berdyaev, \emph{The Russian Idea}, trans. R. M. French (Boston: Beacon Press, 1962), 2.}

\subsection*{Jewishness in Russians, Russianness in Jews}\label{jewishness-in-russians-russianness-in-jews}

\noindent Who in the world can understand the Russians better than the Jews? Both are warm, soulful peoples. They share a past history of hermetic isolation from other peoples (the Pale of Settlement and the Iron Curtain), and through their experience of centuries of solitude, of being cut off from the world, they accumulated a great deal of internal warmth.

Let us set out the most striking of the traits that they have in common:

\begin{enumerate}[nosep]
\def\labelenumi{\arabic{enumi}.}

\item The religious strivings of Jews and Russians. Vladimir Solovyov adduced three reasons why the Jewish nation begot Christianity, and to a certain degree they can be also applied to the Russian nation: 1) the intensity with which religious preoccupations pervade national life; 2) the spontaneity and independence of the nation, its vitality and purposefulness; 3) religious materialism\textemdash a system of religious rules separating the pure from the impure. A profound relationship to sanctity, including the sanctity of the material aspects of existence\textemdash the routine of church rituals extended to everyday life, fasting, utensils; the laws of the Torah and the kashrut, of ritual cleanliness. The warmth of the material aspects of life in God. I quote the words of the Russian thinker and scholar of spiritual life Georgy Fedotov: "Christianity \ldots{} is steadily converted into a religion of sanctified matter: icons, relics, holy water, amulets, communion bread and \emph{kulich} {[}Easter cake{]} \ldots{} This is ritualism, but ritualism of a terribly demanding and morally effective kind. In his religious ceremonies, the Muscovite finds support for heroic feats of self-sacrifice, as the Jew does in the Law."\footnote{Georgy P. Fedotov, "Pis′ma o russkoi kul′ture," in \emph{Sud′ba i grekhi Rossii. Izbrannye stat′i po filosofii russkoi istorii i kul′tury}, 2 vols. (St Petersburg: Sofia, 1991), vol.~1: 174, 175.}

\item A tendency to utopian thinking, the subjection of all life to unified principles and ideals which are destined to be attained in the distant future: "live for the sake of one's children, for the sake of the happiness of future generations" and so forth. As Dostoyevsky observed, "universal happiness is something indispensable for the Russian wanderer if he is to find peace: he will not settle for less\textemdash naturally, while the matter remains in the realm of theory" ("Pushkin Speech," 1880). The enormous importance of fantasy, fairy tales, miracles, dreams in the national cast of mind. Both Jews and Russians are peoples of inspiration and Revelation, rather than of empirical ratiocination.

\item Adherence to the ideal of social equality and justice, readiness to root out aristocratic privileges even at the cost of individual freedom. A socialistic, egalitarian outlook, a thirst to reorder the entire world in accordance with religious-social teachings. Millennialism, eschatologism. A striving for the end of history, for the eternal kingdom of Truth. A revolutionary outlook. A sense that there is nothing that could not be parted with, that there is nothing to lose but one's chains. Marx and Bakunin, their mutual suspicion of one another as prudent revolutionary-accountant and revolutionary-anarchist, preparing himself and the entire world to be turned upside-down. And although Marxism triumphed in Russia, Bakuninism emerged from it and under the name of "Leninism" reconciled it with itself.

\item A feeling for the national element with which personality becomes merged and thereby finds itself. The tradition, long-preserved well into the twentieth century, of communal life\textemdash by the world of the Russian peasant mir community and the Jewish shtetl. The collective farm and the kibbutz. The abundance of legends and anecdotes, rumours and gossip, communal living, whispering in corners, the habit of discussing everyone, neighbourliness, the small town and the village. \emph{Skaz}, tales, oral narratives, the strong folkloric basis for culture.

\item A tendency to wander, to rootlessness, the nomadic element in Russian civilisation which was conditioned by the sheer extent of the territory. To quote Chaadayev: "In our homes it is as if we are assigned billets; in families we look like a kind of stranger; in cities we are like nomads \ldots{}"\footnote{Peter Chaadaev, \emph{Filosoficheskie pis\textquotesingle ma} (1836). Letter 1. \url{https://www.vehi.net/chaadaev/filpisma.html}.} Or Dostoyevsky: "These homeless Russian wanderers continue right up to the present day to wander, and it seems that this practice will not die out for a long time" ("Pushkin Speech"). The fate of the Wandering Jew is close to the Russian heart. The tendency of Russians to disperse is also attested geographically in the twentieth century by streams of emigrants, refugees, and defectors. These constitute one of the largest and culturally rich diasporas in the work, alongside the Jewish diaspora. The Jewish diaspora acquired in Russia the first homeland of socialism in the world\textemdash and the Russian diaspora, having been ejected from its homeland by the victory of socialism, went on to create a diaspora all over the world.

\item Universalism, the easy assimilation of cultural customs, of the scientific and technical achievements of other peoples, imitativeness, a flexibility of mind conditioned by the great historical experience of merging, co-existing with other nations. More than a hundred nations settled in Russia, and the Jews in the course of their wanderings settled on the lands of many nations. Both the Russians and Jews, like gifted actors, are adept at taking on alien roles.

\item Centuries-long experience of suffering and persecutions (the Egyptian and Babylonian captivities, the Tatar-Mongol yoke, the Pale of Settlement, serfdom \ldots{}). From this, a tendency to melancholy, to despair, to grieve. In contrast to Europeans, they willingly disclose and share negative emotions, complain about life, and even flaunt their failures. (But the Jews, all the same, are more closed and guarded\textemdash it is not done to talk aloud about death or serious illnesses.)

\item The combination of melancholy with gentle humour, lyrical enthusiasm, an acceptance of life\textemdash laughter through tears (as in Gogol, Chekhov, Babel, Sholem Aleychem, Perets Markish). Irony and self-irony. The habit of laughing at everything\textemdash and first and foremost at oneself, but all the while preserving the sacred in one's soul.

\item Psychological openness, sociability, fondness for conversation, emotionality, animation, heartiness, being easily roused to the point of being highly strung, a readiness to share one's feelings, to trust what is innermost, to discuss one's personal life. Credulousness and gullibility\textemdash with the difference that the Jews tend to believe other Jews, and the Russians to believe foreigners.

\item Both Jews and Russians are logocentric, "literary" nations, for whom the Book, the written word, is the source of higher religious and moral authority. In Russia, the classics\textemdash Pushkin, Gogol, Dostoyevsky, Tolstoy\textemdash assume the role of "Sacred texts"; they contain commandments and homilies and provide spiritual guidance both in social and personal life. The whole culture of both nations is founded on the Word\textemdash "holy Russian literature," holiness and the commandment-like nature of the word. As Chernyshevsky remarked, "literature is the textbook of life." To become absorbed in a book to the point of frenzy, distraction, of voluptuous lassitude. The world in a fog of words.

\item The love of the Jews for Russian world of nature and for poetry, which they treat as their spiritual property. Breadth and spaciousness\textemdash in these there is something fated for Jews in their wanderings and dispersals; not a foreign land, but a new homeland, which must become the entire world. The melancholy of the Russian autumn, the purity of tints, the transparency of the air, the nakedness of the plain and the immensity of the sky\textemdash all of these are consonant with the Jewish heart, a fact which finds incomparable expression in the work of Isaak Levitan. Nature, purged of a sense of confinement and of heat, the open heavens and the shining of its stars, the light, airy, and not exalted, humbly prostrating itself before God. This world is quite different to the biblical one but is nonetheless dear to that part of the Jewish soul which is in quest for itself in the diaspora and in dispersal, and partially finds itself in Russianness.

\item A musicality of soul, a love of folk melodies, for choral singing, for folk songs and dances. In the kibbutz, round-dances are performed as in Russian villages. Chekhov's \emph{Rothschild's Violin}\textemdash the Russian bequeaths to a Jew his music and his sorrow. (Amongst Russians, the guitar is more popular, one encounters more drunken sincerity and love-songs; amongst Jews, the violin tugs at the heartstrings.) Dunayevsky, the brothers Pokras, Fel′tsmann, Basner, Blanter and other Jewish composers turned out to be the creators of the genre of the popular song in the USSR.
\end{enumerate}

In spite of all the persecutions of the Tsarist and Soviet times, the Jews nonetheless took root in Russia, in the midst of a reckless, "irregular" people, where their love for the winding paths of life could manifest itself fully. It comes naturally to the Jews to do everything in a way that is not entirely free from cunning, with a "catch" somewhere, with evasions and "notwithstandings"; but in Russia, this was the honest and normal means of living, so they found themselves in their native element, where they could pin their hopes not on law and order, but only on the keenness of their wits and on God's grace. God and nation, Tsars and prophets, wars and executions, prayers and miracles, cruelty and clemency, intemperate violence and unrestrained repentance\textemdash here the Jew found a passionate, fierce Biblical world which had long since disappeared in the enlightened West, but which was in a strange way revived for them in the Russia of the nineteenth and twentieth centuries. In this immense country, the Jews found their Leviathan, their fear and trembling, their beast from the abyss described in the book of Job as being "above God's ways" and simultaneously as a pitiless monster: "all hope is in vain: shall you not die from a single glance of his?" In Russia it is easier than in any other country to sense the quaking of all the foundations of existence, the upheavals of the historical soil, as before an earthquake\textemdash the rumble travels all over the earth, announcing the tread of a jealous and vengeful God. This constant presentiment of catastrophe, this menacing emanation from other worlds formed part of the Jewish fate in biblical times, and Russia, even when Jews were consciously repelling it, turned out to be the embodiment of an their archetype of the "Austere Judge" and a "strange attractor" of their collective unconscious.

\subsection*{The Jewish-Russian Atmosphere}\label{the-jewish-russian-atmosphere}

\noindent The unique phenomenon of the Russian intelligentsia who are uprooted from their native soil and separated from their people is comparable to the Jewish experience of diaspora. It is precisely the Russian intellectual, who lives as an internal exile in his own country, is completely marginalised, is critical of everything, and is given to utopian dreams, that exhibits the greatest spiritual kinship with the Jew. The Jews assisted the growth of the Russian intelligentsia, and that intelligentsia was responsible for the deep assimilation of Jewish influences that proved decisive for Russia's culture and destiny. The two intermingled and merged. As Fedotov remarked, "It was no accident that once the Jews began to emerge from their ghetto in the 1880s, we can observe the closest comingling of the Russian-Jewish intelligentsia not only in revolutionary activities, but also in all kinds of passionate spiritual movements; and most importantly, in their fundamental existential outlook: in their ardent state of groundlessness and eschatological prophecies. \ldots{} Spengler saw in Russian intellectual circles a continuation of the Talmudic tradition and spirit."\footnote{Georgy P. Fedotov, "Rossiia i svoboda," in \emph{Sud′ba i grekhi Rossii. Izbrannye stat′i po filosofii russkoi istorii i kul′tury}, 2 vols. (St Petersburg: Sofia, 1991), vol.~1: 285.}

From a western viewpoint, Jews and Russians are seen as lacking in self-restraint, \emph{schlimazel} and unbridled. Neither nation is seen as being \emph{comme il faut} or being from "good society"; both are felt to emerge from different heights and depths of life. "Out of the depths, I cry unto Thee, O Lord" and Pushkin's evocation of being "at the edge of the gloomy abyss" are experiences known to both of them. Both are inclined to give themselves up to their emotions and fantasies, to violently surpassing limits, and at times to succumb to hysteria. They are unaccustomed to the happy medium; they have no respect for form, or any feeling for it. The content boils over and overflows the rim of the vessel.

Jews and Russians incite one other, lead one another on; their steps ring out so harmoniously when crossing the bridges of history that they destroy them with their resonance. Precisely such a mutual incitement and leading on occurred with the 1917 Revolution and the communist experiment. The Jew wants to show the Russian that he is more Russian even than he\textemdash that he is fearless, unyielding, that he will stop at nothing (consider the phenomenon of Trotsky). The Russian wants to show the Jew that he is even more of a Jew: far-sighted, canny, that he acts in a fully conscious and systematic fashion and can subdue the elements (the phenomenon of Lenin). In Fadeyev's novel \emph{Defeat}, Levinson manifests both Russian inflexibility and Jewish awareness in his fight with spontaneity. As a result, the Jewish qualities of shrewdness and being systematic assume Russian characteristics of daring, recklessness and lack of restraint\textemdash such as the Soviet regime displayed, with its mass of plans, its daredevilry in systematisation itself (the pressure to complete projects ahead of schedule \ldots{} to deliver the Five Year Plan in four years \ldots{} the proclamation that the present generation of Soviet citizens would live under communism). The ecstasy and hysteria of planning, the mania for figures, the reckless bookkeeping\textemdash such were typical phenomena of Soviet bureaucracy. The outlook conveyed in the proverbial Russian exhortation, "Strain every muscle to cut the hay," was reinforced by Jewish intellectual stubbornness, meticulousness, and the drive to bring things to their logical conclusion, even at the risk of failure: if, according to our calculations, the bright day of socialism should dawn, then it will do so\textemdash and if it does not, then so much the worse for reality.

The Talmudism of the Soviet regime, its passion for words and the tiniest ideological nuances of their meaning, its faith in the self-sufficient and self-fulfilling power of words which will render reality more joyous and more beautiful\textemdash this is both Jewish and Russian. As was the regime's faith in drafts on paper, and the Talmudic disputes over interpretations of their wording, and the heated discussions of deviations, of "right" and "left" ideological leanings, and of the subtlest details in the employment of citations from the fundamental texts of Marxism. They treated the latter writings as Jews treated holy writ, as being susceptible of multilayered interpretations and explications, like the Torah, the Mishnah, the Gemara, the Kabbala \ldots{} There was a similar Talmudic hierarchy: Marx, Engels, Lenin, then Stalin, then subsequent foreign adherents to Marxist thought\textemdash the "outstanding representatives of the Party and government," the "founders of Socialist Realism," the "writers from fellow Communist states" \ldots{} Every level was endowed with its own degree of authority, which could be measured by the length and frequency of citations from their work, and the relative significance of which could be gauged with minute accuracy from the number of column inches allotted them in the press.

It is said about the Russians that they are a people given to extremes and polarities. In \emph{The Russian Idea}, Berdyaev offers the following examples: despotism\textemdash and anarchism; ritualised belief\textemdash and the search for truth; the search for God\textemdash and flagrant atheism; slavery\textemdash and mutiny \ldots{} The Jews are also a nation of extremes: hoarders and people who are above financial self-interest; realists and fantasists; capitalists and communists; commonsensical and prone to eccentricity. In western nations, there is a wide zone of middle ground for a neutral and secular way of life\textemdash which is only natural and is justified by human nature. Jews and Russians constantly set themselves unrealisable tasks, and both surpass average norms and fall short of them. The inclination to irrationalism, existentialism and personalism in philosophy is exemplified by Rozanov, Berdyaev, Shestov, Rozenzweig, Buber, and Levinas.

Dostoyevsky's "man from the underground," and indeed Dostoyevsky himself are "Jews" in their extravagant thriftinesss, their hyperactivity, their impulsiveness (Sholem Aleichem's Menakhem-Mendl): everything is calculated, put in the kitty for the sake of a certain win\textemdash only to lose ignominiously. Dostoyevsky's teenage "yid" dreams of becoming a Rothschild. Tolstoy remarked on Dostoyevsky's Jewishness. The existential situation of the underground, the state of "nowhereness" characteristic of Jews were carried across into Russian literature by Dostoyevsky.

In general, strange as it may seem, the "anti-Semites" Gogol and Dostoyevsky were the first to introduce the Jewish spirit into Russian literature. With both men, Russian society itself becomes bourgeois, Chichikovised; it is possible that their Ukrainian and Polish roots made them sensitive to the spirit of this derivative milieu, dominated by preoccupations with money, gambling and speculation, in which the Jews took root. In the case of both writers, nature recedes into the background and "Jewish" ideas come to the fore: getting rich, possessing the signs and symbols of power. A noisome urban milieu predominates\textemdash a St Petersburg "ghetto." It would be easy to imagine Akakiy Akakiyevich\textemdash a small man in a small world\textemdash as a denizen of a shtetl or some small town in the Pale of Settlement. Both the tailor Petrovich and the "important personage"\textemdash such are the different ranks of "authority" before whom Akakiy Akakiyevich is reduced to a trembling creature. Similarly, Raskolnikov's way of life and mental outlook also mark him out as a "trembling creature," though he also imagines himself to "have entitlements." As has long been remarked, in Dostoyevsky's depictions of St Petersburg, the colour yellow, traditionally associated with Jews, predominates. "This black and yellow colour, this joy of Judea" (Mandelstam).

In both nations can be observed a combination of two extremes: spontaneous emotionality and abstract ratiocination. Inspiration and "gut feeling," a simple trust in God, and childlikeness are reflected in Hassidism, an emotional, charismatic sect within Judaism. To live by one's heart, to see God in small and everyday things. Hassidism, which originated precisely on Jewish-Slavonic soil, links these nations. As Gershom Sholem writes, "the greater part of Russian and Polish Jewry were drawn into the orbit of this movement (Hassidism) \ldots{} but this form of mysticism was unable to take root anywhere outside the Slavonic countries and Russia."\footnote{Gershom Scholem, \emph{Major Trends in Jewish Mysticism} (New York: Schocken, 1995), 325.} Ukraine is considered to be only second in the world to Israel for having the greatest number of Jewish holy places. It was proposed to build the biggest synagogue in the world with a capacity of five thousand worshippers in Uman, the burial place of Rabbi Nakhum. An unschooled, unbookish relationship to God\textemdash out of an overflowing heart, in the expectation of joy.

And yet, both nations are bookish and are devoted to the word and to literary sermonising. Reflectiveness and premeditation. Action often dissolves in self-reflection, in the construction of multiple points of view, in the art of commentary and interpretation. Job and Dostoyevsky, Buber and Bakhtin, the dialogic character of both nations, emphatically expressed gesticulation (Russian gestures are sweeping, Jewish ones are hasty), a love for endless conversations and verbal sparring, the loss of a feeling of reality. A tendency to self-awareness and to experiment on oneself is predominant. Although one of these nations possessed the most expansive territory in the world and the other remained homeless for many centuries, both these extremes fostered the development of an approach to existence that was at once utopian and schematic. Rabbis, scribes, wise men, people well versed in the scriptures, fantasists and explicators set the tone for Judaism, all having little contact with the reality of the society that surrounded them. Russia was created and recreated to plan: the Christianisation of Rus by Prince Vladimir, the Europeanisation of Russia by Peter the Great and Catherine II, the socialist revolution and the building of communism under Lenin and Stalin, Gorbachev's \emph{perestroika}\textemdash all of this was imposed from above, and originated in the heads of rulers. Ideas and projects always came first\textemdash things were not introduced and brought into being in a way that arose organically out of national customs and ways of life, but rather out of a fertile-brained "head with a little organ" (Saltykov-Shchedrin), obsessed with the next transformative idea. This accounts for St Petersburg being, in Dostoyevsky's phrase, "the most abstract and premeditated city in the world" (\emph{Notes from Underground}), and the spectre of communism, which sprang from the head of Marx and was realised precisely in Russia, which has always been so hospitable to all kinds of speculative abstractions.

This combination of Hassidism and Talmudism, of heartfelt faith and bookish wisdom is characteristic of both nations. There are Talmudic traits in Solovyov and Mandelstam; and Hassidic ones in Rozanov and Pasternak. Holy foolishness, spontaneity\textemdash and cultural mediatedness. Both stem from idealism: the world is reduced to the heart or the book. The relation of both nations to reality is a difficult one, for the European opposition of subject and object is alien to them, their epistemology is weakly developed, as well as its attendant search for reliable criteria for establishing objective truth. From the excessively German Kant, the Jews (Hermann Cohen, Ernst Cassirer) moved towards neo-Kantianism, which is preoccupied with collective subjectivity, a common faith which is jointly shared rather than subjected to rational scrutiny. The proclivity of the Jews for philosophical constructivism and intuitivism is evident: the pupils of Cohen included Pasternak and Matvey Kagan, an older friend and accomplice of Bakhtin. Kant's sharp and consistent dualism is softened in neo-Kantianism in favour of "things-for-us." And the Russians were also drawn to neo-Kantianism\textemdash Lunacharsky, Gorky, the revisionists of Mach (Bogdanov and Yushkevich)\textemdash and to the construction of belief from human commonality, the totality of faiths, and not from rational scrutiny seeking to establish objective truth. The collectiveness of experience, the spirit of togetherness {[}\emph{sobornost′}{]} as the decisive factor in one's orientation towards~the~world~\ldots{}

"Yes, there was such a particular Jewish-Russian atmosphere of which one Jewish poet said: 'Blessed is he who once breathed it.'\,"\footnote{Fedotov, "Rossiia i svoboda," 285.}

\subsection*{The Fourth Jerusalem}\label{the-fourth-jerusalem}

\noindent Messianic strivings and the belief in being God's chosen people simultaneously unify and divide both nations. One of them, in accordance with their forefathers' covenant with God, is obliged to preserve its spiritual isolation; the other to embrace all other nations with Christian love. Both Israel and Russia are "holy lands." This is not merely a typological likeness, but also a conscious historical continuity. The first \emph{attempts} to liken Russia to the New Israel and Moscow to the New Jerusalem are found in Russian sources that date back to the late fifteenth century; and Russians at times called themselves "the true Israelite proselytes." Thus, the subsequent tense relations between both nations can be explained as a kind of messianic jealousy about which was God's chosen people, and whose time had come and whose time had passed. Both messianic aspirations fused for a time in the Bolshevik Revolution.

The notion that Moscow was the Third Rome, which inspired Russian religious self-imagin\-ing since the sixteenth century, was in essence a metaphor or a parable of another, more profound, though unarticulated formula\textemdash namely, Moscow as the Fourth Jerusalem. For Rome was the second, both in temporal succession and in importance as a sanctuary of religious belief, after Jerusalem. The third Jerusalem was Constantinople, and Moscow was consequently the fourth. Why did the importance of Jerusalem itself, the beginning of the beginnings, wane in this messianic succession?

The answer, of course, is that Jerusalem was unsuited to being the centre of a newly-created Christian \emph{empire}, because it had never been an imperial capital. The magic of the number three, sanctified by its association with the Trinity, also played its role. But one can also adduce yet another reason: the connection with Jerusalem, and with God's chosen people, was so important for Russia that it remained in the depths of its religious unconscious; it also determined the strange and rationally inexplicable intensity of its relationship with the historically dispossessed and politically powerless Jews. Rome and Byzantium were empires, but the Roman and the Byzantine nations were not God's chosen people. Russia, on the other hand, laid claim not only to being an empire\textemdash and thereby being the successor of Rome and Constantinople\textemdash but also to the fact of being God's chosen people. In this respect, it was the direct successor of the Jewish nation. This messianic component of the "Russian idea" was scarcely no whit less important than the imperial one, and concealed far greater ambitions, for it pertained not to political horizontals\textemdash that is, to world conquest\textemdash but to religious verticals, namely, to being chosen by God. The Jewish nation was also chosen by God\textemdash but it did not found an empire. The Romans (both western and eastern) founded an empire, but they were not chosen by God. The Russian nation, however, was called to the higher synthesis of both missions, the "Jerusalemite" and the "Roman."

This fermentation of the "Russian idea" with Jewish leaven did not merely remain in the realms of wishful thinking. Within Russian Orthodox Christianity, there is genuinely something close to the religious psychology of the Jew, and correspondingly, in Judaism there is something akin to the religious psychology of the Russian. This fact explains the Judaizing sects in Rus\textemdash the "Judaizers" and the "Sabbatarians." It also explains the frequent conversations of Jews to Russian Orthodox Christianity in the late Soviet epoch\textemdash which did not take place for careerist reasons, as was often the case in tsarist Russia, but due to spiritual attraction, as conversion to Russian Orthodox Christianity did not remove, but doubled and tripled the burden of life in the atheistic State. To persecution directed at Jews was now added persecution directed at believers, as well as suspicion of the very presence of a Jew inside the interior of an Orthodox church: converts had to contend with the mistrust of their fellow communicants towards non-Russians and the mistrust of their fellow Jews towards voluntary converts to Christianity.

At a meeting with rabbis in New York on 13 November 1991, Patriarch Aleksey II (Ridiger) declared in a speech characteristically entitled, "Your prophets are our prophets":

\begin{quote}
The union of Judaism and Christianity has a real basis in spiritual and natural affinity and positive religious interests. We are at one with Jews, not giving up Christianity, not in spite of Christianity, but in the name and strength of Christianity; and Jews are at one with us not in spite of Judaism, but in the name and strength of true Judaism. We are thus set apart from the Jews by the fact that we are "not fully Christian"; while the Jews are set apart from us because they are "not fully Jewish." For the fullness of Christianity embraces Judaism, and the fullness of Judaism is Christianity.\footnote{The full speech of Patriarch Aleksey II is reproduced here: \url{https://mparchiv.narod.ru/alravvin.html}.}
\end{quote}

Whatever one might think of these sentiments, they are characteristic of Orthodox Christianity, especially in its Russian manifestation, to a much greater measure than is the case with other major Christian denominations. In an Orthodox church, the sanctuary is set apart and concealed from parishioners, just like the "Holy of Holies" in the Jewish Temple. In Catholic churches, and even more so in Protestant ones, the altar is brought forward, corresponding to the incarnation of God in man and His visible and tangible manifestation in the world. When Christ was crucified, the veil which set apart the secret part of the temple was rent, for the secret of God was disclosed to the people when he took on human form and died on the Cross. But there was of God something that remained out of bounds and which did not merge with man. And thus, as if as a sign of the non-merger of both natures in Christ, the Russian Orthodox Church from the fourteenth and fifteenth centuries onwards erected the iconostasis between the worshippers and the altar so that the preparation for the mystery of Communion could take place in secret from the worshippers themselves. This separation is also a sign of the "out-of-bounds-ness" of God for man, the incomprehensibility of his mystery that is also emphasised in apophatic (negative) theology, which principally developed in Eastern Christianity\textemdash that is, the knowledge of God in his inaccessibility by the negation of all his names and likenesses.

Is this not also the purpose of the iconostasis\textemdash not merely to bring God closer to the believer through images, but also to set him apart by mean of the very veil of these images? The iconostasis simultaneously makes visible (the image of the Son) and conceals that which ought to remain invisible (the action of the Holy Spirit on the altar): it is not only a window onto another world, but also a veil between two worlds. The icon is a visual paradox of concealment-by-disclosure, which in Judaism is solved in favour of concealment, but in western Christianity in favour of disclosure. Thus, the icon is of such importance in Russian Orthodox Christianity, which mediates between Judaism, with its understanding of the out-of-bounds-ness of God, and western Christianity, with its emphasis on the human and the taking on of human form in God. The same dual relationship ought to be maintained towards the icon, the contemplation of which should set bounds to the act of contemplation itself. It is impossible to approach it not only as a self-sufficient sacred object (that is, as an idol), but also as a transparent likeness of sacred persons (as a picture). For faith, in the words of St Paul, is 'confidence in the invisible'. The invisible has been made visible in other to conceal the invisible more effectively and to strengthen our faith in it.

The same "Jewishness" of Russian Orthodox Christianity\textemdash in comparison with Catholicism\textemdash is also discernible in their Creeds. The dividing term is the \emph{filioque} (and the Son)\textemdash it would seem as a purely formal obstacle. Is this a significant difference\textemdash whether the Holy Spirit proceeds only from the Father, as the Orthodox Creed implies, or additionally from the Son (\emph{filioque}), as the Catholic Church maintains? But precisely in this incommensurability between the Paternal and the Filial can be discerned the closeness of Russian Orthodoxy to the Jewish notion of the out-of-bounds-ness of God. It would seem that for Christianity the entire plenitude of God takes human form as manifest in Christ\textemdash and this divine assumption of human form is emphasised in Catholicism and to an even greater degree in Protestantism, simultaneously with the move of the altar closer to the worshippers. But just as the "Holy of Holies" is protected in an Orthodox church, in the Orthodox Creed, God the Father fully contains within Himself the mysterious source from whence proceeds the Holy Spirit, not transferring it to the Son\textemdash it is elevated above both other hypostases.

Meanwhile, Catholicism also took the next step towards the abolition of that mysterious veil. Not only is the entire plenitude of the power of God the Father transferred to the Son, but also all the power of Christ is transferred to his sole representative on Earth\textemdash the Pope. In Catholicism, the Pope is the Vicar of Christ, His deputy on Earth\textemdash that is, not only does the God-man fully partake of the Divine, but what is more, the human fully partakes of the God-man. The Pope is endowed with complete infallibility when he pronounces \emph{ex cathedra}.

Therefore, the two fundamental differences between Catholicism and Russian Orthodox Christianity\textemdash the \emph{filioque} and the papacy\textemdash come down precisely to the abolition of the transcendence of God, first in the fusion of the human and the divine and then in the incarnation of the divine in the human. Russian Orthodox Christianity, of all Christian faiths, adheres most closely to Judaism, preserving the most precisely defined hierarchy, the subordination of the hypostases and emphasising the out-of-bounds-ness of God to all humanity and even to the Godman himself.

In the work of the Russian philosopher Semyon Frank, an ethnic Jew who converted to Russian Orthodox faith, Christianity is understood in an apophatic spirit which is in general closer to Eastern than to Western Christianity. This apophatism is a Jewish trait inside Christianity: the notion that God did not become fully incarnate in human form and remains inaccessible to comprehension. Frank adheres to the doctrine of Nicholas of Cusa about knowing unknowingness or "learned ignorance"\textemdash that is, that one can only know God through one's ignorance of Him. Frank's most important work, \emph{The Incomprehensible}, in essence propounds apophatism as being half way between Christianity and Judaism: "being conscious of and feeling 'my God,' I cannot and do not have the right to lose sight of that depth of His in which He is the unspeakable, unknowable, and terrible (in His unknowableness) Divinity, which essentially transcends all definition.\hspace{.1em}\ldots{}"\footnote{S. L. Frank, \emph{The Unknowable: An Ontological Introduction to the Philosophy of Religion}, trans. Boris Jakim (Brooklyn, NY: Angelico Press, 2020), 242.} In this sense, Frank is a Jewish-Christian thinker.

The Orthodox priest Aleksandr Men′ (1930\textendash 1990), an ethnic Jew, was the "godfather" of the Soviet intelligentsia, which accounts for his transformative role in the development of post-Soviet Orthodoxy as opposed to its pre-Revolutionary form. This form of Orthodoxy became imbued with that "internationalism" which emerged during the Soviet period from Russian-Jewish atheism\textemdash but bestows on this union a religious dimension. By this I do mean neither the Christianisation of Judaism and the Judaization of Orthodox Christianity, nor the reciprocal religious conversions between these faiths, but rather the establishment of intermediate, transitional degrees of a common tradition of spiritual life. Take the baptism of Russian Jews at the end of the twentieth century: this did not separate many of them from Jewishness, but rather brought them closer to it\textemdash into the bosom of Abraham and David from which Christ and the apostles came forth. It may be that precisely Russian Jewry afforded both Christianity and Judaism an opportunity to discover a mode of psychological relatedness to one another, to draw closer voluntarily as never before, except in the times of the apostles, and in the presaging of the end of time, when the coming of the Messiah is promised to both Jews and Christians.

And so one can understand why the purely negative experience of communism proved instructive for both nations, and stimulated them to try to become closer on account of religious and not atheistic affinities. Formerly, it was easiest for Jews and Russians to converge in their idealistic materialism, in stubborn and fanatical atheism, in disbelief as a variety of belief. Their atheism, in contradistinction to western atheism, which is rational and enlightened, was itself a form of faith, a utopian one, and their denial of the existence of God was itself religious. Communism was the convergence of Christians and Jews on the paths of negation equally of Christianity and Judaism\textemdash and thus, having made evident the negative and self-destructive results of such a renunciation, prepared the conditions for a positive religious convergence of both nations. This is why Berdyaev and Buber, who had first-hand experience of Russian communism and Jewish socialism and found themselves on the paths of Christianity and Judaism, were inwardly akin to one another.

In his collection of essays \emph{Shield} of 1915, Vyacheslav Ivanov, observing the evolution of "Aryan" ideology, expressed the thought that anti-Judaism always becomes transformed into anti-Christianity as a matter of course, as the subsequent history of Nazism confirmed. In its turn, the history of communism confirmed that, conversely, anti-Christianity eventually transforms itself into anti-Semitism. Judaism and Christianity are not only linked historically, but also bound up with one another as systems of beliefs and values; and denying one will sooner or later cause the other to be denied too. In the Russian nationalist movement of the 1980s and 1990s, hostility to Judaism and Russian Orthodox anti-Semitism quickly led to the denial of Christianity from the position of Vedantism, neo-paganism and "native faith" {[}\emph{rodnoveriye}{]}.

\subsection*{Philias and Phobias}\label{philias-and-phobias}

\noindent The fact that the founder of Communism Karl Marx was both an anti-Semite and a Russophobe was a good reason for the two nations to draw closer together, as they both managed to survive communism and free themselves from the yoke of Marxism. This judgement can only be made out of love for both nations, from a standpoint of both Russophilia and Judeophilia. Unfortunately, terms of hatred still predominate in Russian. When one does a Google search for information in Russian about the word "Judeophilia," the automated message appears "Did you mean Judeophobia?" In fact, on Russian-language internet sites, the word "Judeophobia" occurs eighteen times more frequently than "Judeophilia" (90,000 results and 5,000 results respectively). If one compares the data for the search terms "anti-Semitism" and "philo-Semitism," then the difference is even more pronounced: 4,000,000 results as opposed to 2,000. Should one infer from this that anti-Semites are two thousand times more numerous than philo-Semites? I will refrain from drawing simplistic conclusions about attitudes, but even in relation to Russians, the language of hatred predominates: the search term "Russophobia" returns 1,400,000 results, and "Russophilia" merely 20,000.

My personal position in relation to the two vitally important national questions can be summed up in one word: \emph{Judeo-Russophilia}. I propose this term for general use. It is especially helpful in a situation where Russophiles are usually considered (and indeed sometimes actually are) Judeophobes, and Judeophiles are considered Russophobes. The term usefully emphasises the compatibility of these philias, which are understood by many\textemdash without any foundation\textemdash to be mutually hostile and mutually exclusive.

What is at issue here is love for another nation, and not one's own. Love for one's own nation and culture usually goes without saying, and it is even inappropriate to express it. It would be odd for a Russian to be a Russophile, a Jew to be a Judeophile, a German to be a Germanophile, and so on. It would be like answering the question "who do you love?" with "myself." Just as it befits a Russian by and large to be a Germanophile, a Polonophile, a Judeophile or an Anglophile, but not a Russophile, it similarly befits a German to be a Russophile or a Francophile. However, in the nineteenth century, Slavophiles appeared amongst the Slavs themselves, just as Germanophiles appeared amongst Germans, and this was naturally a sign of trouble: only someone for whom things are going badly starts to declare love for himself. One has to take pity on him and understand the reason for his recourse to self-love, when his love should be directed at others.

Thus, by the loftiest standards, it does not become Russians or Jews or Russian Jews to experience Judeo-Russophilia\textemdash that is something for the Germans, the Italians or the Americans, if they are roused to particular interest in and favour towards Russia, Judaism and the unique intersection of their destinies. But in the real and very unfavourable situation in which we find ourselves, a state of schism between a multiplicity of national self-loves and mutual xenophobias, it would be greatly advantageous even if one reinforced the feeling of self-love with the love for a culture which is closely related and parallel, but which in essence is even perpendicular. Under such conditions, \emph{Judeo-Russophilia} is not an egocentric idea, but an eirenic one, which prompts us to understand and accept these "fateful crossings."

The end of the twentieth century was the time when the two nations parted, and perhaps also the time of their new mutual recognition. Once liberated from the equally onerous yoke of Communist rule which they had created jointly by their own efforts, they could now, at the point of leave-taking, take a close look at one another and understand why fate had brought them together, and what lesson the world could draw from their encounter and leave-taking. Russians and Jews\textemdash of all nations\textemdash turned out to be (and, indeed, were from the very beginning) the most Soviet. They laid claim to their own particular national terrain to a lesser extent than any other nations in the USSR, for it was they who embodied the internationalism of the new order. Other nations had their own territories (in the sense that they kept them or created them anew), their own customs, traditions, national festivals; but the Russians and Jews were permitted almost nothing of the kind, so that they would embody the very idea of Soviet rule\textemdash a self-denial of a separate past in the name of the brotherhood of the workers of the world. The Jews had a nominally autonomous region in Siberia, Birobidzhan, but they did not live there; the Russians nominally had their own republic, the Russian Soviet Federative Socialist Republic, but its administrative organs were weakened by the fact that they largely overlapped with the overarching Soviet ones (the so-called "Soviet-Republic Ministries"). Russians and Jews were not nations, but rather the flesh and blood of the new transnational commonwealth of the Soviet people. There were doubts as to whether the Jews could even be considered a nation, or whether, in the absence of a territorial community, they should be considered a "pseudo-nation." Russians were also considered to be transnationals, "the elder brother" in the family of Soviet nations\textemdash that is, as if they were responsible for the entire family in the absence of the deceased father (God and the Tsar). Jews were "an insufficient nation" and Russians were a "trans-nation," but in both roles they had to suppress their own individual nationality\textemdash most of all, when it came to religion.

Therefore, the collapse of the USSR meant first and foremost the sundering of Russians and Jews, who returned to their own national existences. But it was precisely then that the time came for deeper and freely creative relations between the two nations, now that they were no longer bound by the yoke of their combined fates and the injuries caused by a difficult way of life. In a certain sense, the USSR did not fall asunder into Russia and the other constituent republics, but into Russia and the diaspora: the Russian diaspora and the Jewish one. The republics adjoined other areas that were more capacious religiously and culturally, and to which they had formerly partially belonged (the central Asian republics to Islam, the Baltic republics to Western Europe \ldots{}). But the USSR was created to a significant degree by the interaction of Russian and Jewish religious utopianism, precisely on that common international and atheistic soil, where they rejected their own religions and thereby their nationality. With the collapse of the USSR, the messianic and totalitarian era in Russian history came to an end, and each nation has been given back its separate fate.
\vspace{2em}
\begin{center}
  \includegraphics[width=0.75cm]{articlend.png}
\end{center}

\vfill
\biobox{\textbf{Mikhail N. Epstein}~is the Samuel Candler Dobbs Professor Emeritus of Cultural Theory and Russian Literature at Emory University, where he has taught since 1991, following his emigration from the USSR. From 2012 to 2015, he served as Professor and Founding Director of the Centre for Humanities Innovation at Durham University (UK). He has authored 40 books and more than 800 articles, some of which have been translated into 26 languages. His latest books include~\emph{Ideas Against Ideocracy: Non-Marxist Thought of the Late Soviet Period (1953\textendash 1991)}~(2022); \emph{The Phoenix of Philosophy: Russian Thought of the Late Soviet Period, 1953\textendash 1991}~(2019);~\emph{A Philosophy of the Possible: Modalities in Thought and Culture}~(2019);~\emph{The Irony of the Ideal: Paradoxes of Russian Literature}~(2017); and \emph{The Transformative Humanities: A Manifesto}~(2012). He is а recipient of the Andrei Bely Award (St.~Petersburg, 1991), the International Essay competition award (Berlin-Weimar, 1999), and the Liberty Prize (New York, 2000). His most complete bio\textemdash and bibliography is in:~\emph{Homo Scriptor. Sbornik statei i materialov v chest\textquotesingle{} 70\textendash letiia Mikhaila Epshteina}~(\emph{Homo Scriptor: A Collection of Articles and Materials in Honor of Mikhail Epstein\textquotesingle s 70th Anniversary}) (Moscow: Novoe literaturnoe obozrenie, 2020), 688 pp.}

\label{sec:epstein}

\section{Sergii Bulgakov, Socialism, and the Church}


\fancypagestyle{chaptercontentpage}{
  \fancyhf{} % Clear all header and footer fields
\fancyhead[CE]{%
  \fontsize{11}{11}\leftmarkfont%
  \addfontfeature{LetterSpace=10.0}%
  \textit{\MakeUppercase{Sergii Bulgakov, Socialism, and the Church}}%
}
  \fancyhead[CO]{\authorheadfont\addfontfeature{LetterSpace=10.0}\fontsize{11}{11}\selectfont\textbf{{\uppercase{Rowan Williams}}}}
  \renewcommand{\headrulewidth}{0pt} % No header rule on content pages
  \fancyfoot[RE]{\thepage}
  \fancyfoot[LO]{\thepage}
}

\abstractbox{Sergii Bulgakov, Socialism, and the Church}{}{Rowan Williams}{After 1907, Sergei Bulgakov would not have called himself a socialist. Yet he continued trying to understand "socialism" as a phenomenon that needed analysis in terms of its spiritual presuppositions. In "The Soul of Socialism" (1932), Bulgakov argues that socialist politics presumes an anthropology, a doctrine of the human\textemdash that is, a soul. He believed that socialism's soul was an inverted image of the soul of the Church; this conception allowed him to define the Church in a way that was neither subservient nor hostile to the modern epoch. Socialism, according to Bulgakov, is too reductive to reconcile individuals to actual, spatiotemporal existence. He saw the soul of socialism\textemdash immaterial, ahistorical, depersonalized\textemdash as a pseudo-soul, one that functioned, moreover, as a pseudo-Church. He assumed that socialism's failings could be found elsewhere, especially in capitalism, because they were downwind of the modern desire to construct a perfectly managed environment or even to effect the "end of history." Rather than jettison the insular, pseudo-Churches of modernity, however, the real Church could work alongside them, cultivating itself as an alternative community of discernment and learning. Such patience was imperative in order for the Church to fully signify and embody a network of relations\textemdash relations with God, the world, and other subjects\textemdash in which the human person is maximally free for gift, love and mutuality, and thus unwilling to accept any narrower vision of common life.}{Sergei Bulgakov, socialism, capitalism, anthropology, ecclesiology, modernity, matter, history, sophianic harmony, discernment}

\chaptertitle{Sergii Bulgakov, Socialism, \\ and the Church}{}{Rowan Williams}

\fancypagestyle{chaptertitlepage}{
  \fancyhf{} % Clear all header and footer fields
  \fancyhead[L]{\begin{minipage}[t]{0.7\textwidth}\publisher\end{minipage}}
  \fancyhead[R]{\begin{minipage}[t]{\textwidth}\raggedleft \datefont\fontsize{10}{11}\selectfont Volume 1 (2024): \thepage\textendash\pageref{sec:williams}  \\ \doi{10.71521/vtze-5s98} \end{minipage}}
  \renewcommand{\headrulewidth}{0pt} % No header rule on title pages
  \fancyfoot[LE, RO]{\thepage} % Left on even pages, right on odd pages
}
\thispagestyle{chaptertitlepage} % Apply custom page style for chapter title
\addcontentsline{toc}{chapter}{Sergei Bulgakov, Socialism, and the Church \\ \emph{by} Rowan Williams}
\setcounter{footnote}{0}

\noindent Sergii Bulgakov's writings in the first dozen years of the twentieth century sketch his journey from the Marxism of his student and postgraduate years towards Christian commitment and a growing and passionate interest in Christian philosophy and theology.\footnote{Especially the articles contained in his two collections, \emph{Ot marksizma k idealizmu} [\emph{From Marxism to Idealism}] (St Petersburg, 1903), and \emph{Dva grada} [\emph{Two Cities}] (Moscow, 1911). (The present essay was originally published by Volos Academy Publications, an imprint of Volos Academy for Theological Studies, Melissiatika, Volos, Greece, 2023. Republished here with the kind permission of Volos Academy.)} One dimension of this, evident in these early essays, is his concern to bring his aspirations for a more open and participatory society into dialogue with the Christian tradition. His "Christian Socialist" period was confined, strictly speaking, to his abortive involvement in national politics in the middle of the decade. His deeply disillusioning experience as a deputy in the Second Duma gave him a lasting aversion to the kind of revolutionary maximalism that ignored practical and achievable reform in the name of theoretical purity and absolutist demands.\footnote{He writes in his \emph{Avtobiograficheskie zametki} {[}\emph{Autobiographical Fragments}{]} (Paris: YMCA Press, 1946), 80\textendash 82, about the disillusioning effect of his participation in the Duma; the experience is reflected in the magisterial essay on \emph{"Geroizm i podvizhnichestvo"} {[}"Heroism and the Spiritual Struggle"{]} contributed to the symposium \emph{Vekhi: Sbornik statei o russkoi intelligentsii} {[}\emph{Landmarks: A Collection of Essays on the Russian Intelligentsia}{]} (Moscow, 1909). There is an English translation of Bulgakov's \emph{Vekhi} essay, with introductory commentary, in Rowan Williams, ed., \emph{Sergii Bulgakov: Towards a Russian Political Theology} (Edinburgh: T. and T. Clark, 1999), 69\textendash 112. The autobiographical notes are translated into German in Sergij Bulgakov, \emph{Aus meinen Leben: Autobiographische Zeugnisse}, ed.~Barbara Hallensleben and Regula M. Zwahlen (Münster: Aschendorf Verlag, 2017), along with other significant fragments, including "\emph{Fünf Jahre (1917\textendash 1922)\textemdash Agonie,}" 73\textendash 93, which also has some bitter comments on the Russian politics of the first decade of the century.} But he continued to be deeply engaged in the attempt to understand "socialism" as a phenomenon that needed analysis in terms of its spiritual and imaginative presuppositions; and he never lost his concern to find ways of articulating a viable theological foundation for some kind of Christian political witness. Throughout his life in exile, he maintained his criticisms of simple political reaction, of nostalgia for an autocratic patriarchy; his brief flirtation with a mystical monarchism in the years of the First World War and the revolutionary era does not leave much of a mark in his later writing. But it would not be accurate to think of him as continuing to profess anything that he himself would have been happy to call "Christian Socialism." It is clear that "socialism" for him had come to designate not so much a political program as an "ideal type" of human self-understanding, always on the verge of becoming antithetical to the Church to the degree that it refused to ground itself in the reality of the Church\textemdash a point Bulgakov was already making in his "Christian Socialist" days.\footnote{See, for example, \emph{Dva grada}, 307.} This essay will look in detail at his most extended later discussion of socialism, the essay on "The Soul of Socialism" (\emph{Dusha sotsializma}) published in the émigré periodical \emph{Novyi grad} in two parts (1932\textendash 3), and will attempt to clarify how he sees the "socialist" consciousness relating to ecclesial reality.\footnote{English translation in Williams, \emph{Sergii Bulgakov}, 237\textendash 267. Bulgakov had published a sort of summary "position paper" on "Christianity and Socialism" in 1917, which opened up some of the themes of his later discussions. See \emph{Khristianstvo i sotsializm} (Moscow: Educational Commission of the Provisional State Assembly, 1917), reprinted in S. N. Bulgakov, \emph{Khristianskii sotsializm} {[}\emph{Christian Socialism}{]}, edited and introduced by V. N. Akulinina (Novosibirsk: Nauka, 1991), 205\textendash 233.}

\subsection*{The Soul of Socialism}

Bulgakov's theological thinking had already embraced the conviction that this ecclesial reality was above all something very much more than an historical and human institution. The theme is highlighted in what can be considered his first real theological "manifesto," \emph{Svet nevechernii} in 1917, and is stated with clarity in his writings in exile.\footnote{\emph{Svet nevechernii. Sozertsaniia i umozreniia}. For the English edition, see Sergius Bulgakov, \emph{Unfading Light: Contemplations and Speculations}, trans. and ed.~Thomas Allan Smith (Grand Rapids, MI: Eerdmans, 2012), especially 354\textendash 358, 416\textendash 424.} "The Church is both created and uncreated," he wrote\textemdash rather startlingly but very characteristically\textemdash in his Hale Lecture of 1934 on "Social Teaching in Modern Russian Orthodox Theology":\footnote{Williams, \emph{Sergii Bulgakov,} 273\textendash 286.} ultimately it is no less than the entire creation restored to its essential nature as the self-revealing of God, creation unified in and transfigured by uncreated "Sophia," the divine Wisdom which is, for Bulgakov, the inner form of the divine life that is shared by the divine persons, the object of the selfless self-love which the divine persons eternally enact. The Church is creation fully transparent to the creator\textemdash in all its relations and activities, human and non-human. It is the moral and spiritual "shape" of all properly \emph{human} agency; and properly human agency is whatever brings the world closer to its "sophianic" identity. "History is the self-definition and self-revelation of the human," Bulgakov writes in the essay on "The Soul of Socialism."\footnote{Ibid., 244.} This "self-definition" is emphatically a historical task for Christians, a task requiring decision, intelligence and energy, not a retreat towards an imagined and idealized past; but it must be distinguished from any ideas about irreversible natural progress in history. The unfolding of history brings us closer to the apocalypse, and so to the revealing of Antichrist as well as the coming-again of Christ. What develops in history is not the steady advance of Christian triumph or the control of circumstances by the community of believers, but a steady growth in discernment between good and evil; the trajectory of history is not towards a guaranteed victory \emph{within history} for the Kingdom of God, but towards a more and more intensified and purified prayer for the coming of Christ. That intensification and purification will not happen without our conscious commitment here and now to God's future, to the transparent and reconciled cosmos in which Wisdom prevails; and this is, Bulgakov argues in the Hale Lecture, the priority in the Church's approach to public and social matters. As long as history continues, the Church is "growing and ripening" in its discernment and so in its openness to its own character as sophianic; but this will not guarantee a unified and stable world.\footnote{Ibid., 281\textendash 283, 285\textendash 286.} Part of Bulgakov's concern, as we shall see, is that the pressure to \emph{secure} such a world is one of the risks that the "socialist" mindset invites.

But to return to the long essay of 1932\textendash 1933, Bulgakov begins by distinguishing between the socialism of public policy\textemdash the protection of the rights of labor, the public control of the unbridled freedom of capital\textemdash and socialism as a spiritual phenomenon. About the former he is startlingly direct, almost casual: of course the Church must support such protections and restrictions; there is nothing new about this.\footnote{Ibid., 238.} The new, distinctive and difficult problem is the latter, the challenge posed by socialism's "soul" rather than its "body" (this "body" being the ensemble of legal policies needed to guarantee common prosperity and security in a society). Socialism has an anthropology, a doctrine of the human, shaped by what Bulgakov calls "sociologism" and "economism." Sociologism is defined as a discourse that takes for granted the reality and even priority of collective identities (ethnic, class-based or whatever), economism as a discourse preoccupied with how humans manage and overcome their radical dependence on the natural world. Once again, these are not in themselves hostile to Jewish-Christian categories: the Bible regularly presents history in terms of personified collectivities; and the vocation of human beings to make both sense and manageable resource out of the material world is built in to the Christian view of the human role in creation. But the problem with sociologism is that it obscures the creativity of the unique person; and the risk in economism is that in a fallen human environment we lose sight of what the full sense is that has to be made of the world; by casting our economic life in the social-Darwinist terms of a struggle for existence, we identify our necessary, transformative and creative labor simply with a battle against death. Thus, ironically, economic life becomes an enslavement \emph{to} death, because it is driven by a central fear of losing our place within a cosmos that has become intrinsically dangerous to us as humans, rather than a cosmos whose sophianic interdependence assures our life. Bulgakov counterposes Marx's narrative of a long "prehistory" that is about to come to an end in the timeless rationality and justice of the post-revolutionary world with the speculations of Nikolai Fyodorov, the wildly idiosyncratic nineteenth century Russian thinker who defined the "common task" of humanity as the resurrection of the dead\textemdash the transformation of the natural world in such a way that all the ancestors are returned to a world and a life shared with us. We may, as Bulgakov indicates, take the details of Fyodorov's ideas with a pinch of salt; but he represents a powerful symbol of what Marx denies and Christianity affirms\textemdash that history has already begun, that all its existing subjects have worth and dignity, and that our labor in the world must be directed to the maximal degree of hospitality towards this human community in its full reality, extended in time as well as space.\footnote{Ibid., 244\textendash 246 (on Fyodorov, cf.~ibid., 283\textendash 284, from the Hale Lecture).}

\subsection*{Matter and Materialism}

Bulgakov moves on to reflect on the ambivalence of the revolutionary psyche. On the one hand, the nihilism of revolutionary violence, the obsessive destruction of what has been inherited (including religious culture and institutions), is a sickness; but it is a sickness produced by passions and longings deeply ingrained in the human subject\textemdash the utopian strain which does indeed react to what is simply historically "given" with a sentiment of global challenge or protest.\footnote{Ibid., 248\textendash 249.} It is the spirit that refuses to accept that "whatever is, is right," and this is in itself a creative thing. But the difficulty arises when this sentiment is fleshed out in terms of the literal destruction of real persons for the sake of an imagined future. Bulgakov observes how in some sorts of Marxist rhetoric the dream of a scientifically managed future becomes (again ironically) yet another variety of hyper-spiritualized utopianism\textemdash a choice for the unreal over the real, though in the name of a materialist reductionism.\footnote{Ibid., 249.} The real is not good enough, and has to be replaced by what the mind approves. It echoes the comment made a few pages earlier,\footnote{Ibid., 241.} where he notes that the Promethean scientism that seeks to convert the entire material environment into a humanly controlled system in fact \emph{reduces} the scope of matter itself, because it shrinks the material world to the dimension of what can be successfully managed by human minds. There is, in other words, a "materialism" whose effect is to alienate us from matter, partly by alienating us from time and narrative (Iain McGilchrist's recent monumental work on \emph{The Matter With Things} is a formidable riposte to such a schema).\footnote{Iain McGilchrist, \emph{The Matter With Things: Our Brains, Our Delusions, and the Unmaking of the World}, 2 vols. (London: Perspectiva Press, 2021).} But Bulgakov's point here is to underline the risks of any utopian program that drifts away from real identifiable jobs to be done and actual tasks to be completed, tasks that require the specific historical resources and free decisions of personal agents in the present.\footnote{Williams, \emph{Sergii Bulgakov,} 248.} And this utopian seduction will always be a snare so long as human agents are not aware that they are always already involved in an exchange of energy and information through both historical and "natural" processes, an exchange whose direction is towards the maximal harmony and transparency already mentioned. They are always caught up in "sophianic" processes.

Secularism\textemdash or more accurately in Bulgakov's eyes, neo-paganism\textemdash is the \emph{Weltanschauung} in which the goal is a totally managed environment (human and material), and in which this goal acts as an absolute formative force in present decision-making. The exhaustive rationalization of the physical world, the exhaustive account of human motivation and imagination in narrowly physicalist terms, the creation of a wholly predictable and controllable environment\textemdash all these become the encompassing constraints within which we plan and project the future. And this future is understood in bewilderingly contradictory terms\textemdash \emph{both} as the inevitable outcome of a mechanical temporal process \emph{and} as the vision for whose realization we must struggle and sacrifice (a paradox in the revolutionary mindset which Bulgakov had already discussed in some of his much earlier work, like the famous \emph{Vekhi} essay of 1909).\footnote{Ibid., e.g.~85\textendash 88.} This unhelpful tension between determinism and "heroic" individualistic voluntarism is one of the deep dangers he identifies in the "soul" of socialism; but\textemdash strikingly\textemdash he identifies it in the logic of advanced capitalism as much as in the communist worldview.\footnote{Ibid., 262.} There is exactly the same aspiration to contain the material environment, and to chart and police the exercise of human activity in terms of its performance of determined functions; both systems lack a vocabulary for the personal, both regard actual physical reality as a potential enemy to be subdued and totally instrumentalized.

\subsection*{Human Labor Against Utopia}

This is why Bulgakov's critique of "socialism" is not a defense either of market capitalism or of pre-modern social forms. There is a proper fidelity to and respect for what has been inherited; but allegiance to tradition is not "loyalty to immobility."\footnote{Ibid., 255.} It is the plain exercise of human memory in its proper function of attending to the past, and allowing it the dignity of having made the present possible. Things do change; social possibilities alter radically, and it is no part of the Christian calling to turn back the clock. Bulgakov distinguishes\textemdash though in characteristically complex ways\textemdash between the idea of meaningful intentional activity by human agents, with cumulative effects, and the fantasy of an "end of history" that can be attained by such activity.\footnote{Ibid., 257.} Theology allows us to make sense of this distinction; it helps us to pursue and justify purposeful historical action, while retaining a skepticism about any notion that we could guarantee an ideal future. Indeed, one way of reading Bulgakov's scheme is to see him as showing how theology refuses two opposite fantasies of \emph{immobility}\textemdash the idealized past of the conservative, and the unimprovable future of the utopian revolutionary. In contrast, we are necessarily involved in meaningful labor within history\textemdash learning how to discern what does and does not serve the sophianic Kingdom. This entails attention to the needs of the present moment and a commitment to prosaic and long-term change (a theme he discusses in detail in his \emph{Vekhi} essay); and it also involves a willingness to learn how our previous understandings may need some rethinking while retaining a total commitment to serving the same unchanging direction of divine will in a different social climate.

\subsection*{Church, the Self, and Society}

That climate today is above all one of \emph{depersonalization,} Bulgakov argues, whether this is coming from "right" or "left"; and this is where the developments we associate with modernity have a genuinely mixed character.\footnote{Ibid., 261\textendash 263.} The apparently self-evident, inherited understanding of social roles characteristic of pre-modern society has gone forever; and this means that aspects of the biblical ethic that depend for their working on the benign operation of a patriarchal system (masters being kind to slaves, for instance) cannot now be treated as definitive (once again a theme foreshadowed in \emph{Svet}).\footnote{Ibid., 261 (cf.~\emph{Unfading Light}, pp.~419\textendash 420).} We can recognize this without simply ignoring or condemning those inherited understandings. There is no point in castigating pre-modern discourse for being pre-modern, Bulgakov seems to imply. But this dissolution of older, "organic" social patterns, while it may have shattered some kinds of solidarity, has also made possible new perspectives on personal dignity or liberty. One aspect of modernity is the growth of a new depth of understanding of what it is in humanity that resists the totalizing pressures of functionalism and rationalism. It is yet another paradox in our current historical situation that the emergence of the "modern" self has brought both Promethean ambitions for human control over human destiny, \emph{and} has also generated an enhanced sense of the mysterious inwardness and inaccessibility of the human subject. So the challenge for the Church is to affirm the purposiveness and creativity of the modern self\textemdash but, in so doing, also to orient that purposiveness towards the sophianic goal of creation as a whole.\footnote{Ibid., 259\textendash 260; and cf.~pp.~278\textendash 283 (from the Hale Lecture).} The Church, in other words, should not be wasting its energies resisting modernity as such, but must address the central \emph{deficit} in modernity (its reductionist fantasies of control) \emph{from within the cultural framework} \emph{modernity itself has shaped}.

\subsection*{The Church and the Transformation of Society}

However, this is not to conclude that the Church must simply adapt its perspectives and imperatives to this cultural framework. It is crucial, in Bulgakov's understanding, that the Church continue to see itself as more than a social agent among others, and that it refuses to be co-opted by any social or national polity as merely a contributor to "other and higher values."\footnote{Ibid., 263.} It is a position that has something significant in common with the views outlined by Dietrich Bonhoeffer in his last writings. In the fragmentary chapters of his \emph{Ethics}, still being written at the time of his arrest and imprisonment, Bonhoeffer insists that the Church should never seek to defend a position \emph{within} any social territory or state apparatus. The Church both signifies and embodies a defining network of relations, relations with God, the world and other subjects, in which the human person is maximally free for gift, love and mutuality, and so cannot let its life be defined by any narrower vision of common life.\footnote{Dietrich Bonhoeffer, \emph{Ethics}, English translation edited by Clifford J. Green (Minneapolis: Fortress Press, 2005), e.g.~62\textendash 64, 95\textendash 98, 339\textendash 350 (interestingly, Bonhoeffer refers on 341 to Soloviev).} Bulgakov, as we have seen, goes rather further in his concern to locate our ecclesial activity within the ongoing sophianic flow of divine action. For him, the Church's job in relation to the social enterprises and ideals that surround it in a modern and pluralist world is essentially one of discerning and evaluating how far this or that social project, this or that legal reform enables (or disables) human growth towards sophianic maturity, and thus towards the eschatological communion which is sacramentally present in the Church's life. But in this interpretative and discerning role, it cannot look to be a coercive decision maker for society at large.\footnote{Williams, \emph{Sergii Bulgakov}, 264\textendash 265.} It has to work at the transforming of motivation and vision; it has a creative role within any social order (mere ascetical withdrawal into uncompromised purity is not a sophianic option), in that it seeks to set out a vision for the human that will steadily press towards structural changes. Thus\textemdash to go back to some of the questions raised by New Testament ethics\textemdash St Paul's injunctions to slave-owners about how they are to view their slaves will gradually dissolve the structure of slavery. And we begin to change the class-based structures of modern society not by class warfare in the usual sense but by intensifying an awareness of the mutual dependence and mutual duties of different social classes so that we stop thinking of class in terms of superiority and inferiority, control and submission\textemdash once again, a theme that Bulgakov was already adumbrating in his work before the First World War.\footnote{Ibid., 265, and cf.~the earlier essay on "The Economic Ideal," ibid., pp.~27\textendash 53.} ~ ~

Bulgakov does not tackle the problem which more recent social theologies, especially the Latin American "theologies of liberation," have foregrounded: can we be certain that lasting structural change will come about if we persuade ourselves that the existing system (slavery, capitalism, legal discrimination against women \ldots{}) is being benignly administered? Is this not potentially an alibi that allows us to postpone the hard work of structural reform? But although this is not explicitly addressed, it is important to note that Bulgakov regularly deplores a purely "internalizing" response to social injustices; as we have noted, he assumes that there is work to be done both internally and externally, and his stress on transformational labor as the distinctive vocation of human beings goes some way to offset the risks of falling back into a static conception of social order. He is not indifferent to the need for transformed structures; but he \emph{is} skeptical of impatient programs for such transformation. "We must move away," he writes, "from a passive-quietist, conservative-as-similationist relationship to the work of society."\footnote{Ibid., 266.} And this is necessary in part so as not to leave a dangerous vacuum. He has already noted the risk that secular/pagan society has the potential to become a "pseudo-Church."\footnote{Ibid., 263.} What exactly does he mean by this? He is pointing to the way in which society can come to define itself as a \emph{comprehensive moral community}, a community in which the relationships that exist with and within \emph{this} group (ethnic, linguistic, class-based or whatever) or polity (free-market, communist or whatever) become the defining marks and boundaries of human obligation and aspiration as such. In such a setting, the one unsurpassable moral priority is to fulfil the expectations of this particular human collectivity. And if the Church has no perspective from which to ask questions about these priorities or expectations, it will not only betray its calling but tacitly collude with the claims of the collectivity to final authority. It will surrender to "other and higher values."

The point being made is directed not only against the naked political totalitarianisms of Bulgakov's era but also, as he clearly states, against the "soft totalitarianism" of managerial technocracy combined with a successful consumerizing of human leisure and culture.\footnote{Ibid., 266.} The Church is called on to resist marketized and functionally reductive models of human work and human connectedness. And it will do so above all\textemdash here we may recall the emphasis of Bulgakov's Hale lecture\textemdash by maintaining its theological self-understanding, its self-definition in relation to the sophianic and eschatological reality which it both partially embodies and entirely points to. This again has significant points of convergence with Bonhoeffer's ecclesiology, though the latter has a far less ambitious cosmological dimension and a less explicitly sacramental idiom. And it sets Bulgakov at odds with two kinds of Christian response to a secular or pagan modernity that have become more evident in recent decades.

\subsection*{The Church and Modern Challenges}

In the first place, Bulgakov's skepticism about any return to pre-modern categories or disciplines is plain. The laborious, agonized, diffuse reflections of the dialogues from his initial period of exile in the Crimea show how he turned away from the Slavophil idealization of the mediaeval\textemdash or more recent\textemdash Russian past. One of the participants in the long-unpublished "Chersonese" dialogues from these years in the Crimea is a Russian parish priest who is scathing in his depiction of a morally apathetic and half-Christianised Russian peasantry\textemdash much to the scandal of the Slavophil "lay theologian" who is arguing with him.\footnote{Bulgakov wrote \textit{U sten Khersonisa} in 1919\textendash 1920. For recent editions, see Bulgakov, \textit{Trudy po sotsiologii i teologii} (Moscow: Nauka, 1997), vol. 2: 126\textendash 133; and Serge Boulgakov, \textit{Sous les remparts de Chersonese}, translated and edited by Bernard Marchadier (Geneva: \textit{Ad solem}, 1999).} It is hard to see Bulgakov having the least sympathy with what has lately become familiar as the ideology of a \emph{Russkii mir}, or with the mythology of a straightforwardly messianic role for the Russian people, suffering but triumphant in conflict with Godless enemies.

Secondly, we might note that, while his insistence on the need for the Church to combat a pseudo-ecclesiology of the omnicompetent state might at first sight be called on to support Christian resistance to the contemporary state's moral agenda (typically to the liberalization of legal attitudes to sexual minorities, to abortion or to assisted dying, but also, in some contexts, as the last couple of years have shown especially in the USA, to state directives about health or education), Bulgakov is not in fact all that comfortable an ally for this variety of culture wars skirmishing. He is not, of course, concerned with the culture war questions of recent decades, the problems of gender and sexuality or the beginning and end of life. He is certainly not likely to have been a "revisionist" in respect of most if not all of these contested questions. But the core of his critique would, I suspect, not be the specific moral questions involved so much as the claims (explicit and implicit) made for the absolute authority and public unchallengeability of any new liberties granted by the state. His conviction that the Church should not be trying to restore past models of social control suggests that he does not regard public battles about legislation as the most significant site for Christian activism; and his clear opposition to anything resembling coercive religious uniformity implies that he has no difficulty in envisaging a society in which the Church does not have to win such legislative battles in order to sustain its integrity and be faithful to its foundation in Christ. He takes it for granted that the Church will be living in a largely desacralized or disenchanted environment; its task is not resacralizing but fidelity, persuasion, and what he calls "social creativity," a manifest willingness to work along with others for social transformation.

\subsection*{Church vs.~"Pseudo-Church"}

But the image of the "pseudo-Church" is one that deserves some further discussion in his context. We have begun to see how Bulgakov's engagement with the idea of the "\emph{soul} of socialism" is substantially a warning about the risks of accepting uncritically a scheme of underlying assumptions about human nature standing behind some kinds of social program and of regarding any social program as in itself bearing some kind of sacred and final significance. In an important sense, it is crucial for Bulgakov that socialism as a political practice \emph{does not have a "soul" of its own}\textemdash does not have, that is to say, a metaphysical and spiritual rationale distinct from and in competition with the cosmic hope which the Church represents. If it has a "soul" other than what the Church offers, it will inevitably become a pseudo-Church. It is clear enough that the secularized apocalyptic of Marxist-Leninism is a major target, as it was consistently for Bulgakov from 1905 onwards.\footnote{The celebrated essay of 1906 on "\emph{Karl Marks kak religioznyi tip}" {[}"Karl Marx as a Religious Type"{]} in \emph{Dva grada}, vol.~1, 69\textendash 105, is one classic locus for Bulgakov's exploration of revolutionary Marxism as secularized apocalyptic.} But, just as consistently, he has sketched a critique of what results when social goals and projects of any sort acquire a definitive and unarguable character. And he believes that they can do this just as much through the exhaustive reduction of human behavior to function, or to the mechanisms of desire and gratification, as through overt totalitarianism. Technocracy generates its own absolutism. Once functions have been defined and allocated in this shrunken world, no appeal is possible. And also, in a disenchanted world, once perceived problems of injustice have been resolved to the satisfaction of an established majority, there can be no quarrel with the legal settlement prescribed; the gap between a legal liberty conceded by social authority and a universal moral judgement about that concession begins to disappear. Thus, a society may legislate, say, to permit physician-assisted suicide; it may create a new legal liberty. It may then, tacitly or not so tacitly, organize itself so that the duty to facilitate this, or even the duty not to challenge its morality, is effectively enshrined in social practice and culture and is treated as a clear moral imperative in itself, since any such challenge may then fall under the rubric of potentially offensive speech which disadvantages or marginalizes others.

I think that something like this is what Bulgakov has in mind in his rather throwaway remark about the dangers of a "pseudo-Church"\textemdash a social order from whose purely legislative and administrative determinations there is no appeal. The possibility of intelligent dissent is treated by state and/or public opinion simply as something that seeks to de-legitimize certain legal developments; it becomes harder for the critic to say (or to be heard to say), "I accept that x is legal \emph{and} reserve the right to ask whether it is moral." The state's attempt to finalize issues, to close down moral debate, is in fact an aspect of that "end of history" mentality which Bulgakov regards as one of the characteristic temptations of the contemporary mind. And for contemporary moral and political theologians, one of the most difficult areas to negotiate is how to articulate the need for this critical space without simply buying in to a contrarian and reactionary agenda of the kind Bulgakov cautions us against. To put it provocatively (and I do not suggest that Bulgakov would have expressed his view in such terms or in relation to such topics), a provision like the legal recognition of persons who have undergone gender-modifying surgery may or may not be agreeable to a Christian anthropology; there is a serious discussion to be had in the context of Christian theology, as the issue is not crystal-clear for all Christians. But it is a mistake to suppose that, because of that uncertainty as to how Christian discourse might handle the question, a committed campaign to combat or reverse such legal recognition is the best use of Christian energy. The real and more intellectually tough issues are about the assumptions being made in the wider culture (social and legal) about the ideal uniformity of its moral judgements and the illegitimacy of debate, assumptions that would imply that when the legal issue is resolved there is nothing more for anyone to discuss. The theologian will argue that conferring certain rights to civil liberty does not foreclose such debate for good and all\textemdash so that the actual issue of conceding those rights is not the all-important matter. Indeed, it may be right in some circumstances for a Christian to argue for the legal protection of persons whose actions may be questionable but whose security from persecution and violence needs to be safeguarded. There may be a risk of apparently colluding with practices that might be theologically challenged, but there is equally a risk in being too ready to deny some civic dignities or liberties to some categories of person on the grounds of theological anthropology.

A pseudo-Church is a Body of Christ without Christ. The biblical language of the Body establishes the ideal of a community in which there is mutual recognition and gratitude, and a strong awareness of the shared character of the social good. But in the Church, what most deeply defines us\textemdash and therefore what we are most deeply accountable to\textemdash is our relation to Christ, and \emph{through} Christ to all other finite beings, not simply a pattern of human mutuality. We are creatively connected with all other beings, responsible for their growth and well-being and integrity, \emph{because} we are all equally related to the eternal Word that is incarnate in Jesus. This universal relation guarantees universal equality and mutuality; but it also, crucially, grounds the dignity of every finite subject or substance in a relation that is prior to any relation completely internal to the finite universe. The determinative focus of relation is not one element of the created order but a reality quite outside it\textemdash for Bulgakov, the Sophia which holds together the processes of the finite world as a mirror and medium for divine love or beauty, the Wisdom that is always seeking maximal realization within that finite world. Without this external focus, what controls or determines the values and goals of the elements of the finite world will be the resolution of tensions by law, which is necessarily dependent on consensus; and the trouble with consensus is that it so readily masks the relations of power between majorities and minorities. In this connection, what unifies a community and what secures equality will in fact be one finite power among others, the power of secular government; and the legitimacy of purely secular government can come only, in the last analysis, from force. In contrast, the Church in Bulgakov's perspective is constantly in the business of shaping a "culture" in which free persons acknowledge one another because of their recognition in one another of the divine image and their awareness of the sophianic calling they share. A Church that simply deployed "counterforce" in such a context would be stepping away from its essential calling of engaged and critical witness, the continued labor of discernment and the creative formation of a sophianic culture.

\subsection*{The Social Call of the Church}

For Bulgakov's social theology, the enemy is always the temptation to declare that history is over\textemdash whether that declaration is made by a theocratic Church or a determinedly and exclusively secular state in which public dissent is marginalized or silenced. And if history is \emph{not} over, if all historical action contains within it the prayer of longing for the full coming of Christ and the full realization of sophianic harmony, the task of the Church here and now is to work out which historical actions are in tune with that prayer, which actions open the world more fully to its \emph{telos} in Wisdom. This discerning labor is not the same as a series of campaigning programs, negative or positive; it is a clarifying of possibilities, whose outworking will continue to be argued over. But in this connection, one aspect of the Church's calling may be to seek out and support some of those dissenting voices which are active outside the standoffs of political struggle. These represent the perspectives that come from humanistic learning, the arts and, for that matter, the sciences: perspectives which in diverse ways insist that our human capacity and creativity cannot be reduced to performing predetermined function, or to systems of wanting and getting. Intellectual and imaginative creativity, as well as the social creativity Bulgakov underlines, will resist the seductive finalities of secular "ecclesiologies," ideological or managerial. The conviction that humanity is answerable to more than what currently suits a majority (or a majority government) is one of the things that preserves the possibility of the kind of cultural discussion which actually changes perceptions and opens doors to reframing questions. Cultures defined by absolutism and uniformity are in fact eccentric simply \emph{as} cultures because they foreclose the possibility of genuine learning which is at the heart of living culture. Perhaps if we wanted to characterize Bulgakov's vision for the calling of the Church in society, we could see it as a calling to be an agent precisely of \emph{learning,} witnessing to the truth that what we know of humanity before God and in relation to the rest of creation is always capable of deepening and enhancement and so must always be hospitable as well as critical.

It may sound odd to say that a central element of the Church's job in society is to desacralize its habits of thought. But Bulgakov's argument in these texts from the early thirties seems to be that the "socialist" mindset he identifies in a range of twentieth century phenomena carries with it an implied anthropology which can all too readily be treated as possessing ultimate, "sacred" authority\textemdash a mindset in which a failure to recognize the dignity of the person, and a set of assumptions about the determined nature of economic needs and functions, and their dominant importance in culture, end up trapping us in a world in which certain particular social programs cannot rationally or morally be resisted. This is what constitutes the pseudo-religious dimension that Bulgakov sees as the "soul" of contemporary anthropology; and it is what theology and ecclesial practice must continue to resist. In one sense, what Bulgakov argues is that the only legitimate "soul" that socio-political activity can possess is the genuine ecclesial vision, the sophianic hope of the renewed cosmos. Any other kind of soul is dangerously inhuman in his eyes. The Church is called on to "ensoul" the projects of the society in which it finds itself, according to its sophianic discernment\textemdash and in so doing, gradually to displace the ersatz "soul" of modernity, its reversion to paganism (which is always the assimilation of the sacred to what is visible and powerful). The Church announces, in its sacramental (and for Bulgakov, its iconographic) life, in its active diaconal witness and in its theological self-accounting, that humanity is already connected with its entire cosmic environment in more diverse and complex ways than we could have imagined; that its capacity for contemplative joy is always in excess of any satisfactory performance of functions prescribed by others; that its summons to mutual enrichment and nurture constantly puts our localisms and minor loyalties in question.

\subsection*{In Place of a Conclusion}

The Church listens and discerns; and it also asks to be listened to\textemdash listened to, not obeyed; but also listened to, not merely tolerated as a private eccentricity. It assumes freedom to engage in the social conversation. Bulgakov does not offer any schematic suggestions for what that might look like; his chief concern is that the Church should not lose sight of its own cosmic and eschatological role, or dissolve this into something instrumental to national, partisan or pragmatic agendas. And I suspect he assumes that if the Church does this with adequate robustness, it will command something more than grudging tolerance in society. For one thing, it will\textemdash ideally\textemdash show what authentic relation to the sacred looks like, as opposed to the awkward, potentially repressive, potentially contradictory discourses of the sacred that are generated by secular cultures anxious about the ground on which they stand. It will clarify what it might mean to speak of "soul" in politics without sentimentality or ideological ambition. We must not, says Bulgakov, give way to "eschatological panic": "The task is to educate the man who has been partially liberated from economic captivity, and who now faces the danger of spiritual repression in the wake of his liberation from the curse\textemdash which is also, though, just as much a blessing\textemdash of slavery to labor."\footnote{Williams, \emph{Sergii Bulgakov,} 266.} Modernity has broken out of a cycle of subsistence-based labor, and we cannot imagine simply reverting to a world in which this is the norm. Developments which may at first seem ambiguous and threatening are also pregnant with possibilities for new dimensions of creativity and new sensitivity about human mystery or dignity. Bulgakov offers us a nuanced and resourceful perspective on how the Church's future is bound up with its willingness to exemplify an anthropology capable of doing something like justice not only to human dignity but to the plain inexhaustibility, the excess, of the person-in-relation as that is uncovered precisely in the shadowed and dangerous post-Enlightenment world; it is a perspective that has not exactly dated in the nearly one hundred years since the publication of the texts we have been examining.

\begin{center}
  \includegraphics[width=0.75cm]{articlend.png}
\end{center}
\biobox{\textbf{Rowan Williams} is Honorary Professor of Contemporary Christian Thought in the University of Cambridge. From 2002 to 2012 he served as the 104th Archbishop of Canterbury. From 2013 to 2020 he was the 35th Master of Magdalene College, University of Cambridge. He was made a life peer in 2013, becoming Lord Williams of Oystermouth.}

\label{sec:williams}

\part{Research Articles}

% First article
\newpage
\abstractbox{Every Blade of Grass:}{The Divine Language of Nature in Dostoevsky}{Amy Singleton Adams}{This essay shows how Dostoevsky’s ideas about the natural world before and during his Siberian exile develops into an eco-aesthetic theology in his literary work and later informs his understanding of the ways in which art and artists perceive and represent the sacred in nature. It argues against misperceptions that nature plays no role in Dostoevsky’s work and points out key moments where the natural world becomes the setting for heightened religious experiences. Approaching the phrase "blade of grass" as shorthand for complex ideas about the divine language of nature, this study investigates Dostoevsky’s engagement with the Church Fathers (particularly St. Isaac the Syrian and Maximus the Confessor) in the period that began immediately after his release from prison. It examines how patristic thought about creation and the logoi of being are expressed starting with Dostoevsky’s early works like "The Little Hero" and continuing through his novels. The essay also contextualizes Dostoevsky’s understanding of nature as revelation and redemption within the thought of his European and Russian contemporaries\textemdash with special focus on the work of landscape painters Carl Gustav Carus and Alexandre Calame\textemdash who were also exploring the importance of noetic perception and approaching nature as a sacred, theophanic text.}{Dostoevsky, Siberia, nature, divine language, patristic texts, Church Fathers, The Little Hero, Zosima, Carus, Calame}

\section{Amy - Every Blade of Grass}

\fancypagestyle{chaptercontentpage}{
  \fancyhf{} % Clear all header and footer fields
\fancyhead[CE]{%
  \fontsize{11}{11}\leftmarkfont%
  \addfontfeature{LetterSpace=10.0}%
  \selectfont\textit{\MakeUppercase{\leftmark}}%
}
  \fancyhead[CO]{\authorheadfont\addfontfeature{LetterSpace=10.0}\fontsize{11}{11}\selectfont\textbf{{\uppercase{Amy Singleton Adams}}}}
  \renewcommand{\headrulewidth}{0pt} % No header rule on content pages
  \fancyfoot[LE, RO]{\thepage}
}

\chaptertitle{Every Blade of Grass}{The Divine Language of Nature in Dostoevsky}{Amy Singleton Adams}
\fancypagestyle{chaptertitlepage}{
  \fancyhf{} % Clear all header and footer fields
  \fancyhead[L]{\begin{minipage}[t]{0.7\textwidth}\publisher\end{minipage}}
  \fancyhead[R]{\begin{minipage}[t]{\textwidth}\raggedleft \datefont\fontsize{10}{11}\selectfont Volume 1 (2024): \thepage\textendash\pageref{sec:adams}  \\ \doi{10.71521/0zas-g023} \end{minipage}}
  \renewcommand{\headrulewidth}{0pt} % No header rule on title pages
  \fancyfoot[LE, RO]{\thepage} % Left on even pages, right on odd pages
}

\addcontentsline{toc}{chapter}{Every Blade of Grass:\\ The Divine Language of Nature in Dostoevsky \\ \textit{by} Amy Singleton Adams}

\epigraph{}{
  \textit{Эти бедные селенья, \\
Эта скудная природа,\textemdash \\
Край родной долготерпенья, \\
Край ты Русского народа! \\
Не поймет и не заметит \\
Гордый взор иноплеменный, \\
Что сквозит и тайно светит \\
В наготе твоей смиренной. \\}
\textbf{Fedor Tiutchev, 1857}}

\vspace{2em}
\setcounter{section}{0} % Reset the section counter
\setcounter{footnote}{0}

\noindent Swiss pastoral novelist Charles Ramuz once observed of Dostoevsky that "[t]here is no writer who shows greater poverty in 'landscapes' (\emph{paysages}). Nature is missing altogether from his works."\footnote{Charles Ferdinand Ramuz, « À propos de Dostoïevski », \emph{Aujourd'hui}, Lausanne, 13 novembre 1930, 291\textendash 292.} Some readers since then have acknowledged the importance of the natural world in Dostoevsky as an indicator of thematic structures, psychological states, moral vision, and even social status.\footnote{For examples, see the following: Dmitrii S. Likhachev, \emph{Poeziia sadov. K semantike sadovo-parkovykh stilei. Sad kak tekst} (Leningrad: Nauka, 1982); Iurii M. Lotman, "Obrazy prirodnykh stikhii v russkoi literatury (Pushkin\textemdash Dostoevsky\textemdash Blok)," in \emph{Biografiia pisatelia; stat'i i zametki, 1960\textendash 1990} (St.~Petersburg: Iskusstvo- SPB, 1995), 814\textendash 820; I. V. Gracheva, "Floristika v romanakh F. M. Dostoevskogo," \textit{Russkaia slovesnost'}, no. 6 (2006): 20\textendash 26; R. Kidera, "Zvezdy i tsvety v romane F. M. Dostoevskogo 'Brat'ia Karamazovy,'" \emph{Voprosy russkoi literatury} 78, no. 21 (2012): 106\textendash 113; L.V. Timerbaeva, "Floristicheskie obrazy v povesti F. M. Dostoevskogo 'Belye nochi,'" \emph{Mezhdunarodnyi studencheskii nauchnyi vestnik} 1, no. 5 (2015): 28\textendash 30.} Nevertheless, almost one hundred years later, Ramuz's observations are still widely shared. But such persistent misperceptions about Dostoevsky and the natural world risk fundamental misreadings of his work's religious and philosophical foundations. Instead, it is important to observe how, in his novels, stories, and essays, nature provides the setting and sometimes the catalyst for heightened and revelatory religious experiences. For instance, the dynamics of these spiritual epiphanies are evident in Dostoevsky's vision of the peasant Marei in the field, Raskolnikov's conversion experience on the Irtysh, Father Zosima's vision of paradise in the garden before his duel, and the star-gazing Alyosha Karamazov's realization about the profound unity of heaven and earth. The natural world also becomes the setting for the kind of spiritual struggle and striving we see in Myshkin's tearful "biblical" episode in the Swiss Alps or in the ways Versilov and the Ridiculous Man grapple with the limited promises of the Golden Age as earthly paradise.\footnote{Elizabeth Welt Trahan, "The Golden Age\textemdash Dream of a Ridiculous Man?" \emph{Slavic and East European Journal} 3, no. 4 (Winter, 1959): 349\textendash 371. See also Alexander Boyce Gibson, \emph{The Religion of Dostoevsky} (Eugene, OR: Wipf \& Stock: 2016) (previously published 1973), "Variations on Earthly Paradise," 154\textendash 168.} Given the widespread resurgence of discussion and discovery of patristic Christian texts during Dostoevsky's lifetime and his demonstrated interest in the works of the Church Fathers, it is also not surprising to find their ideas about the sanctity of the natural world in the words of characters like Father Zosima and the wanderer (странник) Makar, who both perceive the "mystery of God" as manifest and intelligible in every part of the natural world, even in "every blade of grass" (в каждой былинке, всякая-то трава).\footnote{On the patristic revival, see Patrick Lally Michelson, \emph{Beyond the Monastery Walls. The Ascetic Revolution in Russian Orthodox Thought}, \emph{1814\textendash 1914} (Madison: University of Wisconsin Press, 2017). Other sources of Dostoevsky's understanding of the sanctity of the created world include his lived experience of the liturgy and well as childhood reader. See I. S. Iarysheva, "Religioznaia zhizn' sem'i Dostoevskikh (1867\textendash 1881) v memuarakh A. G. Dostoevskoi," \emph{Problemy istoricheskoi poetiki} (2011): 233\textendash 242 and Gary Rosenshield, "Dostoevskii and the Book of Job: Theodicy and Theophany in \textit{The Brothers Karamazov}," \emph{Slavic and East European Journal} 60, no. 4 (Winter 2016): 612.}

But these major themes come later, after Dostoevsky's imprisonment and
exile in Siberia, where his understanding of the natural world as a
divine language took shape and where, by 1856, he had already formed a
vision of what he called "the mission of Christianity in
art."\footnote{F. M. Dostoevsky, Letter to A. E. Vrangel', 13 April
  1856, \emph{Polnoe sobranie sochinenii v 30-ti tomakh}, tom 28
  (Moscow: Nauka, 1985), 229.} The present study traces the beginnings
of what might be called the eco-aesthetic theology in Dostoevsky's work
to his early understanding of the natural world that is based in
engagement with certain patristic texts and ideas. It considers how the
Church Fathers like St.~Isaac of Syria, Maximos the Confessor, Gregory
Palamas, and their proto-hesychastic predecessors informed Dostoevsky's
view on the way art and artists perceive and represent the sacred in
nature. The repeated phrase "every blade of grass" becomes shorthand
for the complex ideas about the divine language of nature in early
Church writings that echo throughout Dostoevsky's work as it conveys the
idea itself.\footnote{Similarly, as Konstantin Mochulsky notes, by the
  time Dostoevsky wrote \emph{The Brothers Karamazov}, references to
  "little, sticky green leaves" had become proof for his characters of
  the existence of God and the eventual transfiguration of the world
  (\emph{Dostoevsky. His Life and Work} {[}Princeton: Princeton
  University Press, 1967{]}, 135).} Within such theology, all matter
exhibits and contains the mysterious energy and beautiful rationality of
God's creation. And, as the Cappadocian St. Basil the Great argued, the
smaller the object, the greater one's admiration for its Creator:
"Therefore, when you see the trees in our gardens, or those of the
forest, those which love the water or the land, those which bear
flowers, or those which do not flower, I should like to see you
recognizing grandeur even in small objects, adding incessantly to your
admiration of, and redoubling your love for the Creator."\footnote{Philip
  Schaff and Henry Wace, eds., \emph{Basil: Letters and Select Works}
  (Grand Rapids, MI: Eerdmans Publishing Co., 2002), 272.}

Versions of the phrase "every blade of grass" appear in most of
Dostoevsky's larger works, notably in \emph{The Adolescent}, \emph{The
Idiot}, and \emph{The Brothers Karamazov}, where the notion that the
natural world may offer a path toward forgiveness or redemption becomes
thematically important. But the key to understanding the import of the
phrase lies in Dostoevsky's earliest work and ideas that developed just
before and after his imprisonment in Siberia. Equally important to
identifying the first expressions of the relationship between the
natural world and the sacred in Dostoevsky is acknowledging their source
in theologies of creation from the patristic texts Dostoevsky so
admired, read, and collected.\footnote{See \emph{Biblioteka F. M.
  Dostoevskogo. Opyt rekonstruktsii. Nauchnoe opisanie} (St.~Petersburg:
  Nauka, 2005).} Only then can we dispel notions that the representation
of nature in his work is sparse, perfunctory, or even mythical in its
origins. To do so, I investigate how Dostoevsky's vision of the natural
world as iconographical developed early on and, nearly forty years
later, continued to play a significant role in his mature work and the
intellectual life of his day. Like much of what Dostoevsky created, the
theophany of the natural world is polyphonic\textemdash visual, spoken, and
textual\textemdash echoing Biblical encounters of God in nature, through His
voice, and in scripture. It is not surprising, then, that, along with
his stated interest in the works of the Church Fathers after his release
from the Omsk prison in 1854, Dostoevsky also demonstrated concern for
the fate of the only piece of writing he completed during his detention
in St.~Petersburg's Peter and Paul Fortress in 1849\textemdash "The Little
Hero" (Malen'kii geroi, a.k.a., A Child's Tale {[}Detskaia skazka{]}).
In this "little thing," as he called it, we see his first fictional
representation of the natural world as a manifestation of the divine and
the first instances of "verbal" paintings that show how he was already
engaged with ideas about landscape as an iconographical or sacred visual
language. His request from Siberia for the theoretical writings of
German landscape painter Carl Gustav Carus (1789\textendash 1869) underscores this
concern with the visual representation of nature as a kind of theophany,
an idea that Dostoevsky explored further through the work of Swiss
painter Alexandre Calame (1810\textendash 1864).\footnote{See Dostoevsky's request
  for the work of Carus and those of the Church Fathers in his letters
  to his brother Mikhail in January and February of 1854, in
  \emph{Polnoe sobranie sochinenii v 30-ti tomakh}. Tom 28. Pis'ma
  1832\textendash 1859 (Leningrad: Nauka, 1985), 171\textendash 84.} The aim of this study is
to explore the representation of the natural world in Dostoevsky as a
complex and multi-valent expression of aesthetic theology and, in doing
so, offer new ways of reading his work. Exploring the Orthodox regard of
creation as sacred may also be as important to our world\textemdash a world that
seems as bent on self-destruction as some of Dostoevsky's characters\textemdash as it was to his essentially spiritual intellectual project.

The importance of Orthodox thought to Dostoevsky's work and his vision
of Russia is difficult to overstate, although how the patristic vision
of creation informs the representation of nature in Dostoevsky's work
has rarely been studied.\footnote{Of particular note is Bruce V. Folz,
"Shook Foil and Trodden Sod: Nature, Beauty, and the Holy,"
\emph{Environmental Philosophy} 1, no. 1 (Spring 2004): 47\textendash 57 and
\emph{The Noetics of Nature. Environmental Philosophy and the Holy
Beauty of the Visible} (New York: Fordham University Press, 2014). See
also Robert Louis Jackson, "The Root and the Flower. Dostoevsky and
Turgenev: A Comparative Esthetic," \emph{The Yale Review} LXII, no. 2
(December 1973): 228\textendash 250 and David J. Leigh, "The Philosophy and
Theology of Fyodor Dostoevsky," \emph{Ultimate Reality and Meanin}g
33, no. 1\textendash 2 (March 2010): 85\textendash 103.} In fact, the link between the
natural world and Orthodox spirituality in Dostoevsky's work is often
regarded as a form of nature-mysticism far removed from the tenets of
formal Christian thought on which they are based.\footnote{Sergei
Hackel's 1983 reading of Dostoevsky's treatment of the natural world
as a "culture of the earth" and "little more than nature
mysticism" ("The religious dimension: vision or evasion? Zosima's
discourse in \emph{The Brothers Karamazov}," \emph{New Essays on
Dostoevsky}, ed.~Malcolm Jones and Garth Terry [Cambridge: Cambridge
University Press, 1983], 144, 164) was later echoed in the work of Stephen
Cassedy (\emph{Dostoevsky's Religion} [Stanford, CA: Stanford University Press,
2005], 156) and amplified by others. For example, see Julian W.
Connolly, "Dostoevskij's Guide to Spiritual Epiphany in 'The Brothers
Karamazov,'\," \emph{Studies in East European Thought} 59, nos. 1\textendash 2
(June 2007): 39\textendash 54 and Gary Rosenshield, "Dostoevskii and the Book of
Job: Theodicy and Theophany in 'The Brothers Karamazov,'\,"
\emph{Slavic and East European Journal} 60, no. 4 (Winter 2016):
609\textendash 632.} (Even A. E. Vrangel', who noted Dostoevsky's Christological
fervor during their time in Siberia, regarded the writer's faith to be
somewhat pantheistic.)\footnote{See Vrangel's letter to his
father, cited in Joseph Frank, \emph{Dostoevsky: The Years of Ordeal,
1850\textendash 1859} (Princeton, Princeton University Press, 1983), 193.} To
better understand the distinction in Dostoevsky's work between nature
worship and the expression of a patristic understanding of creation, we
recall the words of the fourth-century Cappadocian St.~Basil, whose name
(along with those of his fellow Cappadocian and brother St.~Gregory of
Nyssa) Dostoevsky calligraphed in his notebook in 1868 while working out
ideas for \emph{The Idiot}.\footnote{Konstantin Barsht, "Zhitie i
tvoreniia Grigoriia Bogoslova v tvorchestve F. M. Dostoevskogo,"
\emph{Kul'turnyj palimpsest. Sbornik statei k 60-letiiu V. E. Bagno}
(St.~Petersburg: Nauka, 2011), 45\textendash 62.} In his first sermon on
creation (\emph{Hexaemeron}), St.~Basil entreated his listeners to
approach the natural world as the visible manifestation of an
"incomprehensible and invisible" God that is suffused everywhere with
the divine energy (\emph{energeia)} of creation: "I want creation to
penetrate you with so much admiration that everywhere, wherever you may
be, the least plant may bring to you the clear remembrance of the
Creator. {[}\ldots{}{]} A single plant, a blade of grass is sufficient to
occupy all your intelligence in the contemplation of the skill which
produced it."\footnote{Schaff and Wace, \emph{Basil}, 265.}
This idea that the contemplation of a single blade of grass may lead to
divine revelation dramatically shifted the paradigmatic relationship
among humans, their gods, and the natural world familiar to Classical
and Hellenistic cultures.\footnote{On the objectification of nature in Greek poetry and philosophy, see Mark Payne, "The Natural World in Greek Literature and Philosophy," \emph{Oxford Handbook Topics in
Classical Studies} (online edn, Oxford Academic, 1 Apr.~2014),
https://doi.org/10.1093/oxfordhb/9780199935390.013.001, accessed 23
May 2024. In his study, "Nature and the Greeks" (\emph{Nature and
the Greeks and Science and Humanism} [Cambridge University Press,
2014], 3\textendash 102), physicist Erwin Schrödinger emphasized the
subject-object relationship between humans and nature in Classical
Greek thought, concluding that the resulting ideas about the
knowability of nature informs the modern scientific view of the
natural world.} With the rise of ancient Christianity,
environmental ethicist Bruce Folz points out, nature ceases to function
as the "mask for some shrouded deity" and, newly regarded as "deeply
revelatory of divine truth," takes on its own ontological significance
and its own relationship with its Creator.\footnote{Foltz,
\emph{Noetics of Nature}, 11, 236.} "This relationship to nature,"
Pavel Florensky writes in \emph{The Pillar and Ground of Truth},
"becomes conceivable only when people saw in creation {[}\ldots{}{]} an
independent, autonomous, and responsible creation of God, beloved of God
and capable of responding to His love."\footnote{Pavel Florensky,
\emph{The Pillar and Ground of the Truth} (Princeton: Princeton
University Press, 1997), 210.}

For Dostoevsky, the perception of the "grandeur" of the Creator in an
object as small as a blade of grass required a special kind of vision.
Casual admirers of landscapes (Vrangel' was sometimes frustrated by
Dostoevsky's indifference to the "marvelous" {[}дивный  пейзаж{]} views
of the Siberian steppes), those who study the natural world
scientifically or, like \emph{The Adolescent}'s Versilov, embraced the
"Geneva" idea of Godless nature, all lacked the ability to detect the
divine in the natural world.\footnote{On the antithetical relationship
  between the patristic approach to nature and "scientism," see
  Elizabeth Theokritoff, \emph{Living in God's Creation. Orthodox
  Perspectives on Ecology} (St.~Vladimir's Seminary Press, 2009), 44. On
  how this tension plays out in \emph{The Adolescent}, see Charles
  Arndt's "Wandering in Two Different Directions: Spiritual Wandering
  as the Ideological Battleground in Dostoevsky's \emph{The
  Adolescent}," \emph{The Slavic and East European Journal} 54, no. 4
  (Winter 2010): 607\textendash 625.} Instead, to perceive God's mystery in all
things, one needs to develop the kind of "spiritual sight" described
by St. Isaac of Syria, a spiritual descendent of the Cappadocians and
whose homilies Dostoevsky knew well.\footnote{St.~Isaac of
  Nineveh, \emph{The Ascetical Homilies}, "The Second Part," Chapters
  IV-XLI, Scripores Syri, Tomus 225, trans. Sebastion Brock (Lovani,
  Belgium: Aedibus Peeters, 1995), 84.} As Foltz points out, those who
enjoy this type of sight are often the pilgrim, seeker, or ascetic who
"leaves behind, even if temporarily, the conventions and convenience of
the urban world and becomes immersed in nature."\footnote{Foltz,
  \emph{Noetics of Nature}, 140.} Analogous characters in Dostoevsky can
convey their vision in ways that echo such ideas in the early Christian
texts. In the patristic experience and understanding of the mystery of
the natural world, for example, each created being constantly
demonstrates its intimate relationship with God. Using the labor-loving
honeybee to model this interrelation between God and the natural world,
Gregory of Nyssa notes how the creature joyfully demonstrates the
intelligence of its own creation. "{[}E{]}verything orients itself
perfectly toward celebration and rejoices. Look, what we can see!
{[}\ldots{}{]} Already the industrious honeybee, spreading its wings and
leaving the hive, demonstrates its wisdom" (« \ldots{} все прекрасно
собирается к торжеству и радуется. Смотри, каково видимое! (\ldots{}) Уже
трудолюбивая пчела, расправив крылья и оставив улей, показывает свою
мудрость»).\footnote{Grigory Bogoslov.~Slovo 44, \emph{Tvoreniia sviatykh ottsov v russkom perevode s Prebavleniami dukhovnogo soderzhaniia, izdavaemye pri Moskovskoi dukhovnoi akademii} (Мoscow, 1844). Part 4, 150.} In Dostoevsky, these thoughts are most clearly
developed in the character of Father Zosima who contemplates the "great
mystery" of creation in one well known example and observes how "Every
little blade of grass, every little insect, ant, and little golden bee
to the point of amazement knows its own path, and not possessing human
intelligence, witnesses the divine mystery" (Всякая-то травка,
всякая-то букашка, муравей, пчелка золотая, все-то до изумления знают
путь свой, не имея ума, тайну божию
свидетельствуют).\footnote{F. M. Dostoevsky, \emph{Brat'ia
Karamazovy}, \emph{Polnoe sobranie sochinenii v 30-ti tomakh}, tom 14
(Leningrad: Nauka, 1976), 267.} Dostoevsky's "wanderer" Makar also
detects how the mystery of creation is expressed in natural forms,
explaining to the "adolescent" Arkadii how "Every tree, every blade
of grass (в каждой былинке) contains the mystery (эта самая тайна
заключена) and indescribable beauty {[}of creation{]}
itself."\footnote{F. M. Dostoevsky, \emph{Подросток}, \emph{Polnoe
  sobranie sochinenii v 30-ti tomakh}, tom 13 (Leningrad: Nauka, 1975),
  287.}

The Desert Father and early founder of monastic practice St.~Anthony the
Great was perhaps the first Christian thinker to envision nature as a
sacred and theophanic text. Asked once how a thoughtful man such as he
could live without books, he answered, "My book is this created nature.
It is always with me, and when I wish I can read in it the words of
God."\footnote{Quoted in Foltz, \emph{Noetics of Nature}, 237.} Using
similar imagery, St.~Isaac the Syrian regarded the natural world as
"the first book given by God to rational creatures" and emphasized
how, in the Sermon on the Mount and through parables, Christ himself
"confirmed all spiritual things by examples from the things in nature"
as "windows upon the workings of God in the world."\footnote{Foltz,
  \emph{Noetics of Nature}, 194\textendash 96.} Like the visually discursive icon,
the natural world provides a material and sensible guide for
understanding God's creative will (Logos)\textemdash the \emph{logoi} of being
or "inner essence of things," which St.~Maximos the Confessor
envisioned as a link between two worlds (an idea reflected in Father
Zosima's notion of God's "seeds from different worlds").\footnote{David
  Bradshaw, "The \emph{Logoi} of Being in Greek Patristic Thought," in
  \emph{Toward an Ecology of Transfiguration. Orthodox Christian
  Perspectives on Environment, Nature, and Creation}, ed.~by John
  Chryssavgis and Bruce V. Foltz (New York: Fordham University Press,
  2013), 9, 17.} "The whole intelligible world," he reasoned, "seems
mystically imprinted on the whole sensible world in symbolic forms, for
those who are capable of seeing it, and conversely the whole sensible
world subsists within the whole intelligible world, being rendered
simple, spiritually and in accordance with intellect, in its rational
principles (\emph{logoi})."\footnote{Bradshaw, "\emph{Logoi} of
  Being," 16.} When Alyosha Karamazov suddenly realizes that "that the
mystery of the earth was one with the mystery of the stars," for
example, he uses Maximus's metaphor of the connectedness of God's
\emph{logoi} as a circle from which all things radiate: "It was,"
Alyosha thinks, "as if threads from all those innumerable worlds of God
all came together in his soul (нити ото всех этих бесчисленных миров
божиих сошлись разом в душе), and it was trembling all over, 'touching
other worlds.'\,"\footnote{F. M. Dostoevsky, \emph{Brat'ia Karamazovy},
  \emph{Polnoe sobranie sochinenii v 30-ti tomakh}, tom 14 (Leningrad:
  Nauka, 1976), 328. See Torstein Theodor Tollefsen, \emph{The
  Christocentric Cosmology of St. Maximus the Confessor} (Oxford: Oxford
  University Press), 50.}

For St.~Maximos, the creative "words" (\emph{logoi}) of God expressed
visibly in every living thing make up an intelligible discourse or
"interface," as Elizabeth Theokritoff calls it, between everything and
its Creator and become a way for creatures to know God.\footnote{Theokritoff,
  \emph{Living in God's Creation}, 55.} Those "capable of seeing" or
otherwise sensing the \emph{logoi} of being in creation can develop this
awareness through the spiritual contemplation of nature (\emph{theoria
physike}). For St.~Isaac of Syria, \emph{theoria physike} was a state of
consciousness that resulted from ascetic practice and revealed "God's
creative power and the beauty of created things."\footnote{Quoted in
  Foltz, \emph{Noetics of Nature}, 200.} For St. Maximos the Confessor,
it was a form of spiritual practice designed to perceive the spiritual
principles of created beings.\footnote{Metropolitan Jonah (Pafhausen),
  "Natural Contemplation in St. Maximus the Confessor and St.~Isaac the
  Syrian," in \emph{Toward an Ecology of Transfiguration}, 48\textendash 49.} As
Vrangel' described, the Siberian night sky had a similar effect on
Dostoevsky in the mid-1850s.

\begin{quote}
One of his favorite pastimes on a warm evening was to spread out on the
grass and, lying on our backs, to gaze on the myriad stars flickering in
the depths of the dark-blue sky. These moments gave him peace. The
contemplation of the majesty of the Creator (созерцание величия Творца),
the all-knowing and all-powerful Divine force instills in us some kind
of tenderness, the consciousness of our own insignificance, and somehow
soothed our souls.\footnote{A. E. Vrangel, \emph{Vospominaniia o F. M.
Dostoevskom v Sibiri 1854\textendash 56} (St.~Petersburg: Tipografiia A. S.
Suvorin, 1912), 52.}
\end{quote}

\noindent Like St. Isaac of Syria, who defined true compassion as a "heart on
fire for all creation," Dos\-to\-ev\-sky's experience suggests that
\emph{theoria physike} is a kind of loving discourse, whereby the
"divine force" that instills "some kind of tenderness" in the human
soul is also the object of love itself. As Father Zosima exhorts, love
for all creation reveals the "divine secret" inherent in everything:
"Love all of God's creation (всё создание божие), the whole of it as
well as every grain of sand. Love every little leaf, every divine ray of
light (каждый луч божий). Love the animals, love the plants, love each
thing (всякая вещь). If you love each thing, then you will understand
the divine mystery in all things (тайну божию постигнешь в
вещах)."\footnote{F. M. Dostoevsky, \emph{Brat'ia Karamazovy},
  \emph{Polnoe sobranie sochinenii v 30-ti tomakh}, tom 14 (Leningrad:
  Nauka, 1976), 289.}

In several places in \emph{A Writer's Diary}, Dostoevsky equates the
diminishment of faith with a lost connection with this language
(\emph{logoi}) of nature.\footnote{See "In Lieu of an Introduction On
  the Big and Little Dipper, On the Prayers of the Great Goethe and
  About Nasty Habits in General" (January 1876); "Don Carlos and Sir
  Watkin. Once More About 'Signs of the End'" (March 1876), \emph{Polnoe
  sobranie sochinenii v 30-ti tomakh}, T. 22 (Leningrad: Nauka, 1981),
  5\textendash 6, 91\textendash 97. In \emph{Noetic of Nature}, Foltz notes how in
  Dostoevsky "corruption of the heart is connected to an alienation from
  nature, and {[}\ldots{}{]} redemption is often\textemdash as is the case with
  Raskolnikov\textemdash heralded by a restored relation to nature" (189).}
Indeed, he notes how the very idea of allowing oneself to open up
(раскрыться) to or dissolve (ратвориться) into nature in the way
St.~Basil entreated or the way Dostoevsky himself did in Siberia would
be regarded as "indecent" (неприлично) in society.\footnote{"What
  Really Helps at Spas\textemdash The Waters or the Bon Ton?" (July and August 1876),
  \emph{Polnoe sobranie sochinenii v 30-ti tomakh}, T. 23 (Leningrad:
  Nauka, 1981), 84.} On his deathbed, Father Zosima's brother Markel
repents of this fashionable atheism, asking "God's little birds" to
forgive him for refusing to see the divine in nature: "{[}T{]}here was
such divine glory (божия слава) all around me: the birds, the trees, the
meadow, the sky (небеса), but I alone lived in disgrace (в позоре), I
alone dishonored everything (обесчестил), and never noticed the beauty
and the glory at all (красы и славы не приметил вовсе)."\footnote{F. M.
  Dostoevsky, \emph{Brat'ia Karamazovy}, 263.} Society's ironic attitude
toward the natural world is challenged in Dostoevsky's work by a child's
perspective. The reminiscence that imaginatively lifts Dostoevsky out of
the grim reality of the Omsk prison, "The Peasant Marei" (1876), has
the then nine-year old Dostoevsky describe in loving detail the insects,
creatures, and plants in a birchwood fragrant with a groundcover of dead
leaves. This mental contemplation of the natural world of the past
presages the story's well-known spiritual revelation. But more than
twenty-five years before he published "The Peasant Marei," Dostoevsky
wrote another story from prison about a child and the divine language of
the natural world. "A Little Hero" (written 1849, published 1857)
provides the first glimpse of the patristic roots of Dostoevsky's
eco-aesthetic theology.

\subsection*{Say It With Flowers: "The Little Hero"}

\noindent "The Little Hero" was the only work that Dostoevsky completed during
his detention in the Alekseevskii Ravelin of the Peter and Paul Fortress
in the summer of 1849. The draft he produced there seems to have
remained in the forefront of his mind throughout his Siberian
imprisonment. In his first letter after his release from the Omsk prison
in 1854, he instructed his brother Mikhail (who received and kept the
manuscript and other papers after Dostoevsky's sentencing) not to show
the story to anyone, hinting that he intended to rework the
text.\footnote{See V. D. Rak, "Kommentarii, 'Malen'kij geroi.'\,"
  \emph{F. M. Dostoevskii. Sobranie sochinenii v 15-ti tomakh}. T. 2
  (Leningrad: Nauka, 1988), 576\textendash 578.} "You have to agree," he wrote to
his brother later in 1857 as he anticipated the restoration of his right
to publish, "that the fate of the little thing (вещица), 'A Child's
Tale,' is interesting to me for many reasons."\footnote{Letter to M.M.
  Dostoevskii, 9 March 1857. \emph{Sobranie sochinenii v 15-ti tomakh},
  T.15 (Leningrad: Nauka, 1996), 173.} One of those reasons may have
been Dostoevsky's well substantiated belief that the story may have been
the last "thing" he would ever publish (he was sentenced to death in
November of the same year). It is not unreasonable, therefore, to read
the work as a kind of profession of faith. Another reason for the tale's
importance may be that the process of its composition allowed Dostoevsky
to endure and\textemdash as in "The Peasant Marei"\textemdash even transcend the
conditions of his imprisonment. "When I found myself in the fortress,"
he told the novelist Vsevolod Solov'ev in 1874, "I thought it would be
the end of me, that I couldn't last three days and\textemdash suddenly I found
myself completely at peace. Do you know what I did? \ldots{} I wrote 'The
Little Hero.'\,"\footnote{Rak "Kommentarii, 'Malen'kii
  geroi,'\,"576\textendash 577.} Reimagining the enchanted landscape of his
family's provincial estate, Dostoevsky could escape through the
consciousness of a child to a festive summer gathering. "I, of course,
chase all temptations from my imagination, but, at other times, I'm
unable to manage and my former life breaks into my soul and the past
lives again," he wrote to his brother at the time.\footnote{Letter to
  M.M. Dostoevsky, 18 July 1849. \emph{Polnoe sobranie sochinenii v
  30-ti tomakh}, T. 28 (Leningrad: Nauka, 1985), 156.}

Like his writer's forays into his remembered past, reading materials
from his brother and the fortress library provided Dostoevsky aesthetic
relief from the reality of his prison cell. With these sources, he
thought out themes that later developed in his work through subtexts
that include significant works of European literature. Considering the
conditions under which Dostoevsky wrote "The Little Hero," the
peaceful country setting and the joyful theatricality of the gathering
(gently disturbed only by the angst of entangled love plots) are
striking; they evoke Shakespeare's "Much Ado About Nothing," Molière's
"Tartuffe," and Mozart's "The Marriage of Figaro." A lesser-known
literary model is Tasso's sixteenth-century fictionalized telling of the
First Crusade, \emph{Jerusalem Delivered}. Tasso's text provides the
story with its chivalric imagery and the "steed" Tancred, the
dangerous mount that allows the hero, as Barry Scherr notes, to change
his symbolic social status from enamored page-boy (паж) to "knight"
(рыцарь).\footnote{On the Tasso subtext and themes of chivalry, see
  Tatiana G. Magaril-Iliaeva, "Rasskaz F. M. Dostoevskogo 'Malen'kii
  geroi' kak initsiaticheskii tekst," \emph{Dostoevskii i mirovaia
  kul'tura} 18, no. 2 (2022): 18\textendash 39. On the significance of the horse
  Tancred (named for a leader of the First Crusade) see Barry P. Scherr,
  "Dostoevskii on Horseback," \emph{The Slavic and East European
  Journal} 61, no. 4 (Winter 2017): 674\textendash 695.}

The courtly language of flowers was widely understood and practiced by
the European and Russian upper classes through the nineteenth century
and becomes an indicator of this emotional transition in Dostoevsky's
"little hero."\footnote{On the semantics of emotion expressed through
  garden design and flora, see Dmitrii S. Likhachev, \emph{Poeziia
  sadov.}} Although he is perplexed by the sentimental intricacies of
Madame M.'s secret love affair, the garden scene late in the story shows
how the young hero begins to understand the non-verbal social language
of the adult world, although his gesture\textemdash delivering in a bouquet of
flowers a letter from Madame M.'s lover, despite the boy's own affection
for her\textemdash remains chaste and chivalric. The boy hero hurries to the
fields to gather the bouquet of wildflowers. The list of specific blooms
(a rarity in Dostoevsky's work) includes wild roses, field jasmine,
cornflower, forget-me-nots, campanula, dianthus, lilies, violets, along
with stalks of rye, tall grasses, and maple leaves. Whether understood
via heraldic symbolism or the Romantic language of flowers, the bouquet
symbolically underscores the purity of the boy's love. But the flowers
\textemdash and the letter concealed in it\textemdash go unnoticed by Madame M. until a
"blessed event" (благословенный случай) takes place. A "large golden
bee" (большая золотая пчела) seems to purposefully (будто нарочно)
bother Madame M. until, using the bouquet to wave the bee away, she
discovers the letter. The scene is revelatory as the boy gives himself
over to a "new consciousness, the revelation of {[}his{]} heart, and
the first, unclear insight into {[}his own{]} nature."\footnote{F. M.
  Dostoevskii. "Malen'kii geroi," \emph{Polnoe sobranie sochinenii v
  30-ti tomakh}. T. 2 (Leningrad: Nauka, 1972), 393.} The boy does not
engage with Madame M. beyond her grateful kiss and carefully notes that
he never looks at her after she opens the letter. Instead, feigning
sleep on the grass, he seems to dissolve into the natural world\textemdash he
imagines that he becomes a "bird caught by a curly-haired village lad"
and "a blade of grass" (былинка)\textemdash as he experiences an
"unforgettable moment" of sweet suffering.

Considering the role of the bee as witness to the divine mystery of
creation expressed in patristic thought and in Father Zosima's words,
the presence of the golden bee in this final scene suggests that\textemdash
beyond the end of his "first childhood"\textemdash the boy has discovered
through the natural world a more spiritual understanding of love and
sacrifice. As the deathbed confession of Zosima's brother Markel shows,
this kind of spiritual understanding of the natural world is uncommon.
Dostoevsky makes this point explicitly in the summer of 1876, writing in
\emph{Diary of a Writer} how the social language of flowers is often
limited to insincere communication among people, allowing "contact
with nature only as far as politeness and \emph{bon ton} will permit"
(они соприкасаются с природой лишь насколько позволяют приличие и
хороший тон).\footnote{"What Really Helps at Spas\textemdash The
  Waters or the Bon Ton?" (1876), \emph{Polnoe sobranie sochinenii v
  30-ti tomakh}, T. 23 (Leningrad: Nauka, 1981), 84\textendash 88.} In his critique
of this trend, Dostoevsky invokes the Sermon on the Mount (Matthew 6:
28\textendash 29), entreating his readers to recall the passage he quotes as:
"Do not worry about what you wear, look at the flowers of the
field, even Solomon in all his glory was not dressed like one of these,
will not God dress you better?" (Не заботьтесь во что одеться,
взгляните на цветы полевые, и Соломон во дни славы своей не одевался как
они, кольми паче оденет вас Бог).\footnote{"What Really Helps at Spas\textemdash
  The Waters or the Bon Ton?" (1876), 87. Translation, King James Bible.} 
Although he inaccurately quotes the
passage\textemdash leaving out "they toil not, neither do they spin"
(ни трудятся, ни прядут) and substituting the more general "flowers"
for "lilies"\textemdash Dostoevsky ensures his readers that this
passage expresses "the entire truth of nature" (вся правда природы), a
statement that makes a similar reference to the Sermon on the Mount in
"The Little Hero" seem more intentional and meaningful to the
tale.\footnote{In his reading of one nature scene in "The Little Hero"
  (Malen'kii geroi), Joseph Frank detects a vague "religious meaning,"
  but conjectures that what seems to him a sudden and incongruous
  "prayer of nature" may simply be the "accidental result" of
  Dostoevsky's prison readings, which included the New Testament and the
  life of St.~Dmitrii of Rostov (\emph{Dostoevsky. The Years of Ordeal,
  1850\textendash 1859} {[}Princeton: Princeton UP, 1983{]}, 27\textendash 31).}

Before he gathers his bouquet, the boy spends the morning alone in the
riverside glades then returns to the more manicured gardens closer to
the house. He finds Madame M. sitting under an elm tree surrounded by
ivy, jasmine, and wild roses. Here, the narrative pauses to present the
surrounding landscape through the boy's eyes.  The language of
this passage recalls pastoral landscape paintings of the early
nineteenth century and anticipates the spiritual symbolism of visual art
we see in Dostoevsky's later work.

\begin{quote}
The sun was already high and was floating gloriously in the deep, dark
blue sky, as though melting away in its own light. The mowers were by
now far away; they were scarcely visible from our side of the river;
endless ridges of mown grass crept after them in unbroken succession,
and from time to time the faintly stirring breeze wafted their
sweet-smelling exhalations (благовонная испарина) toward us. The never
ceasing concert of those who "sow not, neither do they reap" (не жнут
и не сеют) and are free as the air they cleave with their sportive wings
was all about us. It seemed as though at that moment every flower, every
blade of grass (каждый цветок, последняя былинка) was giving off a
sacrificial aroma (\foreignlanguage{russian}{курясь жертвенным ароматом}), saying to its Creator,
"Father, I am blessed and happy" (\foreignlanguage{russian}{говорила создавшему ее: "Отец! я
блаженна и счастлива!")}\footnote{F. M. Dostoevskii. "Malen'kii geroi,"
  \emph{Polnoe sobranie sochinenii v 30-ti tomakh}. T. 2 (Leningrad:
  Nauka, 1972), 364.}
\end{quote}

\noindent The boy does not regard the natural surroundings as an objective
observer or casual admirer; rather, he perceives in it the kind of
cosmic liturgy captured in the reference to Matthew 6:26 (a passage read
in the Orthodox liturgy just before the July setting of the tale) about
the birds who, in contrast to the mowers in the distance, "sow not,
neither do they reap."\footnote{Matthew 6:26. Matthew 6: 22\textendash 33 is read
  on the third Sunday after Pentecost, which, between 1849 and 1857,
  fell in late June/early July.} In Dostoevsky, as in the Sermon on the
Mount, it is the pure of heart who "see" God. Here, the young boy
detects what Gregory of Nyssa described as the "trace of divine
fragrance" in creation, noting the "sweet exhalations" of fresh cut
grasses as well as the "aroma of sacrifice" given off by "every blade
of grass" that joyfully witnesses its own creation.\footnote{See Bruce
  V. Foltz, "Traces of Divine Fragrance: Droplets of Divine Love: On
  the Beauty of Visible Creation," \emph{Toward an Ecology of
  Transfiguration}, 324\textendash 336.} In this passage, the doxological nature of
the created world combines with the sacramental nature of the
relationship between the Creator and the created. As the gifts of bread
and wine are offered back to God, so, too, the sacrificial aroma
(жертвенный аромат) of "every blade of grass" "entails a constant
connection with the Giver" and "a cause of endless
gratitude."\footnote{Theokritoff 79.}

The boy's ability to perceive the "truth" of nature as a divine rather
than social language results from the period of contemplation of nature
(\emph{theoria physike}) that precedes his encounter with Madame M.
Rising early that morning, he leaves the garden behind, seeking out the
thickest parts of the groves near the river (где гуще зелень), where
"the resinous scent of the trees was strongest and rays of sunlight
rejoiced that they were able to penetrate the shady gloom of the thick
leaves" (где смолистее запах деревьев и куда веселее заглядывал
солнечный луч, радуясь, что удалось там и сям пронизать мглистую густоту
листьев). In this "wondrous morning" (прекрасное утро), the young hero
gradually surrenders his awareness of self as sensory experience gives
way to the contemplation of his natural surroundings, and he
"unconsciously" (незаметно) walks farther and farther into the copse.
There, "staring" (засмотрелся) at the light playing off of the mowers'
scythes, he loses himself in contemplation (Уж не помню, сколько времени
провел я в созерцании) until the sound of hoofbeats brings him back to
his senses (вдруг очнулся). As Dostoevsky noted from the confines of his
prison, immersion in the natural world\textemdash even in memory\textemdash can be
transcendent and transformative.

Such descriptions of the outdoor setting in "The Little Hero" reflect
the growing awareness of the contemplation of nature as a path toward
revelation that was transforming the art world in Dostoevsky's lifetime.
The characterization of Madame M. as living Madonna and, as Tat'iana
Magaril-Il'iaeva notes, the implicit description of Titian's "Sacred
and Profane Love" (1514) in the garden scene show Dostoevsky already
turning to the visual arts as a means of expressing his eco-aesthetic
theology.\footnote{Tat'iana Georgievna Magaril-Il'iaeva, "Rasskaz F. M.
  Dostoevskogo 'Malen'kii geroi' kak initsiaticheskii tekst,"
  \emph{Dostoevskii i mirovaia kul'tura} 18, no. 2 (2022): 18\textendash 39.}
Within this framework, European landscape painting played a particularly
important role as a way to perceive and represent the holy in nature.
This idea seemed to fascinate Dostoevsky after his release from the Omsk
prison. In addition to his requests for works by the Church Fathers and
his focus on "The Little Hero," Dostoevsky asked for the study
\emph{Psyche: On the Development of the Soul} (1841), a work by Carl
Gustav Carus, in which the Dresden painter argued that the life of the
soul is inextricably bound to both God and nature.

\subsection*{Art and Soul: The Visual Language of the Divine}

\noindent Dostoevsky's deep interest in the visual arts is well documented in his
letters and in his wife's journals that describe their tours through
European galleries and museums. In his work, the word-image reached beyond 
ekphrastic description or\textemdash as it concerned the natural world\textemdash the
pathetic fallacy, revealing his search for ideal visual forms as an
aesthetic and spiritual journey.\footnote{See, for example, Robert Louis
  Jackson, \emph{Dostoevsky's Quest for Form: A Study of His Philosophy
  of Art} (New Haven: Yale University Press, 1966) and the work of
  Tat'iana Kasatkina, especially \emph{Sviashchennoe v povsednevnom:
  Dvusostavnyi obraz v proizvedeniiakh F. M. Dostoevskogo} (Moscow: IMLI
  RAN, 2015).} Dostoevsky was born at an important time in the
development of European painting, when landscape was being reevaluated
as an object with religious and moral significance rather than the mere
backdrop of human experience.\footnote{Timothy Mitchell, "Caspar David
  Friedrich's Der Watzmann: German Romantic Landscape Painting and
  Historical Geology," \emph{The Art Bulletin} 66, no. 3 (Sep., 1984),
  452.} Artistic images of the natural world also offered the
possibility of transcending the limitations of language to express the
ineffable, if not the divine. The rise of science and technology in the
West in the fourteenth and fifteenth centuries, Lynn White argues,
disrupted ways of knowing the world of creation as deeply revelatory of
the divine truth, as described in the works of the Church Fathers and
adhered to more closely but not exclusively by the Orthodox
East.\footnote{Lynn White, "The Historical Roots of the Ecological
Crisis," \textit{Science}, Mar.~10, 1967, 1203\textendash 07.} But in the work of some
Western artists Dostoevsky saw a striving for the "truth of nature"
reflected in the "mystery" of their art that aligned philosophically
with the teaching of the Church Fathers on the perception of nature as a
divine language. Prominent among them was Carl Gustav Carus, whose work
Dostoevsky was thinking about as early as 1853 when he requested that
his brother send him \emph{Psyche} and a German dictionary to translate
it.

At the turn of the nineteenth century in Europe, a number of theoretical
treatises, discourses, and "letters" on what was considered to be the
"new" art of landscape grappled with ideas about the artists' vision
and the representation of the sublime, mystical, and, eventually, the
divine in the beauty of nature as a "poetic"
ideal.\footnote{For an overview of this movement, see Oskar
  Bätschmann's "Introduction" in Carl Gustav Carus, \emph{Nine Letters
  on Landscape Painting, Written in the Years 1815\textendash 1824; with a Letter
  from Goethe by Way of Introduction} (Los Angeles, CA: Getty
  Publications, 2002), 1\textendash 76.} Confronted by the multiplicity of created
forms and phenomena, but unsatisfied with the Enlightenment's explanations
through science and reason, landscape painters like Carl Gustav Carus
and, later, Alexandre Calame (1810–1864)\textemdash who was part of the Dusseldorf
School that inherited and developed many of the ideas of Carus's Dresden
School\textemdash rediscovered the natural world as a great theophany.
The work of both men stood in sharp contrast to the strivings of the
post-Renaissance West to produce "technically perfect, mathematically
accurate, or near photographic images of the world."\footnote{Valentina
  Anker, \emph{Alexandre Calame. Vie et œuvre. Catalogue raisonné de
  l\textquotesingle œuvre peint} (Fribourg: Office du livre, 1987), 310.}
Carus instead imagined how the artist's eye must perceive "the true and
wondrous life of nature" while the artist's hand must be trained to
"do the soul's bidding, quickly, easily, and beautifully."\footnote{Carus,
\emph{Nine Letters}, 30.} Within the context of these
discussions, Carus's \emph{Nine Letters of Landscape} and his paintings
reflected the belief among the German Romanticists that the landscape
artists  must have "sufficient intelligence and sensibility to
feel the spirit and soul of the matter that lies before
him."\footnote{Johann Georg Sulzer, \emph{General Theory of the Fine
  Arts}, quoted in Carus, \emph{Nine Letters}, 12.} For one of the most
well known of the German Romantic painters and Carus's close friend
Caspar David Friedrich, "nature was the book of God and revealed his
active presence through the language of the moon, clouds, trees, and
rocks."\footnote{Mitchell, "German Romantic Painting," 458.}

For Dostoevsky, this ability to see the natural world as the
\emph{logoi}, or the visual language of God, was key to his
understanding of the role of art and artists. Dostoevsky's simultaneous
requests for patristic texts and Carus's \emph{Psyche}\textemdash which
envisions a unity among the human, the natural, and the divine\textemdash
suggests that he sensed a common understanding of the natural world in
both.\footnote{George Gibian, "C.G. Carus' \emph{Psyche} and Dostoevsky," \emph{The American Slavic and East European Review} 14, no. 3 (October 1955): 371\textendash 382.} \emph{Psyche} builds on Carus's
previous work on landscape, positing that the unconscious mind can know
the spiritual principle inherent in natural forms. In the book, Carus
connects the unconscious, visual images, and the sense impressions
created in the mind through the act of looking.\footnote{James Hillman,
"Introductory Note," in \emph{Psyche: On the Development of the
Soul} (Thompson, CT: Spring Publications 2017), 13.} "The subtle,
mysterious nature of this link between the visual, the spirit, and the
emotions," Elisabeth Stopp observes, "was Carus' main and
life-long preoccupation" and he worked on it through his scientific
inquiries and his landscape painting.\footnote{Elisabeth Stopp, "Carl
Gustav Carus' Emblematic Thinking," \emph{Bulletin of the John
Rylands Library} (1989), 22.} As in Dostoevsky, love of the natural
world acts as a powerful and transformative force for Carus. Of Carus,
George Gibian writes, "His praise of love serves as a meeting of the
conscious and unconscious parts of the soul, through which human beings
repudiate their selfishness, overcome their isolation, and are enabled
to return to community with other human beings as well as to arrive at a
supra-rational awareness of the universal and the divine."\footnote{Gibian, 
"C.G. Carus' \emph{Psyche} and Dostoevsky," 373.}

The merging of the conscious and unconscious that informs this awareness
of the divine in nature develops in what Carus called "earth-life
painting" (\emph{Erdleben-Bildkunst}).\footnote{\emph{German Masters of
the Nineteenth Century} (New York: The Metropolitan Museum of Art,
1981), 15.} This theory of landscape painting posits the unity of
science and art so that, as artists visualize the inner workings of
landforms over time and perceive its geological history ("the true and
wondrous life of nature"), they also understand and represent nature as
the "true bodily revelation or\textemdash in human terms\textemdash the language of
God."\footnote{Carus, \emph{Nine Letters} 29, 39.} Any dichotomy
between natural science and landscape painting is resolved for Carus by
the "eternal, supreme, infinite unity" that underlies the goals of
science and art: to recognize nature as the manifestation of the
divine.\footnote{Carus, \emph{Nine Letters}, 38.} For Carus, the sacred
role of art and the artist is to "speak the language of nature," in
other words, to perceive, represent, and convey the mystery of Creation:

\begin{quote}
When the soul is saturated with the inner meaning of all these different
forms; when it has clear intimations of the mysterious, divine life of
nature; when the hand has taught itself to represent securely, and the
eye to see purely and acutely; and when the artist's heart is purely and
entirely a consecrated, joyous vessel in which to receive the light from
above: then there will infallibly be earth-life paintings, of a new and
higher kind, which will uplift the viewer into a higher contemplation of
nature. These works will truly deserve to be named mystic and
orphic.\footnote{Carus, \emph{Nine Letters}, 30\textendash 31.}
\end{quote}

\noindent The striking similarities between Carus's description of his philosophy
of art and, for instance, St.~Basil's appeal to immerse oneself in
nature in order to sense its Creator may explain Dostoevsky's deep
interest in Carus and other landscape painters. Indeed, Carus's
earth-life painting reiterates the importance of \emph{theoria physike}
as a way of perceiving the "mysterious, divine life of nature" while
also emphasizing that, in addition to purity of the heart and eye, the
artist's hand must be able to represent the natural world in ways that
"uplift the viewer" into the same kind of contemplation.

Ultimately, it was not the Dresden painters but the Dusseldorf school
and its vision of the sublime in the grandeur of mostly unpeopled
landscapes that dominated European painting throughout Dostoevsky's
lifetime. In particular, the work of Swiss painter Alexandre Calame, its
leading artist and the intellectual descendent of Carus, played a
prominent role in Dostoevsky's understanding about the visual
representation of the sacred in the natural world. Like Carus, Calame
broke with the conventions of his day, committing what one art historian
calls the "historical transgression" of reaching back before the
Renaissance to the tradition of treating landscape and the natural world
as sacred soil.\footnote{Valentina Anker, \emph{Calame}, 310.} Calame's
work introduced "visual writing" to landscape painting, an approach
that regarded natural spaces as the visible manifestation of the
invisible God.\footnote{A. E. Cherniavskaia and A. V. Martynova, "F. M.
  Dostoevskii i shveitsarskii peizazhist A. Kalam," \emph{Vestnik
  Omskogo universiteta}, No.~4 (2007): 79.} Calame saw nature as
"subjugated to divine laws and to rules of harmony" and a
"composition for which God was ultimately responsible."\footnote{Alberto
  de Andrés, \emph{Alpine Views. Alexandre Calame and the Swiss
  Landscape} (New Haven: Yale University Press, 2006), 11.} His
paintings "explore his belief in a divine presence in the rugged open
spaces of his native country."\footnote{de Andrés, \emph{Alpine Views},
  11.} Calame aspired to find the visual language of the sublime (which
Valentina Anker describes as his "uniquely mimetic iconography" and a
"grammar of a mythical cosmology") and to express in his paintings the
"divine dimension" he perceived in the "immense majesty of
creation," especially in the craggy summits of the Swiss landscape,
which, despite serious health risks, he insisted on painting \emph{en}
\emph{plein air}.\footnote{Anker, \emph{Alexandre Calame}, 310.} For
Calame, the process of painting \emph{en plein air} rather than
producing a studio copy was both an act and profession of his own faith.
Calame was certain, writes Anker, that the contemplation of natural
spectacles, especially after a physically challenging climb into the
mountains, could induce in the soul a sense of supreme harmony with the
divine. It was precisely this kind of privileged moment that Calame saw
reflected in the natural world and what he wanted to capture and convey
in his work.\footnote{Anker, \emph{Calame}, 304\textendash 08.}

The effect of his paintings on viewers was meant to be equally
transfiguring. Throughout the 1860s, art reviewers (including
Dostoevsky) distinguished Calame's work from more "photographic"
renderings of nature, noting how Calame's work could "wake the souls of
the spectators."\footnote{"No," wrote Paris critic Henri Delaborde in
  1865, "Calame does not simply photograph the Alps. He knew, the
  master that he was, how to appropriate the picturesque elements and
  have them serve to produce impressions that can wake the souls of the
  spectators" (quoted in Anker, \emph{Calame}, 224\textendash 26).} For
Dostoevsky, writing in 1861, for example, Calame's \emph{Lake of the
Four Cantons} (1852) bears the mark of the "true artist" (истинный
художник).\footnote{F. M. Dostoevskii. \emph{Polnoe sobranie sochinenii
  v 30-ti tomakh}. T. 19 (Leningrad: Nauka, 1979), 150.} Its depiction
of nature shows neither the photographic mirror that signals for him an
"absence of art" (отсутствие художеств) nor does it conceive of nature
as the "implacable dumb beast" (неумолмого и немого зеря) or senseless
and crushing "machine" (машина) imagined in Ippolit's description of
Rogozhin's painting.\footnote{F. M. Dostoevskii. T. 19 (Leningrad: Nauka, 1979), 149.}
Rather, it achieves the "something" that Dostoevsky describes as
"different, more, wider, deeper" (кое-что другое, больше, шире,
глубже) than mere photographic realism.\footnote{F. M. Dostoevskii. T. 19 (Leningrad:
  Nauka, 1979), 150.} In his rejection of views of nature as
"intelligible apart from grace" (photographic) or "autonomous and
self-contained" (like Rogozhin's painting), Dostoevsky highlights the
iconographical qualities of Calame's work wherein nature becomes the
"medium" through which the "magician of a painter" conveys the
experience of his soul.\footnote{F. M. Dostoevskii. T. 19 (Leningrad: Nauka, 1979), 
150.}

Calame's project constantly searched for a worthy subject to demonstrate
the kind of "perfection" of nature that reveals its Creator\textemdash the
"mighty God"\textemdash in the "magnificent language of the
soul."\footnote{Anker, \emph{Calame}, 40\textendash 43.} For Calame, as with
Dostoevsky, the true "teacher" of this kind of painting is nature
itself.\footnote{Anker, \emph{Calame}, 43.} "Nothing elevates the soul
like the contemplation of the snowy heights and craggy peaks," wrote
Calame in 1851 about painting the Alps. In phrasing that echoes the
effect of the Siberian night sky on Dostoevsky, he continues: "At that
moment, you find yourself immersed in immense solitude, alone before
God. You reflect on the insignificance of man, on his irrationality."
Continuing, the Protestant Calame expresses a deeply iconographical
understanding of creation as he nearly paraphrases the writings of
Church Fathers: "The whole earth glorifies the power and endless
goodness of God, which are evident in every blade of grass, in each
hardly noticeable insect. You feel that you are infinitely elevated by
this unsurmountable power and a cry is torn from your soul, from the
very depths of this experience, which my brush is powerless to
convey."\footnote{Anker, \emph{Calame}, 208.}

Dostoevsky's praise of Calame's 1852 \emph{Lake of the Four Cantons}
marveled at the "secret" by which the "magician painter" conveyed
this revelatory experience to the viewer. In \emph{The Idiot}, a verbal
correlate of such a painting takes shape in Myshkin's description of his
experiences in the Alps that treats the natural setting not only as
iconographical space, but as an icon. Doing so, Dostoevsky inserts a
non-verbal sacred text into this highly visualized work that, as
Tat'iana Kasatkina shows, is thematically structured on the interplay
between the "icon" as window onto the divine and eternal and
"painting" as the mirror-like reflection of material
reality.\footnote{Tat'iana Kasatkina, "Ob odnom svoistve epilogov piati
  velikikh romanov Dostoevskogo," \emph{Dostoevskii v kontse XX veka}
  (1996): 94\textendash 116. On the creation and meaning of the "verbal icon" see
  Valerii Lephakin, "Basic Types of Correlation Between Text and Icon,
  Between Verbal and Visual Icons," \emph{Literature and Theology} 20,
  No.~1 (March 2006): 20\textendash 30.} Kasatkina notes how Prince Myshkin is
always seeking the "soul" of a landscape (or person), as if looking
into "another world" (как в иной мир) that is "closer to an
icon."\footnote{Kasatkina, "Ob odnom," 100.} Recalling Florensky's
belief that "icon painting is the occupation of a person who see the
world as sacred" allows us to see more clearly what Valerii Lepakin
terms the "verbal icon" (словесная икона) that Myshkin creates to
explain how the natural setting in the Swiss mountains made him feel
simultaneously "good," "oppressed," and "uneasy."\footnote{Pavel
  Florensky, \emph{Iconostasis} (Crestwood, N.Y.: St.~Vladimir's
  Seminary, 1996), 78. On the "verbal icon" in Russian literature, see
  Valerii Lepakhin's \emph{Ikona v russkoi khudozhestvennoi literature:
  Ikona i ikonopochitanie, Ikonopis' i ikonopistsy} (Moscow: Otchii Dom,
  2002).} More specifically, the scene suggests the icon of the
Transfiguration of Christ, with Mount Tabor transposed onto the Alps and
Myshkin's experience echoing that of the witnessing apostles as the
natural world becomes a window onto divine eternity.\footnote{Such
  transformative dynamics are explored by Aleksei Lidov, "Hierotopy:
  The Creation of Sacred Spaces as a Form of Creativity and Subject of
  Cultural History," in A.M. Lidov, ed., \emph{Hierotopy: The Creation
  of Sacred Spaces in Byzantium and Medieval Russia} (Moscow, 2006),
  9\textendash 31 and Sergei Avanesov on "cultural-semiotic transfer" in
  "Sakral'naia topika russkogo goroda (2). Sofiiskii sobor: Sintaksis i
  semantika," \emph{PRAXIMA: Problemy visual'noi semiotiki} 9, no. 3
  (2016): 25\textendash 81.}

Like passages in "The Little Hero" and anticipating Alyosha's
stargazing in \emph{The Brothers Karamazov}, Myshkin's descriptions of
his walks in the Alps ("where sky and earth meet") immediately
introduce the idea of the sacred into the natural surroundings. The
first instance, Sarah Young shows, underscores Myshkin's experience of
seeing nature as "a higher reality, a different spatial dimension beyond
the horizon where the true nature of the universe, and man's place in
it, are revealed."\footnote{Sarah J. Young, \emph{Dostoevsky's} The
  Idiot \emph{and the Ethical Foundations of Narrative: Reading, Narrating,
  Scripting} (London: Anthem Press, 2004), 78. See also, Tat'iana
  Kasatkina, "Smert', novaia zemlia i novaia priroda v romane F. M.
  Dostoevskogo 'Idiot,'" \emph{Dostoevskii i mirovaia kul'tura,} no. 3
  (2020), 16\textendash 19.} In the "terrible silence" (тишина страшная) of the place
that evokes the "terrible God" (Бог страшен) of the Psalms, Myshkin
envisions the iconographical in the natural world, engaging in what
Florensky calls "beholding that ascends."\footnote{Pavel Florensky,
  \emph{Iconostasis} (Crestwood, NY: St.~Vladimir's Seminary Press), 72.}
The second description of his solitary walks in the Alps represents more
distinctly Myshkin's entrance into iconographical space, signaled by 
the shift from the first-person narration in the first
instance to third-person, which recalls the lack of coherent speech
Myshkin suffered in Switzerland. But the narrative loss of Myshkin's
voice at this point seems the best response; rational thought and words
are inadequate to comprehend or describe the divine and its presence is
often represented through silence.\footnote{Foltz, \emph{Noetics of
  Nature}, 126.}

As I have discussed elsewhere, Dostoevsky sometimes depends on readers
to mentally complete his verbal icons when he evokes iconographical
principles whose elements may exist outside the narrative
frame.\footnote{See, for instance, Amy Singleton Adams, "Learning to
  Look: The Meaning of Unseen Icons in Dostoevsky's \emph{The Idiot},"
  \emph{IKON. Journal of Iconographic Studies} 9 (2016): 363\textendash 374.} In
this instance, the icon of the Transfiguration of Christ can be inferred
as Myshkin, in his first description of his experience in the Alps,
raises his hands to the bright light all about him on the mountain.
Understanding the correlation with this icon allows us to better
understand the "good" yet "oppressed" effect the natural world has
on Myshkin as \emph{penthos} ("joyful sorrow"), which St.~John
Climacus considered a sign of mourning in a truly contrite heart that at
once perceives its alienation from heaven\textemdash envisioned biblically by
Myshkin as a "great everlasting feast"\textemdash and presages the joy of
baptism into "new life."\footnote{Anthony M. Coniaris,
  \emph{Philokalia. The Bible of Orthodox Spirituality} (Minneapolis,
  MN: Light \& Life Publishing Co., 1998), 173\textendash 175.}

\begin{quote}
Once he went into the mountains on a clear, sunny day, and wandered
about for a long time with a tormenting thought that refused to take
shape (\foreignlanguage{russian}{с одною мучительною,
но никак не воплощавшеюся мыслию}). Before
him was the shining sky, below him the lake, around him the horizon,
bright and infinite, as if it went on forever. For a long time, he
looked and agonized (терзаться). He remembered now how he had stretched
out his arms to that bright, infinite blue and wept. What had tormented
him (Мучило его) was that he was a total stranger to it all. What was
this banquet, what was this great everlasting feast, which has no end
and to which he had long been drawn, always, even since childhood, and
which he could never join?\footnote{F. M. Dostoevskii. \emph{Polnoe
sobranie sochinenii v 30-ti tomakh}, Tom 8 (Leningrad: Nauka, 1973),
390\textendash 391.}
\end{quote}

\noindent At the same time, Myshkin's prayerful suffering and torment in the
natural setting points to the fundamental understanding in Orthodox
theology of the economy of salvation: like the incarnation (воплощение\textemdash 
which echoes Myshkin's inability to give shape to his tormenting
thought) of God's divine world, the natural world will also be
transformed at the Transfiguration of Christ.\footnote{See Pavel
Evdokimov, "Nature," \emph{Scottish Journal of Theology} 18, issue 1
(1965): 1\textendash 22.}

Myshkin's second description of the Alps follows Ippolit's "Necessary
Explanation" and serves as a rebuttal of the latter's sense of being
locked in a material world that denies the truth of immortality and
resurrection, an idea visually represented by Rogozhin's painting of the
dead Christ in the grave. Myshkin even cites Ippolit's image of the
buzzing fly (which reappears in Nastasya Filippova's death scene) to
make the opposite point\textemdash that nature is not only the visible
incarnation of the divine, but offers a glimpse into eternal space and
time.\footnote{Kasatkina, "Smert'," 16.} Once again, the image of the
"blade of grass" that demonstrates an awareness of its own incarnation
(and thus sanctification) becomes for Dostoevsky a shorthand for the
sacred language of the natural world.

\begin{quote}
Every morning the same bright sun rises; every morning there is a
rainbow over the waterfall; every evening the highest snow capped
mountain, there, far away, at the edge of the sky, burns with a crimson
flame; every "little fly that buzzes near him in a hot ray of sunlight
participates in this whole chorus: knows its place, loves it, and is
happy"; every little blade of grass grows and is happy (каждая-то травка
растет и счастлив)!\footnote{F. M. Dostoevskii. \emph{Polnoe sobranie
sochinenii v 30-ti tomakh}, Tom 8, 391.}
\end{quote}

\noindent Ultimately, Myshkin's two-part description of time spent on the mountain
creates a verbal icon that is particularly relevant to Dostoevsky's
eco-aesthetic theology, suggesting, as Calame sensed, the promise of
transfiguration or \emph{theosis} for those with the spiritual (or
aesthetic) vision to behold the natural world as sacred.\footnote{On
  this process in the work of the Church Fathers and in Dostoevsky's
  fiction, see Bruce Foltz, \emph{The Noetics of Nature,} especially
  11\textendash 12, 113\textendash 202.}

Dostoevsky's fascination with both landscape and religious painting
finds expression in his representation of the natural world as a sacred
language or "words" (\emph{logoi}) of God. Although this understanding
of the natural world took shape early in Dostoevsky's life and career,
it can also be viewed as part of broader discussions in the late
nineteenth century about the Forest Question, for example, or Russia's
cultural affinity with Western Europe.\footnote{On the environmental
  concerns in Russia in the late nineteenth century and the aesthetic
  response, see Jane T. Costlow, \emph{Heart-Pine Russia. Walking and
  Writing the Nineteenth-Century Forest} (Ithaca: Cornell University
  Press, 2013).} Dostoevsky was still exploring the "secret" of the
type of art that might "utter the ultimate word of great, general
harmony" to unite the intellectualized wanderers (like Pushkin's
unrooted Aleko and Onegin, whom Dostoevsky described as "no more than a
blade of grass {[}былинка{]}, torn from its stem and carried off by the
wind") with the religious wanderers\textemdash like the hesychastic Makar\textemdash
who are "firmly rooted" in their native Russian soil, have faith in
God, and perceive the beauty of His creation.\footnote{F. M.
  Dostoevskii, \emph{Sobranie sochinenii v 15 tomakh}, Tom 14 (1976),
  439.} As John McGucken points out, for the Church Fathers, the
experience of beauty in the world can be understood as "an epiphany of
the underlying energy of the Logos who had made the world."\footnote{John
  McGucken, "The Beauty of the World and Its Significance in
  St.~Gregory the Theologian," in \emph{Towards an Ecology of
  Transfiguration}, 35.} Perhaps it is the \emph{logoi} of the created
world that becomes "the ultimate word" that Dostoevsky imagined would
come out of Russia to "enfold" its European "brethren" with
"brotherly love." Of course, it is difficult at this historical moment
to imagine Russia as the source of "brotherly love." But important
elements of Dostoevsky's eco-aesthetic theology clearly ring in the
response of the Eastern Christian church to today's climate crisis. His
All Holiness Ecumenical Patriarch Batholomew I continues to urge
citizens of all nations and faiths to share the fundamental Orthodox
belief in the sacredness and beauty of all creation and to regard our relationship with the natural environment as sacramental.\footnote{Faith~and Environment:~An~Inspirational~Perspective"~(2014). \url{https://www.orth-transfiguration.org/wp-content/uploads/2016/05/Lecture_HAH-2014-04-Utrecht-The-Netherlands-April-24-14.pdf} (Accessed 26 May 2024).} It is not, he emphasizes, the scientific
study of the environment that will cause the dramatic shift of mind and
change of heart to transform attitudes and actions in the
world.\footnote{Ibid.} Perhaps, as Dostoevsky noted, the world will be
saved only when its beauty is widely recognized as sacred.

\vspace{2em}
\begin{center}
  \includegraphics[width=0.75cm]{articlend.png}
\end{center}

\biobox{\textbf{Amy Singleton Adams} is a professor of Russian Studies at the College of the Holy Cross in Worcester, Massachusetts. She received her B.A. in Russian Language and Literature at Dartmouth College, and her M.A. and Ph.D. in Slavic Languages and Literatures at the University of Wisconsin-Madison. Her research focuses on the non-ecclesiastical use of icons and creation of iconic or sacred space in Russian literature, art, and society. She co-edited and contributed to the volume \textit{Framing Mary: The Mother of God in Modern, Revolutionary, and Post-Soviet Russia} and recently published an essay on Vladimir Putin’s use of Orthodox iconography as a form of political discourse. The present essay is part of a broader study of sacred landscapes and the divine language of nature in Dostoevsky’s work.}

\label{sec:adams}

\section{Denis - On the Absence of Ethics in Dostoevsky}
\fancypagestyle{chaptertitlepage}{
  \fancyhf{} % Clear all header and footer fields
  \fancyhead[L]{\begin{minipage}[t]{0.7\textwidth}\publisher\end{minipage}}
  \fancyhead[R]{\begin{minipage}[t]{\textwidth}\raggedleft \datefont\fontsize{10}{11}\selectfont Volume 1 (2024): \thepage\textendash\pageref{sec:zhernokleyev}  \\ \doi{10.71521/p0s0-f043} \end{minipage}}
  \renewcommand{\headrulewidth}{0pt} % No header rule on title pages
  \fancyfoot[LE, RO]{\thepage} % Left on even pages, right on odd pages
}
\fancypagestyle{chaptercontentpage}{
  \fancyhf{} % Clear all header and footer fields
\fancyhead[CE]{%
  \fontsize{11}{11}\leftmarkfont%
  \addfontfeature{LetterSpace=10.0}%
  \textit{\MakeUppercase{On The Absence of ethics in dostoevsky}}%
}
  \fancyhead[CO]{\authorheadfont\addfontfeature{LetterSpace=10.0}\fontsize{11}{11}\selectfont\textbf{{\uppercase{Denis Zhernokleyev}}}}
  \renewcommand{\headrulewidth}{0pt} % No header rule on content pages
  \fancyfoot[RE]{\thepage}
  \fancyfoot[LO]{\thepage}
}

\newpage
\abstractbox{On the Absence of Ethics in Dostoevsky}{}{Denis Zhernokleyev}{Dostoevsky is not a novelist in the conventional sense of the word. Unlike Tolstoy or Turgenev, his purpose is not to describe social life in nineteenth-century Russia, with its moral challenges and potential ethical solutions. The characters in his novels are not "real" people, and their experiences never amount to a reliable psychological portrait. When they speak, they do not reveal much about themselves. Their confessional outpourings can rarely be trusted. Instead, Dostoevsky’s characters are what Bakhtin calls "coordinates" of metaphysical realms. Like sorrowful masks in ancient Greek drama, reverberating with profound yet never fully graspable primordial wisdom, these characters resist being placed within a rigorous ethical system. The saintly life of the prostitute Sonya Marmeladov in Crime and Punishment is only the most memorable example. In this resistance to an ethical reading, the Dostoevskian novel does not dismiss the validity of an ethical framework for the moral life, but it profoundly restricts its metaphysical ambition. Ultimately, suffering in the world for Dostoevsky remains infinite and incomprehensible.}{Dostoevsky, Bakhtin, ethics, morality, psychological, theology, Johannine, apophatic, Raskolnikov, Karamazov}

\chaptertitle{On the Absence of Ethics \\ in Dostoevsky}{}{Denis Zhernokleyev}
\addcontentsline{toc}{chapter}{On the Absence of Ethics in Dostoevsky\\\emph{by} Denis Zhernokleyev}
\setcounter{footnote}{0}

\noindent For all the moral intensity of Dostoevsky's world, the ethical value of his novels can be difficult to ascertain. The instances of humility and suffering are often so grotesque that they refuse to serve as paragons of virtue. Sonya Marmeladova's radical kenosis in \emph{Crime and Punishment} is a prime example. The disturbing implication of her self-abnegation is not so much that humiliation is a Christian necessity but rather that this humiliation seems incompatible with even a modest affirmation of individual will.\footnote{Henry M.W. Russell, "Beyond the Will: Humiliation as Christian Necessity in \textit{Crime and Punishment}," in~\emph{Dostoevsky and the Christian Tradition}, ed.~George Pattison and Diane Oenning Thompson (Cambridge: Cambridge University Press, 2001), 226\textendash 36.} This absoluteness of self-renunciation draws Raskolnikov to Sonya, who believes he discovered in her a kindred spirit in nihilism. We know from Dostoevsky's notebooks that his original plan was indeed to depict Sonya as a Russian nihilist\textemdash a representative of the 1860's-70's cultural movement that advocated for a radical form of individualism and rejected traditional values and institutions, including the state, religion, and family.\footnote{For an overview of nihilism in 19\textsuperscript{th} century Russia, see Richard Peace, "Nihilism," in~\emph{A History of Russian Thought}, ed.~William Leatherbarrow and Derek Offord (Cambridge: Cambridge University Press, 2010), 116\textendash 40.} In the final version of the novel, however, Dostoevsky distances Sonya from any association with that political movement. Michael Katz explains this fundamental alteration of the original vision as a part of Dostoevsky's social critique of nihilism wherein Sonya's religiosity serves as a viable ethical alternative.\footnote{Michael Katz, "The Nihilism of Sonia Marmeladova," \emph{Dostoevsky Studies} 1 (1993): 25\textendash 36.} But such a practical reading of Sonya's virtue, in my view, fails to appreciate the extreme nature of her self-sacrifice, her "radical hospitality."\footnote{For the relevance of this theological notion, see Valentina Izmirlieva, "Hosting the Divine Logos: Radical Hospitality and Dostoevsky's \textit{Crime and Punishment}," in \emph{The Routledge Companion to Literature and Religion}, ed.~Mark Knight (London: Routledge, 2016), 277\textendash 88.}

Raskolnikov is not wrong to sense in Sonya a nihilistic tendency. When he unceremoniously suggests to her that it would be "a thousand times better and wiser, to plunge into the water and end it all" (309; VI, 247), he is astounded to see that the thought has been on her mind all along and that only the need to care for her loved ones has prevented her from taking this step.\footnote{For the English translation, see Fyodor Dostoevsky, \emph{Crime and Punishment}, trans. Jessie Coulson (Oxford: Oxford University Press, 2008). For the Russian original throughout this essay, see F. M. Dostoevskii\emph{, Polnoe sobranie sochinenii,} 30 vols., \emph{Khudozhestvennye proizvedeniia,} ed.~F. M. Fridlender et al.~(Leningrad: Nauka, 1972\textendash 90). Henceforth, in all references to Dostoevsky, the first number refers to the English translation of the cited novel, the second to the Russian original.} While there can be no doubt that selling herself to support the family has had a profound effect on Sonya, the precise psychological nature of this effect remains a mystery, for Raskolnikov as well as for the reader (who sees much of the novel through Raskolnikov's aestheticizing gaze). Sonya's guilt is as incomprehensible as it is infinite. When she confesses to Raskolnikov that she is "a great, great sinner," Raskolnikov rushes to supply the ethical framework within which her "great sin" should be understood: "That you are a great sinner is true {[}\ldots{}{]} but your greatest sin is that you have abandoned and destroyed yourself \emph{in vain}" (309; VI, 247; emphasis in original). Although the moral worth of Sonya's sacrifice is not for Raskolnikov to decide, his sense that Sonya's suffering lacks ethical purpose is fundamentally correct. And he seeks to exploit this absence of a clear ethical framework by violently subjecting Sonya's imagination to a narrative, a plot of her life carefully fashioned by him.

Even a seemingly pious scene, where Raskolnikov has Sonya read a fragment from the gospel, might be seen as part of his attack on her devout imagination.\footnote{Eric Naiman, "Gospel Rape," \emph{Dostoevsky Studies} 22 (2018): 11\textendash 40.} Raskolnikov's theodicean stream of images, where both Sonya's own childlike suffering and the innocent suffering of the children she cares for, becomes the ground for a spiteful reproach of the divine, foreshadows the reel of "little pictures" with which Ivan Karamazov attempts to shake the religious resolve of his younger brother Alyosha. Unlike Alyosha, who briefly succumbs to Ivan's despair, Sonya remains remarkably impervious to Raskolnikov's ethical calculations. Her relationship to suffering is mystical, which is to say that she requires her suffering neither to be explained, nor to be explainable. Secular readers tend to reduce Dostoevsky's religious mysticism to the notions of delirium, confusion, or mere puzzlement, thus implying that the absence of a comprehensible ethical framework in his novels is only seeming and will manifest itself as soon as the logical coherence of the narrative is revealed.\footnote{For an example of such a secular reduction of Dostoevsky's mysticism, see Michael Holquist, "Puzzle and Mystery, the Narrative Poles of Knowing: \textit{Crime and Punishment}," in Holquist, \emph{Dostoevsky and the Novel} (Princeton: Princeton University Press, 1977), 75\textendash 101.} It could be argued, however, that Sonya's role is exactly to deny the novel its narrative integrity. Her radical \emph{outsidedness} (to use a Bakhtinian term) places her not only beyond the reach of Raskolnikov's moral imagination but beyond the reach of the descriptive ambition of the novel itself. Hence Sonya's miraculous agency in Raskolnikov's moral transfiguration finds its full disclosure only in the epilogue. Unable to reconcile the mystical denouement of the epilogue with an ethical reading of the main body of the novel, many critics have understandably dismissed the epilogue as aesthetically unpalatable because psychologically unbelievable.\footnote{Ernest J. Simmons, \emph{Dostoevsky: The Making of a Novelist} (Oxford: Oxford University Press, 1940), 165.}

It is my argument in this article that the inaccessibility of Sonya's suffering to ethical analysis illustrates a critically important aspect of Dostoevsky's poetics, not only in \emph{Crime and Punishment} but also in subsequent novels where ethical judgment ultimately exists only as a temptation of the Euclidean mind. My goal is not to advance an amorality thesis in Dostoevsky but to suggest that his world does not know an autonomous ethics, which assumes self-sufficiency in the individual's moral awareness and decision-making.\footnote{On the absence of autonomous ethics in Orthodox theology, which deeply informs Dostoevsky's worldview, see Sergii Bulgakov, "Etika v pravoslavii" in \emph{Pravoslavie: ocherki ucheniia pravoslavnoi tserkvi} (Paris: YMCA Press, 1965), 324\textendash 30. For the English translation, see Sergii Bulgakov, "Orthodox Ethic," in \emph{The Orthodox Church}, revised translation by Lydia Kesich (Crestwood, NY: St.~Vladimir's Seminary Press, 1988 {[}1932{]}), 153\textendash 55.} As Robert Louis Jackson points out, the tragic paradox of Raskolnikov's crime is that it naturally follows from his rage against evil, which, at least in its inception, is fueled by ethical motivations.\footnote{Robert Louis Jackson, "Philosophical Pro and Contra in Part One of \textit{Crime and Punishment}," in~\emph{The Art of Dostoevsky: Deliriums and Nocturnes} (Princeton: Princeton University Press, 1981), 197.} The deeper tragedy of \emph{Crime and Punishment} is the tragedy of individual goodness, which remains vulnerable to the morally distorting isolation of pride and self-will. Raskolnikov's problem is not immorality but alienation from human community, an alienation that begins with the temptation to affirm the autonomy of the self through ethical judgment.\footnote{Randall Havas,~"Raskolnikov Beyond Good and Evil,"~in \emph{Dostoevsky's} Crime and Punishment: \emph{Philosophical Perspectives}, ed.~Robert Guay~(Oxford: Oxford University Press, 2019), 150\textendash 72.} Of course, Dostoevsky's point here is aimed not against ethics, or the law as such\textemdash categories rarely of direct interest to Dostoevsky\textemdash but against the arrogance of that quasi-Kantian self-consciousness that an ethical judgment for Dostoevsky inevitably presupposes. In this respect, self-condemnation can be equally self-indulgent, a point vividly illustrated by the confessional outpourings of the Underground Man. Hence Dostoevsky insists that Raskolnikov's regeneration in the epilogue happen suddenly, without any preceding ethically rigorous discernment. After all, Raskolnikov is a moral corpse, a Lazarus in the grave, and therefore does not possess the ability to \emph{rise} from the dead. He must \emph{be risen}, called forth, out and into the open from the pit of his decomposing interiority.

\subsection*{Inwardness in Dostoevsky}

\noindent To put this emphasis on outsidedness is not to suggest that there is no inwardness in Dostoevsky. On the contrary, Dostoevsky's world is the world of infinite inwardness (\emph{internum aeternum}) and therefore of boundless suffering. In a recent study, Yuri Corrigan has explored suffering in Dostoevsky through the psychologically realistic language of trauma. On this reading, while the "psychic wound" that Dostoevsky inflicts on all his characters cannot be alleviated by the Freudian sort of therapy, it can be addressed through the "inner work" of nurturing the soul's landscape of memory to the point where it becomes a reliable ground for a robust subjectivity.\footnote{Yuri Corrigan, \emph{Dostoevsky and The Riddle of the Self} (Evanston: Northwestern University Press, 2017), 3\textendash 15, 132.} While I agree that suffering in Dostoevsky is all-pervading and inescapable and ultimately constitutes the path to salvation, I am less optimistic about the moral productivity of the inward turn. Dostoevsky's anthropology is fundamentally \emph{ecstatic} in its orientation. Hence Alyosha Karamazov's epiphany takes place not only outside of the room that contains the decomposing body of his elder but more importantly outside of the gloomy depth of his despairing soul.

Retreat into oneself, though ubiquitous in Dostoevsky, does not receive epistemological legitimacy of the Augustinian kind and tends to lead his characters to demonic doubling, not to moral awakening or revelation. The endpoint of Raskolnikov's inwardness, the horizon that lures him forward into the inward depth, is Svidrigailov. Joseph Frank observes that this enigmatic character prominently enters the novel only at the beginning of Part IV, where he appears to be stepping out of Raskolnikov's delirium right when Raskolnikov begins to accept the impossibility of using his egoism in the service of moral ends.\footnote{Joseph Frank, \emph{Dostoevsky: A Writer in His Time} (Princeton: Princeton University Press, 2010), 499.} Indeed, Svidrigailov seems to be an apparition, a successful rendition of Raskolnikov's own disintegrating life. Not only does he escape legal consequences for the murder of his wife, but he experiences no moral qualms about it. The ease with which Svidrigailov indulges in acts of generosity illustrates not his compassion but his utter lack of emotional attachment to the world. To the Svidrigailovian \emph{nihilistic inwardness} of Raskolnikov Dostoevsky offers Sonya's \emph{radical outwardness}. She has given herself to the world completely, to the point of self-destruction, without attempting to process her pain. In her outwardness, Sonya embodies for Dostoevsky the epistemic modality of the icon, which, unlike a Renaissance painting, refuses to entertain the mind of the spectator with naturalistic depictions of reality.\footnote{On the importance of the icon for \emph{Crime and Punishment}, see Tatyana Kasatkina, "The Epilogue of \textit{Crime and Punishment}," in \textit{Fyodor Dostoevsky's} Crime and Punishment: \textit{A Casebook}, ed.~Richard Peace (Oxford University Press, 2005), 171\textendash 87. For a theological discussion of the fundamental opposition between the icon and the Renaissance painting, see Pavel A. Florensky, \emph{Iconostasis}, trans. Donald Sheehan and Olga Andrejev (Crestwood, NY: St.~Vladimir's Seminary Press, 1996).}

In my reading, the ethical value of suffering in Dostoevsky remains ultimately ungraspable, both for the characters and for the reader. In fact, the desire to capture the image of suffering aesthetically in words and imagination, constitutes in Dostoevsky the greatest temptation. We might call it the temptation of natural (secular) salvation, which assumes that suffering can be grasped (remembered/communicated) and therefore can serve as an epistemically reliable ground on which to erect a sentimental metaphysic. When in \emph{The Idiot}, Prince Myshkin kisses a photograph of Nastasya Filippovna, he does not so much venerate her suffering (at this point in the novel he has not yet met Nastasya in person) but his own Romantic hope that suffering can be reliably captured in a trustworthy image and thus be made perpetually available to his insatiable compassion (94; VIII, 68).\footnote{Fyodor Dostoyevsky,~\emph{The Idiot,} trans. by David McDuff (London: Penguin Books, 2004).} The novel will render Myshkin's sentimental hope tragically false. He will fail both to save Nastasya and to comprehend the nature of her suffering. Dostoevsky's polemic is not with compassion as such but only with the absolute status it assumes for the sentimental metaphysic. Compassion tends to trust its own perception entirely and through this epistemic self-reliance, it risks turning subjective human experience into a sole source of truth. Within this natural economy of salvation, justice becomes a prerogative of the compassionate self, who is believed to be capable of adequately assessing the imbalance of suffering in the world and thus of knowing how justice could be restored, or at least improved.

\subsection*{The Illusion of Justice}
{\tolerance=1000 \emergencystretch=5em \hyphenpenalty=10000

It is in the name of justice that Ivan Karamazov subjects Alyosha to an assortment of grotesque images of human suffering in the famous dialogue embedded in \emph{The Brothers Karamazov}. Though some of Ivan's stories have a basis in history, his true reason for employing them has less to do with their historical veracity than with their literary force.\footnote{For a reading that attributes greater weight to the historical and, therefore, ethical aspects of Ivan's argument, see Susan McReynolds, \emph{Redemption and the Merchant God: Dostoevsky's Economy of Salvation and Antisemitism} (Evanston, IL: Northwestern University Press, 2011), 157\textendash 98.} Hence, Ivan refers to these facts and illustrations in the diminutive, as mere "little facts" and "little anecdotes" (239, XIV, 218).\footnote{Fyodor Dostoevsky, \emph{The Brothers Karamazov}, trans. Larissa Volokhonsky and Richard Pevear (New York: Vintage Books, 1991).} The embellished nature of the evidence is not a problem for Ivan because his rhetorical strategy is ultimately emotional, invested in the game of effect. He uses his expertly pre-arranged reel of graphic images as a psychological bludgeon aimed at unsettling Alyosha's faith. The toppling of his brother's faith rather than the tears of children themselves is his final evidence against God's omnipotence. All Ivan needs is for Alyosha to absorb the "alluring (little) pictures" (\emph{kartinki prelestnye}) and, under their influence, decide to take vengeance into his own hands: "\,'Well \ldots{} what to do with him? Shoot him? Shoot him for our moral satisfaction? Speak, Alyoshka!' 'Shoot him!' Alyosha said softly, looking up at his brother with a sort of pale, twisted smile" (220, 221; XIV, 241, 243). Ivan succeeds in provoking Alyosha to a condemnation and call for violent action, because he is canny enough to exploit the weakness of human nature\textemdash its inability to withstand the image of human suffering.}

Despite triumphantly bringing Alyosha to this point, Ivan himself does not believe that taking vengeance into one's own hands can deliver meaningful justice. This much follows from his theodicy, wherein Ivan rhetorically inquires whether divine retribution delivered after the children "have already been tortured" has any equity, any moral weight (245, XIV, 223). If divine vengeance is impotent to restore justice, surely human retribution would accomplish even less. But if Ivan believes that evil is ultimately unanswerable, why does he nudge Alyosha in the direction of violence? The stated reason is to test Alyosha's religious resolve, to take a novice into a true existential wilderness where his commitment to God's goodness and love could be put under proper pressure. The unstated reason is Ivan's desire to share with his brother, and indeed to force his brother to share, the overwhelming despair within which Ivan himself dwells. Ivan's theodicy captures the root of this despair as the impossibility of resolving the tension between the reality of evil on the one hand and the impossibility of justice on the other. For of course, Ivan's aestheticization of evil through literary means cannot hide the fact that innocent suffering is terrifyingly real in the world.

George Steiner identifies this unresolvable moral tension which gives rise to Ivan's despair as the core principle of Dostoevsky's poetics. The aesthetic trial to which Ivan subjects his brother is in fact Dostoevsky's challenge to his reader to resist the urge to resolve the moral ambiguity of reality through an ethical solution. For to appeal to some "unconscious rite of expiation" would inevitably "debase the great terror and compassion" of existence to which Dostoevsky seeks to expose his reader.\footnote{George Steiner, \emph{Dostoevsky or Tolstoy: An Essay in the Old Criticism} (New Haven: Yale University Press, 1996), 204.} To read Dostoevsky, therefore, is to read him with a continuous awareness of the epistemic vulnerability of imagination to images of suffering. As sympathetic as we legitimately feel towards the ethical quandaries of such characters as Ivan Karamazov, Raskolnikov, or Stavrogin, it is our hermeneutical responsibility to refuse the aesthetic manipulation they carry. This demand of Dostoevsky's poetics constitutes to be the greatest challenge for a modern reader, especially in the West, where the primary task of reading has often been defined in thoroughly Kantian terms as an exercise in mimetic illusion undertaken in the service of a social obligation.\footnote{For a critical analysis of this tendency in contemporary Western literary theory and how the Russian novel challenges it, see Chloë Kitzinger, \emph{Mimetic Lives: Tolstoy, Dostoevsky, and Character in the Novel} (Evanston: Northwestern University Press, 2021).} On this view the task of the novel is to model for us the ethical environments we face in real life and thus to prepare us for real moral decisions. Dostoevsky's novel, however, will evade such an ethically oriented approach, inviting the charge that he was on a mission to undermine the novelistic genre as such.\footnote{Peter A. Jensen, "Some Remarks on the Russian 'Anti-novel,'" in \emph{Celebrating Creativity: Essays in Honour of Jostein Børtnes}, ed.~Knut Andreas Grimstad and Ingunn Lunde (Bergen: University of Bergen, 1997), 135\textendash 50.}

\subsection*{St. John \textit{vs.} Rousseau on Nature}

\noindent It bears repeating that the absence of a robust ethical framework in Dostoevsky does not mean amorality. For Dostoevsky, the moral sensibility not only constitutes a fundamental part of human experience but is a manifestation of the divine in humanity. To put it in the language of St.~John, whose theology underwrites Dostoevsky's worldview, \textbf{"}God is love. Whoever lives in~love~lives~in God,~and God~in him" (1 John 4:8).\footnote{For the Johannine context of the Dostoevsky novel, see Nikita Struve, "Dostoïevski et l'Évangile selon Saint Jean," in \emph{Dostoïevski: Cahiers de la Nuit surveillée}, no. 2, textes rassemblés par Jacques Catteau et Jacques Rolland (Paris: Verdier, 1983), 205\textendash 10.} The key point of this Johannine formula is that God is the ultimate source of all love, including love for one's neighbor. Without God (if such a purely natural state could even be imagined in John), human nature remains morally insufficient. This absolute grounding of love in the divine weakens the idea of moral agency to such a profound degree that we could justifiably wonder if the Gospel of John is compatible with ethics.\footnote{Ruben Zimmermann, "Is There Ethics in the Gospel of John?" in \emph{Rethinking the Ethics of John: 'Implicit Ethics' in the Johannine Writings}, ed.~Jan G. van der Watt and Ruben Zimmermann, WUNT 291 (Tübingen: Mohr Siebeck, 2012), 44\textendash 88.} In Dostoevsky the Johannine idea of the divine origin of morality exists in irreconcilable tension with the secular notion of morality as a natural state. The main source of this Romantic doctrine in Dostoevsky is Rousseau and his postulate of "the state of pure nature" (\emph{de la pure nature}). Thus, if for John we become more moral (\emph{holy}) through greater participation in the divine (\emph{theosis}), in the Rousseauan worldview we restore our moral sensibility through reconnecting with our purely natural self.\footnote{For the Russian reception of the patristic notion of \emph{theosis} and its importance for Dostoevsky, see Ruth Coates, "Deification and The Long Nineteenth Century," in \emph{Deification in Russian Religious Thought: Between the Revolutions, 1905\textendash 1917} (Oxford: Oxford University Press, 2019), 55\textendash 81.} Hence the Romantic cult of nature and natural beauty through which we aesthetically attune ourselves to nature and thus induce greater moral sense in ourselves.

The Romantic metaphysic receives its most thorough exploration in \emph{The Idiot.} There its naturalistic ideal is embodied in Prince Myshkin\textemdash who descends from the idyllic Swiss plains to make moral goodness attractive to the fallen world of St.~Petersburg. The city will reject Myshkin but not because it is averse to Romanticism. On the contrary, Dostoevsky's St.~Petersburg is an embodiment of the Romantic metaphysic, which is why, despite all his social uncouthness and ultimate failure, ideologically Myshkin fits right in.\footnote{Donald Fanger, "The Most Fantastic City: Approaches to a Myth," in \emph{Dostoevsky and Romantic Realism: A Study of Dostoevsky in Relation to Balzac, Dickens, and Gogol} (Cambridge, MA: Harvard University Press, 1965), 129\textendash 51.} Myshkin feels most organically at home in Pavlovsk Park\textemdash a beautiful natural space just south of St.~Petersburg where much of the novel unfolds. Developed in the tradition of English gardens, this space, designed as a nature park in the wild, was created to afford weary Petersburgers, especially people of means, an opportunity to restore their moral equilibrium after having it rattled by the inescapable immoralities of city life. In \emph{The Idiot} Pavlovsk Park functions as the Romantic alternative to the biblical garden of Eden, or indeed, the Johannine Garden of resurrection, where Mary Magdalene witnesses the risen Christ's annunciation of the Edenic nature restored (John 20). Reenacting this famous \emph{Noli me tangere} (do not touch me) scene from the Fourth Gospel, Part III of the novel has Myshkin meeting Nastasya on the park road at night. Unlike the Johannine narrative, which radiates with joy, however, the scene in the novel is drenched in endless pain and paralyzing despair: "She sank to her knees before him, right there in that street, like one demented; he stepped back in alarm, but she caught his hand in order to kiss it, and, just as before, in his dream, tears now shone on her long eyelashes" (533; VIII, 381\textendash 382). If Magdalene does not touch Christ's transfiguring paschal body, Nastasya seizes Myshkin's hand and "presses it hard." There are no miracles in Pavlovsk Park\textemdash only compassion, only the helpless, despairing body, pure nature.

Given the metaphysical significance of Pavlovsk Park, it is fitting that the famous Dostoevskian dictum\textemdash "Beauty will save the world"\textemdash is pronounced precisely here. Contrary to popular belief, the phrase does not come as an affirmation but as a question addressed to Myshkin by the terminally ill young man Ippolit: "Is it true, Prince, that you once said the world will be saved by beauty? \ldots{} What sort of beauty will save the world?" (446; VIII, 317).\footnote{On the centrality of beauty in Orthodox theology, see Kallistos Ware, "Beauty Will Save the World," \emph{Sobornost} 29, no. 1 (2007): 7\textendash 20.} The question is rhetorical, a part of Ippolit's comprehensive attack on Myshkin's unstated mission to save the world through goodness. The juvenile spitefulness of Ippolit's remarks should not hide the metaphysical seriousness of his critique. It is not an accident that, at one point in earlier drafts, Dostoevsky considers the possibility of placing Ippolit at the novel's axial core (IX, 277). Ippolit constitutes if not an authorial intervention in the logic of the narrative, then at least an internally precipitated disruption. Robin Feuer Miller observes that Ippolit's main monologue is featured at just that point when the narrator's ability to hold the narrative together begins to dissipate.\footnote{Robin Feuer Miller, \emph{Dostoevsky and The Idiot: Author, Narrator and Reader} (Cambridge, MA: Harvard University Press, 1981), 200\textendash 201.} A similar observation can be made about Dostoevsky's own sympathy towards Myshkin which, though sustained throughout the novel, significantly diminishes immediately after Ippolit's monologue.\footnote{Sidney Monas, "Across the Threshold: The Idiot as A Petersburg Tale," in \emph{New Essays in Dostoyevsky}, ed.~Malcolm V.~Jones\emph{~}and Garth\emph{~}M\emph{.~}Terry (Cambridge: Cambridge University Press, 1983), 67\textendash 93.}

Despite adolescent anxiety and self-inflation, Ippolit's argument is remarkably intelligent and articulates the main antithesis to Myshkin's Romantic metaphysic of compassion in the novel. It is not that Ippolit disagrees with the doctrine of "pure nature" or with the moral value of compassion. He is no stranger to generosity himself. Instead, Ippolit's issue is with the messianic ambition of Myshkin's compassion, with the false promise of its natural theology, which, like the Pavlovsk trees, impotently disguises the full truth about nature's murderous brutality. As the culmination of this theological point, Ippolit offers an extended ekphrasis of Holbein's painting "The Body of the Dead Christ in the Tomb" (1520\textendash 1522), which puts the Romantic doctrine of "pure nature," or as Ippolit calls it, "only nature" (\emph{odna priroda}), in a direct confrontation with Johannine Christology: "Here there is \emph{only nature}, and this is truly what the corpse of a man, whoever he may be, must look like after such torments" (476; VIII, 339; emphasis is mine). Ippolit's analysis of Holbein directly foreshadows the tragic finale of the novel, where Myshkin's mind will be overwhelmed with terror at this confrontation with the novel's ultimate Holbeinian truth. The Johannine idea of nature will remain present in the novel only apophatically\textemdash as an idea whose absence is made painfully resonant through the tragic triumph of Rousseau's "pure nature."

\subsection*{Nastasya in Absentia}

Ippolit will eventually realize that his publicly read confession\textemdash "My Necessary Explanation"\linebreak\textemdash has failed to communicate, both to himself and to the world, the truth of his suffering. The ridicule of the crowd after his botched public suicide will seem a fair finale for his immature performance that night. He will even accept Myshkin's compassion and stay on in Pavlovsk. Although Ippolit's final weakness takes nothing away from his courage and perseverance, it does put into perspective the dramatic rejection of compassion by Nastasya. The radical extreme to which Nastasya takes her rebellion against society is difficult to explain simply as prideful unwillingness to repent her sin. Her flight from Myshkin's compassion towards the tragic finale constitutes the central problem of the novel. While it is tempting to think of Nastasya as a conventional novelistic heroine, whose purpose is to edify the reader with some nuanced psychological insight into the human condition, it is worth resisting this impulse. All psychologizing of her suffering is bound to yield meager results. Hence, in the scholarship of the last two decades, attempts to explain Nastasya psychologically have begun to subside. In their place, there is a growing realization that the heroine is radically \emph{absent} from the novel.\footnote{Sarah J. Young, "The Disappearing Heroine," in \emph{Dostoevsky's} The Idiot \emph{and The Ethical Foundations of Narrative: Reading, Narrating, Scripting} (London: Anthem Press, 2004), 28\textendash 74. In a subsequent reading of the novel, Olga Matich echoes Young. See her "Time and Memory in Dostoevsky's Novels, or Nastasya Filippovna in Absentia," \emph{Slavic and East European Journal}, 60, no. 3 (2016): 397\textendash 421.} Indeed, essentially everything we learn about Nastasya comes to us in the form of unreliable accounts, most of which are outright gossip and slander. Moreover, not only does Nastasya directly encourage the spread of these slanderous rumors, deliberately reinforcing her image as a brazen courtesan; she actively invites her own murder at the hands of Rogozhin. What is the significance of this self-defamation and what is the purpose of her ultimate self-destruction? If we approach Nastasya as a conventional character who must have some psychological motive for her behavior, it will be difficult for us to see her death as more than spiteful vindictiveness, or suicidal derangement.

However, once we abandon portraying Nastasya's suffering psychologically, the religious depth of her drama begins to emerge. Her death may constitute a refusal not so much of Myshkin's kindness but of any Romantic pseudo-salvation through moral acquittal. Ironically, as kind as Myshkin's compassion might be in its generosity, its emphasis on innocence only validates society's judgment-driven ethical metaphysic. Could it be that the meaning of Nastasya's name\textemdash \emph{anastasis (resurrection)}\textemdash is neither random nor merely metaphorical? By committing herself (and with herself, the novel) to a tragic finale, Nastasya leaves only one possible concept of salvation\textemdash Johannine resurrection. She can be saved only in Johannine terms. This Christological significance of Nastasya's death is underscored by its evocation of the Holbein painting, accentuated by her deathbed's location in the deep interior of Rogozhin's house, where the painting is displayed. The messianic mission in the novel is transferred from Myshkin to Nastasya, from the compassionate gaze to the apophatic anticipation of resurrection.

\subsection*{Zosima's Total Guilt}

The resistance of the Dostoevskian notion of guilt to ethical appropriation finds its ultimate expression in \emph{The Brothers Karamazov} in Elder Zosima's famous expression: "Each of us is guilty before everyone, for everyone and everything" (164; XIV, 149). The echoes of this formula reverberate throughout the novel, emerging as one of its binding ideas. The most memorable instance appears in Alyosha's compilation of Zosima's biography, where the origin of the saying is attributed to Zosima's brother Markel, a seventeen-year-old young man who died of tuberculosis when Zosima was nine. The affinity between Markel and Ippolit from \emph{The Idiot}, who are both dying from the same illness and at the same age, is noteworthy. Like Ippolit, Markel is a rebellious spirit, who in his mutiny is willing to go as far as scorning the church. Shortly before his death, however, Markel undergoes a profound transformation, partakes of the sacraments and, during the Easter season, experiences a mysterious spiritual transfiguration. It is during this time that he begins to teach his family that "each of us is guilty in everything before everyone, and I most of all" (289; XIV, 262). This teaching prompts an ethical objection from Markel's mother: "How can it be that you are the most guilty before everyone? There are murderers and robbers, and how have you managed to sin so that you should accuse yourself most of all?" (289; XIV, 262). Markel does not answer his mother's question directly, instead repeating his wisdom in increasingly endearing tones as if trying to enchant her: "Dear mother, heart of my heart {[}\ldots{}{]} my joyful one, you must know that verily each of us is guilty before everyone, for everyone and everything. I do not know how to explain it to you, but I feel it so strongly that it pains me" (289; XIV, 262).

In its most elaborated form, the teaching that "everyone is guilty for everyone and everything" appears in the chapter "From Talks and Homilies of the Elder Zosima," also compiled by Alyosha. Here Zosima teaches the brethren that they should "not be afraid of men's sin" but rather "love man also in his sin" (318; XIV, 289). A monk must remember especially not to become the judge of anyone: "For there can be no judge of a criminal on earth until the judge knows that he, too, is a criminal, exactly the same way as the one who stands before him, and that he is perhaps most guilty of all for the crime of the one standing before him" (320\textendash 321; XIV, 291). While admitting that his teaching on the interconnectedness of the world through guilt borders on the absurd, Zosima defends it with what is arguably the most rational explanation of its truth in the novel: "For if I myself were righteous, perhaps there would be no criminal standing before me now" (321; XIV, 291).

Despite this bit of rationalization coming to us through Alyosha's rendition of Zosima's teaching, the wisdom remains fundamentally ambiguous. Rowan Williams has suggested that perhaps the word "guilty" is not "the most helpful of translations" of the Russian original (\emph{vinovat}), recommending the word "responsibility" as a better alternative.\footnote{Rowan Williams,~\emph{Dostoevsky: Language, Faith and Fiction}~(Waco, TX: Baylor University Press, 2008), 168.} Such a recommendation accords with the decision of many Dostoevsky translators, including the influential Constance Garnett, who likewise prefer "responsibility" to "guilt." But \emph{guilt} remains a more accurate rendition of the Russian word. While the idea of responsibility does preserve a degree of uncertainty, and covers a broad range of meanings, it risks introducing into the novel a more rigorously ethical attitude than the original permits. Caryl Emerson defends the use of "guilty" over "responsible" precisely for the level of ambiguity it preserves, because, she writes, "true relation and causality\textemdash dotted lines we draw between bounded events that permit us to pass critical judgment\textemdash are so completely hidden from human perception that my action cannot be clarified (or pardoned) by its immediate context. I must answer for acts of omission as well as commission. Just as the absolute worth of my own actions is unknowable, so will I never know the full context of the other's act."\footnote{Caryl Emerson, "Zosima's 'Mysterious Visitor': Again Bakhtin on Dostoevsky, and Dostoevsky on Heaven and Hell," in~\emph{A New Word on} The Brothers Karamazov, ed.~Robert Louis Jackson (Evanston: Northwestern University Press, 2004), 157.} This explanation elegantly captures the illusiveness of moral responsibility in the novel. The central moral question\textemdash Who killed the Karamazov father?\textemdash does not receive a definitive answer, at least for those who read the novel as more than a detective story.

\subsection*{Bakhtin on the Suprajuridical Crime}

For Mikhail Bakhtin, the question "Who has killed?" (\emph{Kto ubil?}) cannot be answered because it constitutes the novel's "carnival mystery" (\emph{karnavalnaya taina}) (5:42\textendash 43).\footnote{Mikhail M. Bakhtin, \emph{Sobranie sochinenii v semi tomakh}, ed.~Sergei G. Bocharov et al. (Moscow: Russkie slovari, 1996).} Inside the carnival world moral responsibility is fundamentally displaced. Here the murderer is released, while the innocent atones for the crime. The Christological rendering of Dmitrii Karamazov, who is falsely accused of murdering his father, is entirely consistent with this carnival reading of the novel.\footnote{Carol A. Flath, "The Passion of Dmitrii Karamazov," \emph{Slavic Review} 58, no. 3 (1999): 584\textendash 99.} Least of all is Bakhtin interested in discussing the moral responsibility of Smerdyakov, the actual executioner of the old Karamazov. A child of the coition between a sinner and a holy fool, he is not a moral coordinate but an emanation of moral ambivalence, the putrid smell (the meaning of his name) secreting from the carnivalesque body of the saint. The greatest moral weight in the novel falls on Ivan, but not because he is uniquely responsible for the murder of his father. Rather, he appears morally answerable because he is the one in the novel carrying the burden of humanity's "deep conscience" (\emph{glubokaya sovest’}, 6:284).

For Bakhtin, patricide in Dostoevsky, as in Shakespeare, is the altogether "suprajuridical (\emph{nadyuridicheskoe}) crime of any self-asserting life" (527; 5:85).\footnote{For the English translation of this passage, see Sergeiy Sandler, "Bakhtin on Shakespeare: Excerpt from 'Additions and Changes to Rabelais,'\," \emph{PMLA: Publications of the Modern Language Association of America} 129, no. 3 (2014): 522\textendash 37.} While all individual life shares in this tragedy, within Dostoevsky's dialogic novel, it is Ivan's mission to resist the (Kantian) temptation of allowing universal guilt to resolve itself in the ethical self-condemnation "I have killed" (\emph{ya ubil}). However, the pull towards an ethical judgment over oneself, because of its false promise of autonomy, is so powerful that it is impossible for an individual life to resist it. Salvation from demonic self-condemnation cannot come to Ivan from within his inwardness, even though his inner voice continues to remind him of the dialogic nature of all truth and, therefore, of the illusion of individual responsibility. For salvific grace to become effective it must come to him from the outside, through the voice of the other that lovingly "intersects" Ivan's inner dialogue and thence cancels the isolating demonic power of self-condemnation. Thus the loving word of Alyosha to Ivan: "It was \emph{not you} who killed father {[}\ldots{}{]} You've accused yourself and confessed to yourself that you and you alone are the murderer. But it was not you who killed him, you are mistaken, the murderer was not you, do you hear, it was not you! God has sent me to tell you that" (601\textendash 602; XV, 40).

Does not this authoritatively delivered word of Alyosha contradict Zosima's teaching that "everyone is guilty for everyone and everything"? It does not if we remember not to confuse the universal guilt that Zosima has in mind with self-reliant ethical responsibility. Ivan is not individually responsible for the crime precisely because he is totally guilty. And this truth is not negotiable, hence the authoritativeness of Alyosha's word to Ivan. Dialogue in Bakhtin is often misunderstood as simply an everyday life conversation between two individual consciousnesses respecting each other's subjective autonomy. The authority with which Alyosha enters Ivan's inner dialogue rules out such a naturalistic understanding of the dialogic mode in Bakhtin. Alyosha does not ask Ivan to consider modifying his ethical deliberations, he commands him to stop them altogether. The loving tone of Alyosha does not minimize the forcefulness of his gesture. Just like Markel's enchanting words to his mother, Alyosha is not asking here but is instead actively bringing Ivan out of his subjectivity and into an ontology of radical openness.

\subsection*{Conclusion}

Dostoevsky is not a novelist in the conventional sense of the word. Unlike Tolstoy or Turgenev, his purpose is not to describe social life in nineteenth century Russia, with its moral challenges and potential ethical solutions. His characters are not "real" people, and their experiences never amount to a reliable psychological portrait. When they speak, they do not tell us much about themselves. Their confessional outpourings can rarely be trusted. Dostoevsky's characters are what Bakhtin calls "coordinates" of metaphysical realms (5:99). Like sorrowful masks in ancient Greek drama, reverberating with profound yet never fully graspable primordial wisdom, these characters resist being placed within a rigorous ethical system. Their moral example, even when as radiantly beautiful as the life of Sonya, cannot be followed. In this resistance of an ethical reading, the Dostoevskian novel does not dismiss the validity of an ethical framework for the moral life, but it profoundly restricts its metaphysical ambition. In the ultimate sense, suffering in the world for Dostoevsky remains infinite and incomprehensible.

\vspace{2em}
\begin{center}
  \includegraphics[width=0.75cm]{articlend.png}
\end{center}

\biobox{\textbf{Denis Zhernokleyev} is a senior lecturer at Vanderbilt University and a research scholar at the Northwestern University Research Initiative in Russian Philosophy, Literature, and Religious Thought. He holds a doctorate from Princeton University and a master’s degree from Yale Divinity School. His research interests encompass Russian literature, with a particular focus on Dostoevsky, Tolstoy, and Chekhov, as well as the literary philosophy of Mikhail Bakhtin and Russian religious philosophy. He is currently authoring a book, provisionally titled \emph{Dostoevsky’s Apophatic Novel}, which explores the poetic function of Dostoevsky’s apocalypticism.}

\label{sec:zhernokleyev}

\section{Winsky-Ferapont}

\abstractbox{The Ferapont Paradox}{Orthodox Hesychastic Practice and the Poetic Structure of \\ \emph{The Brothers Karamazov}}{Peter Gregory Winsky}{This article examines how Orthodox Hesychasm shapes the poetics and narrative structure of Dostoevsky's \emph{The Brothers Karamazov}, particularly in light of the early critical debate between Leont'ev and Rozanov regarding Dostoevsky's fidelity to institutional Orthodoxy. It argues that an antinomial approach to Orthodoxy, represented by the paradox of rigid dogmatism and piety in Fr.~Ferapont and his engagement with the self-practice of spiritual ascent, provides deeper insight into Dostoevsky's worldview and the structure of the novel. Central to this analysis is Dostoevsky's vision of Higher Realism, which is conveyed through both the positive and heroic portrayals of Orthodox ideals and the satirical critique of its misapplications through fanaticism. By situating the novel within this dual framework, the study highlights the pivotal role of Orthodox principles in shaping its thematic and aesthetic complexity.}{Dostoevsky, orthodox hesychasm, minimal religion, Konstantin Leont'ev, Vasily Rozanov, apophaticism}

\fancypagestyle{chaptercontentpage}{
  \fancyhf{} % Clear all header and footer fields
\fancyhead[CE]{%
  \fontsize{11}{11}\leftmarkfont%
  \addfontfeature{LetterSpace=10.0}%
  \textit{\MakeUppercase{\leftmark}}%
}
  \fancyhead[CO]{\authorheadfont\addfontfeature{LetterSpace=10.0}\fontsize{11}{11}\selectfont\textbf{{\uppercase{Peter Gregory Winsky}}}}
  \renewcommand{\headrulewidth}{0pt} % No header rule on content pages
  \fancyfoot[RE]{\thepage}
  \fancyfoot[LO]{\thepage}
}

\fancypagestyle{chaptertitlepage}{
  \fancyhf{} % Clear all header and footer fields
  \fancyhead[L]{\begin{minipage}[t]{0.7\textwidth}\publisher\end{minipage}}
  \fancyhead[R]{\begin{minipage}[t]{\textwidth}\raggedleft \datefont\fontsize{10}{11}\selectfont Volume 1 (2024): \thepage\textendash\pageref{sec:winsky}  \\ \doi{10.71521/9w86-tt26} \end{minipage}}
  \renewcommand{\headrulewidth}{0pt} % No header rule on title pages
  \fancyfoot[LE, RO]{\thepage} % Left on even pages, right on odd pages
}
\chaptertitle{The Ferapont Paradox}{Orthodox Hesychastic Practice and the Poetic Structure of \emph{The Brothers Karamazov}}{Peter Gregory Winsky}
\addcontentsline{toc}{chapter}{The Ferapont Paradox:\\Orthodox Hesychastic Practice and the Poetic Structure of \textit{The Brothers Karamazov}\\\textit{by} Peter Gregory Winsky}
\setcounter{footnote}{0}
\subsection*{Introduction}

The question of the orthodoxy of Dostoevsky's Orthodoxy in his post-Siberian texts, and particularly in \emph{The Brothers Karamazov}, continues to perplex scholars. Despite his constant defense of his faith, Dostoevsky's works still draw accusations that his depiction of the life of the Church is too moral, lacks representations of liturgical life, and generally adheres too closely to secular trends rather than the reality of late nineteenth century Russian Orthodoxy. Is it possible, however, that Dostoevsky faithfully represented Orthodoxy in his work without recounting the liturgical and monastic traditions? Despite overwhelming engagement with secular and non-institutional religious trends and ideologies, Dostoevsky's texts are not only faithful to, but dependent upon Orthodoxy as the foundations of the author's realism "in a higher sense."\footnote{F. M. Dostoevskii, \emph{Polnoe sobranie sochinenii, 30 vols.,} vol. 26, \emph{Khudozhestvennye proizvedeniia}, ed. F. M. Fridlender et al. (Leningrad: Nauka, 1972\textendash 90), 65. Henceforth citations for volumes of these texts will be written as Dostoevsky, \emph{P.S.S.} (volume number).} To demonstrate the relationship between Dostoevsky's worldview and his poetics, I will trace the foundations of Orthodox monastic traditions as coded into the gloomy, rancorous, and fanatical Father Ferapont. Despite his aversion to eldership, Ferapont illuminates Dostoevsky's commitment to institutional Orthodox practice, albeit subtly. Dostoevsky creates what I call the 'Ferapont Paradox': a simultaneous lionization and denunciation of institutional Orthodox theology and practice which antinomically points towards Dostoevsky's epistemological, ontological, anthropological, and poetic philosophies. I utilize the correspondence between Konstantin Leont'ev and Vasily Rozanov regarding Dostoevsky's Orthodoxy, which exposes the roots of the synthesis between the fear of God and the active, Incarnational love preached by Zosima, in order to shed light on the crucial role the spiritual practice of hesychasm plays in Dostoevsky's work. Reading Ferapont through the lens of Leont'ev and Rozanov's discourse grants a nuanced understanding of the harsh yet necessary institutional elements of Orthodoxy as presented by Ferapont and allows for a deeper appreciation of the originality and complexity of Dostoevsky's poetics.

There are abundant charges in critical and scholarly engagements with Dostoevsky's work, and particularly in discourse surrounding \emph{The} \emph{Brothers Karamazov}, that Dostoevsky is separated from or that he even explicitly rejects the traditions and practices of what Malcolm Jones terms "institutional Orthodoxy."\footnote{Major critical works which have discussed a the question of institutionalized religion in Dostoevsky include Malcolm Jones, \emph{Dostoevsky and the Dynamics of Religious Experience} (London: Anthem Press, 2005);  {Sven Linner,} \emph{Starets Zosima in} The Brothers Karamazov\emph{: A Study of the Mimesis of Virtue} (Stockholm: Almqvist \& Wiksell International, 1975);  {Gary Saul Morson, "The God of Onions: \emph{The Brothers Karamazov} and the Mythic Prosaic,"} in \emph{A New Word on} The Brothers Karamazov, ed. Robert Louis Jackson (Evanston, IL: Northwestern University Press, 2004);  {Diane Oenning Thompson, The Brothers Karamazov \emph{and the Poetics of Memory} (Cambridge, Cambridge University press: 1991); Mark G. Pomar, "Alesha Karamazov's Epiphany,"} \emph{Slavic and East European Journal} 27, no.1 (1983); {Valentina Vetlovskaya, "Alyosha and the Hagiographic Hero,"} trans. Nancy Pollak and Susanne~Fusso, in~\emph{Dostoevsky: New Perspectives}, ed. Robert Louis Jackson (New York: Prentice Hall, 1984); George Pattison and Diane Oenning Thompson, eds., \emph{Dostoevsky and the Christian Tradition} (Cambridge: Cambridge University Press, 2001). Of particular interest is the recent study of the Orthodox roots of Zosima in the context of the Rozanov/Leont'ev correspondence in Alexander A. Medvedev, "The Elder Zosima as a Renovation of Orthodox Tradition (K.N. Leontiev and V.V. Rozanov's Polemic about the Novel by F.M. Dostoevsky 'The Brothers Karamazov')," \emph{Journal of Siberian Federal University. Humanities \& Social Sciences} 3 (2018).} Jones writes

\begin{quote}
Dostoevsky consciously and repeatedly chose to reveal his glimpses of salvation by means of a narrative structure that might have almost been designed {[}\ldots{}{]} to destabilize and to subvert them. {[}\ldots{}{]} He could have restored the semblance of control by choosing omniscient narrators. {[}\ldots{}{]} Or he could have given his 'saintly' characters an aura of imperturbability and spiritual peace that protected them from worldly shocks and human cynicism.\footnote{Jones, x\textendash xi.}
\end{quote}

\noindent Jones implies that a lack of concrete literary or liturgical inter-/sub-textual linkages with an institutional Orthodox position, or granting any control over the chaos of the novel to this institution, obligates a reading of "minimal religion" in \emph{The Brothers Karamazov}\textemdash essentially implying that only minimal religion shapes its poetic structure.\footnote{For more on "minimal religion" see Mikhail Epstein, "Minimal Religion," in \emph{Russian Postmodernism: New Perspectives on Post-Soviet Culture}, ed. Mikhail Epstein, Alexander Genis and Slobodanka Vladiv-Glover (New York: Berghann Books, 1999).} This claim presupposes that minimal religion is truer to Realism because it prioritizes individual experience over dogma and tradition in its "direct impact of lived, personal experience."\footnote{Jones, xi.} However, just as institutional contexts do not negate personal elements of a religious experience, Dostoevsky\textquotesingle s verisimilar depiction of the world does not mean that he rejects or subverts dogmas and traditions.

\subsection*{The Leont'ev/Rozanov Analysis}

The minimal religion Jones sees in Dostoevsky is precisely the "personal Christianity," of which Leont'ev accused Dostoevsky in his article "On Universal Love" and, more pointedly, in his correspondence with Rozanov. In these texts Leont'ev addresses the transgression of prioritizing "harmony," over the "fear of God" (\emph{strakh Bozhii}), which is the source of "\emph{religious} wisdom" according to the teachings of the Orthodox monastic spiritual practice.\footnote{V. V. Rozanov,~\emph{Sobranie~sochinenii v 30 tomakh}, ed. A. N. Nikolyukin (Moscow: Respublika, 1995), vol. 3: 334. All translations are mine. Certain words have also been presented in the original Russian when appropriate for context. All emphases are in the original texts except where noted.} By stressing fear over love, Leont'ev posits a rejection of the reality of lived experience and the mystical spirit of the Church, and claims that a lack of fear leads to "\emph{transcendent} (not earthly, beyond the grave) \emph{egoism}."\footnote{Ibid., 334.}

For Leont'ev, who was living out the end of his life at Optina Pustyn, institutional Orthodoxy and its mystical elements were non-existent in the works of Dostoevsky. Rather, he claimed Dostoevsky's Christianity was driven by a devotion to secular, particularly German idealist and socialist, conceptions of morality and love. Leont'ev's perspective was that this trend was widespread in Russian literature of the late nineteenth century and was especially problematic in Dostoevsky's work. His view was couched in the context of institutional Orthodoxy, and, in this sense, he lends support to Jones' proposition that Dostoevsky functions outside the boundaries of the institutional church.

Drawing on his experiences and monastic studies on Athos and at Optina, Leont'ev claims that Dostoevsky's presentation of Christianity was completely incompatible with dogmatic Orthodoxy. Leont'ev's praise of a priest who rejected Zosima's call to universal love, referring to him as a "\emph{survivor} of Dostoevsky," highlights the extreme danger that Dostoevsky's Orthodoxy represented, in his perspective. In the same letter to Rozanov, he writes that:

\begin{quote}
In Optina \emph{The Brothers Karamazov} is not recognized as a truly Orthodox text, and the elder Zosima does not resemble Father Ambrose in either his teaching or his character. Dostoevsky described only his appearance, but he made {[}Zosima{]} speak in a way that was completely opposed to how Ambrose did, not at all in the style in which he expressed himself. For Fr. Ambrose \emph{church mysticism} was strictly before all else and only then did he apply morality. For Zosima (through whom Fyodor Mikhailovich himself speaks)\textemdash morality {[}was primary{]}, love, love, etc., and then mysticism, but only very weakly.\footnote{Ibid., 337.}
\end{quote}

\noindent In Leont'ev's fanatical devotion to the orthodoxy of Orthodoxy there was no room for a living religion outside the walls of the monastery. Salvation could only be found within its ramparts, living the rigorous, dogmatically strict lifestyle of spiritual warfare. After all, this monastic mode was his personal life. As Sergei Bulgakov relates, Leont'ev's fierce defense of Orthodoxy, and particularly the \emph{most} \emph{stringent} elements of its practice on Athos and at Optina, led him to a sort of Nietzschean decadence:

\begin{quote}
In Leont'ev's religiosity, two features should be particularly emphasized: its depth and seriousness, along with its particular coloring. \ldots{} Orthodoxy became for him a personal \emph{podvig} {[}\emph{spiritual struggle}{]}, of heavy chains imposed on an impulsive mind and passionate will, as sackcloth for pacifying lust. And the more painfully these chains cut, the more exhausting the \emph{podvig} was, the more significant and authentic his religious comprehension became. Therefore, his observations of Athos and Optina are so truthful and wish. \ldots{} However, it is only possible to use Leont'ev's writings as a source for the comprehension of Orthodoxy in the context of his own personal coloring. While blaming his contemporaries, Dostoevsky, Vl. Soloviev, and in part Tolstoy, as preachers of "rosy" Christianity due to the fact that they "keep silent about one thing, ignore the other, and completely reject the third," Leont'ev himself was undoubtedly guilty of the same. Christianity is many-sided and diverse, and within certain limits it gives scope to personal nuances, even presupposes them. Dangerous deviations appear only when {[}the personal nuances{]} are identified with the supra-personal essence of Christianity, and Leont'ev, intolerant, exclusive, fanatical, undoubtedly sinned in this.\footnote{S. N. Bulgakov, "Pobeditel'\textemdash Pobezhdennyi," in \emph{Konstantin Leont'ev:~Pro et Contra}. \emph{Antologiia}, book 1 (St. Petersburg: Izdatel'stvo Russkogo Khristianskogo Gumanitarnogo Instituta, 1995), 385\textendash 6.}
\end{quote}

\noindent Leont'ev's "deviation"\textemdash assigning the most severe elements of Orthodox monasticism to all Orthodox practice\textemdash mirrors Dostoevsky's Fr. Ferapont. This Pharisaic element, or \emph{prelest'} ("spiritual delusion"), entails taking a personal \emph{podvig} assigned to one person by an elder and applying it to all Christianity.\footnote{For more on the importance of \emph{prelest'} as "spiritual delusion" in Dostoevsky's works see Peter Gregory Winsky, "Dostoevsky through the Lens of Orthodox Personalism: Synergetic Anthropology and Relational Ontology as Poetic Foundations of Higher Realism"~(PhD diss., University of California Los Angeles, 2021), 183\textendash 6, Proquest ID:~Winsky\_ucla\_0031D\_19734.}

This personal \emph{podvig} leads Leont'ev to claim that outside rigorous spiritual practice, there is no real Orthodoxy, and therefore it does not exist in \emph{The Brothers Karamazov}. This fanatical rigorism parallels the scholarly conclusion that institutional Orthodoxy is not present in the novel. Strangely, in both contexts, there seems to be tacit agreement that because one strain of institutional Orthodoxy is not present, or if it is vilified, it negates institutional Orthodoxy's presence in the novel. Furthermore, scholars tend to agree with Leont'ev, at least with the claim that the practices of Zosima are not Orthodox. Jones again provides the best example of the trend:

\begin{quote}
We have already referred to {[}Alyosha's vision of Cana{]} more than once, but it is worth quoting it again in the full light of Epstein's model {[}of Minimal Religion{]} and we shall note, as we have before in a different context, that all but token references to Orthodox ritual and traditional Christian doctrine has been suppressed; perhaps it would be more appropriate to say that, though channeled through it, Alyosha's religious experience has been liberated from Orthodox ritual and traditional Christian doctrine, at a moment when it might be expected to overwhelm him, at a moment when a performance of that ritual and a proclamation of that Gospel coincides with a profoundly emotional religious experience that seems to cry out for such an interpretation.\footnote{Jones, 133.}
\end{quote}

\noindent The claim that Zosima, Alyosha, and Ferapont are "liberated from Orthodox ritual and traditional Christian doctrine" is tenuous at best, especially considering that their entire formation and existence is shaped by and predicated upon their belief in that very system and practice, which becomes evident through a more rigorous analysis of the interplay between liturgical/ dogmatic elements of institutional Christianity and the necessary personal quality they contain.

\subsection*{The Split between Institutional and Personal Christianity}

Turning to the monastic anthropocosmos of the novel\textemdash its spiritual and symbolic center in the monastery\textemdash one quickly finds both institutional and minimal religion/personal Christianity. The most obvious source for these categories is Zosima. He lives within the monastery, practices the monastic mode of spiritual growth toward theosis and partakes of the sacraments. And although Dostoevsky makes clear in his notebooks and letters that Zosima was not meant to be a mimetic depiction of any living monastic, he is both within and a product of institutional Orthodoxy despite also existing outside its stringent legalistic strains.

On the one hand, Dostoevsky wants to depict a positive image of institutional Orthodoxy. He wrote to Mikhail Katkov about the foundations for the character of the retired Bishop Tikhon in \emph{Demons} who would then metamorphose into Zosima:

\begin{quote}
There will be bright faces {[}as well as gloomy ones in the novel{]}. Generally, I'm afraid that I don't have the strength to do much. But for the first time I want to touch on one category of people who haven't been touched on in literature. I take Tikhon of Zadonsk as the ideal of this type of person. \ldots{} I will compare him to the hero of the novel. But I'm very afraid; I have never attempted it, but I know something about this world.\footnote{Dostoevsky, \emph{P.S.S.} XXIX (1), 142.}
\end{quote}

\noindent When Dostoevsky mentions the "one category of people who haven't been depicted in literature," he is referring specifically to the type of monastic in the world that Tikhon of Zadonsk represents\textemdash a true monastic. On the other hand, Dostoevsky wants to represent someone that was not the type of stringent priest he knew and despised in Siberia, but rather a depiction of the institution and liturgical reality personally and in the world. The first to notice this desire to break free of the shackles of rigorous, fanatical monasticism and institutionalism was Rozanov, who, in his response to Leont'ev, expressed the truth of Dostoevsky's vision of Tikhon in both \emph{Demons} and \emph{Brothers Karamazov}:

\begin{quote}
There appeared, for example, a type of monk\textemdash the rector of the {[}monastic{]} institut\-ion\textemdash who simply is not aware of his own personal life, personal interest; who lives among his disciples precisely like a father among children. If this type did not correspond to the type of Russian monasticism of the eighteenth and nineteenth centuries (Leont'ev's words), then maybe, and even probably, it corresponded to the type of monasticism of the fourth to ninth centuries. That's what Leontiev did not take into account, the "new key" of the harp that consisted in the tone of meekness, replacing the tone of indignation, contempt, and mockery.\footnote{V. V. Rozanov, "Russkaia tserkov'," in \emph{Pravoslavie Pro et Contra} (St. Petersburg: Izdatel'stvo Russkogo Khristianskogo Gumanitarnogo Instituta, 1906), 337.}
\end{quote}

\noindent Rozanov's defense of Dostoevsky's monastic and institutional Orthodoxy allows for an apophatic analysis of Orthodoxy in the texts.

An apophatic analysis, in this context, is meant not in the sense that the liturgical and monastic elements are beyond positive knowledge or definition, but rather in the sense utilized by S.L. Frank to discuss the nature of reality: "The true import of {[}Pseudo-Dionysius'{]} 'mystical theology' was not mere negation of earthly conceptions as applied to God, but a certain unity or combination of affirmation and negation, transcending the habitual logical forms of thought."\footnote{S.L. Frank, \emph{Reality and Man: An Essay in the Metaphysics of Human Nature}, trans. Natalie Duddington (New York: Taplinger Publishing Company, 1966), 40.} In Dostoevsky's Orthodox characters, the lived, experiential elements of the liturgical and monastic become syncretic with the institutional and dogmatic in a way that transcends the boundaries of simple correlative modes of analysis. Therefore, when reading the positive figures of Christianity in Dostoevsky's novels, as well as the fanatics like Shatov or Ferapont, we must look for that which is not \emph{explicitly} or \emph{positively} expressed. These elements are often glossed over or ignored by scholars because the living liturgical and monastic elements do not satisfactorily fulfill the expectations of readers who are either outside of or too rigorously legalistic about the dogmatic and ecclesiastic structures because they are an antinomic synthesis of both. Scholars, in this sense, often look for Leont'ev's vision of Orthodoxy, and when they do not find it, claim there is no explicit Orthodoxy.

Through a synthetic, antinomic, and apophatic analysis, we find what is "missing" from \emph{The} \emph{Brothers Karamazov}: the explicit liturgical context of Orthodoxy and the personal spirit of hesychastic monasticism. To refute Jones' proposition that the text only contains personal experience and nothing institutional, Bulgakov's claim in \emph{Unfading Light} is instructive: "every authentic, living religion knows its own objectivity and rests on it\textemdash it deals with Divinity, utters YOU ARE to it. \ldots{} Therefore, there is absolutely no need for everyone in their personal experience of faith to have \emph{the whole} content of a given religious doctrine."\footnote{Sergii Bulgakov, \emph{Unfading Light}, trans. Thomas Allan Smith (Cambridge: Wm. B. Eerdmans Publishing Co., 2012) 105, 187.} Following Bulgakov, this analysis concludes that simply because the whole content of a dogmatic structure is not present does not mean it is not there, or that it is impersonal in its content. The same response can be given to Leont'ev's charge that there is no solid ground of monasticism in the novel; hesychasm can influence the text without being explicitly depicted, much less to be institutional. Ferapont, and his closeness to Leont'ev, explicitly expresses these institutional elements. The rigid structure of monasticism is present in the text and vital to its central moment\textemdash the Wedding in Cana. Therefore, rather than rehashing the abundant analysis of "rosy" love and harmony in Zosima, this analysis prioritizes the centrality of Leont'ev's fear of God. Despite the obviously antagonistic and problematic qualities that arise in Ferapont's fanaticism, the foundation upon which that fervor stands is not rejected or abandoned, but the \emph{prelest'} is exaggerated in order to highlight the underlying elements of the hesychastic tradition.

We must then determine if Ferapont is a representation of Orthodox monasticism, positive or otherwise. Linda J. Ivanits remarks: "Commentators have almost unanimously perceived Ferapont as evil. Most studies have considered him an unhealthy, somewhat comic double looming behind the saner figure of the charismatic elder, and at least one study has specifically linked him to the devil."\footnote{Linda J. Ivanits, "Hagiography in \emph{Brat'ja Karamazovy}: Zosima, Ferapont, and the Russian Monastic Saint," \emph{Russian Language Journal} 34, no. 117 (1980): 110.} Even if Ferapont is evil, he could still represent institutional Orthodoxy, one who resembles the rigid type of Russian, and particularly Siberian, priests who, according to Aleksandr Vrangel and Nikolai Lossky, Dostoevsky particularly loathed:

\begin{quote}
Baron Vrangel, describing his correspondence with Dostoevsky in Siberia between 1854\textendash 56, says: "We spoke little of religion. He was rather pious, but went to church rarely and did not love priests, particularly Siberian ones. He spoke of Christ with delight." Dostoevsky disliked Russian priests for quite a long time afterwards.\footnote{Nikolai Lossky. \emph{Dostoevskii i ego khristianskoe miroponimanie} (New York: Izdatel'stvo imeni Chekova, 1953), 66.}
\end{quote}

\noindent Dostoevsky's feelings make Ivanits' interpretation of Ferapont as a critique of rigid clericalism highly plausible. If Dostoevsky truly despised such priests and their dogmatic rigor, Ferapont could be read be read as satire, suggesting the author's rejection of this institutional fanaticism.

But Dostoevsky's polyphonic artistry allows for truth to exist as antinomy, using contradictory truths to reveal deeper realities of higher realism. Therefore, we should approach Ferapont as if through a glass, darkly, in the context of antinomic apophaticism, or as Carol Apollonio contends, as a reversal or negation:

\begin{quote}
Dostoevsky communicates his religious message in ways that are artistically consistent, though complex. A trope of reversal or negation is at work\textemdash the more appealing or seductive the arguments or images on the surface of the text, the more likely it is that they are false\textemdash not in a primitive factual sense, but in the sense that their seductiveness leads away from the truth. Conversely, an ugly or dirty surface may very well serve as a conduit to revelation.\footnote{Carol Apollonio, "Dostoevsky's Religion: Words, Images, and the Seed of Charity," \emph{Dostoevsky Studies}, New Series, vol. XIII (2009): 24.}
\end{quote}

\noindent Taking this advice on reading religious themes and applying it to Ferapont exposes Dostoev\-sky's poetic mastery in creating a paradox in institutional and personal Christianity.

\subsection*{Dostoevsky's Relationship to and Representation \\ of Institutional Orthodoxy}

To unearth this subversive reading of both the text and Dostoevsky's relationship to Orthodox ontological and anthropological trends, we should read the ugliness of Ferapont and his fanaticism not as a blight eclipsing the foundations of institutional Orthodoxy, but rather as a perverse lionization of positive institutional Orthodox practice. And while Dostoevsky's hesychastic foundations in Ferapont do not excuse or justify the problematic elements of the antagonist, the hesychastic monk illustrates the paradoxical nature of ideology and practice and highlights the value of Orthodox theological praxis. By reading Ferapont's monastic practice as one which parallels and inverts Zosima's rather than contradicting it, it becomes clear how his, and institutional Orthodoxy's, role in the structure and scope of the novel is just as vital as Zosima's and "minimal" religion.

Both monks are practitioners of the hesychastic method, the system of spiritual ascension cultivated throughout the history of Orthodoxy. Sergei Khoruzhii describes hesychastic practice as "the tradition occupied exclusively with creating and then keeping and reproducing identically the hesychast practice or the spiritual art of 'Noetic Practice' (\emph{Praxis noera}, in Greek), a holistic practice of man's complete self-transformation, in which an adept of the practice, advancing step-by-step, ascends to theosis, the union with God in His energies."\footnote{Sergei Khoruzhii, "\emph{The Brothers Karamazov} in the Prism of Hesychast Anthropology," \emph{Institut Sinergiinoi Antropologii Digital Library}, (2008): 4, \url{https://synergia-isa.ru/wp-content/uploads/2011/08/hor_karamazov_boston_2008_eng.pdf}.} This mode of spiritual practice is based, in part, on the teachings of Isaac the Syrian and John Climacus, and is central to the soteriological and eschatological visions of Orthodoxy.

Dostoevsky was well aware of the hesychastic tradition\textemdash if not explicitly through its teachings then at least through his exposure to it throughout his life, including his frequent visits to monasteries with his mother during his youth, his readings, and his trip to Optina with Vladimir Soloviev. There is significant textual evidence of Dostoevsky's awareness of the writings and teachings of Isaac and Climacus, as well as the most widely available contemporary source on Mt. Athos, \emph{The} \emph{Legend of the Pilgrimage and Journey through Russia, Moldavia, Turkey, and the Holy Land of the monk Parfenii of the Holy Mount Athos}. For example, in the notebooks to \emph{Demons}, Shatov claims to have read the \emph{Tale}: "I once read the book by the monk Parfenii about his travels on Athos\textemdash how the monk Nikolai had the gift of tears\textemdash and here you are, the monk Nikolai, who has the gift of tears."\footnote{Dostoevsky, \emph{P.S.S.} XI, 76.} What's more, this "gift of tears" is mentioned frequently in Hesychast texts, including the seventh step in Climacus' \emph{The Ladder of Divine Ascent}, as well as in \emph{The Ascetical Homilies} of Isaac the Syrian.

{\tolerance=1 \emergencystretch=5em
Furthermore, Dostoevsky paraphrases another step of \emph{The Ladder} in the same notebooks when he writes "An angel never falls \ldots{} the person falls and arises," which N.F. Budanov, in the commentary to the notebooks, claims is most likely reproduced from memory, rather than the direct source: "The expression goes back to the aforementioned text \emph{The Ladder}. Dostoevsky inaccurately (obviously from memory) quotes a saying from the ladder that he liked."\footnote{Ibid., and Dostoevsky, \emph{P.S.S.} XII, 352.} Despite the fact that it is unclear whether Dostoevsky knew that the gift of tears and process of spiritual regeneration were explicitly hesychastic, he was clearly aware of these elements as steps toward the Orthodox goal of theosis\textemdash which is the hesychastic process.}

Clearly, Dostoevsky was familiar with the monastic and hesychastic elements from both these textual sources and their praxis in the liturgical cycle. We know that Dostoevsky not only read the lives of the saints and other Orthodox books in his childhood, but his mother frequently took the family to church and would make a yearly five- to six-day pilgrimage to the Trinity-Sergeev Lavra, where St. Sergius of Radonezh applied and instilled the hesychast mode of spiritual practice. And it was to this foundation that he returned in Siberia and brought back with him in both his life and writings. As the narrator, mirroring Dostoevsky's thoughts while in exile, says in \emph{Notes from the House of the Dead}:

{\begin{quote}
I very much liked Clean Week. Those who fast were excused from work. 
We went to the church that was not far from the prison, two or three 
times a day. I had not been to church for a long time. The Lenten 
service, so familiar from my distant youth, in my parent's home, 
the solemn prayers, the prostrations to the ground\textemdash{}all 
stirred up the long-distant past in my soul, reminded me of the 
impressions of my childhood years.\footnote{Dostoevsky, \emph{P.S.S.} IV, 176.}
\end{quote}}

\noindent In these memories of the first week of Lenten services, Dostoevsky reveals not only a childlike connection to the liturgical life of the Church but also a direct connection with the hesychastic tradition as it seeps into the everyday life of the Orthodox believer.

Dostoevsky's appreciation for Clean Week\textemdash the first week of Lent, during which the laity engage most closely with the hesychastic tradition through the institutionalized liturgical cy\-cle\textemdash is crucial for understanding the Ferapont paradox and its connection to the religious themes in \emph{The Brothers Karamazov}. During the first four evenings of Clean Week, the penitential canon of Andrew of Crete is performed. Dostoevsky was introduced to two key aspects of institutional Orthodox tradition that informed his writing of \emph{The Brothers Karamazov}. The first is the codification of the hesychastic tradition for a non-monastic audience. Andrew of Crete conveys his study and practice of hesychastic, cenobitic monasticism that he learned at one of its principal centers, the Monastery of St. Sabbas in the Kidron Valley.\footnote{For more on the explicit relationship between the Canon of Andrew of Crete and hesychasm see Krzysztof Leśniewski, "The Great Canon of St. Andrew of Crete. Scriptural, Liturgical and Hesychastic Invitation for an Encounter with God,"~\emph{Vox Patrum}~69 (2018), https://doi.org/10.31743/vp.3268.} The canon outlines the monastic principles by which an individual assumes responsibility "for all and on behalf of all." Second, the canon incorporates the lives of St. Mary of Egypt and the monk Zosima, both of whom are referenced explicitly and implicitly in \emph{The Brothers Karamazov}. Although Dostoevsky did not possess the texts themselves, the monastic tradition's lived presence during Lenten services profoundly shaped his understanding of their themes, and clearly take shape in the novel. Recognizing this interplay between the personal and institutional aspects of religion is essential when considering how Dostoevsky portrays the practical application of monastic principles in his works.

But there is another layer of value in understanding Dostoevsky's reliance on institutional Orthodoxy to portray a deeply personal engagement with it. If hesychasm is, at its foundation, a tradition of the practical method of achieving spiritual regeneration, cleansing the eye of the heart, and moving the person closer to theosis (the mode of being-in-relation with the Divine), does one need to explicitly follow each step accordingly in order to achieve salvation? According to Khoruzhii, although the practice requires concrete and repeatable modes, the "tradition is a community, united on the basis of a certain practice {[}\ldots{}{]} For reaching its goal, this anthropological practice should have precise plan and method, which means that it should be based on reliable anthropological knowledge."\footnote{Khoruzhii, 3.} He further clarifies how, within the communities in which hesychasm is practiced, a complete engagement with this way of life is not necessary. Even beginning to engage with it brings about a greater awareness of its thrust and telos:

\begin{quote}
At a first glance, describing human being in reference to the steps of the practice, even connecting the human constitution with this practice, it would connect hesychast anthropology exclusively to the small community of adepts cultivating this practice; which implies that it is an especially narrow, specialized anthropological conception of no value for general anthropology. However, it turns out to be of much wider extent and greater importance. \ldots{} Its spiritual and moral authority produces anthropological implications. In any Orthodox society there emerges always some circle of people, for whom the integral way of life created by the tradition (\emph{bios hesychastos}, the "hesychast life") that becomes the model and reference point for their lives. They do not become "full-time adepts" of hesychasm and members of hesychast tradition, but nevertheless they adhere to the "hesychast life" in various degrees and forms: they adopt its attitudes and values, learn some elements of its school of prayer, assimilate some of its behavioral patterns, etc. etc. In short, they conform to the tradition and are orientated towards it in their way of life, both inner and outer.\footnote{Ibid., 7.}
\end{quote}

\noindent This understanding Orthodox monastic practice as fluid, flexible, and personal\textemdash as paradoxical as that statement might be\textemdash is essential to the Ferapont paradox. In the same way in which one does not need to be within the "small community of adepts" to practice the ascetic lifestyle, so too can the reader of Dostoevsky's Orthodoxy engage with the hesychastic qualities of the novel as they can bear it. As one "adopts {[}the{]} attitudes and values" of Dostoevsky's Orthodox vision, they move closer to a revelation of the higher reality in the novel, without needing to entering fully into the religious praxis of hesychasm. It is revealed in itself, but focuses more sharply with a proper lens.

When reading Dostoevsky with an eye toward the theological or religiously institutional context, one should take care to remember that he was not writing explicitly theological or liturgical texts.\footnote{The glaring exception here is the \emph{zhitie} (\emph{life}) of Zosima, penned by Alyosha, in \emph{The Brothers Karamazov}. However, this \emph{zhitie} explicitly trends toward a more literary, non-institutional mode as it is written in the first-person. Typically, \emph{zhitiia} are written from the perspective of the biographer, with the invaluable exception of the \emph{zhitie} of the Archpriest Avakuum, head of the then-schismatic Old Believers movement, which is written in the first-person and penned by Avakuum himself.} However, he wrote for an audience who shares his background and knowledge base\textemdash his audience would have grown up with the dogmatic and institutional elements ingrained in them through lived, synergetic experiences rather than as texts or manuals. But they likely wouldn't have an immediate or "adept" grasp of them. Therefore, we should understand that Dostoevsky's writing is more in line with the institutional yet personal hesychast anthropology as outlined above than the dogmatic. He encodes these institutional elements in the world of the monastery which then expand into the "secular" world of the novel. In this way the institutional "plan and method" of hesychastic practice is located in both worlds, and Ferapont is key to this subtle poetic and moral structure.

\subsection*{The Hesychastic Method and Its Representation in the Novel}

Dostoevsky masterfully encodes the spiritual self-practice of hesychasm into his narratives, and specifically in \emph{The Brothers Karamazov}. As one delves deeper, Dostoevsky's inversion of the institutional traditions of Orthodoxy to highlight their significance sharpens into focus. Therefore, it is necessary to "conform to the tradition and {[}orient{]} towards it," through an understanding of the steps of hesychastic practice and their representation in \emph{The Brothers} \emph{Karamazov}.\footnote{Khoruzhii, 7.} Doing so will allow the reader to see clearly that which Dostoevsky has purposefully placed on the edge of their perspective: the Ferapont paradox. John Climacus' \emph{The Ladder of Divine Ascent} offers the most detailed manual on hesychasm, but for clarity, its thirty steps can be grouped into four stages\textemdash the \emph{Spiritual Gate} of \emph{Metanoia}, \emph{Unseen Warfare}, \emph{Hesychia}, and \emph{Noetic Vision}\textemdash will suffice to allow the general reader more deeply into the circle of "full time adepts."

The first stage, known as the \emph{Spiritual Gate}, emphasizes repentance and the beginnings of an existential change. While many of Dostoevsky's characters, such as Raskolnikov in \emph{Crime and Punishment}, experience the beginnings of \emph{metanoia}, only Alyosha and Zosima embody the experience of the \emph{Spiritual Gate} as a mode of being. Isaac the Syrian, whom Dostoevsky explicit references in multiple texts, defined \emph{metanoia} as the opening of the person to love and perfection and connected it with the gift of tears.\footnote{As quoted in Hilarion Alfeyev, \emph{The Spiritual World of Isaac the Syrian} (Kalamazoo, MI: Cistercian Publications, 2000).} Dostoevsky's depiction of the gift of tears, which accompanies the penitent as they experience the first level of spiritual ascent, reflects his understanding of this initial stage of spiritual ascent. The first explicit mention of these tears is relegated to the notebooks for \emph{Demons} by Shatov.\footnote{See footnote 20.} However, Dostoevsky depicts Alyosha concretely experiencing this state following the death of Zosima:

\begin{quote}
He did not know why he was embracing {[}the earth{]}, he did not give himself an answer as to why he so uncontrollably wanted to kiss it, to kiss it thoroughly, but he, weeping, kissed it, sobbing and pouring out his tears, and frantically swore to love it, to love unto ages of ages \ldots{} What was he crying for? O, he wept in his rapture and for the stars, which shone for him from the abyss, and "he was not ashamed of his ecstasy." It was as if the threads from all these countless worlds of God converged at once in his soul, and {[}his soul{]} trembled all over, "touching other worlds." He wanted to forgive everyone, and for everything, and ask for forgiveness. Oh! not for himself, but for everyone, for everything and for everything, and "as others are asking for me," rang again in his soul. But every moment he felt clearly and, as if palpably, as something solid and unshakable, like this vault of heaven, descended into his soul. Some sort of idea reigned in his mind\textemdash and remained, for the rest of his life and to ages of ages. He fell to the earth a weak youth, and arose an unyielding fighter for the rest of his life.\footnote{Dostoevsky, \emph{P.S.S.} XIV, 328.}
\end{quote}

\noindent Here Dostoevsky inverts the first stage of the hesychastic practice, depicting his "full-time adept" Alyosha experiencing its first fruits \emph{after} he has risen up to the higher rungs. In doing so, he displays to the reader the paradoxical or suprarational elements of "higher" realism come to light through the fact that hesychasm is not experienced in a purely linear manner. Hesychasm, and Dostoevsky's project as guided by it, is a fluid state in which the spiritual struggle occurs, sending the reader toward a telos of unifying the world as-it-is with the world as-it-will-be\textemdash toward a communion with the Divine.

This first stage is evident in Ferapont's ascetic dedication as a "great faster and hesychast" (\emph{velikii postnik i molchal'nik}), though his spiritual focus is prone to wavering.\footnote{Ibid., 151.} Ferapont's physical feats of spending "full day(s) in prayer" without moving and the keeping of the fast\textemdash to the extent that he barely eats \emph{prosphora} (blessed bread) and survives on mushrooms and water\textemdash demonstrates the physical act of self-emptying that begins \emph{metanoia}.\footnote{Ibid., 152.} In these exploits he begins the turn from a fallen mode of being for the self to a mode of being-in-relation with the Divine and is able to inhabit other steps of hesychastic practice. As Isaac the Syrian relates in \emph{The Ascetical Homilies}:

\begin{quote}
Fasting, vigil and wakefulness in God's service, renouncing the sweetness of sleep by crucifying the body throughout the day and night, are God's holy pathway and the foundation of every virtue. Fasting is the champion of every virtue, the beginning of the struggle, the crown of the abstinent, the beauty of virginity and sanctity, the resplendence of chastity, the commencement of the path of Christianity, the mother of prayer, the well-spring of sobriety and prudence, the teacher of stillness, and the precursor of all good works.\footnote{As quoted in Alfeyev, \emph{The Spiritual World of Isaac the Syrian}, 85.}
\end{quote}

\noindent Fasting aids the ascetic in their attempt to cultivate and sustain repentant \emph{metanoia}. Ferapont finds the precursor in a physical mode, and through it he is able to maintain the rigorous, institutional manner of living the hesychastic method.

This devotion to fasting leads Ferapont to the second stage of hesychastic practice, \emph{Unseen Warfare}\textemdash the struggle with the Passions. Spiritual struggle is marked by the awareness of personal responsibility for fighting against the internal and external influences that leads a person away from unification with the Divine. Khoruzhii indicates that "the very first tasks of the ascetic practice include the removing and uprooting of the Passions \ldots{} then the ascetic proceeds to create 'preventive reaction,' i.e. the suppression of any incipient passion {[}\ldots{}{]} leading him to \emph{dispassion}."\footnote{Khoruzhii, 5.} In this way, \emph{Unseen Warfare} as an internal struggle with egocentric desire is often depicted as a struggle with the demonic. Ferapont's engagements in the novel with both the demonic and the Divine are therefore a representation of his own passions and the temptations of the world around him. His constant awareness of and conversations with little demons, the Holy Spirit, and the "Holispirit" are proof of this step on hesychastic practice.\footnote{Dostoevsky, \emph{P.S.S.} XIV, 153\textendash 4, 302\textendash 304. Konstantin Mochulsky claims that the appearance of the "Holispirit" to Ferapont\textemdash rather than the "Holy Spirit"\textemdash represents a bastardization of Dostoevsky's "bright Christianity" as represented in Zosima. However Orthodox monastic thought does not claim that a misunderstanding of the nature of demons negates the spiritual battle and striving for holiness of a monastic. Rather, Ferapont's vision marks the heatedness of his spiritual battle, as John Climacus, one of the pillars of monastic thought, notes: "An active soul is a provocation to demons, yet the greater our conflicts the greater our rewards" (John Climacus, \emph{The Ladder of Divine Ascent}, trans. Colm Luibheid and Norman Russell {[}London: Paulist Press, 1982{]}, 251). What's more, Mochulsky points to the roots of Ferapont in Father Pallady and other contemporary monks of Optina at the time of Dostoevsky's writings as a sign of Ferapont's non-institutional Orthodoxy (Konstantin Mochulsky, \emph{Dostoevsky: His Life and Work}, translated by Michael Minihan {[}Princeton: Princeton University Press, 1967{]}, 581\textendash 2). However, the zhitie of the hierdeacon Pallady concludes with a description that is closer to Zosima than Ferapont in demeanor. The chronicler notes that strict fasting and silence led Pallady to be amazed by everything in the forest, and that when he would approach trees\textemdash much like when Ferapont sees Christ in the trees (Dostoevsky, \emph{P.S.S.} XIV, 154)\textemdash he would be overcome with the glory of the Divine in nature (\emph{Istoricheskoe opisanie Kozel'skoi Vvedenskoi Optinoi pustyni: Opisanie monastyria}, 4th ed., ed. L. Kavelin {[}Moscow: Typografiia M.G. Volchaninova, 1885{]}, 240). If Mochulsky's analysis of Komarovich's genealogy of Ferapont is correct, then we have yet another subtle yet potent inversion of institutional Orthodoxy.} In Ferapont, Dostoevsky integrates one of his most common themes, the problem of the overabundance of consciousness, in the mode of institutional Orthodoxy by representing it as not merely a sociological or biological problem, but as the struggle with the Passions. While the reader might be mollified into finding Ferapont's struggle as unremarkable, Dostoevsky is refamiliarizing these issues in the context of the most common state of spiritual existence according to the hesychastic method. Institutional religion is both the solution to and a cause of Dostoevsky's primary concern regarding Russian nineteenth century issues: hyper consciousness.\footnote{Dostoevsky calls attention to the problem of hyperconsciousness in the author's footnote to the title of Part I of \emph{Notes from the Underground} (Dostoevsky, \emph{P.S.S.} X, 99).}

\emph{Hesychia}, the third stage of monastic practice, involves overcoming spiritual attacks and moving into a space of silence and constant prayer. This stage is described as "one of 'sacred silence', tranquility, quiet concentration and integration."\footnote{Khoruzhii, 6.} The practitioner learns to contemplate and pray in silence by overcoming the passions, utilizing prayer to progress further in \emph{Unseen Warfare}. The victory over demonic presences is predicated on silence and fasting, according to Orthodox tradition, based upon Christ's saying that certain demons can "only be driven out through prayer and fasting" (Mt. 17:21).

Dostoevsky signals Ferapont's engagement with this level of ascetic practice by naming him a "hesychast" (\emph{molchal'nik)}. While the Slavonic term differs from the Greek ἡσυχία ("hesychia"\footnote{While the term ἡσυχία is also present in the Russian as \emph{исихиа}, the two are nearly interchangeable.  {\emph{Молчальник} remains primarily from its use in the Kyiv Caves Paterikon and Dostoevsky most likely uses it rather than the Greek so it would be more familiar to those unfamiliar with non-Slavic Orthodox terminology and tradition.}}), both imply the concept of sacred silence and the need for it in ascetic struggle. In Zosima's homilies and talks, he repeatedly references monks with the same terms used by the chronicler/narrator to describe Ferapont\textemdash fasters and hesychasts (\emph{postniki i molchal'niki,}).\footnote{Dostoevsky, \emph{P.S.S.} XIV; 285, 298, 301.} Zosima never speaks of monks without referencing these two elements of ascetic struggle, thereby reinforcing Ferapont's connection to both institutional and personal religion as well as the inverted parallel Dostoevsky creates between the two monks. A similar parallel is created with Alyosha, who is called a "fighter" after he weeps and kisses the earth following his moment of \emph{metanoia} on the monastery grounds.\footnote{Ibid., 328.} Herein, Dostoevsky again inverts the monastic process in Alyosha, making him into a spiritual warrior only after he has achieved spiritual silence. Thus, reinforcing the connections between the protagonists journey and the struggle of hesychastic practice, specifically by mirror-imaging the route as presented by Ferapont.

\emph{Noetic Vision,} the fourth stage of hesychastic practice, is what Khoruzhii describes as the "ontological unlocking of anthropological reality."\footnote{Khoruzhii, 5.} This stage represents a moment in which the person transcends ordinary modes of consciousness, either sub- or hyper-, and achieve a higher engagement with reality\textemdash a cognitive mode of supraconsciousness that surpasses typical perception. This level of spiritual attunement opens the person to reality without linear time or space\textemdash an infinitude of timelessness and spacelessness marked by communion with Divine being. This achievement of a moment of authentic being changes the ontological boundaries of the world for the person. In this moment, the transcendent briefly becomes tangible, sensible, open to rational contemplation, and leads to a fullness of the mode of being-in-relation with the world both as-it-is and as-it-should-be.\footnote{For more on the concept of being-in-relation see Christos Yannaras, \emph{Relational Ontology}, trans. Norman Russell (Brookline, MA: Holy Cross Orthodox Press).} Through \emph{Noetic Vision} the adept is capable of seeing future events (Zosima's foresight of Mitya's great suffering), a sympathetic and uninhibited mode of autocommunication beyond verbalized communication (Alyosha's compassionate realization of Grushenka's thoughts during their first meeting), or even visions of supra-phenomenal beings. Ferapont's visons of Christ and devils exemplify the latter category of supraconscious perception, and are essential to understanding his complex role in the novel, showcasing Dostoevsky's exploration of spiritual extremism within his literary and moral framework.

It is necessary, however, to first address a major argument against a hesychastic reading of Ferapont. Are not biological or other non-metaphysical reasons for Ferapont's visions more convincing than a religious reading? Some scholars claim that Ferapont's visions are purely natural, or at least a psychedelic state brought on by consumption of mushrooms. Jones, for example, in stating that Ferapont's devils are "hunger induced visions."\footnote{Jones, 136.} This, and other such claims to non-religious foundations for his visions, implicitly reject the relationship between the institutional monastic practice and a higher mode of consciousness. In claiming that Ferapont's visions are nothing more than physiological manifestations of biological necessity and/or psycho-pharmacological hallucinations critics strip Ferapont of his connection to the anthropocosmos of the monastery\textemdash they take him out of institutional Orthodoxy and out of Dostoevsky's circle of positive protagonists. What's more, the claim that Ferapont is on the hesychastic path does not deny the possibility that Ferapont is at times, if not often, on a psychedelic trip caused by the consumption of mushrooms; hallucinating due to exhaustion and hunger; or suffering from a mental or physical disorder that leads him to such visions. Just as Dostoevsky's narrator provides evidence for natural/biological grounds of seemingly demonic incursions into the novel\textemdash including Ivan Karamazov's conversation with the devil\textemdash there is also sufficient evidence to prove that Ferapont is experiencing \emph{both} natural hallucinations and the end-results of the hesychastic practice. Furthermore, Ferapont's role in the monastery and his fidelity to institutional Orthodoxy are essential to the novel's poetic structure, specifically as it is central to his casting out of Alyosha's demon of \emph{prelest'} during the novel's dénouement.

The application of a critical eye to the Orthodox ontological boundaries of Dostoevsky's texts allows for both an Orthodox reading as readily as it does a materialist or minimalist religious reading. However, there is textual evidence that Ferapont, as a practitioner of the hesychastic method, would experience visions in a real, supraconscious manner. One might argue that Ferapont, as a satire of fanaticism, should or could not reach this state of spiritual progress\textemdash Ferapont's fasting and prayer might not have allowed him to reach \emph{Noetic Vision}. Yet visions of the demonic are not only prescribed to the final state of hesychastic practice, they occur also at the most common stage\textemdash \emph{Unseen Warfare}. But what good would it do for the poetic structure of the novel to allow such a vile figure to experience supra-sensory visions, whether because of his holiness or his failures? In either case, through \emph{Unseen Warfare} and \emph{Noetic Vision} Dostoevsky is establishing a truth of the ontological boundaries of his texts via an apophatic proof of the existence of a higher realm of consciousness\textemdash he is establishing hesychastic practice in the text as an institutional source of Orthodoxy. And while it would be simple for him to establish this Orthodox ontology by repeating the institutional texts or utilizing a hagiographic mode, by posing Zosima's opponent as a ground of truth, Dostoevsky paradoxically creates an even more compelling argument for his understanding of the rightness of the practice of hesychia.

\subsection*{Ferapont's Institutional Foundations as the Key \\ to the Novel's Central Event}

Ferapont, the most fervent embodiment of institutional Orthodoxy in the novel, takes his ascetic practice to extremes, ultimately falling into the spiritual self-delusion of \emph{prelest'}. But his personal failures do not negate his practices or their positive contributions to the value of the hesychastic practice in the text. What's more, his fasting, isolation, and rare attendance at liturgical services should not in any way be interpreted as outside of Orthodox tradition. In fact, monastic saints such as Anthony the Great, Mary of Egypt, and Anthony of the Kyivan Caves Monastery all lived in isolation, mortifying their flesh, fasting, and experiencing demonic invasion; Anthony the great experienced visions of demons, Mary of Egypt communed only twice during her isolation in the desert, and Anthony of the Kyivan Caves was noted by his visitors for his taciturn demeanor and rigorous application of monastic practices. Ferapont, therefore, functions as an inverted-hagiographic figure in the text, mirroring these saints while subverting his spiritual achievement through excessive legalism and fanaticism.

Unlike these saints, however, Ferapont distorts the practice, falling into fanaticism. He becomes so assured of his own righteousness that he becomes the perfect model of satire for what Dostoevsky despised in his encounters with Siberian priests\textemdash the Pharisaic application of rigorous legalism as the sole method of spiritual ascent. Ferapont's unwavering self-righteousness transforms him into this satirical figure, critiquing the rigidity of adherents to a solely fear-based form of dogmatic, institutional Orthodoxy. However, just as Zosima's perceived laxity and "rosy" Christianity does not signify a rejection of institutional Orthodoxy, neither does Ferapont's abuse of it undermine it. Ferapont's complex and antinomial role comes into full force during Zosima's funeral. And by interpreting the miraculous stench of corruption through the lens of hesychastic practices, it becomes evident that this the miraculous occurrence critiques not Zosima's righteousness, as Ferapont insists and the majority of the Russian public comes to believe, but the dangers of Alyosha's fanaticism and the others who demand a miraculous sign from God.

Ferapont jubilantly breaks his silence upon hearing the "good" news that "God's judgment is not as man's, and that it has even forestalled nature."\footnote{Dostoevsky, \emph{P.S.S.} XIV, 300.} Entering Zosima's cell, he begins casting out demons, which he sees crawling in every corner, and which he interoperates as the demonic manifestation of Zosima's pride, lax monasticism, and worldly love over Godly fear. A surface reading might view this moment as the culmination of the religious satire, exposing the absurdity of Ferapont's fanaticism and the pitfalls of \emph{prelest'}. After all, in the end Zosima is resurrected and his teachings and life are institutionalized by Alyosha, as Rozanov prophesied. Yet, ironically\emph{,} Ferapont's accusations carry a kernel of the truth. In the Orthodox ontological context, the demons of pride and misguided "rosy" Christianity haunt Zosima's coffin\textemdash but they belong chiefly to Alyosha as a sign of his own spiritual immaturity.

Ferapont's envy toward Zosima exemplifies a common monastic struggle, reflecting Dos\-to\-ev\-sky's nuanced portrayal of institutional hesychastic monasticism. As John Climacus writes in Step 4 of \emph{The} \emph{Ladder}:

\begin{quote}
Those living in stillness and subject to a father have only demons working against them. But those living in a community have to fight both demons and human beings. The first kind keep the commands of their master more strictly since they are always under his scrutiny, while the latter break them to some extent of on account of his being away. Still, the zealous and the hard-working more than compensate for this failing by their persistence and, accordingly they win double crowns.\footnote{John Climacus, 110.}
\end{quote}

\noindent For the \emph{postnik i velikii molchalnik} (\emph{faster and great hesychast}), the struggle is not with external demonic forces alone, but with the demons of communal life\textemdash a realm unfamiliar to his as an eremitic monk. Unlike Zosima, a cenobitic monk and elder who is\textemdash as Climacus demonstrates\textemdash accustomed to the spiritual weight of his own as well as others passions and demons, Ferapont rigidly applies his solitary practices to as dogmatic, universal edict (\emph{ukaz}).\footnote{For more on \emph{указ} and legalism in Orthodoxy in this context, see Rozanov, "Russkaia tserkov'."} This misattribution of sinfulness to Zosima reflects not a failure of institutional Orthodoxy fails but rather due to his own personal, minimalist abuse of it.

Dostoevsky deliberately complicates the narrative, obscuring the line between spiritual truth and deception\textemdash he makes finding the good more difficult for his hero and readers. At Alyosha's lowest, most de-Orthodoxi\-zed moment, Dostoevsky offers an Orthodox critique of \emph{prelest'} by satirizing his hero's fanaticism through the comically dogmatic, fanatically devout Ferapont. Dostoevsky's satire, therefore, does not merely target Ferapont but also extends to Alyosha and Zosima's other disciples, exposing their coercive need for the miraculous. The narrative parallels the third temptation of Christ in the desert, presenting Alyosha's crisis of (little) faith. By connecting this scene directly to Ivan's \emph{poema} of the Grand Inquisitor, Dostoevsky scatters unmistakable signs of \emph{prelest'} in the most unexpected places.

The first sprouting of the odor of \emph{prelest'} arises in Ferapont's treatment by Fr. Paissy. Paissy greets Ferapont with contempt\textemdash just as Ferapont is treated by scholars who, as Ivanits posits, see him as a minor character\textemdash merely fanatical, perhaps evil, yet essential harmless.\footnote{Ivanits, "Hagiography," 110.} Paissy says "Why for have you come, dear father? Why for are you disrupting the services?" He then accuses his monastic brother of egocentric sinfulness, stating "You drive out the unclean spirit, but perhaps you are serving him?"\footnote{Dostoevsky, \emph{P.S.S.} XIV, 302, 303.} While these accusations might seem to condemn Ferapont, they in fact reveal the Pharisaical tendencies of Dostoevsky's heroic retinue.

Dostoevsky brilliantly encodes an intertextual relationship with Matthew 12: 22\textendash 7. When Paissy wrong-heartedly claims that Ferapont is serving the devil he directly parallels one of Christ's admonitions. Ironically, it is not Paissy who is acting in the image of Christ. Rather he levels the same egocentric charge at Ferapont that the Pharisees level at Christ. In Matthew, they claim that when driving out demons Christ acts through the power of Satan. Here Christ replies: "Every kingdom divided against itself will be ruined, and every city or household divided against itself will not stand. If Satan drives out Satan, he is divided against himself. How then can his kingdom stand? And if I drive out demons by Beelzebul, by whom do your people drive them out?" Dostoevsky, contrary to the reader's expectations, places Paissy in the role of the Pharisees\textemdash he is accusing Ferapont in the same manner in which the Pharisee accuses Christ. Therefore, Ferapont is not seeing demons because he is sinful, but because he is a great faster and hesychast\textemdash he is both the Image and Likeness of Christ, although in an ironic, perverse, realist manner.

However, it is one thing to claim that because he is a great faster and hesychast Ferapont can see demons. But this is not enough to show that Dostoevsky is reifying Ferapont's devotion to the hesychastic practice. He is clearly still a fanatic and incorrectly identifies demons. It is a more difficult question to ask whether or not he might cast them out, as Christ says, "through prayer and fasting" (Mt. 17:21)? Indeed, Dostoevsky positions him to do so, albeit in a delayed and incredibly byzantine manner. Ferapont drives Alyosha from the monastery, his visions of demons frightening Alyosha away from the side of his elder where he enters into his own \emph{Spiritual Warfare}. Alyosha and the reader realize that Zosima was not a transcendent being made of transcendent material, he was human like the incarnate Christ\textemdash and particularly like the Christ of Holbein's \emph{Dead Christ in the Tomb}, about whom Dostoevsky said "anyone could lose their faith" due to the horrors of death and decay.\footnote{Dostoevsky, \emph{P.S.S.} VIII, 182.} Facing the unbearable suffering of death, even armed with an unshakable faith in the presence of resurrection, Alyosha succumbs to temptation and he flings himself into the abyss of sin. The narrator remarks "He loved his God and believed unflinchingly in Him, although he suddenly murmured against him."\footnote{Dostoevsky, \emph{P.S.S.} XIV, 307.} Looking upon the decaying, stinking corpse of the physical incarnation of the Divine makes Alyosha lose his faith, and the reader loses their faith in the novel's authorially proclaimed hero as well. It is this loss of faith that manifests in the demons that Ferapont sees. And when he says "Satan, get thee hence! Satan, get thee hence! \ldots{} Casting I cast out," he casts out Alyosha and the reader's hagiographic expectations.\footnote{Ibid., 302.}

Ferapont catalyzes Alyosha's \emph{metanoic} turn, marking his first authentic steps on the institutional monastic path to theosis. Neither Zosima's assurances in life, Paissy's compassion at the death of their Elder, nor Alyosha's personal religiosity suffices to dispel his youthful self-assurance and demands on the Divine. Instead, Ferapont's exorcism of Alyosha's devils forces Alyosha to confront his responsibility for the miraculous odor of corruption. His descent into sinfulness begins with his journey to Grushenka\textemdash whom he depersonalizes into the demonic personification of lust\textemdash in order to destroy his soul, flinging himself into the abyss of \emph{Karamazovschchina}. However, through an inverted hesychastic process, the beginning of his \emph{Spiritual Warfare}, leads Alyosha to a profound \emph{metanoic} state. He recognizes her personhood and love, and reifies the hesychastic active love of Zosima when he tells Rakitin: "You'd do better to look here, at her: did you see how she spared me? I came here looking for a wicked soul\textemdash I was drawn to that, because I was low and wicked myself, but I found a true sister, I found a treasure\textemdash a loving soul \ldots{} She spared me just now \ldots{} I'm speaking of you, Agrafena Alexandrovna. You restored my soul just now."\footnote{Ibid., 318} Overcoming his \emph{Spiritual Warfare} he sits in \emph{hesychastic}, humble silence, not forcing the truth on her, but allowing her freely to confront her own sinfulness and make a small \emph{metanoic} turn as well. This moment underscores Dostoevsky's engagement with institutional Orthodoxy\textemdash without Ferapont's fear of God, Alyosha could not become the hero Dostoevsky intends. The active love of Zosima can only, at least only in Alyosha's personal space, come \emph{after} the fear of God.

Although Ferapont is so wrapped up in his own \emph{prelest'} that he cannot see his own sins, he is still able to penetrate into the souls of others. He realizes how they have strayed from the active and loving communion\textemdash about which Zosima preaches in Chapter 3 of Book VI "From Talks and Homilies of the Elder Zosima"\textemdash in the name of the miraculous, or at the very least in the name of Divine justice. Ferapont is able to see the spiritual reality of their \emph{prelest'} manifested as demons due to his direct engagement with the institutional aspects of Orthodox eremitic monasticism\textemdash he has cleansed his spiritual eye and obtained \emph{noetic vision} (even if it is imperfectly tuned). But rather than explicitly preaching this point, Dostoevsky uses the actions of an antagonist on the hesychastic path to guide his hero, Alyosha. And while it is Alyosha's realization that he has strayed from the truth of his faith that charge his repentance and the gift of tears with pathos and humiliation, Dostoevsky crafts a moment that feels more relatable and real to the reader, filled with both the personal and lived religious experience but inescapably bound by the context of the tradition and dogmas of the Church in this apophatic and antinomial manner. By dispossessing the reader of a faith in institutional Orthodoxy, paradoxically through the most rigorous and faithful instantiation of it, Dostoevsky is able to produce the dénouement of his poetic and moral project by revitalizing that very corn of wheat, which must die to bring forth much fruit.\footnote{A paraphrase of John 12:24, which is the epigraph to the novel (Dostoevsky, \emph{P.S.S.} XIV, 5).}

\subsection*{The Paradox of Religious Fear and Divine Love}

The Ferapont paradox lies in his unwavering clear devotion to the letter of the law\textemdash the fidelity to edict (\emph{ukaz})\textemdash which paradoxically distances him from the liturgical life of Orthodoxy. For both Ferapont and Leont'ev the fear of God blinds them to the living God revealed in Alyosha and Rozanov. Rather than drawing them into union with Christ, it alienates them from Divine, active love. As Hebrews 10:31 declares, fear is meant to drive a person into the hands of Christ "It is a fearful thing to fall into the hands of the Living God!" (\emph{Strashno vpast' v ruki Boga zhivago}!). While fear is a necessary precursor to communion with the Divine, as seen in Alyosha's noetic vision of Cana, Leont'ev and Ferapont fail to embrace the redemptive love that follows.

Both Ferapont and Leont'ev share visions of Christ shaped by an overwhelming and alienating fear of Him. Leont'ev's autobiographical reflection in his hagiography "Father Kliment Zederogol'm" published in the November/December edition of \emph{Russian Messenger} in January, 1879, reveals the nature of this fear and isolation:

\begin{quote}
Now in winter, when I come to Optina Pustyn, I often have to pass by the path that leads to the large wooden Crucifix of the cemetery at the small skete. The path has been cleared, but the graves are covered with snow. In the evening, on the Crucifix, a lamp is lit in a red lamp, and wherever I return at a late hour, I see this light from afar in the darkness and know what is there, near this crimson, shining spot. \ldots{} Sometimes it seems meek, but sometimes unbearably terrifying in the darkness in the middle of the snow! \ldots{} Terrifying (\emph{Strashno}) for yourself, terrifying for loved ones, terrifying especially for your homeland, when you remember how few people {[}such as Fr. Kliment{]} are in it, and how early they die, not having accomplished even half of what is possible for them \ldots{}\footnote{It is quite a marvelous coincidence that Leont'ev published this work in this edition of \emph{Russkii vestnik}, which also contains Chapters V-VIII of Book VIII of \emph{The Brothers Karamazov.} Konstantin Leont'ev, "Otets Kliment Zedergol'm, ieromonakh Optinoi Pustyni," in \emph{Polnoe sobranie sochinenii i pisem v 12 tomakh} (St. Petersburg: Izdatel'stvo "Vladimir Dal'\," 2000), vol. 6: 351.}
\end{quote}

\noindent Similarly, Ferapont's vision of Christ in Chapter I of Book IV in \emph{The Brothers Karamazov} utilizes the same language and tenor: "It happens at night. Do you see the two branches? At night Christ Himself reaches His hands toward me and searches with those His hands for me, I clearly see and tremble. Terrible, o terrible! (\emph{Strashno, o strashno}!)"\footnote{Dostoevsky, \emph{P.S.S.} XIV, 154.} Both see only the image of Christ crucified\textemdash the fearful inertia of Holbein's \emph{Christ Entombed}. These mortified hands are not the "hands of the Living God," which Alyosha encounters in his noetic vision\textemdash the open arms that parallel those with which the Father forgives the Prodigal son. Alyosha is embraced by the arms of an actively loving God, and as Zosima says: "Don't fear him. Terrible (\emph{Strashen}) is His glory before us, fearful in is His heights, but His mercy is infinite, out of love He has become like us and rejoices with us \ldots{}"\footnote{Ibid., 327.}

In Orthodox theology and Dostoevsky's poetics, both love and fear hold central importance, yet it is love that draws one inward from the "Walls of the Church" without cloistering them to the point of egocentric \emph{prelest'}. Love embraces philosophical skepticism and doubt, nurturing the seed of faith, while fear retreats into rigid dogmatism, clinging to assurance of proper-worship. Love reflects the polyphony of the uniqueness of humanity, created in the Image and Likeness of the Divine, while fear weeps at the tragedy of an ontologically monologic world incapable of freedom and individual beauty. Dostoevsky's vision aligns with institutional Orthodoxy, echoing early Church Fathers like Hermas, who cautioned against overemphasizing fear: "But do you clothe yourself in the desire of righteousness, and, having armed yourself with the fear of the Lord, resist them. For the fear of God dwells in the good desire. If the evil desire shall see you armed with the fear of God and resisting itself, it shall flee far from you, and shall no more be seen of you, being in fear of your arms."\footnote{Hermas is cited in Alexander Roberts, James Donaldson, A. Cleveland Coxe, and Allan Menzies,~\emph{Ante-Nicene Fathers: The Writings of the Fathers Down to A.D. 325} (Peabody, MA: Hendrickson Publishers, 1994), vol. 2: 28.} By rejecting the plurality of human otherness in the world, Ferapont and Leont'ev prioritize the egocentrism they claim to despise, to an "evil desire" for a world without freedom and sin.

Dostoevsky recognizes the necessity of the fear of God while highlighting the dangers of excessive fear, particularly its potential to spiral into fanaticism as epitomized in the Leont'ev-Ferapont extreme. In his works, fear as the dominant source of spiritual well-being clearly leads to coercion. As S.L. Frank observes, writing about the anthropological problem of coercion: "{[}C{]}ompulsion as such is itself an objectively sinful action, even if it proceeds from a subjectively righteous motive, for it is a violation of the God-given freedom expressive of human personality as akin to God."\footnote{Frank, 184.} "The Grand Inquisitor" provides a vivid portrayal of coercion, particularly as the foundation for the sin of so-called "caesaropapism." And yet Leont'ev, in his correspondence with Rozanov, shockingly wrote that:

\begin{quote}
  After all, I confess that although I am not, of course, completely on the side of the Inquisitor, I am neither on the side of the lifeless, all-forgiving Christ, which Dostoevsky himself invented. Both of these are extreme. And according to the Gospels and the Holy Fathers, \emph{the truth is in the middle}. I asked the monks, and they confirmed my opinion. \emph{The actual} Grand Inquisitor, of course, believed in God and Christ \emph{stronger than} Fyo{[}dor{]} Mikh{[}ailovich{]}. Iv{[}an{]} Karamazov, through whose mouth Fyod{[}or{]} Mikh{[}ailovich{]} wants to humiliate Catholicism, \emph{is completely wrong.}\footnote{Rozanov, \emph{Sobranie~sochinenii}, vol. 3: 354.}
\end{quote}

\noindent Leont'ev and the Optina Elders fidelity to the Inquisitor may shock modern readers, but Roza\-nov contextualizes this stance within the broader framework of the fanaticism of idealism prevalent in the late nineteenth century\textemdash a force deeply terrifying to Dostoevsky. His footnote highlights the perilous trajectory of such idealism:

\begin{quote}
We Russians, in general, understand only \emph{the} \emph{type of Russian faith}, the type of faith of a somewhat carefree and unenergetic person. \emph{The idealists} of the French Revolution began the "terror reipublicae" {[}"public terror" (lat.){]}, and \emph{the idealists} of the Christian faith began the inquisition, this "terror fidei" {[}"terror of faith" (lat.){]}. It is amazing that very \emph{serious believers} do not abhor the Inquisition even now! They don\textquotesingle t complain that "it happened;" they don\textquotesingle t write satires or cartoons of the auto-da-fe. This is their silence, their calmness (among our so liberal times!) it shows that \emph{the idealism of faith} really contains an "inquisitorial moment:" a little more, their eyes will become gloomy, their eyelids will drop, their lips will shrink, and they will say "auto-da-fe."\footnote{Ibid., 355.}
\end{quote}

\noindent The danger of the fanatical fear of the living, loving elements of Orthodoxy transforms into an inquisitorial spirit. And it is precisely this jeopardy that Rozanov sees in the necrotic elements of devotion to edict (\emph{ukaz}) that absolutizes legalism over a personal-yet-structured form of spiritual self-practice, such as the hesychasm preached by Zosima.

In \emph{The} \emph{Brothers Karamazov}, Dostoevsky confronts his personal struggles with faith while affirming the truths he holds about death, resurrection, and Orthodoxy. Tempering institutional rigor with personal love, Dostoevsky finds resolution to the question of whether or not he will ever see Masha again.\footnote{A reference to Dostoevsky's journal entry following the death of his wife, in which he states "Masha is lying on the table. Will I see Masha again?" Dostoevsky, XX, 172.} The synergetic engagement between institutional Orthdoxy and its personal expression culminate in Alyosha's noetic vision, offering an apophatic yet positive conclusion to the question of the immortality of the soul. By harmonizing the personal experience of divine love with the institutional rigor of Orthodox practice, Dostoevsky invites readers into a transformative engagement with faith\textemdash one that transcends rigid dogmatism to reveal the profound potential for spiritual renewal and communion with the Divine. Importantly, Dostoevsky does not demand adherence to institutional Orthodoxy or hesychastic practice as the only path to authentic living\textemdash to become a Zosima, a Ferapont, or an Alyosha. Grushenka, in fact, proves that it is possible to achieve \emph{metanoia} through folk practice and belief, through "minimal religion," highlighting the universal nature of Divine love.\footnote{See  {Morson, "The God of Onions."}}

But Dostoevsky's poetic structure requires some measure of institutional Orthodoxy to reach this spiritual vision\textemdash a spiritual polyphony played out in secular literature. Hesychastic practice structures and cultivates the novel's capacity to depict a glimpse of Dostoevsky's truth of higher reality. And unlike his other novels, "higher" realism is seen "no longer as through a glass, darkly," but rather in the full light of noetic vision. Central to Dostoevsky's poetics is the idea that access to higher reality relies on the relational mode of being cultivated through hesychastic practice. For Alyosha, adherence to a structured, institutional practice rejects the legalism that fosters fanaticism\textemdash a danger exemplified by both Ferapont and Leont'ev. Through this intricate synthesis of fear, love, and spiritual discipline, Dostoevsky illustrates the path toward genuine communion with the Divine.

\vspace{2em}
\begin{center}
  \includegraphics[width=0.75cm]{articlend.png}
\end{center}

\biobox{\textbf{Peter Winsky} received his Ph.D. in Slavic Languages and Literatures in June 2021 from the University of California, Los Angeles. His dissertation "Dostoevsky through the Lens of Orthodox Personalism: Synergetic Anthropology and Relational Ontology as Poetic Foundations of Higher Realism" approaches the post-Siberian novels of Dostoevsky within the context of contemporary Orthodox philosophical and theological trends. His research focuses on Russian literature of the 19th century and Orthodox Personalism, as well as Russian Ornamentalist prose of the early 20th century and Yugoslav Blackwave Cinema. He is currently a Lecturer at the University of Southern California, teaching courses on Russian 19th century literature, cultural history, and religious and philosophical thought. Peter serves as moderator/curator of the \textit{Northwestern University Forum for Russian Philosophy, Literature, and Religious Thought} and as editor of \textit{Northwestern University Studies in Russian Philosophy and Religious Thought.}}

\label{sec:winsky}

\fancypagestyle{chaptercontentpage}{
  \fancyhf{} % Clear all header and footer fields
\fancyhead[CE]{%
  \fontsize{11}{11}\leftmarkfont%
  \addfontfeature{LetterSpace=10.0}%
  \textit{\MakeUppercase{\leftmark}}%
}
  \fancyhead[CO]{\authorheadfont\addfontfeature{LetterSpace=10.0}\fontsize{11}{11}\selectfont\textbf{{\uppercase{Jillian Pignataro}}}}
  \renewcommand{\headrulewidth}{0pt} % No header rule on content pages
  \fancyfoot[RE]{\thepage}
  \fancyfoot[LO]{\thepage}
}

\newpage{}
\abstractbox{In Search of a Harmonious Whole}{Nikolai Strakhov's Organicist Critique of Darwinism}{
Jillian Pignataro}{This paper clarifies the Russian naturalist philosopher, Nikolai Strakhov’s critique of Darwinian thought as it was presented in a series of articles written from 1862 to 1896. This paper substantiates Strakhov’s initial praise for \textit{On the Origin of Species} and his sustained interest in the key Darwinian concept that organisms experience changes through time in Strakhov’s broader philosophy. Strakhov’s analysis of Darwinism, as this paper shows, is rooted in his adherence to the philosophical movement of Organicism, which dictates that the world is to be understood as one coherent, interconnected whole. On this basis, Strakhov critiques Darwin’s dethroning of man as the highest organism in the natural world, his interpretation of the world as a series of "random occurrences," and above all else, Darwinism’s ostensible theory of "external teleology," as opposed to the organicist notion that all organisms contain an inner purposiveness. Borrowing from Hegelian philosophy and working with the texts written by his close friend, the Russian philosopher Nikolai Danilevsky, Strakhov purports that in order to understand any organism, one must recognize its internal teleology. According to Strakhov, Darwinian thought, precluding the notion of internal teleology and thus perfectibility, presents an incomplete and even erroneous view of the world.}{Nikolai Strakhov, Organicism, Charles Darwin, Nikolai Danilevsky, Kliment Timiriazev, natural selection, Hegel, teleology}

\section{Jillian Pignataro - In Search of a Harmonious Whole}
\fancypagestyle{chaptertitlepage}{
  \fancyhf{} % Clear all header and footer fields
  \fancyhead[L]{\begin{minipage}[t]{0.7\textwidth}\publisher\end{minipage}}
  \fancyhead[R]{\begin{minipage}[t]{\textwidth}\raggedleft \datefont\fontsize{10}{11}\selectfont Volume 1 (2024): \thepage\textendash\pageref{sec:pignataro} \\ \doi{10.71521/c9nm-dm86} \end{minipage}}
  \renewcommand{\headrulewidth}{0pt} % No header rule on title pages
  \fancyfoot[RE]{\thepage}
  \fancyfoot[LO]{\thepage}
}

\chaptertitle{In Search of a Harmonious Whole}{Nikolai Strakhov's Organicist \\ Critique of Darwinism}{
Jillian Pignataro}

\addcontentsline{toc}{chapter}{In Search of a Harmonious Whole: \\ Nikolai Strakhov's Organicist Critique of Darwinism \\ \emph{by} Jillian Pignataro}
\setcounter{footnote}{0}
\seriffont
\fontsize{12}{18}\selectfont

\noindent In the 1860s, the Russian intellectual landscape was fundamentally altered by the introduction of Charles Darwin's (1809\textendash 1882) distinct evolutionary theories. For Russian materialists, Darwin's \emph{On the Origin of Species} was a triumph, while the Tsarist regime and Russian Orthodox philosophers were largely disturbed by its arrival. As an Orthodox believer and a trained biologist and zoologist, the idealist and organicist Nikolai Strakhov (1828\textendash 1896) was uniquely poised to be the most important polemicist of Darwinism in Russia. Although Strakhov initially praised Darwin's \emph{Origin}, he quickly became one of Darwin's most virulent critics, an "uncompromising anti-Darwinist."\footnote{Alexander Vucinich, \emph{Darwin in Russian Thought} (University of California Press, 1989), 103.} Strakhov's organicism, his belief in the inherent wholeness of the universe, dictated this critique. According to Strakhov, Darwinism, which is mechanistic and not organicist, denies that the universe is interconnected, that organisms are arranged hierarchically, and that man occupies the supreme position in this world. As a consequence, Darwinism deprives organisms of any \emph{a priori} guiding principle, a "goal" ordained in creation, and therefore precludes any theory of "internal teleology."

If one looks at the entirety of Strakhov's criticism leveled against Darwinism beginning in 1862 and continuing until Strakhov's death in 1896, it is clear that his principal issues concern Darwin's belief in "blind chance" and "randomness," Darwinists' degradation of man as the apex of organic life, the pervasive materialism and nihilism among Darwin's followers, and the preclusion of internal teleology in Darwinian thought. The interpretation that Darwin had argued in favor of a world governed by chance was widespread in its time, even from scientists. For example, Darwin's close mentor, the prolific scientist John Herschel (1792\textendash 1871) famously called the theory of natural selection "the law of higgedly-piggedly."\footnote{James G. Lennox and Charles H. Pence,~"Darwinism," \emph{The Stanford Encyclopedia of Philosophy} (Summer 2024), ed.~Edward N Zalta \& Uri Nodelman, https://plato.stanford.edu/archives/sum2024/entries/darwinism/.} In relation to Strakhov, scholars have examined many of the aforementioned points of contention; in particular, Strakhov's anthropocentrism has been emphasized. In this essay, I will provide a broad analysis of an aspect of Strakhov's anti-Darwinism that remains insufficiently studied, his theory of internal teleology. Internal teleology is fundamental to Strakhov's organic worldview, which holds that an organism is a total living system that functions with internal purposiveness, each part interdependent to each other and the whole. Naturally then, internal teleology is a concept Strakhov admittedly borrows not from the natural sciences but from German Idealism, particularly from the work of his most important influence, the organic philosopher G.W.F. Hegel (1770\textendash 1831).

As one scholar notes, Strakhov's organicism "decisively shaped" how he understood Darwinism and provided the springboard by which he subsequently rejected it.\footnote{Brendan G. Mooney, "Strakhov on Darwinism: Humans, Progress, and Organicism" in \emph{Reading Darwin in Imperial Russia: Literature and Ideas} (Lanham, Maryland: Lexington Books, 2023), 104.} Considering organicism is Strakhov's primary philosophical preoccupation from which all of his philosophical arguments derive, I will begin with an overview of this philosophy. Organicism originates in Plato (427\textendash 348 B.C.) and is given broader shape in the work of several German Idealists, including Hegel, especially with regard to his metaphysics of Nature. In organicism, the universe is conceived as one cohesive, living whole, harmonious, and precluding any superfluous elements; all organisms within this whole are microcosms of the macrocosm, or functionally purposive wholes unto themselves.

Strakhov's most robust definition and defense of organicism is to be found in his book \emph{The World as Whole} (\emph{Mir kak tseloe}). Published in 1872, \emph{The World as Whole} is a collection of essays, many of which were written in the 1850s and 1860s, on organic and inorganic nature and man's place in the universe. In this work, Strakhov clarifies what he means by the titular phrase:

\begin{quote}
\emph{The world is a harmonious whole} {[}стройное целое{]} or, as they say, a harmonious, organic whole. That is, the parts and phenomena of the world are not just connected but subordinated; they represent a ladder of right {[}правильную лестницу{]}, a pyramid, and best of all, a hierarchy of beings and phenomena. The world, as an organism, has parts that are less important and more important, higher, and lower; and the relationship between these parts is such that they represent harmony, serve one another, and form one whole in which there is nothing superfluous or useless.\footnote{Strakhov, \emph{Mir kak tseloe}, Predislovie k pervomu izdaniju (Moscow: Ajris press, 2007), 67.}
\end{quote}

\noindent Another fundamental aspect of Strakhov's organicism was its theistic foundation. Strakhov believed that order was given to the universe upon its creation and that the interconnectedness of all organisms was a consequence of the participation of all organisms in a shared, spiritual realm. Because of this belief, Strakhov was strictly opposed to the idea that scientific inquiry and empiricism exhausts one's knowledge of the universe. Strakhov's understanding of the limits of empiricism was also the impetus for his opposition to spiritualism and mysticism. Strakhov's primary critique is that spiritualists were misguided in their attempts to justify their belief in the supernatural by searching for the presence of the divine in the material world. Strakhov even writes that Darwinism is a "delusion" that can be put on par with spiritualism.\footnote{Strakhov, Strakhov, \emph{Mir kak tseloe,} Predislovie ko vtoromu izdaniju, 76.} Considering the extent to which mysticism pervaded Russian idealist philosophy in the early 20\textsuperscript{th} century, particularly with the influence of Vladimir Solovyev (1853\textendash 1900), with whom Strakhov feuded, Strakhov's simultaneous anti-materialism and anti-spiritualism was highly unique. Though resonances of Strakhov's organicism can be seen in many of the Russian neo-idealists, ultimately Strakhov's legacy was carried by his pupil, the controversial religious thinker Vasily Rozanov (1856\textendash 1919). Rozanov's belief in the unity of humanity and the common, divine origin of all organisms was inspired by Strakhov.\footnote{Adam Ure, "Rozanov, the Creation, and the Rejection of Eschatology," \emph{The Slavonic and East European Review}, Vol. 89, No.~2 (April 2011): 224.}

In order to understand Strakhov's critique of Darwinism as it relates to teleological questions, it is important to begin with Strakhov's first encounter with the theory. The Russian translation of Darwin's \emph{On the Origin of Species} was not published until 1864; however, Darwin's discoveries had been discussed and analyzed in Russia for several years prior. Not only was the French translation published in 1862, but Russian thinkers also had access to critical reviews from abroad, which naturally contained summaries of Darwin's seminal text.

Strakhov was the first thinker to introduce Darwin's \emph{Origin} in some capacity to a Russian audience. Strakhov was not an unlikely figure to have brought Darwinism to Russia. Though a seminarist like many of his fellow \emph{raznochintsy}, Strakhov had a consistent interest in science. In fact, shortly before his encounter with Darwin's \emph{Origin}, in 1857, Strakhov completed his master's thesis, titled "On Mammalian Wrist Bones" (\emph{O kostiakh zapiast'ia mlekopitaiushchikh}), which was subsequently published in the \emph{Journal of the Ministry of Public Education} (\emph{Zhurnal ministerstva narodnogo prosveshchenija}). Shortly before Darwin's book was published in London (November 1859), Darwin's mentor, the geologist Charles Lyell (1797\textendash 1875) gave a public lecture to the British association for the Advancement of Science, in which he provided a scant overview of Darwin's discoveries. In January 1860, Strakhov published an abridged translation of Lyell's address. This translation was titled "The Appearance of Man on Earth" (\emph{Pojavlenie cheloveka na zemle}). Unfortunately, Strakhov did not provide much information on \emph{Origin} and Darwinian theory in this article since he had yet to read the book.\footnote{Brendan G. Mooney, "What is in a Word? A History of the Words 'Evolution' and 'Natural Selection' in Russian and of Kliment Timiriazev's Legacy as a Translator and Popularizer of Darwinism" in \emph{Reading Darwin in Imperial Russia: Literature and Ideas} (Lanham, Maryland: Lexington Books, 2023), 13\textendash 60.}

It may be assumed that Strakhov first read \emph{Origin} once it was translated into French in 1862. This is also the year that Strakhov writes his first review of Darwinian theory, an essay titled "Bad Signs" (\emph{Durnye priznaki}), published in the Dostoevsky brothers' journal \emph{Time} (\emph{Vremya}). Despite its title, which alludes to the possible "social Darwinist" conclusions the French translator incautiously attached to Darwinism, "Bad Signs" is a largely positive assessment of Darwin. Above all else, Strakhov was enthusiastic because Darwin's \emph{Origin} definitively showed that organisms are not static entities. The stringent belief in the constancy of organisms may be ascribed to the influence of the French zoologist, Georges Cuvier (1769\textendash 1832), who famously constructed a system of animal taxonomy. Cuvier's theory is rooted in the broader philosophical position known as essentialism. In short, essentialism is the belief in essences and may be traced back to Plato's "theory of forms." Plato and subsequent metaphysical positions founded on his essentialist theory allege that all organisms have been endowed with immutable properties, i.e., essences, or what Plato often termed forms. Plato's theory of forms, laid out in his dialogue \emph{Phaedo}, states that all similar objects participate in a single shared form, a universal "-ness." Plato believes that the world of forms is in a sense more real than the world of empirical objects. Therefore, the universal property, the shared form that an object participates in, is a better reflection, a more "essential" property of the object than any feature detectable by the human eye.

Strakhov does not refer to essentialism as such nor Plato in "Bad Signs"; however, he does describe the "old worldview," that is the view that Darwin's \emph{Origin} combats, as one dependent on a belief in "essences" {[}сущности{]}. Strakhov writes that if, according to essentialism, organisms are given immutable characteristics upon creation, then organisms necessarily go unchanged over time. In this way, this old worldview propounds a world that is a mere "accidental collision" {[}случайное столкновение{]} of these unchanging properties.\footnote{Nikolai Strakhov, "Durnye priznaki (O knige Ch. Darvina "Proiskhozhdenie vidov")" in \emph{Kriticheskie stat'i} (1861\textendash 1894) (Kiev, 1902), 385.} The philosophical tides had been turning for decades if not centuries prior to 1859 and Darwin's discoveries; essentialism was questioned any moment man realized that knowledge was not the mere piecemeal "acknowledgement" of "eternal treasures of truth."\footnote{Ibid.} However, Strakhov believed that Darwin's book represented a "great progress" and "huge step" in the field of natural sciences. It provided a succinct and accessible rebuttal to essentialism and the stagnancy it necessarily propounded. For the first time, Darwin proved the variability of species, that they evolve and gradually degenerate from one form to another; most importantly, Darwin articulated the mechanism by which this occurs, his law of natural selection\textemdash the struggle for the means of existence among all organic beings.

Albeit brief, "Bad Signs" shows how optimistic Strakhov was in 1862 with regard to Darwin and the theories laid out in \emph{Origin}. Nevertheless, over the course of the 1860s, Strakhov cemented his position as one of the most virulent and prolific anti-Darwinists in the history of Russian thought. For instance, in 1873, Strakhov writes that "the rapidity with which Darwin\textquotesingle s theory gained followers does not at all correspond to its intrinsic dignity" and that the current passion for Darwin is "deeply false" and "extremely ugly."\footnote{Nikolai Strakhov, "Darvin," Filosofskaja kul\textquotesingle tura: Zhurnal russkoj intelligencii, no. 2 (July-Dec 2005, St.~Petersburg) https://www.hrono.ru/proekty/metafizik/fk208.html.} Importantly, Strakhov never denies the merit behind Darwin's initial discovery, namely that species evolve, change form, and regenerate from each other. However, according to Strakhov, Darwin never successfully proves his theory of natural selection.

Several scholars have attempted to explain Strakhov's apparent ideological shift from a position supporting Darwin's \emph{Origin} to that of an unequivocal anti-Darwinist. The scholar Alexander Vucinich views Strakhov's shift as a proper conversion. According to Vucinich, Strakhov had abandoned his former belief in evolution and advocated instead for philosophical idealism, a change that was fueled by his support for the Tsarist regime against its radical critics of the 1870s. In 1848, the Tsarist government implemented requirements in natural sciences in all gymnasiums. However, as Russian youth began to adopt radical politics, the regime associated these ideological trends with Darwinism. In short, Darwinism bolstered Materialism which, in turn, substantiated a nihilistic worldview. Notably, in 1871, with the assistance of the Russian historian Mikhail Pogodin (1800\textendash 1875), Strakhov convinced the Minister of Public Education Dmitri Tolstoy (1823\textendash 1889) that Darwinism was a "pseudo-science" rooted in "anti-Russian" sentiment and dogmatic materialism. Tolstoy ultimately revised the requirements for gymnasium curricula, precluding all study of the natural sciences.\footnote{Vucinich, 103\textendash 104.}

Nevertheless, Vucinich's claim that Strakhov launched a campaign against science as such is spurious. According to Vucinich, Strakhov "conducted bitter attacks on science as an archenemy of the sacred values of Russian culture."\footnote{Ibid., 103.} By "sacred values," we may presume Vucinich is referring to Orthodoxy. Science is not at odds with religion; however, scientific inquiry does have its limitations. At the end of \emph{The World as Whole}, Strakhov refers to "real knowledge," which satisfies a human's need to not just understand the material world but to apprehend the spiritual. Naturally, "real knowledge" is not gained from empirical, scientific inquiry alone. In short, Strakhov apprehended the critique of what would be termed naturalism in the 20\textsuperscript{th} century, that is the belief that all investigation of reality and the human spirit can be discerned through scientific means. Vucinich is also reductive when he states that Strakhov believed Russia must abandon Darwinism because it does not align with the "established," Cuvierian science.\footnote{Ibid., 102\textendash 103.} This false claim was echoed by one of Strakhov's ardent ideological opponents, who wrongly suggested that Strakhov "worshipped" Cuvier.\footnote{Kliment Timiriazev "Oprovergnut li darvinizm?" \emph{Izbrannye sochinenija v 4-kh tomakh}, tom. 4 (Moscow: Sel\textquotesingle hozgiz, 1948\textendash 1949).} On the contrary, Strakhov found Cuvier's belief in the "constancy of species" to be illegitimate. Over a decade after writing "Bad Signs," and after pressuring Tolstoy to amend the official gymnasium curriculum, Strakhov writes that Darwinism's "main strength consists in the ingenious hypotheses about the very process of how species change."\footnote{Strakhov "Darvin."} When Strakhov does discuss Cuvier alongside Darwin, he does not praise one or the other. Instead, Strakhov uses Cuvier as an example to denounce the dogmatism of Darwinists. Followers of the aforementioned essentialist worldview did not adhere to the belief that species are created and remain static because they believed in the validity of this theory. On the contrary, scientists followed closely behind Cuvier's essentialism simply because Cuvier said so and, as Strakhov writes, "it was impossible to dare to say otherwise." Strakhov continues, "That's how things go in the sciences. For some reason, some opinions begin to be considered orthodox, and others heretical; then the whole mass of scientists stubbornly and fervently stand for orthodox opinions, while heretical ones hardly dare to speak out and are met with general contempt."\footnote{Ibid.}

Among contemporary Darwinists, Strakhov saw a dangerous and undeserved adherence. Strakhov was undoubtedly concerned about the sweeping, seemingly blind acceptance of Darwin's theories. Even Darwin himself could not have imagined the "revolution" that his own theories galvanized in the field of natural science and beyond. Nevertheless, Darwin is still to blame. Strakhov writes in 1873 that every scientist has a certain prejudice to view their own findings as supremely authoritative, but this prejudice was particularly strong in Darwin.\footnote{Ibid.} Strakhov discusses the hubris of believing that one's ideas are categorically superior to all that came before. Strakhov writes,

\begin{quote}
Highly intelligent historians of recent times, imagining that they themselves walk in the truth, often present the entire history of people as wandering in mistakes, and the entire progress of this history as a gradual liberation from delusions. But if we are convinced that people's opinions have a different meaning beyond the objective truth, then maybe we will not look so arrogantly at past times and will not prematurely boast about the present.\footnote{Ibid.}
\end{quote}

\noindent In regard to Darwinism, though the old, aforementioned essentialist worldview contained very little "truth," it most certainly contained a "higher" and, most notably, more "moral" meaning.\footnote{Ibid.}

In opposition to Vucinich, the scholar Brendan Mooney argues that Strakhov did not experience an ideological conversion, but that his praise for Darwin was made on false assumptions. Strakhov mistakenly believed Darwinian theory was congruent with his own idealistic philosophy, his anthropocentrism, and his progressive and teleological vision of life.\footnote{Mooney, "Strakhov on Darwinism," 102.} Mooney writes, "The history of life on earth, according to Strakhov's view, is driven by an intrinsic teleological, perfecting force that culminates in humanity."\footnote{Ibid., 103.} Strakhov's anthropocentrism, which Mooney stresses, is undeniable. In \emph{The World as Whole}, Strakhov writes that not only is the world one coherent whole, but that man is the center of this world:

\begin{quote}
Man is the peak of nature, the node of being. Man is the greatest mystery and the greatest miracle of the universe. He occupies a central place in all directions of the connections connecting the world into one whole; he is the main essence and the main phenomenon and the main organ of the world. (Preface to the first edition)
\end{quote}

\noindent Materialists and Darwinists throughout the 1870s had abandoned the fact that man is the "knot of the universe." In worshipping atoms, a materialist breaks up the world into its smallest components, he looks for a way out of this whole; he seeks to break the ties connecting him with this world, to "break his umbilical cord."\footnote{Nikolai Strakhov \emph{Mir kak tseloe}, Predislovie k pervomu izdaniju, 68.}

If one analyzes Strakhov's "Bad Signs," it is clear that Strakhov was not necessarily confused by what he knew of Darwin so far but was instead overly optimistic about its eventual development. In reality, when Strakhov praised Darwin, he simply did not have enough knowledge of Darwinian thought to realize how incongruent it was to his own organic thinking. In "Bad Signs," Strakhov reiterates his support for Darwin's theory that all organisms develop through "interaction, reproduction, improvement, and struggle"; however, he admits this "process of \emph{internal} development" is "very complex" and not at all clear.\footnote{Strakhov, "Durnye priznaki."} This presumes that though Strakhov saw merit in Darwin's hypothesis, he expected Darwin to present a more thorough explanation of the theory of the variability of species and evolution more generally.

Before discussing Strakhov's interpretation of Darwin's version of teleology, it is productive to consider the competing critical opinions on this subject. Teleology is the doctrine that there is a design and a purpose to the universe. Aristotle (384\textendash 322 BC) is the progenitor of teleological thinking. In his \emph{Physics}, Aristotle presents his four causes, that is the principal explanations for movement in nature. Aristotle's fourth cause, the "final" cause states that an organism contains within itself a \emph{telos}, that is an end or purpose. Whereas Plato's teleological thinking emphasized the external source of all change, namely the Demiurge, Aristotle believed that an organism's \emph{telos} was immanent in nature.\footnote{Christopher Shields, "Aristotle,"~\emph{The Stanford Encyclopedia of Philosophy}, ed.~Edward N. Zalta \& Uri Nodelman (Winter, 2023) https://plato.stanford.edu/archives/win2023/entries/aristotle/.} Aristotelian thinking was suffused throughout early scientific thinking, was integral to the debates surrounding vitalism and mechanism in the 18\textsuperscript{th} century, and culminated in Darwin's theory of natural selection. Natural theology, a teleological way of thinking which states that the observable function of any organism proves the existence of a Divine creator and his well-thought-out "plan" for the universe, was never advocated by Darwin. Nonetheless, whether Darwin was a "teleologist" of any sort is the subject of a debate that has continued until this day.\footnote{The bioethicist John C. Lennox is the most notorious thinker to argue in favor of the idea that Darwin was a teleologist, notably in his 1993 article "Darwin was a Teleologist." Conversely, scholars such as the biologist Mark Ghiselin have vociferously argued that Darwin dispelled with the notion of teleology entirely.} In his discussion of issues relating to translating Darwin into Russian, Mooney writes, "This confusion over the ambiguity of the term 'teleology' led friends and foes to criticize and praise Darwin for both obviating the need for teleological explanations in science and for putting teleological explanations on a scientific footing."\footnote{Mooney, "What's in a Word?" 16.} As Mooney states, Darwin did not use the term "teleology" in \emph{Origin}; however, he does discuss "final causes," adding to his subsequent followers' and critics' bewilderment. Nonetheless, it is notable that, in 1874, Darwin responds positively to the American botanist Asa Gray's (1810\textendash 1888) evaluation that in effect labeled him a teleologist. Gray is himself responding to widespread accusations in England that Darwin had eliminated "final causes" and purpose from the study of Nature. In his book, \emph{Darwiniana}, Gray writes, "{[}L{]}et us recognize Darwin's great service to natural science in bringing Teleology back to it; so that, instead of Morphology versus Teleology, we shall have Morphology wedded to Teleology."\footnote{Asa Gray, \textit{Darwiniana: Essays and Reviews Pertaining to Darwinism} (New York: D. Appleton and Co., 1889), 288.} Darwin wrote Gray personally expressing his satisfaction with this statement: "What you say about Teleology pleases me especially and I do not think anyone else has ever noticed the point."\footnote{Charles Darwin, \emph{Life of Charles Darwin}, ed.~Francis Darwin (London: William Clowes and Sons, 1892), 308.} In addition to this correspondence, Darwin wrote explicitly in the concluding remarks to \emph{On the Origin of Species} that according to his theory of natural selection, which works for the "good" of all beings, every alteration in an organism's development tends to "progress towards perfection."\footnote{Charles Darwin, \emph{On the Origin of Species By Means of Natural Selection} (London: The Folio Society, 2006), 388.} Despite the various contestations, I will proceed on the assumption that Darwin indeed believed in final causes, that organisms' adaptations are defined by external stimuli and there is a progressive development in this adaptability.

It is also notable that the most vocal proponent of Darwinism in Russia, the preeminent botanist, Kliment Timiriazev (1843\textendash 1920), also believed Darwin was a teleologist. Timiriazev writes in an 1890 article that Darwinism "freed minds" from the "aversion" to everything teleological; Darwin created a new, natural scientific teleology. Timiriazev is especially relevant to any discussion of Strakhov's anti-Darwinism. As I discuss in more detail below, Strakhov dragged himself into a rancorous polemic with Timiriazev, prompted by the former's 1885 article, "A Complete Refutation of Darwinism." This debate was fueled by Strakhov's defense of the work done by his deceased friend and fellow anti-Darwinist, the Russian philosopher, historian, and ichthyologist, Nikolai Danilevsky (1822\textendash 1895). These articles, published during Strakhov's polemic with Timiriazev, provide the most comprehensive look at his anti-Darwinism and specifically his adherence to the theory of internal teleology.

To begin, Strakhov writes in 1887 that Darwinists' consistent error is in not distinguishing between the beginning and the end of a process.\footnote{{\raggedright Nikolai Strakhov, "Vsegdashnjaja oshibka Darvinistov: Po povodu stat\textquotesingle i prof. Timirjazeva: Oprovergnut li darvinizm?" originally published in \emph{Русский Вестник} no. 11, 12, 1887. \url{http://az.lib.ru/s/strahow_n_n/text_1887_oshibka.shtml.}}} In response to the question, "what purpose does such an organ serve?," a Darwinist will answer with an explanation of its current function, entirely ignoring the question of its original purpose. Though, according to Strakhov, Darwin was correct in his hypothesis that all species change, it cannot be claimed that Darwin "proved" or "explained" this theory.\footnote{Strakhov, "Darvin."} Darwinism purports to reveal the reason why one feature or principle of an organism developed over another, why one organism was selected for survival over another, in short, the \emph{origin} of species. However, this system relies on the idea that organisms are perfectly adapted to their external environment, that there is a certain \emph{tselesoobraznost\textquotesingle{}} or purposefulness in the organic world.\footnote{\emph{Tselesoobraznost'} is often translated to "expediency," which I find does not adequately encompass the meaning for Danilevsky and Strakhov. I have chosen "purposefulness" because it is used by both thinkers to mean the extent to which an organism evolves while adhering to a preordained goal or purpose. See Stephen M. Woodburn "Nationality, Philosophy, and Science in Nikolai Danilevsky's Critique" in \emph{Reading Darwin in Imperial Russia: Literature and Ideas,} p.~177, for a more thorough study of the use of this term and its various translations.} To paraphrase Strakhov's interpretation of Darwin's theory, "if a certain arrangement of organisms is necessary for their existence, then this excludes from the organic kingdom everything that is unsuitable and lacking purpose."\footnote{\hspace{-.1em}Nikolai Strakhov, "Polnoe oproverzhenie darvinizma" in \textit{Darvinizm. Kriticheskoe issledovanie N.Ja. Danilev-\\ skogo} (St. Petersburg, 1885), 516.}

The Darwinian model, that organisms are molded to the blind forces of nature, is insufficient for Strakhov. He succinctly asks, "The whole dispute actually boils down to this: where should we look for purposefulness, in organisms, or outside of them?"\footnote{Strakhov, "Vsegdashnjaja oshibka darvinistov."} All Darwinists seek this purpose in circumstances external to the organism, and all anti-Darwinists, including Strakhov himself, believe every organism contains an \emph{a priori} guiding principle (i.e., an internal teleology). Put simply, Strakhov is alluding here to the issue of external and internal teleology. A discussion of internal teleology can go back as far as Aristotle and his concept of "entelechy" {[}\emph{εντελέχεια}{]}; however, Strakhov specifically references Immanuel Kant (1724\textendash 1804) and Hegel, writing that these two philosophers brought the question of teleology to his farthest conclusions. Summarizing both of these thinkers, Strakhov defines internal teleology as such: "Internal teleology discovers goals that lie within the subjects themselves, stemming from their very nature; these goals are always definite, essential, and necessary."\footnote{Strakhov, \emph{Mir kak tseloe}, Part I, Letter VIII, 193.}

Internal teleology should be contrasted with external teleology. Strakhov defines external teleology in a letter written in 1861 and published in \emph{The World as Whole}: "External teleology seeks goals that lie outside the subjects themselves, for which there is no determination within the subject; such goals are always indefinite, relative, and conditional."\footnote{Ibid.} Strakhov consistently believed in the law of internal teleology, and specifically that a "true" {[}\emph{истинный}{]} teleologist looks at the meaning of a single organ and recognizes that the whole, as a "self-constructing" {[}\emph{самостроящегося}{]} entity, consists of many parts striving for one "common goal" {[}\emph{общую цель}{]}.\footnote{Strakhov, \emph{Mir kak tseloe}, Predislovie ko vtoromu izdaniju, 75.} In short, Strakhov advocated for a revised essentialism in which organisms retain their inherent purposiveness but also are eternally in flux.

In Strakhov's judgement, a scientist cannot understand an organism if he ignores its internal teleology. Darwin sought to reduce organisms to the most comprehensible units, which were themselves dictated solely by the laws of heredity and variability; however, he failed to ever explain the origin of species because he failed to seek an organism's essence and resorted to the "ruse" {[}\emph{уловка}{]} that the essence, the very thing necessary for explanation, does not exist at all.\footnote{Strakhov, "Darvin."} Strakhov's denunciation of what he viewed as a lack of internal teleology in Darwin was intertwined with the views of his friend and fellow philosopher, Danilevsky. In 1885, Danilevsky published Volume I of what he had hoped would be a longer series on Darwinism. Simply titled \emph{Darwinism: Critical Research}, this text provided a comprehensive rebuttal to Darwinian theory and cautioned against the rapid acceptance of Darwinism among scientists and philosophers of his day. Danilevsky composed this work in "close contact" with Strakhov and it should be presumed that they shared many of their concerns with Darwinism.\footnote{Gerstein, 159.} In later writings, Strakhov repeatedly praises Danilevsky personally, once describing his "spiritual nobility," and defends the scientific and philosophical merit of \emph{Darwinism}, what he called a "magnificent" {[}\emph{великолепный}{]} book.\footnote{Strakhov, "Polnoe oproverzhenie darvinizma."}

Danilevsky, unlike Strakhov, continued his work in the scientific field after he completed his master's thesis. In the early 1850's, when exiled to the Vologda province for his participation in the Petrashevsky circle, Danilevsky published a series of scientific articles. In 1853, Danilevsky was even awarded half the Geographical Society's annual Zhukov prize for his articles on the climate of the province.\footnote{Robert E. MacMaster, \emph{Danilevsky: A Russian Totalitarian Philosopher} (Cambridge: Harvard University Press, 1967), 98.} In June 1853, Danilevsky was appointed to be the statistician on an expedition to survey fisheries, led by the preeminent embryologist and naturalist Karl Ernst von Baer (1792\textendash 1876). In the remainder of that decade and the next, Danilevsky participated in and led several more expeditions of his own, largely focusing on ichthyology. In short, by the time Danilevsky began writing \emph{Darwinism} in 1870, he had spent many years doing scientific field work. In \emph{Darwinism}, Danilevsky positions himself as an unequivocal opponent of Darwinism, refuting it from a "natural-theological" position. In the text, Danilevsky presents a theory of evolution that has been described as a "borrowing" from his former associate Baer's "neo-Aristotelean teleological" one.\footnote{MacMaster, 148.} This Baerian view of the variability and development of species that is subordinate to a theory of internal teleology is instrumental to Strakhov's anti-Darwinism.

Danilevsky died shortly after the publication of Volume I of \emph{Darwinism}, and Strakhov, having almost been a collaborator in its composition, "made himself responsible" for Danilevsky's book.\footnote{Gerstein, 160.} Strakhov was even responsible for composing a final chapter and index for Volume II, compiled from Danilevsky's notes and unfinished manuscript. In January 1887, Strakhov's audacious essay "A Complete Refutation of Darwinism" (\emph{Polnoe oproverzhenie darvinizma}) appeared in the journal \emph{The Russian Herald} (\emph{Russkii vestnik}). Though the article contained many of the important criticisms which Strakhov leveled against Darwinism, this essay was essentially a scientific defense of Danilevsky's ideas. In "A Complete Refutation," presumably with the intent of bolstering Danilevsky's original argument, Strakhov accuses Darwin of professing a "pseudo-teleology." Strakhov first argues that it is wrong to classify Darwinism as a theory of development. To clarify, Strakhov writes, "By development, we mean a series of changes, one of which must necessarily follow from the other, as if by virtue of a certain, permanent law, even if in essence we did not understand this necessity, as in fact we almost never understand it, and conclude about it only from the constancy of the repetition of a series."\footnote{Strakhov, "Polnoe oproverzhenie darvinizma."} Darwin's schema\textemdash that all organisms are consistently and gradually adapting to their external environment\textemdash does not even attempt to account for a "permanent law" by which all organisms develop.

Again, linking his ideas to Danilevsky's, Strakhov writes that, as organisms move towards the aforementioned "purposefulness," they respond to changes that occur in their external environment. According to Darwinism, it is not because organisms have received a certain property that they thrive in certain conditions; on the contrary, it is only because organisms live under certain physical conditions that they possess the necessary properties most suitable to them. In other words, the properties of an organism that allow it to survive in nature occur \emph{a posteriori} to the very conditions in nature. Darwinism wants to explain all kinds of properties and differences of an organism. But as Strakhov asks, "How can we say that {[}Darwin's book{]} explains the origin of species?" It does not paint a picture of the vegetable or animal kingdom, nor does it explain heredity, sexual differences, or any of the "essential features" of organisms. Darwinism ostensibly suggests an external teleology but even this is spurious.

According to the theory of natural selection, the dominant force acting on each organism throughout its development is its own, individual usefulness {[}\emph{собственная полезность}{]}, "the perfection of the organisms themselves, their harmony with the conditions of existence." An organism takes a series of beneficial steps towards a more perfect form. Though Strakhov considers these benefits which appear in an organism's development as indelible facets of a broader, overarching goal, they are in fact only a condition that determines the formation of this organism, only an instrument, a mere means, not an end. Since Strakhov was explicit about his indebtedness to Hegel, even writing in the preface to \emph{The World as Whole} that the book is largely a reworking of Hegelian ideas, it is \emph{apropos} to consider Hegel's critique of external teleology as it relates to randomness. Hegel believed chiefly in the "internal purposiveness" of all living beings. But, in trying to decipher "final causes" of any feature of an organism, man falls into "trifling reflections." By beginning at the end, the supposed "goal" of an organism is ultimately entirely arbitrary and optional.\footnote{Jeffrey Hamilton Wattles, "Hegel's Philosophy of Organic Nature" (doctoral dissertation, Northwestern University, 1973), 25\textendash 26.}

In Strakhov's understanding, Darwinism claims that "there is no structure in organisms that would properly follow from a certain beginning as its cause {[}\ldots{}{]}. On the contrary, the whole order arises from disorder."\footnote{Strakhov, "Polnoe oproverzhenie darvinizma."} In reducing phenomena to chance, Darwin actually eliminates reason from the universe. In fact, according to Strakhov, naturalists, materialists, positivists, etc. are more hostile than others to the "rational view of things." Scientists who look at organisms \emph{correctly} that is, without reducing their form to chance, will see each step in an organism's development as a "manifestation of the fundamental principles that builds organic forms."\footnote{Strakhov, \emph{Mir kak tseloe,} Predislovie ko vtoromu izdaniju, 75.} Each development of form is "full of the deepest instructiveness in all its particulars."\footnote{Strakhov, "Polnoe oproverzhenie darvinizma."} A correct study of the purposefulness {[}\emph{целесообразность}{]} of organisms is really a study of the organic creation {[}\emph{органическое} \emph{творчество}{]} of organisms. According to Strakhov, nothing profound such as an organism's purpose can ever come from random particulars {[}\emph{случайных частностей}{]}, since from the beginning truth must be seen as a whole.\footnote{Strakhov, \emph{Mir kak tseloe,} Predislovie ko vtoromu izdaniju, 75.}

As a consequence of denying internal teleology, Darwinism also obfuscates the concept of "perfectibility" as it applies to the development of organisms. As evidence shows, organisms that live under the same physical conditions develop dissimilarly. For instance, a tadpole exists in the same environment as any small fish, its external influences are of the exact same nature. However, a tadpole develops from its lower form, into a more perfect form, the frog. Since the tadpole and any other cohabitating fish have qualitatively different developments, Strakhov believes that the tadpole's galvanizing principle must exist not in the external environment, but in the tadpole itself. Strakhov also illustrates this point with a discussion of the development of the human embryo. An embryo itself reaches personhood, just as a person once born develops either into someone insignificant or extraordinary or anything in between. Strakhov vehemently rejects the idea that external environment has a substantial impact on a human being: "It is quite clear that every person can develop only when he develops himself. Upbringing and education, in fact, do not produce development, but only give it an opportunity \ldots{}"\footnote{Strakhov, \emph{Mir kak tseloe}, Part I, Letter VI, 160.} To summarize this point, Strakhov poignantly writes in \emph{The World as Whole}, "Truly human, truly vital phenomena do not consist in blind submission to the environment, but in getting out from under its influences, in the development of a higher life on the steps of a lower one. This is the nature of human life, this is the nature of life in general, the life of all organisms."\footnote{Ibid., 161.} An infant needs to become a man and, with varying degrees of consciousness, he understands this goal and works to achieve it.

Therefore, a Darwinian worldview is erroneous because it claims that there is essentially no reason why an organism ended up in one form over another. There is no qualitative difference between an organism's final form after centuries of adaptation and another potential final form had the external influences been radically altered. As Gerstein puts it, Strakhov concedes that Darwin adequately explains how a man derived from an ape but could not even begin to answer \emph{how}, and I would add \emph{why}, man is different from an ape.\footnote{Gerstein, 159.} External influences exert the pressure needed for organisms to change; however, organisms not only change but they \emph{improve}. In short, Strakhov wants to reinstate the possibility of perfection, to emphasize the organicist position that the world has parts that are higher and parts that are lower, and that these parts represent a "harmony."

In 1887, the Darwinist Timiriazev discernibly asks, is the organic world, even in all its apparent "harmony," not also a collection of chaotic elements? According to Strakhov, Darwinism's ostensible principal failing is that it denies internal teleology and thus tacitly admits that nature is subject purely to chance; however, Timiriazev writes that Strakhov's anger is misplaced. It was not Darwin who opened the door to this conclusion, but the observable natural world that proved it to be so. Is the sun not a "chaos of accidents?" he asks. If you blame a Darwinist for disrupting the incredible harmony of the organic world, its aesthetic and moral sense, then you must also forbid an astronomer to use his telescope and look at the sun.\footnote{Timiriazev "Oprovergnut li darvinizm?"} The same conclusions against harmony will undoubtedly be drawn. Strakhov's response to Timiriazev is simple: "the task presented to us in the organic world is obviously a special and incomparably higher task than the task of astronomy."\footnote{Strakhov, "Vsegdashnjaja oshibka darvinistov."} Because there are unobservable phenomena, changing psychic, moral and mental forms, the task set before the mind is "immeasurably higher" than the task of the Darwinist. In the solar system, the sun, the planets, the comets, etc. change temperature and movement "but none of them, not a single atom in them feels a shadow of delight or anger."\footnote{Ibid.} In other words, man is categorically different than any other organism, and Strakhov refuses to believe that man was subject to the same chance development as organisms as insignificant as mold.

In a poignant passage in "Organic Categories," Strakhov ends his argument in favor of organic as opposed to mechanical categories, with the following idea: "Those who only look for a mechanical connection between phenomena will only find a mechanical connection. Those who are dissatisfied with the mechanical connection explanation will search for something deeper."\footnote{Nikolai Strakhov, "Organicheskie kategorii: Po povodu stat'i g. Edel'sona 'Ideia organizma.'" In \textit{Biblioteka dlia Chteniia}, 3, 1860 (St. Petersburg).} In other words, Strakhov establishes boundaries, and perhaps an insurmountable one, between those who are mechanically minded and those who accept an organic worldview. Needless to say, Danilevsky similarly opposed a mechanical worldview. In an emphatic passage in \emph{Darwinism}, Danilevsky exclaims "{[}H{]}ow pitiful, miserable the world and ourselves seem, in which all harmony, all order, and all reasonableness are only a special case of the senseless and absurd; all beauty is an accidental part of ugliness; all goodness is a direct inconsistency in the universal struggle, and the cosmos is only an accidental private exception from the wandering chaos!"\footnote{Danilevsky, \emph{Darvinizm} quoted in K. Timiriazev's "Oprovergnut li darvinizm?"}

In his book, Vucinich writes that Strakhov's crusade against Darwinism was based on his belief in a "supreme intelligence": "Darwinism was a pseudoscience and a materialistic conspiracy, for it eliminated the role of divine powers in maintaining the regularity of natural processes."\footnote{Vucinich, \emph{Darwin in Russian Thought}, 102.} Though Vucinich is correct that Strakhov's underlying criticism of Darwin was rooted in his belief in the divine, Strakhov's critique was not dogmatically theological. Strakhov rejects Darwinism because a worldview that precludes the transcendent does not even remotely explain man, a uniquely spiritual being. What Strakhov perceives to be Darwinian naturalism is anti-rational, whereas his organic understanding of the world as a whole is a more nuanced, truthful and fuller account of reality. Strakhov's critique was both rational and theistic. In \emph{The World as Whole}, Strakhov writes, "For me, there is no doubt that men of science, pure researchers, not allowing into their work any kind of interference from fantasy or feeling, should without condition recognize the world as whole. This view alone corresponds to the full rigor of the scientific method."\footnote{Strakhov, \emph{Mir kak tseloe}, Predislovie k pervomu izdaniju, 68.} This is substantiated by the Hegelian notion that the truth of an individual feature will always lead one closer to the truth of the whole.

Moreover, it is noteworthy that Strakhov believes a pursuit of rationality, hence of truth, is identical to a pursuit of the Absolute. We can again see the ways in which Strakhov's thinking is indebted to Hegel. Strakhov writes in "A Complete Refutation," "The presence of reason means the presence of the spiritual, divine principle; therefore, rising into this area, we ascend to the very source of our being and knowledge." He also writes that scientific research, when done on the basis of reason and, therefore, from the "correct" organicist position, will lead man to "true teleology," where God is to be sought.\footnote{Strakhov, "Polnoe oproverzhenie darvinizma."} In short, a belief in a supreme being was inherent to Strakhov's organicism. In \emph{The World as Whole}, Strakhov writes, "a unity of the world can be obtained only by spiritualizing nature, recognizing that the true essence of things consists in various degrees of the incarnating spirit {[}\emph{воплощающегося духа}{]}."

In late 1858, Strakhov asked why the natural sciences had gripped the public to such an extent. Though he believed that the rise of materialism and nihilism were at the root of the sudden shift in perspective, he wrote that the overwhelming reason was man's desire to eliminate all doubt, to know for the sake of knowledge itself {[}\emph{знать\textemdash для одного знания}{]}. Strakhov writes, "{[}W{]}e strive to solve the riddle of being, to comprehend the essence of the world among which we are placed and of which we ourselves are a member, to remove the veil from the mysterious and formidable Isis."\footnote{The metaphor of Isis's veil, based upon the statue located in the Egyptian city of Sais, was a common motif in counter-Enlightenment thinking and used to warn of the dangers of attempting to reveal all of nature's secrets. Strakhov here likely has in mind Friedrich Schiller's (1759\textendash 1805) 1795 balled "The Veiled Image at Sais" ("Das verschleierte Bild zu Saïs"), in which "organicist" themes on the wholeness of Truth are explored.} However, natural science has not disentangled the secrets of nature, since it has refused to recognize the supremacy and the mystery of man, that man is the "embodied ideal of animal life," the ultimate goal of the animal kingdom.\footnote{Strakhov, \emph{Mir kak tseloe}, Part I, Chapter V, 250.} Strakhov ultimately rejected the supremacy of empiricism, writing in 1871 that, though empiricism has an "invaluable" quality, it cannot satisfy man's theoretical requirements.\footnote{Nikolai Strakhov, "O chisto-jempiricheskom metode" in \emph{Filosofskie ocherki} (St.~Petersburg, 1893).} Empiricism is insufficient because essences exist which are inaccessible to the human eye, and the internal teleology is dictated by these essences. Strakhov, however, is not simply reinstating the pre-Darwinian, essentialist point of view, considering the fact that he accepts the notion of the variability of species; man's highest aspiration is to adapt, change, and evolve from a lower form into a higher.

More than anything, Strakhov disliked mixing fields of research. A theologian's attempt to comment on natural science was an insult not to natural science but precisely to theology. A theologian should respect his field enough to know that the sciences cannot substantiate belief. As one of the earliest critics of Social Darwinism, Strakhov believes that one cannot draw social and moral conclusions from the discoveries of the natural sciences.\footnote{Gerstein, 159.} Ultimately, Strakhov believes that man transcends the object of knowledge that science claims for itself.\footnote{Strakhov, "Vsegdashnjaja oshibka darvinistov."} Strakhov teaches us that science can only provide man with limited truth. According to Strakhov and the organicist perspective, man is a member of the material world yet also is the center, the "knot" of the universe. A rational understanding of the world inevitably leads man to the recognition of himself as a cosmic being, a being who is not a mere accumulation of changes prompted by external stimuli, but a being motivated by an internal impulse; man is immeasurably higher than nature.

\begin{center}
  \includegraphics[width=0.75cm]{articlend.png}
\end{center}

\biobox{\textbf{Jillian Pignataro} is a PhD candidate in the
department of Slavic Languages and Literatures at
Northwestern University.~Her general research
interests include 19th-century Russian intellectual history, Russian
conservative and religious thought, organic aesthetics,
and the influence of German Idealism on Russian philosophy.
She is currently writing her dissertation on the 19th-century
Russian philosopher, Nikolai Strakhov, and the movement of
Russian Organicism. }

\label{sec:pignataro}

\newpage

\section{Julia Berest - J.S. Mill Logic}

\fancypagestyle{chaptercontentpage}{
  \fancyhf{} % Clear all header and footer fields
\fancyhead[CE]{%
  \fontsize{11}{11}\leftmarkfont%
  \addfontfeature{LetterSpace=10.0}%
  \textit{\MakeUppercase{\leftmark}}%
}
  \fancyhead[CO]{\authorheadfont\addfontfeature{LetterSpace=10.0}\fontsize{11}{11}\selectfont\textbf{{\uppercase{Julia Berest}}}}
  \renewcommand{\headrulewidth}{0pt} % No header rule on content pages
  \fancyfoot[RE]{\thepage}
  \fancyfoot[LO]{\thepage}
}

\fancypagestyle{chaptertitlepage}{
  \fancyhf{} % Clear all header and footer fields
  \fancyhead[L]{\begin{minipage}[t]{0.7\textwidth}\publisher\end{minipage}}
  \fancyhead[R]{\begin{minipage}[t]{\textwidth}\raggedleft \datefont\fontsize{10}{11}\selectfont Volume 1 (2024): \thepage\textendash\pageref{sec:berest} \\ \doi{10.71521/88mj-2b75} \end{minipage}}
  \renewcommand{\headrulewidth}{0pt} % No header rule on title pages
  \fancyfoot[RE]{\thepage}
  \fancyfoot[LO]{\thepage}
}

\newpage
\abstractbox{J.S. Mill's \textit{System of Logic} in Russia}{The Debate on Empiricism in Russian Philosophy, 1860s\textendash 1890s}{Julia Berest}{
\noindent J.S. Mill’s \textit{System of Logic} (1843)—the book that brought him both fame and controversy in Britain—reached Russian readers during the pivotal decade of the 1860s when Russia experienced an unexpected jolt into scientific modernity which came with a renewed influx of Western ideas. As a work that received recognition in Europe for its contribution to the development of scientific methods and the theory of knowledge, the \textit{Logic} found an enthusiastic reception among the Russian intelligentsia who embraced empirical science and its attendant skepticism towards religion. For their opponents on a conservative side, Mill’s naturalist explanation of the world and of human cognition epitomized the positivist worldview which they held responsible for an erosion of religious foundations of life and knowledge. Such contrasting reactions to Mill’s empiricism in Russia reflected the same ideological divide that emerged in Europe around the question of empirical science. However, many Russian critics also exhibited a Slavophile mindset, seeing Mill as a symptom of Western spiritual decadence antithetical and pernicious to native Russian culture. Towards the end of the century, the European debate on empiricism began to show signs of cultural compromise between the two sides, whereas in Russia the divide remained sharp and focused on the question of Russia’s response to Western influences.}{Russian philosophy, science, religion, J.S. Mill, debate on empiricism and logic, Westernizers, Slavophiles}


\chaptertitle{J.S. Mill's \textit{System of Logic} in Russia} {The Debate on Empiricism in Russian Philosophy, 1860s\textendash 1890s}{Julia Berest}

\addcontentsline{toc}{chapter}{J.S. Mill's \textit{System of Logic} in Russia\\The Debate on Empiricism in Russian Philosophy\\\textit{by} Julia Berest}

\setcounter{footnote}{0}
\seriffont
\fontsize{12}{18}\selectfont

\noindent John Stuart Mill (1806\textendash 1873) became known to Russian readers in the 1860s, two decades after his fame rose in Britain following the publication of his first book-length work, \emph{A System of Logic} (1843). A controversial figure from the start owing to his liberal views in politics and utilitarian stance in ethics, Mill's philosophy of radical empiricism was bound to attract attention in Russia, no less than in his home country. For the Russian educated public, the 1860s was a time of unprecedented intellectual revival after thirty years of Nicholas I's reactionary rule and isolationist policies which impeded the development of philosophical studies in Russia. With the start of liberal reforms under Alexander II, it became possible for Russian readers to catch up on the latest Western publications in the field of philosophy and science during the crucial years when the pace of scientific modernization and its attendant cultural change began to intensify in the West.

Once Mill was discovered in Russia, his popularity and influence remained strong up until the end of the tsarist period (and was especially notable throughout the 1860s\textendash 90s), as evidenced by the number of Russian editions of his works and the amount of commentary they generated on both sides of the ideological spectrum.\footnote{See J. Berest, "The Reception of J.S. Mill's Feminist Thought in Imperial Russia," \textit{Russian History} 43 (2016): 101\textendash 41; Julia Berest, "J. S. Mill's \textit{Principles of Political Economy} in Imperial Russia: Publication and Reception," \textit{Modern Intellectual History} 14 (2017): 67\textendash 97; J. Berest, "John Stuart Mill and his \textit{Autobiography} in Imperial Russia," \textit{Journal of Modern Russian History and Historiography [JMRHH]} 10 (2017): 28\textendash 70; J. Berest, "J. S. Mill's \textit{On Liberty} in Russia: Modernity and Democracy in Focus," \textit{Slavonic and East European Review} 97 (2019): 266\textendash 98; J. Berest, "The Theme of Happiness and British Utilitarianism in Russian Thought, 1860s\textendash 80s," \emph{JMRHH} 14 (2021): 5\textendash 68; J. Berest, "Scientific Modernity vs.~Cultural Tradition: N. N. Strakhov, F. M. Dostoevskii and J. S. Mill's \emph{A System of Logic}," \emph{JMRHH} 16 (2023): 5\textendash 49.} \emph{The Principles of Political Economy}, for instance, was published seven times, in addition to a serialized publication in the journal \emph{Sovremennik}, which introduced the book to the Russian public in 1860; \emph{The Subjection of Women} underwent six editions, becoming staple reading for Russian feminists and, by necessity, for their opponents who had to familiarize themselves with the most famous European champion of women's rights; the \emph{Logic} too gained long-term popularity and was published in six editions between 1865 and 1914.\footnote{Dzhon Stuart Mill, \emph{Sistema Logiki}, trans. P. L. Lavrov (St Petersburg: Tip. M. Vol'fa, 1865; 2nd ed.~1878); Dzh. S. Mill, \emph{Sistema logiki sillogicheskoi i induktivnoi} (Moscow: I. N. Kushnerev, 1897\textendash 98); Dzh. S. Mill, \emph{Sistema logiki}, trans. V. N. Ivanovskii (Moscow: Knizhnoe delo, 1900; 2nd ed.~1914). There was also an abridged edition for novice readers published in 1897. See below.} The fact that the first translations of Mill's major works were published by radical and left-leaning writers\textemdash N. G. Chernyshevskii, N. K. Mikhailovskii, P. L. Lavrov and G. E. Blagosvetlov\textemdash was indicative of the reputation that Mill acquired in Russia. He was commonly viewed as a philosopher whose works had practical value and could be useful for Russia's renewed effort at modernization. The \emph{Logic} was no exception despite its recondite language and subject matter seemingly removed from day-to-day concerns.

Unlike logic as we know it today\textemdash a highly specialized discipline which holds more use for computer scientists than philosophers\textemdash logic in the nineteenth century enjoyed wider popularity and was defined more broadly, as a foundational discipline designed to improve one's reasoning abilities and the art of communication. In early nineteenth century Britain, logic was deemed a necessary part of the liberal education popular with the upper classes.\footnote{James W. Allard, "Early Nineteenth-Century Logic," in W. J. Mander, ed., \emph{The Oxford Handbook of British Philosophy in the Nineteenth Century} (Oxford: Oxford University Press, 2014), 29.} Its practical significance rose even further when logic was transformed, later in the century, into "the science of science itself," as Mill called it, with a subject matter that now included the methods of scientific inquiry and the theory of knowledge.\footnote{Quoted in Antis Loizides, "Introduction," in A. Loizides, ed., \emph{Mill's A System of Logic: Critical Appraisals} (New York: Taylor and Francis Group, 2014), 15. See also W. J. Mander, "Introduction," in \emph{The Oxford Handbook of British Philosophy,} 2\textendash 5.} Mill was a major figure who affected this momentous change by developing the inductive logic (as opposed to the traditional logic of the syllogism) in an effort to improve the methodological apparatus of empirical sciences. The latter, in his view, included not only the natural sciences but also the sciences that study society (or the social sciences, in today's terminology)\textemdash an unconventional position in Mill's time, when the nature, methods and scope of science were still the subject of debate.\footnote{D. Cobb, "Mill's Philosophy of Science," in Christopher Macleod and Dale E. Miller, eds., \emph{A Companion to Mill} (Southern Gate, UK: Wiley Blackwell, 2017), 234; Loizides, "Introduction," 1, 16.} Although Mill eschewed any explicit commentary on religion in the \emph{Logic}, he was well aware that his book, which adhered to the principle of a naturalist explanation of the world, implicitly undermined the foundations of the religious worldview and was later used by some of his followers in Britain to challenge theological orthodoxies.\footnote{See, James Livingston, "Sceptical Challenges to Faith," in Thomas Baldwin, ed., \emph{The Cambridge History of Philosophy, 1870\textendash 1945} (Cambridge: Cambridge University Press, 2012), 323\textendash 4; Richard Reeves, \emph{John Stuart Mill}: \emph{Victorian Firebrand} (London: Atlantic Books, 2007), 167\textendash 8.}

By 1865, when the \emph{Logic} became available in Russian translation, Mill was at the peak of his fame in Britain and the practical impact of his "uncompromising empiricism"\footnote{W. J. Mander, \emph{A Study in Nineteenth-Century British Metaphysics} (Oxford: Oxford University Press, 2020), 125.} was in full display, pointing to the ambivalent intellectual atmosphere in which Mill's ideas operated. On the one hand, it was widely recognized that Mill's methods of scientific inquiry helped to move the natural sciences forward, including the field of medicine.\footnote{Loizides, "Introduction," 25; Reeves, \emph{John Stuart Mill},168.} On the other hand, Mill's epistemological challenge to religion was one of the factors that plunged Britain into "the crisis of faith" deeply unsettling to those of his countrymen who worried about the diminishing role of religion in a world increasingly defined by science.\footnote{W. J. Mander, "Introduction," in \emph{The Oxford Handbook of British Philosophy,} 17; James Allard, "Idealism in Britain and the United States," in Thomas Baldwin, ed., \emph{The Cambridge History of Philosophy, 1870\textendash 1945} (Cambridge: Cambridge University Press, 2012), 45.} It was in reaction to the dominance of Mill's empiricism in British academia that a strong Idealist movement arose in Britain in the 1870s, (so unusual for a culture steeped in empiricist tradition\footnote{W. J. Mander, "Hegel's Thought in Europe," in Lisa Herzog, ed., \emph{Hegel's Thought in Europe} (Palgrave Macmillan: New York, 2013), 168.}), as a way of addressing what one Victorian commentator called "the most anxious thought of our time."\footnote{Quoted in Allard, "Idealism in Britain," 44.}

In Russia, where the discipline of logic itself was relatively new but growing in popularity, Mill's book became a staple of university courses by the early 1880s, generating a significant amount of commentary in specialized monographs and textbooks on logic and psychology (which was part of philosophy curriculum at that time). As this essay will show, similarly to Mill's reception in his home country, the \emph{Logic} in Russia became a subject of debate between those who subscribed to Mill's empiricism (often with great enthusiasm) and their opponents, who viewed empirical science as inadequate or fundamentally flawed in its explanation of the world and of the way we cognize it. Given the scope of the topic and the scarcity of studies on the history of logic and epistemological discourse in Russian philosophy,\footnote{There are no monographic studies on the history of Russian logic in Western (English-language) scholarship. The only specialized work available is V. A. Bazhanov, \emph{Istoriia logiki v Rossii} (Moscow: Kanon, 2003). On the questions of epistemology in Russian philosophy, see Thomas Nemeth, \emph{Kant in Russia} (Springer, 2017) and Nemeth, \emph{Philosophy in Imperial Russia's Theological Academies} (De Gruyter, 2023)} this examination will focus on select thinkers whose engagement with Mill's book illustrates the contrasting reactions that his empiricism engendered in Russia. For the majority of them, Mill's conception of causality was the main subject of interest, as it involved broader philosophical questions concerning the origin of the world and the nature of human cognition.

Examining the reception of Mill's empiricism in Russia sheds light on epistemological debates that arose in Russian philosophy with the (abrupt) arrival of modern science in the 1860s. In a recent study of Russian theological academies, Thomas Nemeth has argued that philosophy instructors at the academies proved ill-prepared to meet the challenge of modernity due to the isolationist and dogmatic stance adopted by the Orthodox Church as way of warding off "the pernicious 'subjectivism' of modern Western philosophy." "Whereas in the West the Roman Church confronted the rise of the empirical science and thereby the role of reason in affirming the religious tenets, \ldots{} similar conflicts were largely absent from the Orthodox Church in Russia."\footnote{Nemeth, \emph{Philosophy in Imperial Russia's Theological Academies}, xiii.} It is instructive to see how secular Russian philosophers responded to the same challenge. Although their reaction to Mill's \emph{Logic} had much in common with the reception of his work in Britain, the specific intellectual context to which his ideas were transplanted\textemdash the presence of censorship, the belatedness with which empiricism reached Russia, as well as the impact of anti-Western sentiments emanating from conservative circles\textemdash inevitably influenced the nature of the debate. The story of Mill's reception in Russia is best understood against the background of European intellectual developments that reveal the historical significance of Mill's work but also the inherently controversial nature of the questions involved in the debate on scientific modernity.

\subsection*{\textit{A System of Logic} in Britain: \\ Transformative Influence and Controversy}


\noindent \emph{A System of Logic} was the work that brought Mill widespread recognition not only in academic circles, but also, to his surprise, among the wider British audience. A six-part treatise with a long and rather formidable title\textemdash \emph{A System of Logic, Ratiocinative and Inductive, Being a Connected View of the Principles of Evidence and the Methods of Scientific Investigation}\textemdash this massive study was not an easy read, as Mill himself acknowledged. "I don't suppose people will read anything so scholastic," he wrote shortly after the \emph{Logic's} publication.\footnote{Quoted in Loizides, "Introduction," 5.} It was a pleasant surprise when the book turned out to be a great success. Looking back on those years, Mill wrote in his \emph{Autobiography} (1873): "How the book came to have, for a work of its kind, so much success, I have never thoroughly understood."\footnote{J. S. Mill, \emph{Autobiography} (London: Oxford University Press, 1924), 190.} On this, Mill's recent biographer commented, with a touch of humor, that "the \emph{Logic} was one of those works which became a vital addition to the bookshelves of all self-respecting educated households."\footnote{Richard Reeves, \emph{John Stuart Mill: Victorian Firebrand} (London: Atlantic Books, 2007), 163.}

Within Mill's lifetime, the \emph{Logic} went through eight editions, contributing to the revival of academic logic in Britain and establishing itself as "a decisive work in the historical development of the philosophy of science," according to modern assessments.\footnote{Cobb, "Mill's Philosophy of Science," 234; See also, Loizides, "Introduction," 1, 16.} The growing interest of the general public in the subject of logic was noted by many contemporaries: "Not a month passes which does not bring us new publications on logic," wrote a British commentator in 1854.\footnote{Quoted in ibid., 16.} One of the reasons for that was the role that logic came to play in the development of modern science and philosophy of science. With his focus on methods of scientific investigation, Mill was almost guaranteed to find a large audience in Britain, a country where the scientific revolution made the biggest advancements in his lifetime.

The foundation stone of Mill's philosophy of science was the idea that "all knowledge" "consists in generalization from experience." For Mill, "there is no knowledge \emph{a priori}; no truth cognizable by the mind's inward light, and grounded on intuitive evidence."\footnote{Quoted in R. F. McRae, "Introduction," in John Stuart Mill, \emph{A System of Logic Ratiocinative and Inductive. Collected Works}, 33 vols. (Toronto: University of Toronto Press, 1974), VII: xxii; see also Christopher Macleod, "Mill on Epistemology of Reasons: A Comparison with Kant," in \emph{Mill's a System of Logic: Critical Appraisals}, 152.} Sense perceptions supply the "original data," which are then processed through inductive reasoning\textemdash a mode of reasoning that, according to Mill, is the most natural to human beings, something that we employ spontaneously.\footnote{David Godden, "Mill on Logic," in \emph{A Companion to Mill}, 185.} Long before science emerged, mankind was capable of applying primitive inductions to observed phenomena, thereby learning of causal connections between phenomena. In this manner, early humans discovered that "food nourishes, that water drowns, or quenches thirst, that the sun gives light and heat."\footnote{J. S. Mill, \emph{A System of Logic} (London: Longmans, 1882), 1: 392.} These primitive generalisations laid the foundation for more complex scientific induction. Thus, sense impressions constitute the raw material out of which the mind fashions both empirical generalizations and abstract concepts.

In a move that earned Mill the appellation of "radical" empiricist, he famously applied this principle of cognition to explain the nature of mathematical knowledge, going against the widely shared view that mathematics and geometry are \emph{a priori} sciences.\footnote{See W. J. Mander, \emph{The Unknowable: A Study in the Nineteenth-Century British Metaphysics} (Oxford: Oxford University press, 2020), 110.} He wrote: "it is customary to say that the points, lines, circles, and squares which are the subject of geometry, exist in our conceptions merely \ldots{} {[}that{]} minds, by working on their own materials, construct an \emph{a priori} science \ldots{} which is purely mental, and has nothing whatever to do with outward experience." For Mill, in contrast, geometric definitions reflect real facts about objects amenable to sense perception: "Our idea of a point, I apprehend to be simply our idea of the \emph{minimum visible}, the smallest portion of surface which we can see."\footnote{Mill, \emph{A System of Logic}, 1: 280.}

Since knowledge, according to Mill, ultimately derives from experience, the limits of the knowable are set by the limits of our senses and cognitive abilities as humans. "Of nature, or anything whatever external to ourselves, we know \ldots{} nothing, except the facts which present themselves to our senses, and such other facts as may, by analogy, be inferred from these."\footnote{Quoted in Rae, "Introduction," xxii.} Mill, therefore, acknowledged that there might be things in the world simply inaccessible to human faculties but he did not view this limitation as a serious obstacle to the progress of knowledge.\footnote{Ibid., xxii; Mander, \emph{The Unknowable,} 108.} As W. J. Mander aptly put it, Mill's key epistemological claim amounted to a simple proposition: "empiricist knowledge is all we have, but it is also all we need."\footnote{Ibid, 105.}

At a time when the nature and scope of science was a contestable subject, Mill's commitment to pure empiricism entailed a crucial argument that the goal of science is to explain the world naturalistically, through the discovery of causal laws that operate in nature.\footnote{Cobb, "Mill's Philosophy of Science," 236; William Stafford, \emph{John Stuart Mill} (New York: St.~Martin's Press, 1998), 56.} "An individual fact is said to be explained," Mill wrote, "by pointing out its cause, that is by stating the law of causation" and by "cause" he meant nothing more than a "physical cause" "in that sense alone in which one physical fact is said to be the cause of another."\footnote{Quoted in Cobb, "Mill's Philosophy of Science," 236.} By contrast, Mill's major opponent, William Whewell, maintained that causation involves more than merely an unconditional succession of phenomena in which the cause is a set of invariable antecedent conditions; rather, causation requires an agent that has productive power. Thus, in keeping with the more traditional understanding of the aims of scientistic inquiry, Whewell argued that science ultimately has to lead to an explanation of the First Causes of things. This could be achieved through a combination of inductive reasoning and what he called "a colligatory inference\textemdash an inference uniting diverse observed phenomena according to an \emph{a priori} conception." The concept of causality itself, according to Whewell, is not an inference from experience but an \emph{a priori} idea that connects the human mind to that of God.\footnote{Quoted in ibid., 244\textendash 5. See also, Geoffrey Scarre, "Mill on Induction and Scientific Method," in John Skorupski, ed.~\emph{The Cambridge Companion to Mill} (Cambridge: Cambridge University Press, 1998), 115.} Mill, however, was convinced that the theory of \emph{a priori} knowledge (which he more commonly called "intuitivist") lacked objectivity, operating with mental conceptions that may not be amenable to independent verification. From Mill's standpoint, abstract conception is something implicit in the facts themselves, so the mind only discovers rather than constructs it: "there is in the facts themselves something of which the conception itself is a copy."\footnote{Mill, \emph{CW}, VII: 296. See also David Godden, "Mill on Logic," 184.}

Based on his commitment to the empiricist theory of knowledge, Mill formulated the structure of scientific inquiry as a three-stage process: inductive reasoning (which includes five methods of causal investigation), ratiocination (or syllogistic reasoning) and verification. This was the most famous part of the book.\footnote{Cobb, "Mill's Philosophy of Science," 239.} Although he drew on the insights of the English empiricist school, his account of the methods of induction was original enough for some of his contemporaries to view it as "a kind of revolution in the science of logic."\footnote{Quoted in Loizides, "Introduction," 19. Of the earlier methodological explorations, David Hume's "Rules by which to judge causes and effects" was the most well-known, but as Richard Fumerton notes, "Mill's statement of the methods is perhaps the clearest, most comprehensive and certainly, most influential." See R. Fumerton "Mill's Epistemology," in \emph{A Companion to Mill}, 202.} In the debate between "the defenders of the syllogistic theory and \ldots{} its assailants," as Mill put it\textemdash a controversy of much significance at a time when syllogistic logic (invented by Aristotle) continued to dominate the academic curriculum\textemdash Mill assumed a middle-ground position by arguing that inductive, rather than deductive, reasoning is a major tool of knowledge acquisition while syllogism remains useful as a form of reasoning that explicates inductive generalizations. Some sciences, according to Mill, have greater reliance on the deductive method because of the problem of possible multiple causes and difficulty conducting controlled experiments and observations. Political economy, for instance, deduces its laws from the foundational principles supplied by experimental sciences, such as psychology, which postulates the universal principle of self-interest in economic behavior.\footnote{Allard, "Early Nineteenth-century Logic," 26\textendash 8; Scarre, "Mill on Induction," 112; Godden, "Mill on Logic," 180\textendash 1.}

Despite Mill's reluctance to comment on matters of religion in the \emph{Logic}, his book became, in the words of James Livingston, a "principal guide" to Victorian critics of religious tradition.\footnote{James Livingston, "Sceptical Challenges to Faith," in \emph{The Cambridge History of Philosophy}, 323.} More than anything, it was Mill's general epistemological position\textemdash the idea that any belief requires sufficient evidence and critical inquiry\textemdash that his empiricist disciples in England employed in their attempts to challenge established theological doctrines.\footnote{Ibid., 324.} Mill was well aware of the \emph{Logic's} implicit anti-theological message, and when advising his young followers on how to approach the book, he wrote half-jokingly: "whoever does read any of it must know that she does it at her own risk and responsibility."\footnote{Quoted in Reeves, \emph{John Stuart Mill,} 168.}

The \emph{Logic's} controversial reputation also had to do with the fact that Mill declared in it his support for August Comte's positivist principles which elevated the scientific mode of inquiry to the level of preeminence (over religious or metaphysical thinking) in all branches of human knowledge, including sociology\textemdash a term that Comte invented. In later years, Mill developed fundamental disagreements with Comte over the latter's socio-economic and anti-feminist ideas which appeared dangerously authoritarian to Mill-the liberal reformer. However, his commitment to positivism as a methodological stance remained unchanged.\footnote{Michael Singer, \emph{The Legacy of Positivism} (New York: Palgrave Macmillan, 2005), 51\textendash 74. Nicholas Capaldi\emph{, John Stuart Mill: A Biography (}Cambridge: Cambridge University press, 2004), 171\textendash 75.}

Already by the end of the 1860s, Mill's book dominated British universities to such a degree that some contemporaries found its influence overpowering and exclusionary. John Mozley, a lecturer of history and literature at King's College wrote in 1872: "{[}W{]}e must lament that one of the most useful and distinguished of English universities, the university of London, should have almost formally excluded from the examinations any other philosophy than that of Mr.~Mill and Mr.~Bain {[}Mill's follower and protegee{]}."\footnote{Quoted in Loizides, "Introduction," 24.} By contrast, others credited Mill with bringing about "quite a revolution in the education of Oxford," the highest-ranking university in Britain, which, according to this assessment, was finally in the process of relinquishing "its old ecclesiastical conservatism," owing to Mill's influence, among other factors.\footnote{Quoted in ibid., 24\textendash 5.} Mill's own opinion of the book, as expressed in a letter to a friend, was that it gave him "a certain capital," i.e.~an intellectual capital which he intended to use for promoting other, more radical ideas: "I fully expect to offend and scandalize ten times as many people as I shall please," he wrote.\footnote{J. S. Mill, \emph{The Earlier Letters, 1812\textendash 1842}, in \emph{Collected Works}, XIII: 708\textendash 9.} In this case, he specifically had in mind his \emph{Principles of Political Economy} which he was writing at that time. Apparently, the \emph{Logic's} ideas were not that scandalous by Mill's standards.

\subsection*{Logic as a Philosophic Discipline in Russian Universities}

\noindent Mill's works were slow to reach the Russian readers. In the 1840s and 50s, when Mill gained recognition in Britain for his \emph{Logic} and \emph{The Principles of Political Economy} (1848), Russian intellectual life languished under the iron rule of Nicholas I, who sought to restrict intellectual contact with Western Europe as much as possible. The government Instruction required that logic, as well as other philosophic disciplines (including psychology), be taught according to the principle that philosophic truths were only "relative" compared to the "absolute" truths preserved in Christian religion.\footnote{Quoted in Pustarnakov, \emph{Universitetskaia filosofiia v Rossii} (St.~Petersburg: Izd-vo Russkogo Khristianskogo Gumanitarnogo Instituta, 2003), 112; see also, V. A. Bazhanov, \emph{Istoriia logiki,} 21.} In this restrictive atmosphere, the quality of instruction suffered and it was often difficult to find suitable instructors for courses in logic. Boris Chicherin left in his memoirs a vivid description of how he and his classmates at Moscow University felt about their course on logic taught by Mikhail Katkov. Later known for his journalism and a memorable flip from liberal sympathies to conservatism, Katkov was appointed to teach logic in 1845 even though his academic training was in philology rather than philosophy. Chicherin, a very diligent and talented student, remembered Katkov's course with utmost puzzlement:

\begin{quote}
I have never seen anything like this at the university. I have, on occasion, had courses that were vulgar, stupid, empty, but I have never had a course in which nobody could understand anything. And this was not something unusual, accidental. Kat\-kov was teaching this course for a second year in a row. In the previous class \ldots{} nobody could understand a word of what the professor was saying \ldots{} I diligently went to each lecture, took notes in the most assiduous manner but decidedly did not understand anything, and all my classmates were in the exact same position.\footnote{B. N. Chicherin, \emph{Vospominaniia,} 2 vols. (Moscow: Izd-vo Sabashnikovykh, 2010), 1: 177.}
\end{quote}

Government interference in academic life grew even more oppressive after the European revolutions of 1848\textendash 49. In 1850, the importation of books was severely curtailed, Russian students were forbidden to study abroad and, as an additional precautionary measure, the teaching of philosophy in Russian universities (already much limited) was abolished altogether. In the memorable words of the Minister of Education Count Shirinskii-Shakhmatov, "the benefits of philosophy are not proven but the harm from it \emph{is} possible (\emph{pol'za ot filosofii ne dokazana, a vred ot nee vozmozhen})."\footnote{Quoted in Pustarnakov, \emph{Universitetskaia filosofiia}, 157.} The courses on logic survived, but their teaching was now entrusted to professors of theology with special supervision from the Department of Spiritual Affairs.\footnote{Bazhanov, \emph{Istoriia logiki}, 25.} Since qualified instructors were hard to find, logic as an academic subject, fell into "the most pitiable state," according to contemporary accounts.\footnote{Quoted in ibid., 32.} Alexander Skabichevski noted in his memoirs that the atmosphere of demoralization, which reigned in academia in those years, was the reason for many students losing all interest in serious study and abandoning themselves to a life of "epicureanism." The authorities, according to Skabichevski, did not only look the other way, but "even encouraged" this sort of behavior in students "if only to get {[}their minds{]} off politics" ("\emph{lish' by ne kasalis' politiki}").\footnote{A. M. Skabichevskii, \emph{Literaturnye vospominaniia} (Moscow: Agraf, 2001), 111, 75.}

The situation began to change after Russia's disastrous defeat in the Crimean War (1853\textendash 56) which compelled the new tsar Alexander II to embark upon a series of modernization reforms accompanied by a relaxation of censorship. Many contemporaries on the intelligentsia side described the decade that followed as a time of unprecedented intellectual fervor and optimism, comparing it to "the spring of renewal" after "the endlessly long and harsh winter" of the Nicholaevan years.\footnote{A. P. Pypin, "Belletrist narodnik shestidesiatykh godov. Sobranie sochinenii A. I. Levitova," \emph{Vestnik Evropy} (August 1884): 649. See also Skabichevskii, \emph{Literaturnye} \emph{Vospominaniia}, 125; N. V. Shelgunov, L. P. Shelgunova, M. L. Mikhailov, \emph{Vospominaniia}, 2 vols. (Moscow: Khudozhestvennaia literatura, 1967), 1: 92\textendash 4.} The sudden influx of Western European literature was extremely exciting for the intelligentsia but deeply concerning for their conservative opponents. Imported publications were under the purview of the Foreign Censorship Committee, then chaired by the highly educated and accomplished poet Fedor Tuitchev, who became known for his "liberal" views on censorship. In 1867 he wrote in an official letter to his superior: "if there is one truth \ldots{} which has emerged \ldots{} from the harsh experience of recent years, it is surely this: \ldots{} a too absolute, too prolonged repression cannot be imposed on intellects without resulting in serious damage to the whole organism."\footnote{Quoted in M. Choldin, \emph{A Fence Around the Empire} (Durham: Duke University Press, 1985), 58.} It was under Tiutchev that the first editions of Mill's books were published in Russia, some of them after heated debates in the Committee and with considerable excisions.\footnote{A. Nikitenko, \emph{Dnevnik}, 3 vols. (Moscow, 1955), 628, n.~307; Berest, "John Stuart Mill and His \emph{Autobiography} in Russia," 39\textendash 42; Choldin, \emph{A Fence Around the Empire}, 56.} The disagreements within the Committee reflected the deepening divide in the Russian educated public over the question of intellectual borrowings from the West.\footnote{I am grateful to Susan McReynolds for her insightful questions to the earlier draft of this paper which led me to develop the sections below.}

With the reopening of philosophy chairs in December of 1859 and the new university statute of 1863, academic philosophy in Russia experienced a freer and more vibrant phase of its history than ever before. The statute made it possible for universities to attract a new generation of instructors, some of whom received additional training in Germany and were eager to reignite students' interest in philosophical studies. As we shall see, one of these newcomers, the professor of logic and psychology at Moscow University, Matvei Troitskii (1835\textendash 1899) was an enthusiastic follower of Mill who succeeded in attracting crowds of students to his lectures. However, it is important to note that despite the relaxation of censorship and more liberal university statute, the government continued to keep an eye on courses in philosophy and natural sciences in an effort to guard the traditional religious teaching from the encroachment of materialistic ideas.\footnote{See Alexander Vuchinich, \emph{Science in Russian Culture, 1861\textendash 1917} (Stanford: Stanford University Press, 1970), 56, 62; Daniel P. Todes, \emph{Ivan Pavlov: A Russian Life in Science} (Oxford: Oxford University Press, 2014), 22.}

Compared to the profound epistemological changes and revitalization that the field of logic underwent in Britain beginning in the 1820s (when the empiricist school supplanted the traditional Aristotelian logic in British universities), in Russia the revival of academic logic was belated and modest, with few changes in later decades. British logicians went on to develop mathematical logic which led to the emergence of computer science in the twentieth century.\footnote{See Peter Simons, "Logic: Revival and Reform," in \emph{The Cambridge History of Philosophy}, 119.} Although Mill's emphasis on inductive logic put him in the camp that stood in opposition to mathematical treatments of logic (which derived from syllogistic reasoning), his famous dispute with Sir William Hamilton, a proponent of the logic of quantification, contributed to the mutually simulating dynamic of ideas in British thought that generated innovations in logic and the philosophy of science. The same can be said of the controversy that developed between Mill and Whewell over the method of concept building in the natural sciences.\footnote{Allard, "Early Nineteenth-century Logic," 31, 37; Elliah Millgram, "Mill's and Whewell's Competing Visions of Logic," in \emph{Mill's A System of Logic: Critical Appraisals}, 101\textendash 2.} In Russia, the process of catching up to Western developments in logic and epistemology continued to lag behind. By the time Mill came to occupy a prominent place in Russian academic philosophy, empiricism had already been thoroughly challenged (though by no means replaced) by idealist philosophy in Britain.\footnote{See W. J. Mander, "Hegel's Thought in Europe," 168; David Godden, "Mill's \emph{System of Logic}," in \emph{The Oxford Handbook of British Philosophy}, 66\textendash 7; Philip Perreira, "Idealist Logic," in \emph{The Oxford Handbook of British Philosophy}, 115.} The belatedness of Russian philosophical revival is a factor that should be kept in mind as we assess the Russian reception of Mill's \emph{Logic}.

It is also important to remember that Russian secular philosophy received little stimulus from its theological counterpart at the Ecclesiastic academies. Unlike the secular institutions of higher education, the theological academies were allowed to keep philosophy courses in their curriculum throughout the 1850s, but their freedom of expression remained severely constrained. Thomas Nemeth who recently examined the philosophical output of Russian theologians in the nineteenth century has noted that their publications were marked by "\emph{intended} unoriginality" as one of the consequences of the Orthodox Church's determination to protect Russian thought from the "corrupting influence of modernism and Western Christianity."\footnote{Thomas Nemeth, \emph{Philosophy in Imperial Russia's Theological Academies} (De Gruyter, 2023), X.} "We find within the academies," Nemeth writes, "an incessant distancing of their philosophies from modern Western thought. \ldots{} All philosophical writings emanating from there as a rule had to be harshly critical of Western ideas even while explicitly absorbing many of them."\footnote{Ibid., xi.} When some theologians did attempt to step out of the prescribed conceptual frameworks, the result was usually negative for their teaching career.\footnote{Ibid., 39, 182, 252.}

The works analyzed by Nemeth suggest one more striking feature regarding the philosophical position adopted by the academies: While in Britain the ideas of Kant and Hegel served as a major source of inspiration for philosophers who sought to defend the religious worldview from the impact of empiricist thinking,\footnote{See James Allard, "Idealism in Britain and the United States," 43\textendash 5.} theologians in the Russian academies attacked German idealism even more fiercely than they did the empiricist school.\footnote{For the theologians' response to Mill, see Nemeth, \emph{Philosophy in Imperial Russia's Theological Academies} 84, 194. Hegel was frequently accused by Russian theologians not only of pantheistic but also of "atheistic" tendencies. See ibid., 111, 134, 142.} The reason for this seemingly paradoxical reaction could have been the fact that Kant's "religion based on reason alone" and the Hegelian Absolute were far removed from the traditional concept of God but as philosophies that retained the idea of the Divine they presented competition to the Orthodox teaching, whereas Mill's empiricism, as a secular school of thought, held no appeal to religiously-minded audience.

What little the academies produced on Mill, their publications had no notable impact on Russian secular philosophy.\footnote{Nemeth notes that "there was little communication between the theological and the secular communities" in Russia (320).} The lack of "philosophical vitality" at the academies (to use Nemeth's characterization\footnote{Ibid., 319.}) at a time when secular institutions were prohibited from teaching philosophy altogether was a missed opportunity for Russian thought, delaying the wider dissemination of Mill's \emph{Logic} by at least a decade. Throughout the 1850s, logic in the academies remained "deeply mired in the Aristotelian categories with no hint of the advances" that were taking place in the West.\footnote{Ibid., 116.}

\subsection*{Nikolai Strakhov on Mill's \textit{Logic}}

\noindent Unusually for Mill's reception in Russia, the first commentary on the \emph{Logic} appeared in the press five years before the book was translated into Russian. It came from the pen of Nikolai Strakhov (1828\textendash 1896), who would become the most frequent and severe critic of Mill on the conservative side.\footnote{See, Berest, "The Reception of J. S. Mill's Feminist Thought," 127\textendash 30; Berest, "The Theme of Happiness and British Utilitarianism," 28\textendash 31.} A graduate of St.~Petersburg University with a Master's degree in biology, Strakhov abandoned his unassuming career as a schoolteacher in 1860 to devote himself full-time to publicist work which he pursued with exceptional enthusiasm until his death in 1896, producing a large number of philosophical and literary pieces as well as articles on the question of Russia's quest for national identity that appealed to conservative readers.\footnote{For Strakhov's intellectual biography, see Linda Gerstein, \emph{Nikolai Strakhov} (Cambridge: Harvard University Press, 1971). Although Gerstein's monograph is still useful, there is much in Strakhov's philosophical legacy that is not discussed in the book, including his response to Mill's \emph{Logic}. Moreover, Strakhov appears in this biography much less of a contentious and contradictory figure than he was.} Throughout those years, Strakhov's intellectual mission was determined solely by the Slavophile agenda\textemdash he sought to promote authentic Russian culture and to counter the 'pernicious' influence of Western ideas (philosophical and political) which he called, rather inaccurately, "the European nihilism."\footnote{N. Strakhov, \emph{Bor'ba s Zapadom} \emph{v nashei literature}, 3d ed.~(Kiev: Tip I. Chokolov, 1897\textendash 98), 2: xxii.}

It was Mill's writings (along with Darwin's) that preoccupied Strakhov the most for many years. Following in the footsteps of Ivan Kireevskii, Strakhov took up the question of the relationship between modern science (and more generally, the scientific worldview) and traditional religion\textemdash a question that had received ample attention from religious apologists in the West but was still new in Russia, where scholastic tradition was absent and the Ecclesiastic academies were slow to respond to the rise of modernity.\footnote{See Berest, "Scientific Modernity vs.~Cultural Tradition," 18\textendash 26; Nemeth, \emph{Philosophy in Imperial Russia's Theological Academies}, ix-xiv.} With his background in science, Strakhov developed a special interest in epistemological problems, focusing on the question of the origin and limits of knowledge, which he discussed in a series of articles later collected into two monographic works: \emph{On the Methods of the Natural Sciences and their Significance in the System of Education} (1865), and the more abstract philosophical work, \emph{The World as a Whole} (1872).

Strakhov's commentary on Mill's \emph{Logic} was published in the journal \emph{Vremia,} which belonged to Mikhail Dostoevsky and was edited by his brother Fedor Mikhailovich who shared Strakhov's interest in epistemological questions.\footnote{On Strakhov's philosophical influence on F. M. Dostoevsky, see Donna Orwin, "Strakhov's \emph{World as a Whole}: A Missing Link between Dostoevsky and Tolstoy," in Catherine O'Neil, Nicole Boudreau, Sarah Krive, eds., \emph{Poetics, Self, Place: Essays in Honor of Anna Lisa Crone} (Bloomington, IN: Slavica Publishers, 2007), 475\textendash 7; Berest, "Scientific Modernity vs.~Cultural Tradition," 29\textendash 42.} In June of 1861, \emph{Vremia} published the Russian translation of Hippolyte Taine's article, "\emph{Philosophie Anglaise,}" in which the French philosopher and literary critic analyzed Mill's \emph{Logic} in a form accessible to non-specialized readers.\footnote{Ten, "Sovremennaia angliiskaia filosofiia. Dzhon Stuart Mill' i ego Sistema logiki," \emph{Vremia} 3, no. 6 (1861): 356\textendash 92. Taine's essay was originally published in \emph{Revue des Deux Mondes} 32 (1861). In 1864 Taine included this article into a book which was translated into English under the title \emph{History of English Literature} (1865).} The translation was apparently made by Strakhov who later translated Taine's work \emph{On Intelligence} (1872) and wrote critically on his positivist ideas.\footnote{Strakhov, "Zametki o Tene," in \emph{Bor'ba s Zapadom}, 2: 95\textendash 123.} Attached to the translation (in consecutive pagination) was Strakhov's commentary which addressed both Mill's \emph{Logic} and Taine's response to the book. Although relatively brief, the commentary was significant in that it contained the core ideas of Strakhov's critical view of empiricism that he would develop in more detail over the next two decades.

"John Stuart Mill," Strakhov wrote, "is already well known to Russian readers as one of the progressive English thinkers (\emph{odin iz peredovykh myslitelelei Anglii}) and readers, undoubtedly, will become interested in his logic."\footnote{Ten, "Sovremennaia angliiskaia filosofiia," 391.} It was not, however, Strakhov's intention to promote the \emph{Logic} as something worthy of following. On the contrary, the message that he intended to convey was that empiricist epistemology provided a limited and unsatisfactory picture of the world. Mill's book, Strakhov wrote, "is remarkable to an utmost degree as an extreme expression of a certain worldview. A thought developed to its logical conclusion necessarily exposes its own truth or falsehood."\footnote{Ibid., 392.} Apparently, Strakhov deemed Mill's errors so self-obvious that he did not spend much time proving his criticism. He then asserted that Mill's pure empiricism was not only false but also extremely negatory in its epistemic scepticism: "Mill's system leads directly to the denial of reasoning (\emph{otritsanie} \emph{myshleniia}) {[}that is, \emph{a priori} reasoning{]} and thus to complete scepticism. He almost regrets that we have reason and not just memory. \ldots{} The world for him is a \emph{chaos} in the full sense of the word, and he goes as far as calling an insanity any other understanding of the world," i.e.~an understanding that presupposes the harmony and interconnectedness of phenomena beyond the mechanistic (as Strakhov saw it) force of cause and effect operating in nature.\footnote{Ibid., 392. Italics are Strakhov's. Where exactly Mill expressed such regrets regarding human reason, Strakhov did not specify.}

Although Strakhov ultimately disagreed with Taine on the question of empiricism, it is easy to see why he used Taine's article as a way of introducing Mill to Russian readers. The French philosopher presented Mill's ideas in the form of a Socratic debate, which allowed him to play the role of a critic, asking heuristic, sometimes provocative questions, while his friend from the Oxford university was cast in the role of Mill's follower who defended his theory of knowledge from an empiricist position. With this format, Taine's article provided some of the criticisms against Mill and empiricism in general that aligned with Strakhov's position. Moreover, Strakhov helped his case by taking liberties with the translation\textemdash in some places he rendered Taine's tone more sarcastic and sceptical than it was in the original text.\footnote{See, for instance, ibid., 362. Cf. Hippolyte Taine, \emph{History of English Literature}, 3 vols. (New York: Colonial Press, 1900), 3: 368.}

What Strakhov wanted readers to notice in Taine's article was the discussion of the limitations of empiricist science, which were acknowledged even by Mill's defender. "The major efforts {[}of science{]}," the English interlocutor argues, "are focused on adding one fact to another or on finding the connection between facts." "What we call the essence of a phenomenon is nothing more than the connected set of facts which constitute that being."\footnote{Ten, "Sovremennaia angliiskaia filosofiia," 362, 363.} Mathematical "axioms"\textemdash such as "two straight lines cannot enclose in a space"\textemdash do not derive from pure reason but from "experience, of sorts," even though we do not need to actually experience two straight lines running parallel to one another ad infinitum; at certain point we can use the "imagination" just as we use the telescope to extend our vision. Taine, however, responds by expressing doubt that this mental operation can be called "experience," but the defender explains that the imagination produces "impressions" which are, strictly speaking, sensations observed by the mind.\footnote{Ibid., 367\textendash 8. The last part of the argument was shortened in Strakhov's version. Cf. Taine, \emph{History of English Literature}, 3: 377.}

Moving on to Mill's notion of causality, the defender admits that empiricists are content with the idea that the realm of the knowable is limited to physical causes while "permanent causes" should be set aside as lying beyond science. According to Mill, whose words were cited by Taine's interlocutor, "the Sun, the Earth, the planets, their motion, their chemical composition" belong to those elemental causes which empirical science is unable to explain. "'Why \ldots{} these particular agents appeared originally, why they exist in a certain proportion\ldots{}' \textemdash~this is the question that we cannot answer. \ldots{} Even astronomy \ldots{} is an example of limited science."\footnote{Ten, "Sovremennaia angliiskaia filosofiia," 376.} "We understand a million facts but \ldots{} hundreds of other facts \ldots{} remain inexplicable to us." Taine's friend then admits that "even if the theory of the universe is complete, there will remain in it two big voids: one at the commencement of the physical world and the other at the beginning of the moral world."\footnote{Ibid., 377.}

Taine's own view of the empiricist theory of knowledge appears very critical at first. Empiricism, he argues, signifies an "abyss of chance and abyss of ignorance (\emph{nevedenie})." "By excluding from science the knowledge of first causes, that is divine causes," he tells his opponent, "you reduce man to scepticism, positivism, utilitarianism, if he has a dry mind {[}a hint at Mill himself{]} or you turn him into a mystic, an enthusiastic Methodist, if he is endowed with lively imagination." "In this void of the unknown, which you place beyond your little world \ldots{} man of cold judgment must all sink down to the search for the improvements of material life, having no hope for achieving something more elevated."\footnote{Ibid., 379.} In Strakhov's later works, this was one of the key charges against the intelligentsia, who adopted, as he believed, a strictly utilitarian approach to science. He viewed this tendency as an undue influence of Western, especially English, thought in Russia, which impacted even government education policies.\footnote{See Berest, "Scientific Modernity vs Cultural Tradition," 19\textendash 20.} In Taine's view, both features\textemdash practicality and religiosity tending towards mysticism\textemdash characterized "the English mind" in contrast to German philosophy, which found, according to Taine, a better way "of reconciling religion and science." In the original text, however, Taine's criticism of the utilitarian nature of British thought is softer; the point about "more elevated" goals supposedly lacking in the empirical mind was Strakhov's own embellishment.\footnote{Cf. Taine, \emph{History of English Literature}, 394.}

It was disappointing to Strakhov that Taine's criticism of empiricist theory did not go far enough. Having noted all the limitations of relying on experience alone (and here too, Strakhov took the liberty of adding his own words\textemdash "you distort the human mind"\textemdash when translating Taine's point about the "incomplete{[}ness{]}" of the empiricist system\footnote{Ten, "Sovremennaia angliiskaia filosofiia," 380. Cf. Taine, \emph{History of English Literature}, 395.}), Taine surprises the reader with the following statement: "We believe that there are no essences (\emph{sushchnosti}) but only systems of facts. We regard the idea of essence as a psychological illusion \ldots{} the remnant of scholastic doctrine. \ldots{} {[}T{]}here exists nothing but facts and laws, that is phenomena and the relations between them, and all knowledge consists of adding facts to facts."\footnote{Ten, "Sovremennaia angliiskaia filosofiia," 381.} As Taine goes on to explain, the only difference between his position and that of Mill is the latter's unwillingness to remedy the limitations of pure empiricism with more abstract reasoning that would go beyond factual data. Taine, therefore, proposed to supplement Mill's empirical methods with what he called "abstraction" ("\emph{otvlechenie}"), the mental operation that allows "to isolate elements of facts and consider them separately." As he further explained, using a specific example, "{[}m{]}y eyes follow the outline of a square and abstraction isolates its two properties, the equality of its sides and angles."\footnote{Ibid., 381.}

Whether Taine's method was in fact distinct from Mill's inductive reasoning, Strakhov did not consider. He noted that Taine's "abstraction" was an improvement on the empiricist methodology since it supplied "an organizing, connecting" element of reasoning. "Abstraction indeed brings Taine to {[}the recognition of{]} some connection between phenomena; the world is no longer broken into numberless disparate facts as in Mill." "But in the end," Strakhov added with disappointment, "Taine agrees with Mill." "The world" in Taine's system, "has neither the center nor the boundaries," it is still the world in which "chance" (the notion which Strakhov decried in Darwin) plays the dominant role. "Taine's thoughts" he concluded, "suffer from some vagueness and incompleteness."\footnote{Ibid., 392.}

With the briefness of Strakhov's commentary, he offered too little argument to support his conclusions on the supposed flaws of empiricism, but readers whose interest was piqued by the topic could refer to his book, \emph{The World as a Whole.} Although the book consisted of previously published articles (the earliest one dated 1858), they were apparently little known until Strakhov compiled them together into a book in 1872. Lev Tolstoy read it and left an interesting comment: "My general impression: 1) I learned much that was new. \ldots{} 2) Many questions which had vaguely occurred to me before were posed and resolved clearly, freshly and forcefully. \ldots{} 3) Many, terribly many questions are not resolved \ldots{} and the reader wants to know how the author will resolve them."\footnote{Quoted in Orwin, "Strakhov's \emph{World as a Whole}," 480. Tolstoy's comment also suggests that he did not pay attention to Strakhov's articles until the book came out because "nothing was heard" of them. Ibid., 480.} Tolstoy's last point is especially curious given the fact that Strakhov confidently assured readers that his book is "the easiest to comprehend of all {[}the books{]} devoted to philosophical questions (\emph{predlagaiu chitateliam} \emph{samuiu poniatnuiu iz knig posviashchennykh  filosofskim voprosam})."\footnote{Strakhov, \emph{Mir kak tseloe: cherty iz nauki o prirode} (St.~Petersburg: Tip. Zamyslovskogo, 1872), iii.}

The article in which Strakhov offered to "show the powerlessness of empiricism" was written three years before his commentary on Mill's \emph{Logic}, but its claim was fundamentally the same\textemdash empiricists are unable to resolve "certain questions," including, most importantly, the question of "the essence of phenomena" and the nature of "the connection between spiritual and material substances."\footnote{Ibid., 323\textendash 25.} In the article from 1860 he also charged that "the naturalists subsume all their notions about the world under the conception of causality. \ldots{} {[}F{]}or them \ldots{} the world is nothing more than the endless game of causes and effects."\footnote{Ibid., 107.} Contra the empiricist position, which removed any spiritual meaning from the notion of causality, Strakhov adhered to the view that "the world is a connected whole (\emph{mir est sviaznoe tseloe}) and all its parts \ldots{} are mutually dependent." "{[}It{]} is an organic whole \ldots{} or rather, a hierarchy of beings and phenomena" in which "man is the apogee of nature, the hub of the universe (\emph{uzel bytia}). He is the greatest mystery and the greatest creation of the world."\footnote{Ibid., vii-viii.}

As Strakhov noted in the introduction, his conception of the world was inspired by two sources: one was "mathematical and natural sciences," and the other derived from "Hegelian philosophy," but "not any of its specific ideas," Strakhov added, only "its method, which I consider to be \ldots{} the expression of the scientific spirit."\footnote{Ibid., vi.} With this disclaimer, Strakhov apparently sought to distance himself from Hegel's philosophy of religion, which could be interpreted as a form of pantheism, but it is clear that he owed more to Hegel than just his method.

Strakhov was in fact an early exponent of the idealist current that was yet to emerge in Russia but was already discernable in Britain. His reasons for turning to Hegel in battling the empiricist conception of nature were the same ones that drove his British counterparts to Hegelian idealism. A historian of British philosophy notes that the rising popularity of idealism was a remarkable phenomenon for a country whose philosophical fame rested on its contribution to the empiricist philosophy and "distrust of anything abstract and metaphysical."\footnote{W. J. Mander "Hegel and British Idealism," in L. Herzog (ed.), \emph{Hegel's Thought in Europe} (New York: Palgrave Macmillan, 2013), 165.} Originating in the mid 1860s, "the initial trickle of idealist writings grew during the 1870s into an absolute flood of publications."\footnote{Ibid., 166. See also Mander, "Introduction," 7\textendash 8; John Hedley Brooke, "Evolution and Religion," in \emph{The Oxford Handbook of British Philosophy}, 212.} Among the intellectual factors that triggered the surge of idealism in Britain, W. J. Mander lists "the rapid growth of materialistic science and the advent of evolutionary theory which together constituted a profound challenge to traditional senses of religion, self and morality"\footnote{Mander, "Hegel and British Idealism,"173\textendash 74.}\textemdash the same developments that propelled Strakhov into the discourse on the philosophy of science and fuelled his interest in Hegel. Like Strakhov, British idealist philosophers were searching for "the larger whole" endowed with a "spiritual dimension" in a world that appeared increasingly fractured by clashing ideas and myriad new facts emerging from specialized sciences.\footnote{Ibid., 169, 173; see also W. J. Mander, \emph{British Idealism: A History} (Oxford University Press, 2011), 5}

However, despite adopting the same conceptual foundations, Strakhov, unlike his British counterparts, devoted more energy to attacking the opposing view than to creating his own epistemological theory.\footnote{In considering the epistemological writings published by Russian theologians, Nemeth noted that one of the most talented of them, Mikhail Karinsky, produced extensive commentary on the theory of knowledge in several of his works but "came to a few positive conclusions of his own and certainly to no all-encompassing system." See Nemeth, \emph{Philosophy in Imperial Russia's Academies}, 150. Vucinich made a similar argument about Russian idealist philosophers of the nineteenth century, such as Vvedenskii, Lopatin and Losskii. Although, unlike Strakhov, they were professional philosophers, Vucinich argued that "they spent far more time in pointing out the intrinsic limitations of science than they did in actually building a philosophy that might reveal the ultimate nature of the universe." See Alexander Vucinich, \emph{Science in Russian Culture}, 251.} Rather ironically, \emph{The World as a Whole} was not a unified philosophical exploration but a loose collection of articles (over five hundred pages long) united only by the preface, which gave a strangely confused explanation of the nature of the book, as though Strakhov could not decide whether he had written it from the viewpoint of a philosopher or "a naturalist": having assured readers that the book is "philosophical," he immediately qualified his statement by saying that there are, in fact, "few philosophical terms" in it and "it is written almost completely in the language of a naturalist rather than philosopher"; yet he also insisted that it is not a work seeking to popularize ideas from the natural sciences. "I am almost an unconditional opponent of popularizations," Strakhov wrote, despite the fact that he had previously published in \emph{The} \emph{Journal of the Ministry of Public Education} a number of articles that did precisely that\textemdash explain some of the latest scientific discoveries in more layman language.\footnote{Strakhov, \emph{Mir kak Tseloe}, {[}i.{]} The preface was unpaginated. See, for instance, Strakhov, "Proizvol'nye zarozhdeniia," \emph{Zhurnal Ministerstva Narodnogo Prosveshchenia} (1859): 118\textendash 21.}

Throughout his long writing career, Strakhov indeed tried to speak both as a scientist and a philosopher, but the voice of a feisty publicist driven by a political agenda, was often much louder in his articles, especially in later years. The end result was not very convincing for either side: for Russian scientists who engaged with Strakhov in a polemic on Darwinism, his position was too hostile, lacking in substance yet confident in its truth to the point of unbending dogmatism (the latter criticism, ironically, was the same that Strakhov commonly hurled at his empiricist opponents).\footnote{K. A. Timiriazev, \emph{Bessil'naia zloba antidarwinista} (Moscow: I. Kushnerev, 1889), 6\textendash 7.} In an article entitled "The Helpless Anger of an Anti-Darwinist" (1889), Kliment Timiriazev, a prominent Russian scientist, responded to Strakhov by hinting, rather uncharitably, that his "bitterness" stemmed from his failure to make a career in science: It is the "type of person," Timiriazev wrote, who "believes that it {[}science{]} stopped when he abandoned his books, trying to convince himself and others that science is moved not by scientific works but by a scholastic dialectic."\footnote{Ibid., 7.} Timiriazev also charged that Strakhov's judgment of the Darwinian theory was based on his prejudicial view that "the English," "as empiricists," were "incapable of sound reasoning." "Mr.~Strakhov," Timiriazrev wrote, "tries to convince the reader that {[}Darwin's{]} theory cannot be anything but weak just because \ldots{} it is the work of an Englishman," and he takes this view as an "axiom."\footnote{Ibid., 38. The ellipses were Timiriazev's, employed for stronger rhetorical effect.}

Tolstoy reacted to this exchange by noting in his diary: "I read Timiriazev's article. He is wrong, but \ldots{} it would be a pity to miss it."\footnote{Lev Tolstoy, \emph{Sobranie sochinenii}, 22 vols. (Moscow: Khudozhestvennaia literatura, 1978\textendash 85), 21: 389. This entry was made on June 24, 1889.} In another comment, Tolstoy acknowledged that personal recriminations came from both sides in this heated polemic. With some exaggeration but not inaccurately, he summed it up as follows: "You are a fool; no it is you who is a fool. (\emph{Durak, ty sam durak}.)"\footnote{Ibid. 388.} He also remarked, however, that Strakhov's position on the "limits of knowledge (\emph{predely poznaniia})" remained "vague."\footnote{Ibid.}

As a journalist with good marketing skills, Strakhov was certainly successful in drawing attention to his writings and stirring up controversies. However, his epistemological explorations were not original and indeed, not coherent enough to have a positive impact on Russian religious philosophers later in the century. The most prominent of them, Vladimir So\-lov'ev, engaged in a protracted polemic against Strakhov in the 1880s, which came as an unpleasant surprise to the old Slavophile.\footnote{One of their disagreements concerned Strakhov's Slavophile views and the manner in which he brought them to bear on broader cultural issues, including what he called "\emph{nauchnaia samobytnost'}" (scientific originality), i.e.~the impact of national culture on science. See N. Strakhov, "Nasha kul'tura i vsemirnoe edinstvo," in Strakhov, \emph{Bor'ba s Zapadom}, 2: 234; Strakhov, "Poslednii otvet g. Solov'evu," in ibid., 2: 235\textendash 59; Strakhov, "Novaia vykhodka protiv knigi Danilevskogo," in ibid., 3: 124\textendash 52 (a reply to Solov'ev's last article in this dispute entitled "Mnimaia bor'ba s Zapadom," in \emph{Russkaia Mysl'} (1890).} He was especially hurt and puzzled when some of Solov'ev's accusations were approvingly cited by Timiriazev.\footnote{Strakhov to Tolstoy, June 21, 1889 in L. N. Tolstoy i N. N. Strakhov, \emph{Polnoe sobranie perepiski}, vol.~2, \url{http://az.lib.ru/s/strahow_n_n/text_1878_02_perepiska.shtmlб}} As often, Strakhov sought consolation from Tolstoy but received the following answer: "both of you are right and wrong (\emph{i pravy i ne pravy vy oba})."\footnote{Strakhov to Tolstoy, March 12, 1887; Tolstoy to Strakhov, June 28, 1888, in ibid. Strakhov often complained to Tolstoy about his opponents. In 1888 he wrote that positivist Vladimir Lesevich accused him of showing too much vanity in his articles while Strakhov retorted that Lesevich was apparently incapable of speaking respectfully of anyone other than "Darwin and Mill." See, Strakhov to Tolstoi, February 5, 1888, in ibid.} Although the novelist strongly disapproved of what he saw as a layer of "pernicious Hegelian phraseology" in Solov'ev's religious thinking, he had, from the beginning, disliked the idea of public dispute between these two thinkers and asked Strakhov to abstain from writing critically of Solov'ev\textemdash to no avail, as it turned out.\footnote{Tolstoy to Strakhov, December 23, 1874; March 16, 1878, in ibid., vol.~1. Strakhov agreed that Solov'ev's religious views led him to "pantheism closely resembling that of Hegel's." See Strakhov to Tolstoy, April 9, 1878, in ibid.}

Over the years, despite the rise of empirical science in Europe and Russia, Strakhov remained committed to his view that empiricism provided nothing more than a superficial, "me\-chanistic explanation" of the workings of nature which did not solve "the mystery of life." He conceded that experimental science had made some important discoveries, pointing, as one example, to the works of Pasteur that revealed the role of microbes in organic nature. But Strakhov insisted that an element of "mystery which wraps around all phenomena of life" persisted in the world of microbes, as elsewhere: "Pasteur and other scientists of this school, only put to us new puzzles (\emph{zagadki})" of nature. A fuller explanation of the natural world can only be achieved, according to Strakhov, if science is anchored in "philosophy" by which he meant religiously grounded metaphysics.\footnote{Strakhov wrote: "The rejection of metaphysics, that is, philosophy, proclaimed by Comte, has attracted many minds for whom philosophy was an unbearable burden." Strakhov, \emph{Bor'ba s Zapadom}, 3: 10.} "{[}U{]}nfortunately," he lamented, "the lack of philosophical guidance is the reason for those flaws in the development of the natural sciences that we are talking about."\footnote{Strakhov, \emph{Bor'ba s Zapadom}, 3: 84\textendash 6. Gerstein argues that Strakhov favored the "separation of metaphysics and scientific knowledge," but no textual evidence is provided in support of this claim. See Gerstein, \emph{Nikolai Strakhov}, 151, 154.} "If man and the universe are divine creations, this force must be reflected in everything. But does science strive to discern the imprint of this force?" Strakhov asked in 1891. As a general line of criticism against empirical science, he continued to argue until the end, that by setting aside the question of the ultimate causes and essences of things, empiricists do not make the problem go away.

Strakhov's opinion on Mill's \emph{Logic} also remained unchanged. In one of his last essays, "The Outcomes of Modern Science" (1891), Strakhov reiterated the assessment he had made thirty years earlier: "Mill gave the theory of knowledge a distinct and clear formulation, but it only resulted in the denial of knowledge, not in a new step towards it."\footnote{Ibid., 29. See also ibid., 101.} Expressed in this generalized form, Strakhov's verdict was strikingly simplistic and severe compared to the criticism that Mill sustained in Britain from even the most exacting of his opponents on the idealist side\textemdash T. H. Green. The latter, like other British idealists, adhered to the view that "conceptions," which allow us to have a unified picture of the world, are "immanent in the human mind" and "constituted by the act of conceiving" rather than acquired through inductive reasoning as was Mill's position.\footnote{Thomas Hill Green, \emph{Philosophical Works} (Cambridge: Cambridge University Press, 2012), 2: 196. (It was originally published in 1886, posthumously, on the basis of Green's lectures delivered at Oxford university in the 1870s.)} He judged that Mill's empiricist approach was unable to explain \emph{how} knowledge was possible, but Green's critique never amounted to the kind of reductivism adopted by Strakhov.\footnote{According to a historian of British idealism, Green was "so concerned \ldots{} to come to terms with the philosophers he criticized, especially with Hume and Mill, \ldots{} that he is found qualifying his own affirmations." See H. D. Lewis, "The British Idealists," in Ninian Smart et al, eds., \emph{Nineteenth Century Religious Thought in the West} (Cambridge: Cambridge University Press, 2010), 276.}

There was also in Strakhov's pronouncements on Mill and European thought a heavy admixture of the anti-Westernist bias that detracted from his judgement as a philosopher, at times resulting in simplistic characterizations of a sensationalist type. Describing the state of the European mind in \emph{The World as a Whole}, Strakhov wrote:

\begin{quote}
in present times \ldots{} empiricism reigns almost unchallenged. Natural sciences attract both youth seeking enlightenment and the so-called educated people \ldots{} Balzac has written a novel entitled \emph{The Quest for the Absolute}. With the utmost artfulness \ldots{} he tells the story of a character who searches for the Absolute. But what do you think it is?\textemdash It is only some kind of chemical substance. \ldots{} Dickens in his touching story \emph{Contract with Ghosts}, presents the professor of chemistry as a sage and describes how reverently the audience is listening to his every word. Such dreams are understandable in England, the classic country of empiricism where even physics and chemistry pass for philosophy.\footnote{Strakhov, \emph{Mir kak tseloe}, 326.}
\end{quote}

\noindent In "The Outcomes of Modern Science," Strakhov reaffirmed his claim that the European mindset was characterized by a fixation on experimental sciences and the scientific mode of thinking, which aspired to offer "solutions to all kinds of questions," pushing aside and, in effect, trying to replace religion.\footnote{Ibid., 10\textendash 11.} On its own, this was a valid concern but Strakhov failed to mention that many European theologians and idealist philosophers had long raised such concerns and sought to redefine the role of religion in the modernizing world. They "felt that modern men could do neither without religion nor with it as it was."\footnote{James C. Livingston, "British Agnosticism," in \emph{Nineteenth Century Religious Thought}, 232.} And while they rejected philosophical materialism, there was a good measure of respect for empirical science among idealist philosophers.\footnote{Christopher Adair-Toteff, "Neo-Kantianism: The German Idealist Movement," \emph{The Cambridge History of Philosophy}, 28; see also, Livingston, "British Agnosticism," 232.}

Strakhov, however, assured his readers that "philosophy" (i.e.~metaphysical inquiry) had entirely lost its footing in Europe already in the 1840s. "Afterwards, some thinkers achieved \ldots{} great popularity and were widely read \ldots{} Among them were Schopenhauer, Mill, Spencer and Hartmann. \ldots{} But the success of these writers did not bring about the rise of philosophy." The empiricism of Mill and Spencer, according to Strakhov was "nothing more than the logical development \ldots{} of the ideas of Kant and Hume," which yielded "no new steps" forward in debates on epistemology or ethics.\footnote{Strakhov, \emph{Bor'ba s Zapadom}, 3: 28\textendash 9. Drawing a direct line of continuity between Mill and Kant was a strikingly inaccurate characterization. It is unlikely that Strakhov was unaware of the differences between them but in a manner typical of his publicist style he allowed himself such overgeneralized statements when talking about Western thinkers. See also Berest, "Scientific Modernity vs.~Cultural Tradition," 25, 40.}

Strakhov's conclusion was bleak\textemdash disengaged from philosophy and religion, the European mind had become impoverished morally and intellectually, despite all the material improvements brought about by modern science. He ended the review with a quote from the Russian poet Dmitrii Venevetinov, who had written back in 1827 that Russia would need to "isolate {[}itself{]} from the current influence of other peoples" in order to build its own authentic culture. Strakhov deeply regretted that this solution was no longer feasible. Now that European "books were streaming" into Russia "one after another," he wrote, Russians "are bound to experience all the ills and downfalls of European thought."\footnote{Ibid., 33.}

\subsection*{Peter Lavrov's Translation of Mill's \textit{Logic}}

\noindent A much different view of Mill's \emph{Logic} and its significance for modern intellectual culture was presented by Peter Lavrov (1823\textendash 1900) who published the first Russian translation of the \emph{Logic} in 1865.\footnote{Dzhon Stuart Mill, \emph{Sistema Logiki}, trans. P. L. Lavrov (St Petersburg: Tip. M. Vol'fa, 1865).} Lavrov's introduction assessed Mill's book by situating it in the history of logic and the philosophy of science. Already in the ancient world, Lavrov noted, syllogistic reasoning presented not much more than the gymnastics of the mind, divorced from the "real world" and devoid of any "real content." "There was," he admitted, a "share of pedagogical benefit" in practicing syllogistic exercises as a form of "verbal argumentation"; however, without connection to "the real world," "the scientific value" of formal logic was "negligible," for "it did not \ldots{} explain all of the theory of reasoning."\footnote{{[}P. Lavrov{]}, "Predislovie redaktora," in Mill, \emph{Sistema Logiki}, x.} The treatment of logic by Kant and Hegel, although insightful, took logic, in Lavrov's view, further down the path of metaphysical development and away from the world of natural science. By contrast, English adherents of the empiricist school tended "to deny any scientific value of syllogism and deduction."\footnote{Ibid., xx.} Lavrov credited Mill for attempting to "build the theory of knowledge" which combined the best of the two methods. "He {[}Mill{]} became convinced that in many sciences induction and deduction are intertwined" and that "in some sciences the inductive method is not applicable." "As a theory of knowledge," Lavrov asserted, "this work is virtually unsurpassed in Europe (\emph{edva li etot trud imeet drugoi ravnyi emu v Evrope})."\footnote{Ibid., xxii.}

At the same time, Lavrov noted "some shortcomings in Mill's work" which he attributed to Mill's "strict adherence to positivism." These shortcomings, as Lavrov saw them, specifically had to do with Mill's unwillingness to venture into the realm of metaphysics to expand the scope of logic as a tool of knowledge. "Mill did not solve the problem of creating a science of human reason as objectivizing {[}i.e.~interpreting{]} everything that exists (\emph{vse sushchee})."\footnote{Ibid., xxiii.} Such a grand task, Lavrov admitted, was never and could never be Mill's aim by virtue of his empiricist position, but even within this limited scope, "Mill could have and perhaps, should have explored more closely the psychological processes" involved in the formation of belief as a cognitive instrument. "This discussion would have tied together \ldots{} the theory of knowledge with the theory of things-in-themselves." As a mathematician (Lavrov taught mathematics at two Military Academies in St.~Petersburg), he also found objectionable Mill's empiricist approach to mathematics, which was in fact the most criticized aspect of Mill's book in Europe.\footnote{Ibid.}

Nevertheless, Lavrov's overall assessment of Mill's contribution to logic was more than positive. "What can be said about his {[}Mill's{]} critics?" he wrote in conclusion. "The majority of them are not even worth mentioning, \ldots{} {[}e{]}specially the Germans who stand firmly on their idealistic ground, refusing to even think deeper of Mill's book" merely because he shifted the focus of logic to "the theory of knowledge" instead of the "theory of thinking." "They simply do not understand each other," speaking, as it were, in different epistemological languages. The only critique he could recommend was the one offered by Taine, which "Mill himself acknowledged \ldots{} as the best critique of his book." In Lavrov's view, constructive criticism of Mill's logic could only be made from the same epistemological position and not from the position of idealism.\footnote{Ibid., xxiv.} "One can say," he remarked in conclusion, "that Mill's book belongs to the best literature on logic that has ever been written." He ranked him alongside Aristotle, Descartes, Bacon, Kant and Hegel, emphasizing the fact that Mill tackled "the practical side" of logic.\footnote{Ibid., xxv.}

\vspace{-1em}
\subsection*{Mill's \textit{Logic} at Moscow University:\\M.M. Troitskii and Evgenii Trubetskoi}

\noindent Unlike Mill's other works published in Russian translation, the \emph{Logic} did not get much attention from the popular "thick" journals, apparently because of its specialized nature.\footnote{Only \emph{Vestnik Evropy} offered an article on the \emph{Logic}, as part of the comprehensive review of Mill's thought a year after his death. See Iu. Rossel', "Dzhon Stuart Mill," \emph{Vestnik Evropy} (July 1874): 132\textendash 68.} The book's influence, however, was evident in academic publications on logic and psychology as both disciplines acquired a larger degree of professionalization in Russia by the 1880s.\footnote{The majority of Russian publications in the field of logic appeared only in the 1880s: M. I. Vladislavlev, \emph{Logika} (SPb.: Tip V. Dimakova, 1872); P. F. Kapterev, \emph{Istoriia logiki. Lektsii} (SPb.: Tip Kurochkina, 1880); N. Ya. Grot, \emph{K voprosu o reforme logiki} (Nezhin: Istoriko-filologicheskii institute, 1882); M. Troitskii, \emph{Logika} (Moscow: Tip Pustoshkina, 1884); M. I. Karinskii, \emph{Logika} (SPb.: Tip Eleonskogo, 1884); E. Radlov, \emph{Logika. Lektsii} (SPb.: Tip Dabizha, 1880). See also, V. A. Bazhanov, \emph{Istoriia logiki v Rossii}, 32\textendash 8.}

Within academic philosophy, the influence of Mill's empiricism was especially pronounced in the Moscow University, where Matvei Troitskii, an enthusiastic follower of the British school of empiricism, taught courses on logic and psychology from 1875 to 1899. A graduate of Kiev Spiritual Academy, Troitskii was one of the first beneficiaries of the study-abroad program for graduate students, which resumed under Alexander II. From 1862 to 1864 Troitskii took philosophy courses in Heidelberg, Leipzig, and Jena, including the lectures of the Hegelian follower Kuno Fischer.\footnote{V. A. Volkov, M. V. Kulikova, V. S. Loginov, \emph{Moskovskie professora XVIII-nachala XX vekov} (Moscow: Ianus, 2006), 241\textendash 42. ~} However, unlike many other Russian students in Germany who returned home converted to German idealism, Troitskii came to the conclusion that modern German philosophy had not moved far beyond medieval scholasticism.\footnote{See Thomas Nemeth, \emph{Kant in Imperial Russia} (New York: Springer, 2017), 157.} He presented this bold claim in his doctoral dissertation, "German Psychology in the Present Century," which he defended at the St.~Petersburg university in 1867, after it had been rejected for defense at the Moscow university by the chair of philosophy Pamfil Iurkevich. When Iurkevich passed away, Troitskii assumed his position, drastically altering the direction of philosophical studies in Russia's oldest university.

Students remembered Troitskii as an inveterate critic of Kant whose philosophy presented, in his view, nothing more than a refurbishing of old scholastic ideas with the help of more "abstract" "nebulous terminology."\footnote{M. Troitskii, \emph{Nemetskaia psikhologiia v tekushchem stoletii} (Moscow: Tip. T. Ris, 1867), 628. For student recollections, see E. Trubetskoi, \emph{Vospominaniia} (Sofia: Rossiisko-Bolgarskoe izdatel'stvo, 1921), 72.} The future of "the science of the mind (\emph{dukha})," according to Troitskii, belonged to the English school of empiricism which commenced with Bacon and culminated with Mill: "Apart from its own original qualities, {[}Mill's{]} \emph{Logic} presented in a coherent manner all criticisms of the syllogism since the time of Bacon." Troitskii credited Mill for laying down the empirical foundations of the inductive method of reasoning with "superior" clarity, fully agreeing with the British philosopher that induction rather than syllogism should be the primary mode of reasoning in science.\footnote{Troitskii, \emph{Nemetskaia psykhologiia}, 20\textendash 23. It is curious that despite his empiricism, Troitskii still used the Russian term "\emph{dukh}" for "the mind." Lavrov also occasionally (but not consistently) translated "the mind" in Mill's \emph{Logic} as "\emph{dukh,}" while in later translations the more correct term "\emph{razum}" was used with proper consistency.} In his later works, including the monumental, three volume \emph{Uchebnik Logiki} (1885\textendash 88), Troitskii reiterated his view that Mill was "the creator of modern logic" which helped to move the natural sciences forward.\footnote{M. Troitskii, \emph{Uchebnik Logiki}, 3 vols. (Moscow, 1885\textendash 88), 3: i. He also published \emph{Nauka o Dukhe} (Science about the Mind) in 1882.}

Among Troitskii's students in the 1880s was Evgenii Trubetskoi (1863\textendash 1919), future Mos\-cow University professor and prominent religious thinker who left in his memoirs a vivid account of his intellectual journey from Mill's empiricism to Kantian-inspired idealism.\footnote{For more on Trubetskoi's philosophy, see Teresa Obolevich and Randall Poole, eds., \emph{Evgenii Trubetskoi: Icon and Philosophy} (Eugene, Oregon: Pickwick Publications, 2021).} Remembering his philosophy classes in Moscow university, Trubetskoi wrote that Troitskii's rejection of German metaphysics in favor of British empiricism stemmed from his inability to comprehend the more intellectually taxing Kantian philosophy: "Troitskii \ldots{} was extremely ignorant of the history of philosophy. He peppered his lectures with cheap and vulgar ridicule of the German philosophers while it was clear to me that \ldots{} the foundations of the German philosophy were entirely unknown to him." "Having insulted 'metaphysics' with his vulgar jokes, Matveika {[}a diminutive from Matvei{]} then lectured either on Mill's logic or on modern psychological teachings, mainly English ones, which was all he knew."\footnote{Trubetskoi, \emph{Vospominania}, 73\textendash 4.} Trubetskoi admitted that "{[}y{]}outh \ldots{} flocked to his classes" and "accompanied each of his lectures with a thunderstorm of applause." Trubetskoi, however, was greatly "annoyed" by these scenes, being convinced that Troitskii did not deserve such high regard in the least.\footnote{Ibid., 73.}

Trubetskoi wrote this with the air of a person who was intensely proud that he had outgrown his own youthful fascination with British philosophy and moved on to a higher stage of his philosophical maturation, as he later saw it. "Already in grade six," Trubetskoi recalled, "being only fifteen years old I managed to read and even take summary notes of Mill's \emph{Logic}, along with his \emph{Political Economy}."\footnote{Ibid, 56.} That year Trubetskoi also read Mill's \emph{Auguste Comte and Positivism}, Spencer's \emph{Psychology} and Darwin's \emph{The Origin of Species}. "I remember," he added," that during that period I lived and thought by Buckle, Mill, Spencer; getting out of this hypnosis was then completely out of question."\footnote{Ibid., 57.} The result of Trubetskoi's reading, perhaps too intense for his early age, was the onset of what he called a period of "boundless doubt," which gripped him to such a degree that he felt mentally disturbed at times. The young Trubetskoi came to doubt Mill's postulate of the uniformity of nature and with that, doubts crept in about the existence of physical reality and even his own existence as a self-conscious cognitive agent. "I was frightened and haunted by those thoughts," Trubetskoi remembered, "sometimes I felt close to insanity."\footnote{Ibid., 59.}

What ended this "hypnosis" for Trubetskoi was Kuno Fischer's \emph{History of Philosophy} (available in Russian translation since 1861), the same reading that had, ironically, led Troitskii away from German idealism. After reading the book, as Trubetskoi recalled, "all of a sudden so much of what seemed to be the irrefutable truth in the teachings of Mill and Spencer turned out a long-refuted error! I used to think of Mill's empiricism as the 'last word' {[}in philosophy{]} and suddenly I discovered that this empiricism had been refuted already by Leibnitz in his polemic against Locke. \ldots{} When I got to Kant, I made even greater discovery." "In short," Trubetskoi concluded, "all those formulas in which I believed blindly, dogmatically, were shattered to pieces." He became convinced that Kant's teaching and "all of German metaphysics" had simply escaped the minds of Mill and Spencer.\footnote{Ibid., 57.}

It is curious that Vladimir Solov'ev, a fellow religious philosopher, did not share Trubetskoi's harsh opinion of Troitskii despite the fact that Solov'ev considered himself a student of Iurkevich (Troitskii's former opponent) and had expressed some criticism of Mill's positivist ideas.\footnote{See Thomas Nemeth, \emph{The Early Solov'ev and His Quest for Metaphysics} (Cham, Switzerland: Springer Intl. Publishing, 2014), 30.} In a review article which showed a greater spirit of intellectual tolerance towards diverse philosophical "schools," Solov'ev pondered on the intellectual legacy of three academic philosophers\textemdash Iurkevich, Troitskii and Nikolai Grot, "so different from each other," who nevertheless, had one thing in common\textemdash they had all helped to advance Russian philosophical studies at a time when philosophy was a neglected field. Solov'ev noted that Troitskii's \emph{German Psychology} was a product of Russia's peculiar cultural circumstances in the 1860s\textemdash the need to catch up to the latest European developments in philosophy. "It goes without saying that the European reader would not learn anything from the three works which he {[}Troitskii{]} wrote \ldots{} but to the Russian people, the first of these works {[}\emph{German Psychology}{]} which appeared in the sixties, taught \ldots{} the foundations without which one could not move any further."\footnote{V. Solov'ev, "Tri kharakteriskiki. M. M. Troitskii. N. Ia. Grot. P. D. Iurkevitch," \emph{Vestnik Evropy}, no. 1 (1900): 321.}

Solov'ev admitted that Troitskii got carried away by his enthusiasm for English empiricism and that in later years his teaching acquired "a dogmatic form," a form of "absolute truth" in which he came to believe uncritically. "However," he added, "Troitskii \ldots{} made and helped others make the first step towards serious philosophical studies." In Russia, where "philosophical education" was a noble but "unappreciated activity," Solov'ev thought it important to give credit to Troitskii's devotion to philosophy. He also noted, with a sense of sympathy, that Troitskii's philosophical views almost cost him his position at the university when the suspicion arose that was spreading atheism through his courses. He was forced to go to St.~Petersburg and defend himself before the educational authorities. Whether Solov'ev actually believed Troitskii's assurances of his piety is not entirely clear but he certainly made it clear that government interferences into academia remained an impediment to philosophical pursuits in Russia.\footnote{Ibid., 322.}

\subsection*{Mill's \textit{Logic} at St. Petersburg University: M.I. Vladislavlev}

\noindent At the St.~Petersburg University, M. I. Vladislavlev (1840\textendash 1890), professor of logic, psychology and philosophy, also wrote a popular textbook on logic\textemdash \emph{Logic. An Overview of Inductive and Deductive Methods of Reasoning} (1872), the first of its kind in the post-reform era, in which he discussed Mill's empiricism in a historical and philosophical context. Relatively unknown today, Vladislavlev was an important figure in Russian academic philosophy who contributed to the rise of neo-Kantian studies by producing in 1867 the first Russian translation of Kant's \emph{Critique of Pure Reason}.\footnote{See Nemeth, \emph{Kant in Russia}, 154; Nikolas Lossky, \emph{History of Russian Philosophy} (New York: International University Press, 1969), 163.} His student and successor at the St.~Peterburg University, A. I. Vvedenskii, credited Vladislavlev with raising the status of philosophy as a scholarly discipline after the difficult years of Nicholaevan restrictions. Vladislavlev was also known to contemporaries as a thinker close to Dostoevsky, whose journal \emph{Vremia} published Vladislavlev's article on psychology and the question of free will that became a subject of interest in Russian thought during this period.\footnote{A. I. Vvedenskii, \emph{Nauchnaia deiatel'nost' M. I. Vladislavleva} (St.~Petersburg: Tip. V. S. Balasheva, 1980), 14. On the theme of free will in Russian thought see, T. Nemeth, \emph{The Later Solov'ev} (Cham, Switzerland: Springer, 2019), 45\textendash 71.} Married to Dostoevsky's niece, Vladislavlev enjoyed close proximity to his famous relative even after the closure of Dostoevsky's journal. According to Joseph Frank, Vladislavlev "frequently invited his eminent uncle-in-law to meet the luminaries of the learned world" at his St.~Petersburg parties.\footnote{Joseph Frank, \emph{The Mantle of the Prophet, 1871\textendash 1881} (Princeton: Princeton University Press, 2002), 22.}

In contrast to Troitskii, whose attitude towards English empiricism Vladislavlev deemed lacking in "proper impartiality,"\footnote{M. I. Vladislavlev, "Zavisimost' nemetskoi filosofii ot angliiskoi. (Po povodu sochineniia g. Troitskogo 'Nemetskaia psikhologiia v tekushchem stoletii.')," \emph{Zhurnal Ministerstva Narodnogo Prosveshchenia}, v. 135 (1867): 179.} his own textbook on logic offered a more balanced approach and a chance for readers to judge for themselves the merits of other schools of epistemology, including the Kantian one.\footnote{M. I. Vladislavlev, \emph{Logika. Obozrenie induktivnykh  i deduktivnykh metodov myshleniia} (St.~Petersburg: Tip Demakova, 1872). The book contained an appendix over two hundred pages in length, in which Vladislavlev presented the historical development of various schools of logic and their main ideas.} Nevertheless, Vladislavlev found much in Mill's theory of knowledge that he was willing to accept without objection or with only minor criticism. Regarding the use of syllogism, for instance, Vladislavlev's position was fully consonant with Mill's: "One must admit," he wrote, "that syllogism is not the kind of tool by which we arrive at new knowledge as we do by means of induction."\footnote{Ibid., 204.} "Induction, by contrast, leads to the discovery of new general truths. By means of {[}induction{]} we learn of the connection between cause and effect."\footnote{Ibid., 213.} However, he found more use for syllogistic reasoning in the art of argumentation than Mill had.\footnote{Ibid., 207\textendash 8.}

On the question of the scope of modern science, Vladislavlev endorsed Mill's view that the primary task of scientific inquiry is to uncover the laws of nature by tracing causal connections between phenomena through inductive reasoning.\footnote{Ibid., 223.} Considering that Vladislavlev was a religious person, as evidenced by his comments on Kant's \emph{Critique},\footnote{See Nemeth, \emph{Kant in Russia}, 155.} his thoroughly empiricist approach to science and nature was one of the earliest attempts in Russian philosophy to find a compromise between the religious and empiricist worldviews by delineating their respective spheres as two different but compatible forms of human activity. It was also a sign of seismic changes in Russian academic philosophy, which now manifested a greater diversity of views.

In a chapter entitled "The impossibility of metaphysical explanation of the law of causality," Vladislavlev sided with Mill's empiricist notion of causality against different versions of metaphysical explanation, such as Leibniz's theory of pre-established harmony or Spinoza's notion of the boundless and endless Entity.\footnote{Vladislavlev, \emph{Logika}, 248\textendash 49.} "The question of interconnectedness of bodies or the causal connection between them is not amenable to metaphysical \emph{a priori} analysis," Vladislavlev argued. "It remains to turn to experience and accept it as a fact."\footnote{Ibid., 250.} He then cited Mill's definition of causality with a caveat that "{[}Mill{]} does not claim to explain the origin of causality," that is, the First Causes, but this limitation, according to Vladislavlev, is consonant with the level of knowledge in the natural sciences which only makes "empiricist generalizations" from facts. "We have to be content with that" because metaphysical analysis of causation presents "insurmountable difficulties, whereas \ldots{} the connection between cause and effect in nature is a fact which we have no right to deny despite being unable to explain it."\footnote{Ibid., 251.}

Vladislavlev was even more explicit, albeit diplomatic, in handling the question of religious implications arising from the empiricist theory of causality. This approach, he admitted, "has its thorny side. Apparently, it goes against some necessary religious and moral presuppositions."\footnote{Ibid., 251.} His own answer to this dilemma was an exemplar of intellectual tolerance that conceded some merit to the materialists' point of view but pondered the possibility of drawing the line between religion and science, between empirical evidence and faith:

\begin{quote}
Looking from their own viewpoint, materialists have some logic in their denial of the idea of the Creator \ldots{} they do not accept any proof of knowledge other than experience. However, they mistakenly believe that their viewpoint is the only one that exists \ldots{} and every other view makes no sense. By asserting that the world originated from the will of Creator out of nothing \ldots{} we step outside the boundaries of experience and, apparently, cannot test this supposition through the law of causality.\footnote{Ibid., 252\textendash 53.}
\end{quote}

\noindent However, the conclusion that Vladislavlev made after stipulating this concession did not flow entirely from the premises of the argument and was expressed, somewhat awkwardly, through analogical reasoning that would not have convinced the empiricists and might have been too rationalist for more traditional religious readers: "Therefore," Vladislavlev wrote, "we are warranted to believe that a number of physical causes originate in absolute, all-perfect cause which contains in it everything necessary for producing the material world (\emph{bytie})" in the same way as our free will is the cause of our moral actions.\footnote{Ibid., 253.} Vladislavlev left open the question of how science would be able to accommodate the conception of the divine mind if he himself acknowledged that scientific explanation admits only verifiable knowledge.

Two years after publishing his \emph{Logic}, Vladislavlev wrote a review of Mill's \emph{Autobiography} in which he weighed in on the most controversial aspect of Mill's self-characterization\textemdash his admission that he was raised without any religious education and that his father, the philosopher James Mill, had lost his faith early in life.\footnote{The \emph{Autobiography} was published in 1873, a few months after Mill's death. The Russian translation appeared the next year.} Regarding this startling confession, Vladislavlev commented, with a sense of sincere compassion, that Mill might have embraced religion had he been given the chance for proper religious instruction in childhood: "Mill hardly had a proper knowledge of Christianity. \ldots{} There is no mention {[}in his A\emph{utobiography}{]} that he ever held \ldots{} the New Testament in his hands. \ldots{} If this book had come to his attention in his youthful years, as a person who was impressed by heroes and humanity's benefactors, he would have certainly been impressed by the personality of Christ."\footnote{M. Vladislavlev, "Dzhon Stuart Mill'. Avtobiografiia." \emph{Zhurnal Ministerstva Narodnogo Prosveshcheniia}, 175 (1874): 122\textendash 23.}

\subsection*{K.P. Pobedonostsev contra J.S. Mill}

It is curious how much Vladislavlev's sympathetic comment contrasted with the harsh verdict that Konstantin Pobedonostsev, a top government official and prominent conservative thinker, delivered upon reading Mill's revelations. In 1873 he published a review of Mill's \emph{Autobiography} in the journal \emph{Grazhdanin} which was owned by the arch-conservative Prince Meshcherskii and edited at that time by Dostoevsky.\footnote{Considering that Pobedonostsev's article was published several months before the Russian translation of Mill's \emph{Autobiography}, he must have read the English, and thus, uncensored version of the book. Censorship cuts pertained primarily to Mill's revelations of how his father arrived at religious scepticism. See Berest, "J. S. Mill's \emph{Autobiography}," 38\textendash 40.} "Mill's Autobiography," Pobedonostsev wrote, "expresses the clearest and the most decisive rejection of religious truth and religious feeling, especially Christian one. But it is instructive to see from this book \ldots{} the degree of spiritual deformation which {[}human{]} nature can experience when the mind makes this rejection." For Pobedonostsev, unlike Vladislavlev, the story of Mill's life was one of emotional and spiritual "castration" (\emph{skopchestvo}) which he read with "horror" and "disgust."\footnote{Pobedonostsev, "Kartina vysshego vospitania. Avtobiografia Dzh. Stuarta Millia," \emph{Grazhdanin}, no. 45 (November 1873), 1193.}

In 1878, in an essay titled, "The Ultimate Purpose of Life," Pobedonostsev once again brou\-ght "the life of the most \ldots{} influential atheist\textemdash John Stuart Mill" to readers' attention as "an instructive example." This time the focus of his critique was the ethics of utilitarianism, but similarly to Strakhov, Pobedonostsev saw a close epistemological connection between the utilitarian and empiricist thinking. He posed the familiar question: What is "truth (\emph{istina})" for men like Mill? "They call truth the knowledge of facts and phenomena, causes and effects of the natural order of things based on observation and experience\textemdash in other words, the knowledge of nature. But what is nature from their point of view?" Pobedonostsev was convinced that by deconstructing nature into a set of "chemical and mechanical processes" devoid of religious meaning, scientific thought stripped the world of nature of its sacredness, thereby creating a profound dissonance between nature and human beings as moral creatures: "Mill tells us that nature is a force unworthy of our worship or approval."\footnote{Pobedonostsev's assumption regarding Mill's attitude towards nature was inaccurate. Mill was an armature botanist, an admirer of nature and something of an early environmentalist. See Reeves, \emph{John Stuart Mill}, 233.} Pobedonostsev agreed that "{[}f{]}rom a purely human point of view, nature is a monster. \ldots{} It knows neither a sense of justice nor compassion." He found it hard to understand how "atheists" might find happiness and a sense of moral satisfaction in studying nature if their conception of the natural world revealed nothing but nature's cruelty. By contrast, "the believer can see beyond nature to all-merciful God who mysteriously resolves through his person all contradictions."\footnote{Pobedonostsev, "Konechnaya tsel' zhizni," \url{https://azbyka.ru/otechnik/Konstantin_Pobedonosev/konechnaja-tsel-zhizni/}.}

"Nobody denies," Pobedonostsev admitted, "that there are many facts {[}of nature{]} which everybody needs to understand correctly" but this quest for factual knowledge driven by utilitarian goals has nothing to do with "devotion to truth for its own sake," which derives from religious devotion. It appears however, that Pobedonostsev viewed all scientists as "atheists" by default, with no hope that empirical science and religion might be reconciled. Certainty in the existence of divine truth in this world is achieved, according to Pobedonostsev, "only through faith and nothing else. No observation, no experience can provide it. One can even say that without faith observation and experience can only weaken or destroy this certainty." By the same principle, Pobedonostsev asserted that empirical methods of studying human behavior and morality can never yield a moral ideal that would lead to us to happiness. At most, experiential knowledge provides a guide to material conditions required for the pursuit of happiness, but the essence of happiness is to be found in religion alone. From this Pobedonostsev drew a conclusion that sounded more extreme than the premises of his own argument warranted: "Thus, reason takes away our faith (\emph{razum otnimaet nashu veru}), and with it our moral sense."\footnote{Ibid.}

\subsection*{Boris Chicherin's "Universalism" vs. Mill's Empiricism}

\noindent The most extended discussion of Mill's empiricism came from Boris Chicherin (1828\textendash 1904) in his book \emph{Science and Religion} (1879). It addressed the question of the relationship between modern science and religious faith by contextualizing it within the historical-philosophical dispute between empiricism and rationalism concerning the nature of knowledge.\footnote{For the role of religion in Chicherin's philosophical outlook, see Gary Hamburg, "Boris Chicherin: Christian Modernist," in Paul Valliere and Randall A. Poole, eds., \emph{Law and the Christian Tradition in Modern Russia} (London: Routledge, 2022), 132\textendash 50.} Similar to Strakhov, Chicherin was not content with the epistemological limits of empirical science, which focused, as he believed, exclusively on "the particulars" (\emph{chastnostnoe)}, providing only "mechanistic" explanations and moving "from one mystery to another" (a slightly modified quote from Mill's \emph{Logic}), being unable to resolve questions concerning First Causes and "the inner essence of things."\footnote{Boris Chicherin, \emph{Nauka i religiia} (Moscow: Tip. Martynova, 1879), xiv. The quote to which he refers is contained in a passage where Mill writes that science is unable to explain the "why" questions when it comes to the "general course of nature," only exploring \emph{how} it works. Mill, \emph{A System of Logic}, in \emph{CW}, VII: 471.}

Chicherin was also uncomfortable with the idea that science and religion should be treated as separate fields of knowledge in the manner proposed by Herbert Spencer and many other positivists. Such an idea, he argued, was nothing but an attempt to guard experiential knowledge from the influence of religion, relegating the latter to the realm of feelings and dreams unguided by reason.\footnote{Chicherin\emph{, Nauka i religiia}, 4\textendash 5.} At the same time, Chicherin emphasized that, in tackling these questions, he was speaking from a viewpoint of "science" rather than "theology": "Theological studies and contemplations (\emph{razmyshleniia}) can satisfy only those who are already convinced of religious truths but not those who have left the religious grounds." "Faith," he added, "is not to be taken blindly"; rather, "it is an objective principle" that itself needs to be rationally examined. However, Chicherin's concept of science differed principally from one shared by empiricists such as Mill. Much like Strakhov, Chicherin wrote that "true science (\emph{istinnaia nauka})" should be able to discern the expression of "the Spirit" in the world's phenomena. Only then can science and religion, which had "drifted apart further than ever before," be reunited into a more holistic system of knowledge, which he called "universalism."\footnote{Ibid., xiii, 88.}

What Chicherin proposed was a synthesis of empiricist and rationalist epistemologies, i.e. an attempt to integrate "experiential knowledge" with what the rationalist philosophers traditionally called "innate" concepts, such as time, space, mathematical notions and the absolute, which derive, as Chicherin argued, from "the laws of cognition" "immanent in human reason" ("\emph{korennoe svoistvo razuma"}).\footnote{Ibid., 17.} Taken separately, both methods are"one-sided" and unsatisfactory. Knowledge obtained through the senses does not reach beyond concrete phenomena, as they appear to us. "We can go from one observation to another, from one discovery to another, we can even explore the composition of the Sun" (and Chicherin noted that the natural sciences had indeed achieved "brilliant results" in terms of factual discoveries); however, by relying exclusively on empiricist thinking, "we only add one particular to another, never achieving a general principle that would satisfy our mind" in its quest for "the absolute." To the sceptics who doubted not just the need but the capacity of the human mind to go beyond the limits of experiential knowledge, Chicherin retorted that the very presence in our minds of questions about the origin and the meaning of phenomena suggests that "{[}the human{]} mind is meant for cognizing the absolute," in contrast to animals whose intellectual abilities are limited to the concrete and physical.\footnote{Ibid., xiv, 2, also 87. A similar argument (save the comparison with the animal world) had been previously expressed by theological philosopher Ivan Kedrov (1811\textendash 1846). Chicherin made no references to either Russian theologians or more recent Western literature on the issue at hand. On Kedrov, see Nemeth, \emph{Philosophy}, 46.}

Mill's version of empiricism served Chicherin as a major foil against which he expounded his vision of a "synthesis" between the spiritual and material worlds. Expectedly, his critique was concentrated on Mill's empiricist conception of knowledge, including his notion of causality, which Chicherin, similar to Strakhov, held to be purely mechanistic and narrow, focusing on "how" but not "why" various phenomena occur. He scored the most by attacking Mill's explanation of mathematical thinking, but had some difficulty refuting his view that analytic propositions (such as the law of contradiction) are propositions based in physical reality, which the mind is capable of formulating by means of observing its own mental activity (introspection). Chicherin offered the following example to prove his argument that the empiricist conception of the mind as a mere receptor and processor of experiential data was deeply mistaken:

\begin{quote}
Following Locke, the proponents of pure experience insist that logical propositions derive not from the laws of logic but from immediate introspection. \ldots{} {[}F{]}or instance, when we conclude that AB is equal to CD because both are equal to EF, anyone will agree with that, even though they might have never heard of the general law, according to which two lines equal to a third one are equal among themselves. But why is it that anyone will agree with that? \ldots{} It is because \ldots{} this law has always been immanent in his {[}man's{]} mind.\footnote{Ibid., 18.}
\end{quote}

\noindent According to Chicherin, when empiricists claim that they infer the laws of logic and mathematics by observing the mind's operations, they fail to recognize that the very ability of the mind to discern an instance of some general law in the physical world is possible because this law, in "its pure form," is already present in our minds. In other words, logical laws are innate and exist prior to the mind's experiential activity, not the other way around. The same goes for the "law of contradiction" which is "entirely obvious to us without any observation."\footnote{Ibid., 18\textendash 20.} However, a page earlier Chicherin had noted that "in children" "reason acts unconsciously" (\emph{razum deistvuet bezotchetno}), being unaware of its own laws. The question then arises of how and when exactly the mind becomes conscious of its innate concepts. Contrary to Mill's position that all knowledge is learned rather than innately given, Chicherin's insistence on the self-obvious nature of logical laws implied the equal ability of "anyone" to understand and utilize the laws of the mind in practical experience, but this is far from convincing. Even such a strong believer in innate knowledge as Whewell argued that \emph{a priori} conceptions are "innate \ldots{} {[}but{]} not self-evident."\footnote{Cobb, "Mill's Philosophy of Science," 244.}

In the same manner, Chicherin's argument for the innateness of "the absolute" was vulnerable to the objection already made by John Locke two centuries earlier. "It is well known," Chicherin wrote, "that positivism rejects any possibility of knowing the absolute. \ldots{} Yet it does not explain where our notion of the absolute comes from. If experience is the only source of human knowledge, it is obvious that this notion could never have appeared in human head"; the relative (as opposed to absolute) knowledge that we are able to derive from experience, "can never lead us to the notion of the absolute" or to the notion of "the infinite" since all objects that we observe are spatially defined. "The very existence of the notion {[}of the absolute{]} demonstrates the failure of positivism." The latter cannot explain "why {[}even{]} a little child is concerned with the deepest questions of existence, so far removed from all his sense-experiences."\footnote{Chicherin, \emph{Nauka i religiia}, 92, 94.} To this line of reasoning, long familiar in rationalist discourse, Locke had answered in \emph{An Essay Concerning Human Understanding} that the "Idea of God {[}is{]} not innate" because it is not present in atheists and certain other cultures, as evidenced by pre-Columbian American peoples in which "no notion of a God or religion" was observed upon their discovery.\footnote{John Locke, \emph{An Essay Concerning Human Understanding}, 2 vols. (New York: Dover Publications, 1959)1: 95\textendash 7; See also Markie, Peter and M. Folescu, "Rationalism vs.~Empiricism," \emph{The Stanford Encyclopedia of Philosophy} \url{https://plato.stanford.edu/entries/rationalism-empiricism/} accessed November 20, 2021.}

Despite Chicherin's aspiration to reunite the empiricist and rationalist epistemologies, it is hard to imagine how he could have convinced both sides. Looking at Chicherin's theory of knowledge from an idealist point of view, Nicholas Lossky criticized it for lacking "the suprarational" element that lies "outside the rational and the irrational." "In Hegel, due to his pantheism and his exaggeration of the wholeness of the world linked with it, there is a preponderance of universalism. \ldots{} In Chicherin, who was insufficiently aware of the world's organic wholeness, we find an exaggeration of individualism."\footnote{N. O. Lossky, \emph{History of Russian Philosophy} (New York: International Universities Press, 1972), 140\textendash 41. Lossky also notes that Chicherin's understanding of the process of intuiting ideas resembles Descartes' theory of knowledge, which "after Kant \ldots{} can no longer be accepted," as it lacks proof that man is endowed with the faculty of intuition. Ibid., 139.} From an empiricist standpoint, the fact that Chicherin's system ultimately terminated in a theistic explanation of the world presupposed that empirical science would have to relinquish its major epistemological principle\textemdash the idea that only provable and verifiable knowledge qualifies as scientific. Chicherin must have been aware of that, but he never addressed this issue explicitly, focusing instead on the supposed deficiencies of the empiricist picture of the universe. "For those who limit themselves to the realm of the material," he wrote, "the only force connecting the world \ldots{} is the general law of gravity," but this position leaves "entirely unexplained" why material particles have a tendency towards gravitation. This mystery becomes entirely intelligible, Chicherin argued, if we adopt the view that "all this visible universe is permeated by God's essence (\emph{Bozhestvennaya sushchnost}') which reaches out on all sides and connects it with invisible ties."\footnote{Chicherin, \emph{Nauka i religiia}, 113.} His proof for God's existence was purely metaphysical and well familiar from Western speculative tradition: "if there is being, there must be absolute being (\emph{esli est' bytie, to est' i absoliutnoe bytie),}" "the contingent necessarily presupposes the absolute (\emph{otnositel'noe nepremenno predpolagaet absoliutnoe})." From this Chicherin concluded, that the concept of "self-existing being (\emph{bytie samosushchee})" is the starting point of all reasoning and all being without which the search for causality would continue \emph{ad infinitum}.\footnote{Ibid., 104\textendash 5, 102.} The last argument was apparently directed at such sceptics as Mill who wrote defiantly in his \emph{Autobiography} that the question "who made God?" (along with the problem of evil) had led him away from religion, because his mind was not satisfied with the traditional answers.\footnote{J. S. Mill, \emph{Autobiography} (London: Oxford University Press, 1958), 36; see also Berest, "John Stuart Mill and his \emph{Autobiography} in Imperial Russia," 39.} It is not clear how Chicherin hoped to convince men with mindsets like Mill's to accept metaphysical arguments for God's existence.

Later in the book Chicherin's discussion of the science of psychology also showed that his promised synthesis of methods was in fact heavily titled towards rationalist and theological reasoning, which leading psychologists of his time had already left behind. The year Chicherin published his book, Europe saw the establishment of the first psychological laboratory as a culmination of the intellectual ideas that had their origins in seventeenth century empiricists. By the mid-nineteenth century, "the question on the minds of psychophysicists" was whether "an emerging field of psychology {[}could{]} divorce itself from purely metaphysical speculations about the working of the mind by adopting rigorous measurement and methodology," the same ones that "had worked so well in Newtonian physics, chemistry and other natural sciences."\footnote{Daniel J. Danis and Briana Young, "Methodology in psychology," in Robert J. Sternberg and Wade E. Pickren, eds., \emph{The Cambridge Handbook of the Intellectual History of Psychology} (Cambridge: Cambridge University Press, 2019), 33. Gustav Fechner's \emph{Elements of Psychophysics} was published in1860.} By the late 1870s the works of Gustav Fechner and Wilhelm Wundt transformed psychology into an empirical science, just as Mill had hoped, but Chicherin remained convinced that "purely experiential psychology" "relegates human being to the level of an animal by denying what makes him human and what explains the whole world of human relations," that is, the spiritual principle, which makes possible the existence of free will and morality.\footnote{Chicherin, \emph{Nauka i religia}, 147.}

Chicherin's remarks about psychology closely echoed Strakhov's apprehensions about the cultural effect of the deterministic tendencies implicit in modern science. He read the famous book by Ernst Renan \emph{The Future of Science} wherein the French thinker confessed that his newly found faith in science had led him to abandon his Hegelian-inspired anthropocentric beliefs. Far from being the center of the Universe, Renan wrote, humanity is perhaps as insignificant in the order of nature as "mold or mushrooms"\footnote{Strakhov, \emph{Bor'ba s Zapadom}, 3: 13.} Taking this analogy rather too literally, Strakhov responded by arguing that in Europe the ideas produced by the natural sciences have the effect of diminishing the status of man not only in the world of nature but also in society, going against the once cherished principles of Enlightenment:

\begin{quote}
Not only the successes of political and social sciences are weak, the very foundations of these sciences are being destroyed. The abstract notions of justice, equality, freedom, which have long ignited the minds, are losing their strength, giving place to \ldots{} the ideas of the lower grade. The source of this lowering is the same: the conclusions emanating from the natural sciences. People lowered themselves to the level of animals \ldots{} and even \ldots{} mushrooms.\footnote{Ibid., 27.}
\end{quote}

\noindent This may seem a startling argument for a political conservative who never advocated the principle of rights and freedom in Russia or elsewhere; it is clear, however, that Strakhov's only reason for mentioning it was the desire to pile up as many criticisms against the West as he could possibly think of, disregarding the fact that the rights of the individual in European countries, including the right to vote, continued to grow steadily during the period when the scientific mode of thinking was gaining ascendancy.

By contrast, Chicherin, coming from the camp of the westernizers, tried hard to find something positive to say about the West, pushing the blame to Russia instead. "Having immersed themselves exclusively in the study of particulars (\emph{chastnosti}), Western European communities," he wrote, "retained at least one redeeming feature\textemdash there the intellectual decline (\emph{upadok mysli}) is, to some extent, compensated by the {[}results of{]} the intellectual labor directed towards the study of factual material.\footnote{Chicherin, \emph{Nauka i religia}, xii.} This soothing conclusion fit awkwardly into Chicherin's general line of criticism against empirical science according to which the accumulation of factual data (if unaided by the proper metaphysical frameworks) presented little benefit to humanity and could even be detrimental, as in the case of empirical psychology. More surprising, perhaps, was Chicherin's further claim that Russia could not boast even these meagre results. "If a Western European can be compared to a miner (\emph{rudokop}) who tirelessly works for the future wellbeing \ldots{} we {[}the Russians{]} are content with waiting for the fruit of somebody else's {[}work{]}." "Russian thought has turned into an intellectual desert. \ldots{} {[}S{]}incere respect for science has almost disappeared" and "{[}n{]}ever before has Russian literature stood so low."\footnote{Ibid., xi-xii.} As Chicherin wrote this, we might recall, Dmitrii Mendeleev had already made his transforming discovery in chemistry, Ivan Sechenov had impressed the scientific community with his achievements in physiology, and Russian literature produced such giants as Tolstoy and Dostoevsky.

\subsection*{The \textit{Logic}'s New Editions in the 1890s}



\noindent Mill's \emph{Logic} remained in high demand in Russia until the end of the tsarist period. In 1892, a group of Moscow professors undertook the publishing of a bibliographic index for the benefit of readers seeking to enhance their education through self-study. The index provided guidance on the best scholarly literature in "all major branches of knowledge," including logic, which listed Mill's book, both in Russian and in English. The editorial note stated: "{[}It is{]} a classic work on inductive logic as well as the exposition of the basic principles of empiricist philosophy. It is a book necessary for anyone wishing to acquire philosophical education."\footnote{I. I. Ianzhul, P. N. Miliukov, P. V. Preobrazhenskii, L. Z. Morokhovtsev, \emph{Kniga o knigakh. Tolkovyi ukazatel' dlia vybora knig po vazhneishim otrasliam znanii} (Moscow: Tip. Inozemtseva, 1892), 25.} To make Mill's \emph{Logic} more accessible to novice readers, a new, abridged edition was issued in 1897. The modified and shortened title of this edition was apparently meant to attract the attention of beginners: Mill, \emph{Polozhitel'naia logika: obshchedostupnoe izlozhenie} (\emph{Positive logic in accessible exposition}).\footnote{D. S. Mill', \emph{Polozhitel'naia logika v obshchedostupnom izlozhenii}. S predisloviem A. P. Fedorov (SPb.: Ia. Kutenko, 1897). If anecdotal evidence can serve as an illustration of Mill's popularity among the general public during this period, Anton Chekhov's collection of aphorisms published in 1883 mentioned Mill's \emph{Logic} as the book that one opinionated "\emph{okolotochnyi}" (policeman) was reading (while on duty, apparently). See A. Chekhov, \emph{Polnoe sobranie sochinenii i pisem}, 30 vols (Moscow: Nauka, 1975), 2: 253.}

The 1897 edition was followed in 1900 by a new unabridged translation published by the Moscow professor of philosophy V. N. Ivanovskii. In the introduction, he summed up Mill's contribution to the field in the most laudatory language: "\emph{A System of Logic} by J. S. Mill," he wrote, "made an epoch in the development of logical theories and this merit is recognized by all scholars regardless of their philosophical school."\footnote{V. N. Ivanovskii, "Ot redaktora perevoda," in Dzhon Stuart Mill', \emph{Sistema logiki}. \emph{Sillogicheskoi i induktivnoi}, trans by V. N. Ivanovskii (Moscow: Delo, 1900), iii. This edition was reissued in 1914.} Ivanovskii was convinced that the book "can be and should be used by any scientist, indeed by any thinking person, whatever his profession."\footnote{Ibid., iv.} Ivanovskii himself used Mill's book in his courses and publications on psychology.\footnote{See V. N. Ivanovskii, \emph{Assotsiatsionizm v psikhologii i gnoseologii} (Kazan, 1905); V. N. Ivanovskii, \emph{K vorposu o genezise assotsianizma} (Kazan, 1910)} The 1900 edition was reissued in 1914.

As an indication of the \emph{Logic's} prominence in academic philosophy, an Instruction issued in 1899 to a graduate student at Kazan' University warned against focusing too much on Mill's inductive logic in preparing for final examinations. The student was urged to familiarize himself with more recent developments in logic, specifically mathematical logic, in addition to reading Mill's, book which remained a staple university textbook.\footnote{Bazhanov, \emph{Istoriia logiki}, 48\textendash 9.}

\subsection*{Conclusion}

\noindent In many ways, the reception of Mill's \emph{Logic} in Russian thought is a story of Russia's intellectual response to the challenges of scientific modernity which arrived in a land of autocracy and Orthodoxy with abruptness unseen in the West. Weakened by the years of Nikolaevan isolationist policies and continuously hedged around by censorship, Russian fledgling philosophy was confronted, in the 1860s, with the daunting task of responding to the questions that had preoccupied its Western counterpart for some decades. One of the most contentious of these questions concerned the nature and limits of modern science which no longer stood in a "cozy dovetailed alliance" with religion unlike its pre-modern version, including Newtonian physics, that presupposed the presence of the divine hand behind the laws of nature.\footnote{See Mander, "Introduction," 9; see also Singer, \emph{The Legacy of Positivism}, 19; Lynn S. Joy, "Scientific Explanation from Formal Causes to Laws of Nature," in Katherine Park and Lorraine Daston, eds., \emph{The Cambridge History of Science}, 5 vols. (Cambridge: Cambridge University Press, 2005), 3: 103.}

The response in British thought was complex and dynamic: after a period of empiricist hegemony spearheaded by Mill, the Idealist movement provided a robust philosophical alternative which initiated the process of reconfiguring the role of religion in the modern world. By the early 1900s British Idealism began to fade away as new varieties of realist and empiricist philosophies reasserted themselves, some claiming direct lineage to Mill.\footnote{Christopher Hookway, "Pragmatism," in \emph{The Cambridge History of Philosophy}, 74\textendash 5. The idealist tradition in Europe, while already in decline by the start of World War I, suffered a major blow as a result of the war, which undermined the belief in the concept of harmonious universe. See Leslie Armour, "The Continued Idealist Tradition," in \emph{The Cambridge History of Philosophy,} 428.} At the same time, the purely empiricist paradigm of science received a major correction in the early twentieth century with new developments in physics, especially Einstein's theory, which disproved Mill's claim that scientific knowledge can only be gained through empirical methods of inquiry.\footnote{See Reeves, \emph{John Stuart Mill}, 167; Nancy Cartwright, Stathis Psillos and Hasog Chang, "Theories of Scientific Methods: Models for the Physical-Mathematical Sciences," in Mary Jo Nye, ed., \emph{The Cambridge History of Science}, 5: 22.}

In Russia, Mill and Darwin came to represent major intellectual authorities that challenged the traditional religious worldview. Compared to Britain, the Russian response to Mill did not exhibit the same degree of originality and was more ideologically charged, not least of all because Mill's ideas were commonly perceived through the lens of Russia's cultural relationship with the West. On the one side of the debate there were largely uncritical followers of British empiricism who saw no need to look beyond Mill as late as the 1890s. In the opposite camp, the critics refused to acknowledge Mill's role in advancing the progress of empirical sciences, nor did they have much respect for empiricist methodology itself (if decoupled from religious metaphysics), even while acknowledging recent advances in experimental sciences. It is striking that despite political differences between Chicherin and Strakhov, both of them painted a deeply pessimistic picture of the cultural effects of modern science, pointing to the West as a source of intellectual decline and discordance. Fueling their pessimism was an anxiety that positivistic attitude widely accepted by adherents of empiricism would lead to the complete replacement and diminution of religion by science. This sentiment was shared by many religious apologists in the West throughout the 1860s but as time went on more of them opted for the separation of science and faith in hopes of preserving religion as a spiritual domain. An emerging line of argument held that science and religion were grounded in fundamentally different but equally valid epistemic positions that required different forms of evidential proof and certitude. What counts as evidence in science cannot be applied to religious belief, just as scientific explanation cannot admit arguments that require a leap of faith.\footnote{See Livingston, "The Defense of Faith," 331\textendash 5; Livingston, "The Sceptical Challenges to Faith," 323\textendash 27; Harvey, "Challenges to Religion," 535\textendash 6.} In Russia among those who engaged with Mill's \emph{Logic}, this line of reasoning was discernable in Vladislavlev, if very cautiously (perhaps for reasons of censorship), whereas Chicherin maintained an unrealistic expectation that science should reunite empirical methods with the search for the divine in external reality. Nevertheless, Chicherin's attempt at the "universalist" theory of knowledge was noteworthy for its new epistemological perspective, which provided an alternative to Mill's empiricism, in a manner that resembled Western philosophical developments. His \emph{Nauka i religiia} was also the most systematic and detailed treatment of the question of religion in its relation to science that appeared in Russian pre-revolutionary thought.

With the arrival of the Soviet regime, the long-awaited maturation and diversity of philosophical views in Russia came to a sudden halt. In the first post-revolutionary years, academic philosophy experienced what Bazhanov called "Bolshevik '\emph{filosofitsid}' (philosophycide)"\textemdash the purges of university instructors and exile of famous thinkers in an effort to create new cadres of "red professors" willing to serve the ideological goals of the Soviet power. Since Lenin had a low opinion of Mill's agnostic position on the question of Matter (\emph{materia}), the \emph{Logic} was tossed aside and the discipline of logic itself went into a steep decline in the 1920s.\footnote{Bazhanov, \emph{Istoriia logiki}, 97\textendash 120.}

\vspace{1em}
\begin{center}
  \includegraphics[width=0.75cm]{articlend.png}
\end{center}

\vspace{-1em}
\biobox{\textbf{Julia Berest} is Adjunct Assistant Professor at Western University (Canada). She received her PhD in history from Western University and her BA from Zaporizhzhia State University (Ukraine). Her major field is Russian and European intellectual history with a focus on reception studies. She is the author of a series of articles on the reception of J.S. Mill in pre-revolutionary Russia, and of an intellectual biography of Alexander Kunitsyn, one of the earliest Russian liberals and a champion of Kantian ideas in Russia: \textit{The Emergence of Russian Liberalism: Alexander Kunitsyn in Context (1783\textendash 1840)} Palgrave, 2011.}

\label{sec:berest}



% Define a custom command for chapter titles with the image

\newpage
\fancypagestyle{chaptercontentpage}{
  \fancyhf{} % Clear all header and footer fields
  \fancyhead[CE]{%
    \fontsize{11}{11}\leftmarkfont%
    \addfontfeature{LetterSpace=10.0}%
    \textit{\MakeUppercase{\leftmark}}%
  }
  \fancyhead[CO]{\authorheadfont\addfontfeature{LetterSpace=10.0}\fontsize{11}{11}\selectfont\textbf{{\uppercase{Daniela Steila}}}}

  \fancyfoot[RE]{\thepage}
  \fancyfoot[LO]{\thepage}
  \renewcommand{\headrulewidth}{0pt} % No header rule on content pages
}

\newpage{}
\abstractbox{The Subject and the Ideal}{A Critical Discussion among Russian Marxists before the Revolution}{Daniela Steila}{At the beginning of the 20th century, criticism of both the classical positivist materialism of the orthodox Marxists and the idealist tendencies of the "legal Marxists" was voiced by some important figures in Russian Marxism, whose positions on this aspect anticipated later critical theory. Thinkers such as A. A. Bogdanov and A. V. Lunacharskii interpreted the ideal as an essential dimension for the development of a critical reflection on the present, as a "different" standpoint from which to view and judge the world as it is. For both, such an Archimedean standpoint was neither given by historical determinism nor by transcendent values. Their "realism," which was not a mere recognition of the dynamics of social, economic and political reality, expressed an emotional affirmation of the creative life of humanity and a strong belief in its power. In this perspective, "critical" Marxism represented an alternative to orthodox Marxism. In contrast to contemporary "critical theory," however, it did not question the idea of a powerful human subject that conquers nature and history.}{Marxism, science, free will, determinism, ideals, critical theory, collectivism, individualism}

\section{Daniela Steila - The Subject and the Ideal}

\fancypagestyle{chaptertitlepage}{
  \fancyhf{} % Clear all header and footer fields
  \fancyhead[L]{\begin{minipage}[t]{0.7\textwidth}\publisher\end{minipage}}
  \fancyhead[R]{\begin{minipage}[t]{\textwidth}\raggedleft \datefont\fontsize{10}{11}\selectfont Volume 1 (2024): \thepage\textendash\pageref{sec:steila} \\ \doi{10.71521/q8sc-qk46} \end{minipage}}
  \renewcommand{\headrulewidth}{0pt} % No header rule on title pages
  \fancyfoot[RE]{\thepage}
  \fancyfoot[LO]{\thepage}
}
\chaptertitle{The Subject and the Ideal}{A Critical Discussion among Russian Marxists before the Revolution}{Daniela Steila}

\addcontentsline{toc}{chapter}{The Subject and the Ideal: \\ A Critical Discussion among Russian Marxists before the Revolution\\\textit{by} Daniela Steila}

\setcounter{footnote}{0}
\seriffont
\fontsize{12}{18}\selectfont

\noindent At the turn of the twentieth century, Russian Marxism experienced a
period of profound internal conflict and remarkable creativity. From the
1890s onwards, the "orthodox" interpretation of Marx's and Engels'
views on nature and history, represented by G. V. Plekhanov, was
confronted with the emergence of "legal Marxism," which was more
interested in the development of capitalism and the pursuit of
constitutional freedom than in the prospect of socialist revolution,
which was postponed to a distant future.\footnote{See A. Walicki,
  "Russian Marxism," in \emph{A History of Russian Philosophy.
  1830\textendash 1930. Faith, Reason, and the Defense of Human Dignity}, ed. G.
  M. Hamburg and R. A. Poole (New York: Cambridge UP, 2010), 305\textendash 308.}
At the beginning of the 20\textsuperscript{th} century, the newly formed
Social-Democratic Party split into two factions with different ideas
about political work and revolutionary goals. The Bolsheviks advocated a
more aggressive program of action, whereas the Mensheviks supported the
development of capitalism in Russia as a prerequisite for socialism
itself. The philosophical controversies that emerged further complicated
the landscape. Both the "orthodox" views, which were inspired by
Plekhanov's interpretation of historical and dialectical materialism,
and the "critical" views, which sought to combine Marxism with
contemporary epistemology, were to be found in both political factions.

This article will focus on what can be called "critical Marxism," that
is the diverse group of thinkers who took a common "critical" stance
toward the principles of \hspace{.1em}"orthodox" materialism. It is important to
note that the term "critical Marxists" has sometimes been used to
define legal Marxists because of their interest in
Kantianism.\footnote{See C. Henry, "Sergii Bulgakov's Early Marxism: A
  Narrative of Development," in \emph{Building the House of Wisdom.
  Sergii Bulgakov and Contemporary Theology: New Approaches and
  Interpretations}, ed. B. Hallensleben, R. Zwahlen, A. Papanikolaou, P.
  Kalaitzidis (Münster: Aschendorff Verlag, 2024), 351\textendash 352.} In this
article, the term "critical" is used in a different way, more in line
with the contemporary notion of "critical theory." In his synthesis,
Stephen Bronner asserts that "critically" oriented Marxists in the
West "were from the start dismissive of economic determinism, the stage
theory of history, and any fatalistic belief in the 'inevitable' triumph
of socialism. They were concerned less with what Marx called the
economic 'base' than the political and cultural 'superstructure' of
society," emphasizing the "utopian moment" and "the role of
ideology" in Marxism.\footnote{S. E. Bronner, \emph{Critical Theory. A
  Very Short Introduction} (New York: Oxford UP, 2011), 2.} The aim of
this article is precisely to show such a "critical" approach of
Russian Marxists in their attitude towards the "ideal." This is
manifested in their distancing themselves both from the classical
materialism of orthodox Marxists and from the tendencies of legal
Marxists towards "idealism."

"Critical" Marxism, which differs markedly from the orthodox theory
espoused and developed by Plekhanov or Lenin and subsequently developed
in the Soviet Dialectical Materialism ("Diamat"), has attracted
particular interest among Western scholars since the late 1960s. Jutta
Scherrer, who has written extensively on the subject, acknowledged that
"a new generation of historians, partly influenced by the spirit of
1968 and the search for a 'non-Soviet', 'human-faced' Marxism, sought to
liberate the historiography of social democracy and Russian Marxism from
its unambiguous fixation on Lenin. Bogdanov's collectivist thought was
'discovered' as an alternative to Leninism, and the group around
Bogdanov became known as the 'other Bolsheviks', as described by Robert
C. Williams."\footnote{J. Scherrer, "Ortodossia o eresia? Alla ricerca
  di una cultura politica del bolscevismo," in \emph{Gor'kij-Bogdanov e
  la scuola di Capri. Una corrispondenza inedita (1908\textendash 1911)}, ed. J.
  Scherrer and D. Steila (Rome: Carocci, 2017), 37. The reference is to
  R. C. Williams, \emph{The Other Bolsheviks: Lenin and His Critics
  1904\textendash 1914} (Bloomington-Indianapolis: Indiana UP, 1986).} In the
Soviet Union, Lenin's "rivals" enjoyed a brief period of popularity
during the \emph{perestroika} era for similar reasons. However, the
denunciation of Marxism in general soon erased all ideological
differences, and Marxism as a whole was generally disregarded as a
subject of reflection and interest.

In the historical context, "critical" Marxism did represent an
alternative to orthodox Marxism, which became the official Soviet
ideology. However, it would be wrong to consider it as a complete
alternative concept of socialism. In my conclusions, I will argue that,
unlike contemporary "critical theory," which questions the legacy of
the Enlightenment and modernity, early 20\textsuperscript{th} century
Russian "critical" Marxists could not challenge the idea of a powerful
human subject conquering nature and history. This is the fundamental
limitation of their critical perspective.

\subsection*{Marxism as a Science}

Once the "materialist conception of history"\textemdash as Marxism was often
called to avoid censorship\textemdash took hold in Russia, both in academic
discussions of political economy and among revolutionaries disillusioned
with Populism, one of the features that contributed to its success was
an understanding of the internal dynamics of history according to
principles that were as necessary as the laws of nature. One of the
reasons why many revolutionaries, including Georgii Valentinovich
Plekhanov, the so-called "father of Russian Marxism," abandoned
Populism and turned to Marxism was that it appeared to be a concrete and
"scientific" theory, in contrast to Populism, which relied on human
subjects and their personal decisions. Instead of trying to understand
the laws of history and directing "their revolutionary activity
accordingly," Plekhanov wrote, a typical populist Blanquist "merely
substitutes their conspiratorial skill for historical
development."\footnote{G. V. Plekhanov, \emph{Izbrannye filosofskie
  proizvedeniia} (Moscow: Gospolitizdat, 1956\textendash 1958), vol. 1: 127.} But
only a rigorous scientific explanation of history can lead to successful
practice. In Plekhanov's words: "To discover the laws, under the
influence of which the historical development of humankind takes place,
means to acquire the possibility of consciously influencing the process
of this development; it means to cease being a powerless plaything of 'chance' and to become its master."\footnote{Ibid., vol. 4: 425.}

In Russia, Marxism attracted revolutionary youth because of its
"scientific" form. Many years later, tracing his own philosophical
path, Semyon Frank emphasized that Marxism initially appealed to him as a
"scientific" worldview. He described it as the "idea that the life of
human society could be known in its regularity by studying it, as the
natural sciences study nature."\footnote{S. L. Frank, "Predsmertnoe
  (Vospominaniia i mysli)," in \emph{Vestnik Russkogo Khristianskogo
  Dvizheniia}, I, no. 146 (1986): 110\textendash 111.} In 1922, the poet Vladimir
Maiakovsky observed: "All my life I have been amazed by how Socialists
can disentangle facts and systematize the world."\footnote{V.
  Mayakovskii, \emph{Mayakovsky and His Poetry}, trans. H. Marshall
  (London: The Pilot Press, 1942), 15.} Marxism was conceived as a
science that would provide its adherents with a supposedly correct
understanding of the laws of history and enable them to act accordingly
within history, thereby guaranteeing the ultimate success of their
political actions. In this sense, Lenin claimed in 1913 that "the
Marxist doctrine is omnipotent because it is true."\footnote{V. I.
  Lenin, \emph{Collected Works} (Moscow: Progress Publishers,
  1972\textendash 1978), vol. 19: 23.}

The "truth" of the doctrine gave its adherents both the assurance of
their ultimate triumph and the moral justification to do whatever was
necessary to pursue a goal that was both the culmination of history and
their own. This type of conviction had deep roots in the
nineteenth-century intelligentsia. One cannot help but recall the words
of V. G. Belinskii in his letter to V. P. Botkin dated September 8,
1841: people "are so witless that they must be forcibly led to
happiness."\footnote{V. G. Belinskii, \emph{Polnoe sobranie sochinenii}
  (Moscow: AN SSSR, 1956), vol. 12: 71; trans. in \emph{Russian
  Philosophy}, ed. J. M. Edie, J. P. Scanlan, M.-B. Zeldin, G. L. Kline,
  3 vols. (Chicago: Quadrangle Books, 1965), vol. 1: 311.} Those who
possessed the truth were given the authority to lead or force others to
follow when they were unable to perceive and understand their genuine
interests. Marxism as a science could provide the "true" vision to
lead all humanity to its "true" happiness.

However, viewing Marxism as a science has implications not only for the
justification of the means to be used to achieve the "objective"
supreme goal of history, but also for the role of the individual in
history and their freedom of action. This was a familiar theme in the
history of Russian thought. As early as the 1860s, the debate between
positivists and anti-positivists had already made it clear that a
deterministic worldview would ultimately lead to the surrender of
individual freedom. It can be argued that only if human behavior is
understood as determined by natural laws, by the physiological
constitution of the human body, and by "rational egoism," is it
possible to develop a "human science" that is just as well-founded as
the natural sciences. In Turgenev's \emph{Fathers and Children}, Bazarov
states: "What's important is that twice two is four and all the rest's
nonsense."\footnote{I. S. Turgenev, \emph{Fathers and Sons}, trans. R.
  Freeborn (Oxford: Oxford UP, 1991), 44.} But when determinism pervades
all aspects of human life, the notion of human free will is reduced to a
trifle, with profound ethical consequences. As Dostoevsky observed in
his \emph{Notes from Underground}, a world governed entirely by
deterministic principles would reduce human beings to nothing more than
the keys of a piano or the pipes of an organ played by someone else's
hands, whereas human beings value their freedom above all else. In order
to assert themselves and their freedom, they are willing to renounce any
rational calculation and do something irrational or crazy, even to the
point of denying their own well-being, in order to disrupt the perfect
mechanism of the necessary laws of science and retain a modicum of
unpredictable irrationality.\footnote{F. M. Dostoevskii, \emph{Notes
  from Underground}, trans. M. Ginsburg (New York: Bantam Books, 1974),
  26\textendash 28; D. O. Thompson, "Dostoevsky and Science," in \emph{The
  Cambridge Companion to Dostoevskii}, ed. W. J. Leatherbarrow
  (Cambridge: Cambridge UP, 2002), 191\textendash 211.}

At the end of the nineteenth century, the German philosopher Rudolf
Stammler, in his polemic against Marxism, reiterated the contradiction
between human free will and historical determinism. His essay
\emph{Wirtschaft und Recht} had a great influence on the Russian
progressive intelligentsia, both among Populists and Marxists. The text
was first translated and published in the journal \emph{Severnyi
Vestnik} in 1898, then as a separate book in 1899 and again in two
volumes in 1907.\footnote{R. Stammler, \emph{Khoziaistvo i pravo s
  tochki zreniia materialisticheskogo ponimaniia istorii}, Prilozhenie k
  \emph{Severnomu Vestniku} 1, 10/12 (1898) (St. Petersburg: N. Berezin
  i M. Semenov, 1899; 2-oe izd. St. Petersburg: Nachalo, 1907).}
Stammler criticized Marx's efforts to reduce history to the laws of
economics and thus provide a scientific explanation for historical facts
that turned out to be inevitable and completely independent of human
will. Stammler argued that such a wholly consistent determinism would
lead to the abandonment of all struggle and would in no way inspire
revolutionary enthusiasm. Marx identified historical dynamics that would
have unfolded independently of human agency and commitment. But, as
Stammler's most famous illustration shows, no one would consider forming
a political party or revolutionary movement with the goal of achieving a
lunar eclipse, since this phenomenon depends solely on astronomical laws
that are completely indifferent to human action. If socialism is the
inevitable result of the laws of history, why bother to fight for it?
For Stammler, the fact that the Marxists called people to action
indicated that they themselves, more or less consciously, regarded human
effort toward a goal as a condition for the realization of the goal
itself.\footnote{R. Stammler, \emph{Wirtschaft und Recht nach der
  Materialistischen Geschichtsauffassung. Eine sozial-politische
  Untersuchung} (Leipzig: Veit, 1896), 432\textendash 433.}

In Russia, Pavel Novgorodtsev, a liberal philosopher with neo-Kantian
sympathies, succinctly summed up the central issue as follows: "It is
difficult to exaggerate the combination of fatalism and pragmatism
inherent in Marx's teachings. The fatalistic certainty of the inevitable
affirmation of the perfect condition actually reduces human action to
the level of a simple reflex in the objective course of events. What is
the point of calls to action and struggle if everything is ultimately
determined by the inescapable laws of history?"\footnote{Quoted in M.
  Kolerov, \emph{Idealismus militans: istoriia i obshchestvennyi smysl'
  sbornika "Problemy idealizma,"} in \emph{Problemy idealizma. Sbornik
  statei} (Moscow: Tri kvadrata, 2002), 87.} This question gave rise to
a great deal of lively debate among Russian Marxists, not only in terms
of its philosophical significance, but also in terms of its practical
implications. On the one hand, by providing solid guarantees for the
realization of the ideal, scientific socialism avoided the danger of
dissolving into a mere utopian fantasy. On the other hand, it was
susceptible to the potential pitfall of becoming a mere form of fatalism
that could lead to indifference and passivity.

\subsection*{The "Idealism" of Legal Marxists and Their Orthodox Critics}

Following the publication of Stammler's book, it became, as one
contemporary ironically observed, "impossible to be a Marxist writer
without having one's own philosophy, at least for domestic
use."\footnote{L. Gavrilovich, "Noveishie russkie metafiziki," in
  \emph{Voprosy filosofii i psikhologii} 75 (5) (1904): 647.} The
interest in philosophy, well documented among Russian Marxists since
their early critique of Populism, was now focused on the relationship
between historical necessity and political action.

In particular, the so-called "legal Marxists" sought to understand the
meaning of political action not in its conformity to the supposedly
necessary laws of history, but in its connection to social ethical
values. Sergii Bulgakov traced his intellectual path from Marxism to
Idealism in precisely this way: "The question of the social ideal,
which for me had previously been posed and completely resolved in the
field of positive Marxist sociology, gradually emerged from it and was
formulated more and more clearly as a religious and metaphysical
problem."\footnote{S. N. Bulgakov, \emph{Ot marksizma k idealizmu.
  Stat'i i retsenzii 1893-1903} (St. Petersburg: Obshchestvennaia
  pol'za, 1903), xvi. On the question of the ideal in Bulgakov's early
  works, see Caleb Henry, \emph{op. cit.}, 356\textendash 357.} In the process, he
credited Stammler with shattering the supposed scientific certainty of
Marxism: "As a result of the polemic with Stammler [\ldots{}] it had to
be recognized beyond doubt that the actual ideal of Marxism is not
provided by science, but by 'life', therefore it is \emph{outside} the
purview of science or it is \emph{non}-scientific."\footnote{Bulgakov,
  \emph{Ot marksizma k idealizmu}, ix.} Bulgakov himself and other
Russian Marxists of the time attempted to re-found the "social ideals"
on a moral basis. They did not reject the social ideals themselves;
rather, they sought to give these ideals a new and more solid
foundation. As Bulgakov himself stated: "Idealism aspires to perform
for social ideals the same function that economic materialism plays for
Marxism: it is a kind of new foundation laid under the old
edifice."\footnote{Ibid., vi.}

In a letter to Petr Struve, Nikolai Berdiaev expressed a similar
intention: "My greatest wish is to raise Marxism to the heights of
heaven, to give it an ultimately idealistic character."\footnote{M. A.
  Kolerov, "N. A. Berdiaev v nachale puti (pis'ma k P. B. i N. A.
  Struve)," in \emph{Litso: Biograficheskii al'manakh}, 3 (Moscow-St
  Petersburg: Feniks-Atheneum, 1993), 127.} In 1901, he published an
essay that would have a profound impact on his intellectual development
and that of his contemporaries, titled "The Struggle for
Idealism." Berdiaev identified himself as a "dissident Marxist" and
asserted that "the melody of positivism, naturalism, and hedonism has
been sung."\footnote{N. A. Berdiaev, "Bor'ba za idealizm," in
  \emph{Mir bozhii}, 6 (1901): 2.} Although he conceded that some
positivist claims were unavoidable in the field of natural science, he
rejected them outright in the field of philosophy and ethics. As for
history, Berdiaev later recalled in his autobiography that he never
fully embraced the Marxist view: "I accepted a materialist conception
of history, but I refused to ascribe a metaphysical meaning to it and to
link it to general philosophical materialism."\footnote{N. A. Berdiaev,
  \emph{Samopoznanie. Opyt filosofskoi avtobiografii} (Moscow: Kniga,
  1991), 123.} In his letters to Struve, he summarized the main points
of his "Struggle for Idealism" as follows: "It is we, the
representatives of the progressive social aspirations of modern times,
who must declare the struggle for (social-moral,
philosophical-religious, aesthetic) idealism. We must deny the
reactionary forces their right to idealism, those dark forces that boast
in the mud. We are the only ones who aspire upwards in all
respects."\footnote{Kolerov, "N. A. Berdiaev v nachale puti," 134.}
In his essay, Berdiaev proposed uniting practical and theoretical
idealism in order to "fight together against the social and cultural
bourgeois spirit (\emph{burzhuaznost'}) and prepare the human soul for
the future of society."\footnote{Berdiaev, "Bor'ba za idealizm," 14.}
In contrast to Bernstein's quietism, Berdiaev emphasized the profound
emotional dimension inherent in his own ideas. He wrote: "We need
breathtaking emotions to transcend the ugly vulgarity of the ordinary
gray life, to seize the enthusiasm without which nothing great has ever
been done in history."\footnote{Ibid., 34.} Political activists and
committed intellectuals should consider "the ideal goals that transcend
the material means of struggle."\footnote{Ibid., 23.} According to
Berdiaev, the "great task" of his time was "to introduce a moral
content into the social forms that the progressive forces of society
carry within themselves."\footnote{Ibid., 31.} In essence, Berdiaev
considered himself as a "realist" in his analysis of contemporary
economic processes. However, when he addressed the subject of revolution
and socialism, he did not imply that history would unfold independently.
Here, the intelligentsia was called upon to play a decisive role in
guiding the social movement through actions consistent with ideals.

As a result of his embrace of "idealism," Berdiaev contributed to the
remarkable success of the collection that appeared in late 1902
(officially dated 1903) under the title \emph{Problems of Idealism}.
Under the auspices of the Moscow Psychological Society, twelve authors
of varying renown compiled a collection of essays. They did not always
take the same positions, but shared a common interest in important
philosophical problems, especially ethics, that positivism seemed to
have neglected. In his preface, the editor, Pavel Novgorodtsev,
explained the failure of positivism "in the face of the complex and
ineradicable problems of moral consciousness, philosophical inquiry, and
living creativity. The light of philosophical idealism," he concluded,
"is necessary to meet these new challenges."\footnote{\emph{Problems
  of Idealism. Essays in Russian Social Philosophy}, trans. and ed. R.
  A. Poole (New Haven \& London: Yale UP, 2003), 83.}

For the authors of the collection, the term "idealism" was not used in
the context of abstract theoretical speculation. Their use of the word
had nothing to do with the metaphysical foundations of Hegelianism.
Rather, they regarded Kantianism as the basis for the moral ideals of
the individual and their efforts to realize those ideals in the world.
Instead of the certainty about the ultimate outcomes of history implied
by the fatalistic results of Marxist historical determinism, the Russian
"Idealists" held that individual ethical choices are of paramount
importance in all human action. In Caryl Emerson's summary: "Idealism
is completely alien to those sorts of naïveté that counsel us to await a
change in environment that will then bring about (for the most part
automatically) a change in the self. Such mechanical solutions are
castles in the air. In contrast, living by ideals is supremely
realistic, since coherence or justice is at no point expected from the
outside world or imposed upon it."\footnote{C. Emerson, "Foreword,"
  in \emph{Problems of Idealism}, xii.} In such a perspective, every
social and political commitment must be based on the moral choices of
the individual. Sergii Bulgakov observed that interpreting social
struggle "not merely as a confrontation of hostile interests, but as
the realization and development of a moral idea" does not diminish the
idealistic drive for political action, but rather strengthens it. He
continued: "Our participation in it will be motivated not by egoistic
class interest, but by religious duty, by an absolute order of the moral
law, by a dictate of God."\footnote{\emph{Problems of Idealism}, 118.}
When political commitment is based on ethics and not on historical
materialism, it will be even stronger and more powerful.

In the context of Russian Marxism, a number of highly respected figures
spoke out against the "idealist turn" of the authors of \emph{Problems
of Idealism}, especially those who distanced themselves from their own
earlier Marxist positions. Both Georgii Plekhanov and his follower
Liubov' Aksel'rod proposed a classical, Hegelian, and Spinozian solution
to the contradiction between determinism and freedom by identifying
freedom with the consciousness of necessity. In his 1898 essay on
\emph{The Role of the Individual in History}, Plekhanov was primarily
addressing the Populists, but he also made some remarks about Stammler.
Plekhanov conceded that "a party to facilitate a lunar eclipse could
only exist in a lunatic asylum," because "human action is not and
cannot be among the conditions whose conjunction is necessary for a
lunar eclipse." However, Plekhanov notes:

\begin{quote}
{[}I{]}n order for the example of the lunar eclipse to cease to be
meaningless [\ldots{}], one would have to assume that the moon is endowed
with a mind, and that the position in celestial space that causes its
eclipse is perceived by the moon as the result of the self-determination
of its own will, and not only gives the moon enormous pleasure, but is
essential to its moral calmness, leading it constantly to strive to
occupy that position. Having considered all this, one must ask how the
moon would feel if it finally realized that its motion in the celestial
space is not determined by its own will or "ideals," but rather that
its motion determines its own will and "ideals."\footnote{G. V.
  Plekhanov, \emph{Izbrannye filosofskie proizvedeniia}, vol. 2: 303.}
\end{quote}

\noindent If Stammler's hypothesis were correct, the moon would be paralyzed.
According to Plekhanov, in contrast, the most energetic practical action
can arise from the awareness of the necessity of a certain process.
Plekhanov adopted Spinoza's identification of freedom and necessity:

\begin{quote}
When the consciousness of the non-freedom of the will takes the form of
the complete subjective and objective impossibility of behaving
otherwise than one is currently doing, and when at the same time the
given actions are those that one considers the most desirable of all
possible actions, then necessity is identified in the mind with freedom,
and freedom with necessity. In this case, one is not free only in the
sense that one cannot overturn this identity of freedom and necessity;
one cannot oppose them; one cannot feel the constraint of necessity.
Conversely, such an absence of freedom is at the same time its fullest
manifestation.\footnote{Ibid., 304\textendash 307.}
\end{quote}

\noindent From the perspective of the human subject in history, "the
consciousness of the absolute necessity of a given phenomenon can only
intensify one's own energy if one is in sympathy with that phenomenon
and considers oneself as one of the forces that brought it
about."\footnote{Ibid., 308.} It is in this sense that Marxism should
be considered deterministic. In contrast to the voluntarism of Populism,
Plekhanov adhered to the scientific basis of Marxism and its
implications for individual action and the general philosophy of
history.

It was Plekhanov's follower Liubov' Aksel'rod who first attacked "legal
Marxists" from the standpoint of orthodox Marxism. She had been
concerned about the success of Kantian ideas in Russia since the late
1890s. In a letter to her mentor Plekhanov, while both were living as
émigrés in Switzerland, she expressed her concern: "Those who have
recently returned from Petersburg have informed me that the younger
generation there has simply gone mad over Kant. A considerable number of
clubs have been formed with the specific aim of studying the
\emph{Critique}. I can easily imagine the confusion and disorder that
must reign in their minds!"\footnote{\emph{Literaturnoe nasledie G. V.
  Plekhanova} (Moscow: Gosudarstvennoe social'no-ekonomicheskoe
  izdatel'stvo, 1934), sb. 1: 297.} As a philosophy student, she was
assigned by Plekhanov himself the task of criticizing the new currents
that were becoming so popular in Russia. In Plekhanov's words, Aksel'rod
could "use the philosophical information that {[}she{]} had acquired
and do socialism a great service by ridding it of neo-Kantian
vulgarities" by publishing in both revolutionary
journals abroad and legal journals in Russia.\footnote{Ibid., 283.}

Aksel'rod was convinced that materialism could "explain the real world
in terms of its inherent internal regularity, that is, in terms of
mechanical causation,"\footnote{L. I. Aksel'rod, \emph{O "Problemakh
  idealizma"} (Odessa: Kommercheskaia, 1905), 46.} and could be applied
to both natural and social sciences. However, the claim that every
phenomenon can be explained in terms of objective mechanical causation
does not preclude the possibility of ethical judgment and evaluation.
When Bulgakov observed that "from the perspective of mechanical
causality, one set of phenomena is no different from
another,"\footnote{Ibid., 47.} Aksel'rod responded: "We
{[}materialists{]} consider all phenomena that contribute to the
self-preservation of both individuals and society as progressive, and
all phenomena that delay or hinder such self-preservation as regressive,
although both types occur according to necessary causal
laws."\footnote{Ibid., 48\textendash 49.} Aksel'rod maintains that well-founded
ethical principles can be established within a materialist worldview,
since it is perfectly understandable that human beings develop moral
ideals and concepts throughout history as a result of historical
conditions. She gave several illustrative examples:

\begin{quote}
The pursuit of an ideal is the movement toward a specific, concrete,
more perfect future, the realization of which must take place here on
earth, within history. Morality means solidarity with one's fellow human
beings, the ability and willingness to sacrifice personal interests and
oneself for the benefit of society and future generations. Spiritual
improvement can be defined as the expansion and deepening of one's
spiritual personality to the point of merging one's life, suffering and
triumph with the life, suffering and triumph of the whole of
humanity.\footnote{Ibid., 7\textendash 8.}
\end{quote}

\noindent In Liubov' Aksel'rod's view, socialism did not need a moral foundation
in Kant's imperative in order to develop "ideals."

\subsection*{"Critical" Marxism}

The "orthodox" positions of Plekhanov and Aksel'rod were not the only
responses to the challenges of the new discussions on the "ideal" and
"idealism." Between 1899 and 1902, a particularly remarkable and
innovative group of exiled Marxists met first in Kaluga and then in
Vologda. Later, A. V. Lunacharskii recalled that

\begin{quote}
at that time there were few towns in Russia where such a circle of
Marxist forces could be observed. Besides, we were all united by a
certain original inclination. We were all deeply interested in the
philosophical aspect of Marxism, and at the same time we were eager to
strengthen its epistemological, ethical and aesthetic sides. This was
done, on the one hand, independently of Kantianism, to which a tendency
had already begun to develop\textemdash recently so noticeable both in Germany
and in Russia (Berdiaev, Bulgakov)\textemdash and, on the other hand, without
capitulating to the narrow orthodoxy of the French encyclopedists, on
which Plekhanov tried to base the whole of Marxism.\footnote{A. V.
  Lunacharskii, \emph{Vospominaniia i vpechatleniia} (Moscow: Sovetskaia
  Rossiia, 1968), 26.}
\end{quote}

\noindent The most original and creative thinker among these "unorthodox"
Marxists was Aleksandr Bogdanov, already known at the time as the author
of a very popular \emph{Short Course in Political Economy} and of a
philosophical essay, \emph{Basic Elements of the Historical View of
Nature}, which were intended "to respond to the extensive demands of
our workers for a general worldview."\footnote{\emph{Deiateli SSSR i
  revoliutsionnogo dvizheniia v Rossii} (Moscow: Sovetskaia
  entsiklopediia, 1989), 361.}

When Berdiaev, exiled in Vologda with Bogdanov, Lunacharskii and others,
published his first "idealist" article in 1901, both Bogdanov and
Lunacharskii attacked him as a traitor in lively debates that inflamed
the colony of exiled activists. As a psychiatrist, Bogdanov visited
Berdiaev for some time to examine his psychological condition,
postulating that Berdiaev's idealistic conversion could be explained by
a psycho-physical disorder.\footnote{Berdiaev, \emph{Samopoznanie}, 127.}
After the group of exiled political activists left Vologda, their
polemics continued in the press.

In particular, Bogdanov and Lunacharskii developed a worldview that
sought to offer an alternative to both orthodox Marxism and
"idealistic" Kantian revisionism. In their view, revolutionary values
could be grounded neither in the transcendental world of ethics nor in
the necessary laws of history. However, it was precisely these values
that required a solid foundation, since only a robust and reliable basis
could allow for a critique of reality and the successful implementation
of political and social action. Once both the ethical foundation of
idealist values and the fatalistic outlook of orthodox Marxists were
rejected, the question remained: how to formulate the project of the
future in such a way that it would not become a futile aspiration, but
rather a desirable, compelling goal that would inspire successful
action.

After the publication of \emph{Problems of Idealism}, such a question
became unavoidable. In February 1903, Lunacharskii intervened in the
discussion with a commentary on the recently published collection of
essays titled "\emph{Problems of Idealism} from the Standpoint of
Critical Realism." In this context, he explicitly asserted the
necessity of a "critical" standpoint, situated outside of existing
reality, in order to engage in criticism of reality itself. He wrote:
"In order to oppose reality, one obviously needs a point outside of it,
an Archimedean point on which to rely."\footnote{A. V. Lunacharskii,
  \emph{Etiudy kriticheskie i polemicheskie} (Moscow: Pravda, 1905),
  215.} But he thought that the "Idealists" went too far in putting it
in some metaphysical sphere, ultimately\textemdash according to
Lunacharskii\textemdash because they lacked courage and could not face the
uncertainty of reality itself. Consequently, they imagined a realm of
absolute values. In contrast, Lunacharskii, Bogdanov, and others sought
to establish their ideals on a realistic and "scientific" basis, while
at the same time avoiding any form of "fatalism." This position was
called "realism," and it was the common ground of a collection of
essays that appeared in 1903 in response to \emph{Problems of Idealism}:
\emph{Essays on a Realistic Worldview}. The broad first section included
three philosophical essays by Suvorov, Lunacharskii and Bazarov, all of
whom were very active participants in the discussions within the exile
community in Vologda. The second section dealt with "economic" issues,
with essays by Bogdanov and others. The third section presented a
miscellany, including an essay by N. Korsak, which was another pseudonym
for A. A. Malinovskii, alias Bogdanov.

In the introduction to the first edition in 1904 (probably written by
Bogdanov himself), idealism was rejected as the unhealthy consequence
of, paradoxically, ideal weakness, demoralization, distrust of human
possibilities, and retreat to metaphysical guarantees. The authors'
realism, on the other hand, was characterized by the rejection of all
metaphysical absolutes. This theoretical conception had strong practical
implications. It was a form of disenchanted realism closely linked to a
practical "idealism" of enthusiastic commitment to the fullness of life.
In the preface we read: "Steady consistency in knowledge and steady
consistency in life are two manifestations of the same principle.
Theoretical realism, as the expression of this principle in the sphere
of knowledge, and practical idealism, as its expression in the sphere of
life, are siblings in spirit."\footnote{\emph{Ocherki realisticheskogo
  mirovozzreniia. Sbornik statei po filosofii, obshchestvennoi nauke i
  zhizni}, 2-oe izd. (St. Petersburg: Izd. S. Dorovatovskogo i A.
  Charushnikova, 1905), v.}

The collective volume attempted to propose a new worldview, rather than
merely to criticize idealism. A few months after its publication,
Lunacharskii explicitly asserted the autonomous and original meaning of
the text. "We have come \emph{all by ourselves}\textemdash he wrote\textemdash to present
some problems that, to a superficial observer, appear to be very closely
related to reflections and discourses of the idealists. There is no
doubt that the ideas on which the 'realists' base their work would have
been articulated regardless of the appearance of the 'pious
philosophers' who shine with their halo of sanctity. Their appearance
was only a signal to us that it was time to come out!"\footnote{A. V.
  Lunacharskii, "Zhizn' i literatura," in \emph{Pravda}, 11 (1904):
  261.}

For the authors of the "realist" collection, realism was the
standpoint of a consistent critique of every form of metaphysics,
including the traditional materialist metaphysics of historical
necessity as well as the new metaphysics of transcendent values.
However, this did not imply a rejection of ideals and their significance
for human action. As Bogdanov noted as early as 1901, this form of
realism was opposed to "idolism," the metaphysical absolutization of
values and ideas, but not to "idealism." He explained that "\emph{the
characterisation of 'idealism' is applied to the manifestation of active
psychical life; feelings, desires, and deeds are considered to be
idealistic the more they are socially directed}. At the same time, this
characterisation always presupposes a real or only a conceptual
\emph{clash} between attitudes that are more social and attitudes that
are less social whereby the first is victorious. [\ldots{}] Idealism
signifies a victorious struggle of more social elements of the psyche
with less social elements."\footnote{A. Bogdanov, \emph{Toward a New
  World: Articles and Essays, 1901\textendash 1906: "On the Psychology of
  Society," "New World," and Contributions to "Studies in the
  Realist Worldview}," trans. D. G. Rowley (Leiden-Boston: Brill,
  2021), 21. Italics in the original.}

In discussing Stammler's ideas, which, as we have seen, were so
influential for the Marxists who embraced "idealism," Bogdanov noted
that Stammler's main error was his assumption "that everything other
than 'external norms' is nothing but 'individual' and 'accidental'." In
contrast, realism holds that "collective experience" is a powerful
force within social reality, capable of "bringing regularity into the
social life of people to a much greater degree than external norms."
Bogdanov continued:

\begin{quote}
Collaboration is inseparable from the commonality of experiences. Social
labour means social experience. The human psyche is a product of the
life of social labour, and no matter how
"individual" it is, a multitude of threads continuously tie it
together with the psyches of other people. The basic similarity of
biological organization, the same spontaneous forces of external nature
that people struggle against and overcome, the constant exchange of
thoughts and impressions\textemdash all of these things form a massive amount of
common experiences in the life of any given society.\footnote{Ibid.,
  369.}
\end{quote}

\noindent Ideals are an integral part of this collective experience. They derive
from the power of life itself, from the human desire for a better,
stronger, and fuller life for all the humanity. As expressions of
collective experience, they can direct human action toward their goal.

Lunacharskii also rejected the assumption that positivism \emph{per se}
negates the power of ideals, which was one of the fundamental tenets of
the entire collection of \emph{Problems of Idealism}. On the contrary,
he asserted that "positivism cannot ignore" the existence of the
ideal, which emerges from human life in its constant confrontation with
the natural and social environment. In Lunacharskii's words,

\begin{quote}
humanity aims not only at the knowledge of the external environment, but
also at the clarification of a corresponding program of action: how
should the environment be modified so that all the needs (including the
ever-present need for growth of forces) are satisfied as fully and
luxuriously as possible? How should the forces of humanity be organized
to achieve this goal more effectively? These are the basic questions of
positive idealism.\footnote{Lunacharskii, \emph{Etiudy kriticheskie i
  polemicheskie}, 235\textendash 236.}
\end{quote}

\noindent Since ideals are based on the growth of the vital forces in humanity,
Lunacharskii identified "the criterion for comparing ideals" in "the
fullness of life."

Clearly, the realists' conception of the "ideal" differed markedly
from the transcendent ideals espoused by Bulgakov, Berdiaev, and others.
Lunacharskii emphasized this distinction: "\emph{An ideal before us}
serves as a powerful motivator for action, while an \emph{ideal above
us} eliminates the need to work. It is already there, it exists apart
from us, and it is reached not by knowledge, struggle, or reform, but by
mystical divination, mystical ecstasy, and deep introspection. The more
the idealists strive to illuminate the kingdom of heaven, the more
tragic is the darkness they cast upon the earth."\footnote{\emph{Ocherki
  realisticheskogo mirovozzreniia}, 131.}

The ideal ahead was compatible with determinism, while at the same time
rejecting fatalism. Lunacharskii attempted to draw a clear distinction:
while fatalism is incompatible with freedom, since "it presupposes the
consciousness of a power existing outside us and against us,"
determinism, on the other hand, "does not contradict freedom at all. It
merely analyzes the fact of my freedom, finds that freedom is
\emph{mine}, i.e. it is determined by \emph{my} organism, which in its
turn is connected in a chain of phenomena." Lunacharskii concluded:
"The same determinism teaches that no \emph{action} \emph{can} take
place without consequences, and that we can always rely on certain laws
to produce the desired result of a finalistic impact on the
environment."\footnote{Ibid., 236\textendash 237.} Such determinism provided a
solid foundation for human action and guaranteed its success. However,
it did not contradict the role of ideals, which arise from the human
desire for a fuller and stronger life.

For both Bogdanov and Lunacharskii, probably the most original thinkers
among the non-Orthodox Russian Marxists at the early twentieth century,
the Archimedean point\textemdash which was necessary to develop a critical
attitude toward reality and ultimately to overthrow it\textemdash was not
provided by the certainty of objective structural laws of historical
development, which was the core of Plekhanov's Marxism. Nor was it
provided by the transcendent values dear to the authors of
\emph{Problems of Idealism}.

Nevertheless, Bogdanov and Lunacharskii held different views on the
foundation of their critique of reality. Bogdanov believed that
objective knowledge could be based on the collective experience of
humanity. In his view, "the characterization of 'objectivity'
altogether cannot be based on individual experience\textemdash neither the
stability of its composition nor the harmony between the results of
activity and the data of experience that is the starting point of that
activity. The basis of 'objectivity' must lie in the sphere of
\emph{collective} experience."\footnote{A. A. Bogdanov,
  \emph{Empiriomonism. Essays in Philosophy,} books 1\textendash 3, trans. and ed.
  D. G. Rowley (Leiden-Boston: Brill, 2019), 18. Italics in the
  original.} Although Bogdanov emphasized the collective character of
human experience, it is important to note that "there is no place for
the absolute in the sphere of human experience; everything there is
relative."\footnote{Bogdanov, \emph{Toward a New World}, 29.} In the
course of history, humanity develops worldviews that prove increasingly
effective in organizing human experience. However, none of these
worldviews, including Marxism, can claim absolute truth. According to
Bogdanov, "for a philosophy that takes a historical perspective, there
is neither absolute truth nor absolute error. Such a philosophy is
obligated to find in every error that portion of relative truth which
justified belief in it, just as it strives to find in every truth that
portion of error that requires us to move on from this truth to another,
higher truth."\footnote{Bogdanov, \emph{Empiriomonism}, 133.} Plekhanov
and Lenin both argued that there is an "objective" criterion of truth,
and that this criterion is located not in the subject, but in the
relations that exist in the external world. As Plekhanov summarized:
"Those views are \emph{true} which correctly represent these relations;
those views are \emph{false} which distort them. The theory of natural
science is \emph{true} when it correctly grasps the mutual relations
between the phenomena of nature; a historical description is \emph{true}
when it correctly depicts the social relations existing in the epoch
described."\footnote{Plekhanov, \emph{Izbrannye filosofskie
  proizvedeniia}, vol. 1: 671.} Conversely, Bogdanov proposed a
criterion for selecting the most optimal of various ideals, defined in
the broadest collective sense as the highest ideal. He wrote: "Since
the essence of idealism consists in the social nature of its frame of
mind, the more social the ideals are, the more idealistic they are."
For example, "an ideal that does not go beyond the confines of the
relationship of life of a limited group of people is lower than an ideal
whose content embraces the life of all society."\footnote{Bogdanov,
  \emph{Toward a New World}, 24.}

Lunacharskii believed that the standpoint of effective critique of
reality and the basis of progressive ideals was the tension of life
towards the full realization of its power and strength. Unlike
traditional metaphysical thinkers, who "worship the existing and close
their eyes to everything else," "realists" aim to "enlighten and
unify the whole of reality."\footnote{R. Avenarius, \emph{Kritika
  chistogo opyta v populiarnom izlozhenii A. Lunacharskogo} (Moscow:
  Izd. S. Dorovatovskogo i A. Charushnikova, 1905), 122\textendash 123.} Their
monistic worldview could not be limited to the representation of
reality, but had to become the framework and orientation for human
active intervention in the natural and social environment. According to
Lunacharskii (as well as to Bogdanov), the values that guide human
action in the world cannot be absolute. He wrote: "Nothing in the world
is inherently good or evil. Nothing possesses intrinsic value, outside
of its relation to a sentient organism."\footnote{Lunacharskii,
  \emph{Etiudy kriticheskie i polemicheskie}, 396.} A thing is
considered good for a given subject if it enhances its ability to live
or the depth of its life. Ultimately, the criterion of value must be
one's own pleasure. Lunacharskii conceived of this as "\emph{edoné},
the wavering \emph{joy of life}," "the joyful feeling of the growth of
one's inner strength."\footnote{Ibid., 155, 172.} Inspired by both
Nietzsche and Avenarius, Lunacharskii maintained that "the love of
life, of nature, a boundlessly increasing tendency to happiness" was
the basis of his "assessment of the world." Furthermore, he noted that
this perspective was "completely pagan," since it "has nothing to do
with the morality of duty, since it \emph{does not subject} the human
being \emph{to anything}," but pursues "the fullest and most
harmonious possible existence."\footnote{Ibid., 405. See D. Steila,
  \emph{Nauka i revoliutsiia. Retseptsiia empiriokrititsizma v russkoi
  kul'ture}, trans. O. Popova (Moscow: Akademicheskii Proekt, 2013),
  165\textendash 177; B. G. Rosenthal, \emph{New Myth, New World. From Nietzsche
  to Stalinism} (University Park: The Pennsylvania State UP, 2002).}

\subsection*{An Alternative Marxism?}

The "critical" Marxism proposed by thinkers such as Bogdanov and
Lunacharskii developed as an alternative to both orthodox Marxism and
the "idealism" of the legal Marxists. The concept of the subject and
its capacity to act in the world was central. In contrast to the
individualistic perspective of the legal Marxists, the "critical"
Marxists posited that the individual is inextricably linked to the
collective of humanity, which they identified as the true subject of
history. In contrast to the orthodox Marxists, they argued that absolute
truth was impossible, and that this lack of absolute truth precluded the
possibility of ultimate guarantees, including those that might have been
derived from the necessary laws of history. Nevertheless, the
"critical" thinkers examined here seem to share a fundamental and
unshakable conviction that the pursuit of individual personal
aspirations would not conflict with the needs of collective development.
Lunacharskii asserted that the diversity and plurality of individual
ideas was a necessary condition for the progress of the collective
community. In his 1909 contribution to the \emph{Essays of a
Collectivist World-View}, Lunacharskii declared that "the development
of 'individuality' and spiritual originality cannot but be highly valued
in socialist society for the same reasons that it will never renounce a
certain degree of specialization within various branches of labor. The
presence of a multiplicity of ideas, an abundance of different points of
view, hypotheses, directions, provides the most successful approach,
since the fundamental law by which ideas are improved is their conflict
and the triumph of the most viable."\footnote{\emph{Ocherki filosofii
  kollektivizma} (St. Petersburg: Znanie, 1909), 253.} Bogdanov, for his
part, was aware that there could be a conflict between individual choice
and collective organization, but he was confident that the statistical
laws of large numbers would eventually reconcile freedom and necessity.
In the socialist society of Mars, as depicted in Bogdanov's utopian
novel \emph{Red Star}, the figures provided by the Central Institute of
Statistics indicate the number of workers required in each area of
production, and hundreds and thousands of workers redistribute
themselves accordingly. This ensures that the labor necessary for the
survival and development of society is obtained without forcing anyone
to do a job they do not freely and spontaneously choose. In Bogdanov's
words: "The statistics continually affect \emph{mass} transfers of
labor, but each individual is free to do as he chooses."\footnote{A. A.
  Bogdanov, \emph{Red Star. The First Bolshevik Utopia}, ed. L.R. Graham
  and R. Stites, trans. Ch. Rougle (Bloomington: Indiana UP, 1984), 68.}

The attempt to realize socialism, to which they had devoted their lives,
seemed to disprove the fundamental optimism that the harmonization of
individual and collective development was possible. In the late 1920s,
Bogdanov observed that the supposedly new social system was organized
according to the authoritarian models that he had criticized throughout
his life. Moreover, Bogdanov noted that the triumphant proletariat was
turning into a "herd" of slaves. He wondered: "What did \emph{I} want
to do with Marxism and what did \emph{they} do with it?"\footnote{RGASPI
  (Rossiiskii gosudarstvennyi arkhiv social'no-politicheskoi istorii),
  f. 259, op. 1, d. 48, l. 111. The words in italics are underlined in
  the original.} Instead of the universal organizing class that was
expected to transform the entire world, Bogdanov saw a herd in search of
a leader. He jotted down his thoughts in his notebook: "A class that
needs absolute leaders is still by its very nature a subordinate
class," incapable of assuming responsibility for the new economic,
social, and cultural organization.\footnote{Ibid., l. 43.} Lunacharskii,
who played a more active role in building the Bolshevik regime than
Bogdanov, seemed to question the moral justification of the atrocities
committed in the name of the revolution toward the end of his life.
According to his daughter's account, he expressed bitter disillusionment
in a diary entry dated 1930:

\begin{quote}
Of course, I am a revolutionary on behalf of a tremendous flowering of a
strong, bright, and just culture. But when you chop wood, the chips fly.
Let us suppose that I myself did nothing disagreeable. Even if it were
justified by the revolution, but still disagreeable. However, I cannot
hide from myself the fact that, in the final analysis, I must answer for
everything. [\ldots{}] Yes, it would be impossible to improve this
terrible society without the revolution. But at what price will victory
come? And will it indeed come? The price has been paid,
but \ldots{}\footnote{The quotation from Lunacharskii's diary is provided
  in I. Lunacharskaia, "Why Did Commissar of Enlightenment A. V.
  Lunacharskii Resign?" in \emph{The Russian Review}, vol. 51, n. 3
  (1992): 341.}
\end{quote}

\noindent Both Bogdanov and Lunacharskii considered their ideal society to be a
system in which each individual could find perfect fulfillment in the
wholeness of humanity. Furthermore, both agreed that in order to achieve
such an ideal, humanity must overcome its conflicts and contradictions
with the natural environment. The lively discussions on the subject of
individual death that took place among the so-called "other
Bolsheviks" demonstrate the importance of their belief that even death,
as the ultimate inescapable contradiction between human being and
nature, would be overcome in the context of the collective immortality
of humanity.\footnote{See R. C. Williams, "Collective Immortality. The
  Syndicalist Origins of Proletarian Culture, 1905\textendash 1910," in
  \emph{Slavic Review} 39, 3 (1980): 389\textendash 402; R. Tartarin,
  "Transfusion sanguine et immortalité chez Alexandr Bogdanov," in
  \emph{Droit et société} 28 (1994): 565\textendash 581; D. Steila, "Death and
  Anti-Death in Russian Marxism at the Beginning of the 20th Century,"
  in \emph{Death and Anti-Death}, ed. Ch. Tandy (Palo Alto: RIA UP,
  2003), 101\textendash 130; Rosenthal, \emph{New Myth, New World}, 79\textendash 93; A.
  Bernstein, \emph{The Future of Immortality. Remaking Life and Death in
  Contemporary Russia} (Princeton-Oxford: Princeton UP, 2019), 67\textendash 70.}
However, the transformation of nature by humanity through the
"humanization" of nature, as Marx had conceptualized it, became for
Russian "critical" Marxists a task of domination and conquest, of
technological subjugation of spontaneous and disorganized natural
forces, which were therefore perceived as potentially dangerous.
Lunacharskii was perhaps the most explicit in declaring that the new
worldview takes nature "as a spontaneous force, a half-cosmos, a task,
and a source of strength and joy."\footnote{A. V. Lunacharskii,
  "Budushchee religii," in \emph{Obrazovanie} 11 (1907): 60.} The
contradictions "between the laws of life and the laws of nature" are
overcome by scientific socialism through the introduction of "the idea
of the victory of life, the subjugation of spontaneity to reason through
knowledge and labor, science, and technology."\footnote{A. V.
  Lunacharskii, \emph{Religiia i socializm} (St. Petersburg: Shipovnik,
  1908), vol. 1: 40\textendash 42.} In order to achieve this goal, it is necessary
for humans "to boldly torture nature everywhere, and overcome its
always alleged limitations."\footnote{A. V. Lunacharskii, "Dvadtsats'
  tretii sbornik \emph{Znaniia}," in \emph{Literaturnyi raspad}, kn. 2
  (St. Petersburg: EOS, 1909), 92.} Lunacharskii notes that the new
humanity conceives of its "ideal of happiness and community" not as a
"mystical dream," but rather as "the plan according to which it must
rebuild the world." He continued: "Human beings found themselves as
gods in labor, in technique, and decided to impose their will on the
world. With a hammer, an iron hammer they will destroy what is formless
and evil. With the same hammer they will forge their golden
happiness."\footnote{Lunacharskii, \emph{Religiia i} \emph{socializm},
  vol. 1: 104.}

The assumption that the human subject is the mighty conqueror of nature
and the ultimate ruler of the universe was one that "critical"
Marxists could not criticize. It could be said that this idea is deeply
embedded in the whole of modernity, which Marxism ended up sharing with
capitalism because of the common conviction that science and technology
will eventually overcome the misery and suffering of humanity. Today,
however, as we face the catastrophic consequences of climate change and
as technology is often used to facilitate ever more destructive
aggression and devastation, it becomes imperative to reconsider this
notion of a powerful human subject. A consideration of the reflections
on the meaning of the "ideal" and the relationship between freedom and
necessity that developed in Russian Marxism at the beginning of the
twentieth century reveals that the traditional modern conception of
human subjectivity was not questioned by orthodox Marxists, their
"critics," or even the "idealists." The socialist "dream" became a
nightmare not so much because of the power of the collective over the
individual, but rather because of the acritical assumption of the
modern idea of a powerful, strong, and independent subject, responsible
for dominating the natural and social environment. The tragedies of the
twentieth century, and the continuing horrors of the twenty-first,
suggest that we should see ourselves as vulnerable, weak subjects, in
need of attention and care, and always dependent on others.\footnote{See,
  for instance, J. Butler, \emph{Precarious Life. The Powers and
  Mourning of Violence} (London-New York: Verso, 2004);
  \emph{Vulnerability. Reflections on a New Ethical Foundation for Law
  and Politics}, ed. M. Albertson Fineman and A. Grear (London-New York:
  Routledge, 2016); Adriana Cavarero, with Judith Butler, Bonnie Honig,
  and Other Voices, \emph{Toward a Feminist Ethics of Nonviolence}, ed.
  T. J. Huzar and C. Woodford (New York: Fordham UP, 2021).} By
embracing human vulnerability and exposure as something to be celebrated
and cherished, we may be able to develop ideals that reject ultimate
metaphysical guarantees, but still direct our actions in the world
toward a better future of greater justice and solidarity, precisely the
goal of the Russian revolutionaries' ideal.

\vspace{2em}
\begin{center}
  \includegraphics[width=0.75cm]{articlend.png}
\end{center}

\biobox{\textbf{Daniela Steila} is Professor of History of Philosophy at the University of Turin (Italy). Her scholarly research and writing have focused on Russian philosophy of the 19th and 20th centuries (particularly Russian Marxism and materialism, P. Ia. Chaadaev, S. L. Vygotskii, M. K. Mamardashvili) and the reception of Western European philosophical traditions in the Russian and Soviet context (empiriocriticism, Spinoza, Kant, pragmatism).}

\label{sec:steila}

\section{Brad - Nothing From Nothing}

\abstractbox{Nothing From Nothing}{The Underground in Sergei Bulgakov and Nikolai Berdiaev}{Bradley Underwood}{The Underground Man, in Russian literature and in European philosophy,~is so familiar that~it's easy to take~him for granted. This was not always the case. Two of Russia's greatest minds, Nikolai Berdiaev and Sergei Bulgakov, were keenly aware of his philosophical potential. This article~discusses~their use of~the Underground Man to explore the problems of freedom and evil,~foregrounding~the cosmic backdrop of their inquiries. The Underground Man, in their view, seeks a freedom that resembles the void from which God fashioned humans. Evil is coterminous with the freedom to unmake what God has made. Berdiaev and Bulgakov regarded the Underground Man as a mode of de-creation. His mission is metaphysical suicide. While he fails to achieve his aim, Berdiaev and Bulgakov presumed that he comes sufficiently close to~that goal to~infer the fundamental or primordial elements of the cosmos. Here, Berdiaev and Bulgakov diverge, challenging the perception that they are complementary role-players for the same ideas. Both thinkers differ over the "nothing" towards which the underground is headed. Bulgakov's conception of~a "primal nothing"~was more radical than Berdiaev's,~which better positioned him to argue that~humans have free will but can also be confident that evil will one day be defeated. Evil is too unnatural, too much like nothing, to last.}{Sergei Bulgakov, Nikolai Berdiaev, evil, freedom, creation \textit{ex nihilo}, abyss, soul, nothingness, \emph{mē on, ouk on,} creativity}

\fancypagestyle{chaptertitlepage}{
  \fancyhf{} % Clear all header and footer fields
  \fancyhead[L]{\begin{minipage}[t]{0.7\textwidth}\publisher\end{minipage}}
  \fancyhead[R]{\begin{minipage}[t]{\textwidth}\raggedleft \datefont\fontsize{10}{11}\selectfont Volume 1 (2024): \thepage\textendash\pageref{sec:underwood} \\ \doi{10.71521/s31r-ah05} \end{minipage}}
  \renewcommand{\headrulewidth}{0pt} % No header rule on title pages
  \fancyfoot[RE]{\thepage}
  \fancyfoot[LO]{\thepage}
}
\fancypagestyle{chaptercontentpage}{
  \fancyhf{} % Clear all header and footer fields
  \fancyhead[CE]{%
    \fontsize{11}{11}\leftmarkfont%
    \addfontfeature{LetterSpace=10.0}%
    \textit{\MakeUppercase{\leftmark}}%
  }
  \fancyhead[CO]{\authorheadfont\addfontfeature{LetterSpace=10.0}\fontsize{11}{11}\selectfont\textbf{{\uppercase{Bradley Underwood}}}}

  \fancyfoot[RE]{\thepage}
  \fancyfoot[LO]{\thepage}
  \renewcommand{\headrulewidth}{0pt} % No header rule on content pages
}
\chaptertitle{Nothing From Nothing}{The Underground in Sergei Bulgakov and Nikolai Berdiaev}{Bradley Underwood}

\addcontentsline{toc}{chapter}{Nothing From Nothing:\\ The Underground in Sergei Bulgakov and Nikolai Berdiaev \\\textit{by} Bradley Underwood}

\setcounter{footnote}{0}
\seriffont
\fontsize{12}{18}\selectfont

\epigraph{}{"Nothing will come of nothing. Speak again"\textemdash \textbf{Lear, \emph{King Lear}}}
\vspace{1em}
\noindent Dostoevsky boasted in his notebooks that "I alone deduced the tragic
essence of the underground."\footnote{This essay would not have been
  possible were it not for the immense generosity of Caryl Emerson,
  Randall Poole, Susan McReynolds, and Yuri Corrigan. The epigraph is
  from William Shakespeare, \emph{King Lear}, ed. R.A. Folks (London:
  Bloomsbury, 1997), 164. Fyodor Mikhailovich Dostoevsky, \emph{Polnoe
  sobranie sochinenii v tridtsati tomakh} (Leningrad: Nauka, 1976), vol.
  16: 329\textendash 330. All translations from Russian texts are mine unless
  stated otherwise.} "The underground" and its anonymous inhabitant,
the Underground Man, remain among the most novel contributions of the
Russian intellectual tradition. The Underground Man's liminal
personality and hyperbolic arguments have served as a touchstone for
psychological and philosophical analysis: for psychologists, to
understand the nuances of spite and resentment; and for philosophers, to
develop insights into language and consciousness.\footnote{See, for
  example, Peter Shabad, "Giving the Devil His Due: Spite and the
  Struggle for Individual Dignity," \emph{Psychoanalytic Psychology}
  17, no. 4 (2000): 690\textendash 705; Garry Hagberg, "Wittgenstein
  Underground," \emph{Philosophy and Literature} 28, no. 2 (2004):
  379\textendash 392.} His tragic thirst for freedom galvanized the existentialist
movement in philosophy, which continues to pack introductory courses
with similarly restless undergraduates.

The aim of this article is to delineate how two Russian
philosophers\textemdash Nikolai Berdiaev (1874\textendash 1948) and Sergei Bulgakov
(1871\textendash 1944)\textemdash use the underground to grapple with the problems of free
will and evil. My argument begins with the premise that Berdiaev and
Bulgakov approach the Underground Man as someone whose will is grounded
in "nothing," a concept invested with cosmic significance. The
Underground Man, in their view, seeks a freedom that approximates the
void from which God fashioned humans. For both thinkers, evil is
coterminous with the liberty to unmake what God has made. Yet Berdiaev
and Bulgakov differ in their understanding of the "nothing" towards
which the underground is headed.

The underground thus provides a prism through which to discern vital
differences between two eminent thinkers from the Russian Silver Age
(1890s\textendash 1920s) in Berdiaev and Bulgakov.\footnote{These parameters for
  the "Russian Silver Age" are given by Catherine Evtuhov, \emph{The
  Cross and the Sickle: Sergei Bulgakov and the Fate of Russian
  Religious Philosophy, 1890\textendash 1920} (Ithaca, NY: Cornell University
  Press, 1997), 65.} Such dissimilarities are at times obscured due to
their intellectual and biographical overlap. Berdiaev and Bulgakov made
the transition from Marxism to religious belief after encountering the
philosophical idealism of the Moscow Psychological Society.\footnote{Randall
  A. Poole, "Philosophy and Politics in the Russian Liberation
  Movement: The Moscow Psychological Society and its Symposium,
  'Problems of Idealism,'\," in \emph{Problems of Idealism: Essays in
  Russian Social Philosophy}, trans. and ed. Randall A. Poole (New
  Haven, CT: Yale University Press, 2003), 12\textendash 13.} Despite their
sympathy for modern liberal democracy, each grounded the concept of the
human "person" in religious precepts that they regarded as more
holistic than the conception of the human being advanced by their
liberal, secular peers. In 1922, both were exiled from Russia at the
behest of the more radical secularism of the Bolsheviks. Both lived out
the rest of their days in Paris. While Berdiaev, unlike Bulgakov,
immediately found an audience in the spiritually restive West, Russian
Orthodox theologians tended to criticize both figures as beyond the pale
of Orthodox Christian thought.\footnote{Paul Gavrilyuk and Brandon
  Gallaher challenge the widely accepted dichotomy that Florovsky and
  Lossky drew between their own "neo-patristic revival" and their
  forebears in the Russian religious renaissance. In doing so, Lossky
  and Florovsky sought to liberate Eastern Orthodox theologians from the
  ethos of Western "idealistic" philosophy, ostensibly gnostic in
  orientation and agnostic towards the church, which influenced many
  Russian thinkers, including Berdiaev and Bulgakov, during the Silver
  Age. Gavrilyuk and Gallaher maintain that such binaries\textemdash East versus
  West, patristics versus idealism\textemdash are reductive. They place Florovsky
  and Lossky on a continuum with the Russian religious renaissance
  rather than as a reaction to its excesses. Gavrilyuk traces
  Florovsky's interest in freedom to Berdiaev and his preoccupation with
  patristics to Bulgakov. Paul L. Gavrilyuk, \emph{Georges Florovsky and
  the Russian Religious Renaissance} (Oxford: Oxford University Press,
  2015), 260\textendash 261. Gallaher contends that Lossky drew from the
  sophiology of Bulgakov in a positive rather than reactionary manner.
  See Brandon Gallaher, "The 'Sophiological' Origins of Vladimir
  Lossky's Apophaticism," \emph{Scottish Journal of Theology} 66, no. 3
  (July 2013): 278\textendash 98. Gallaher's claim is especially provocative.
  Florovsky and Lossky dismissed sophiology as a Western innovation that
  blurred the difference between God and the world as well as the two
  natures of Christ. Lossky's criticisms played a crucial role in
  motivating the metropolitan of the Russian Orthodox Church to issue an
  official opprobrium on Bulgakov's sophiological teachings in 1935
  (Gavrilyuk, \emph{Georges Florovsky}, 120\textendash 124, 138\textendash 140). Gallaher
  argues that so long as one focuses on Bulgakov's conception of Christ,
  accusations of heresy carry little theological weight. See Brandon
  Gallaher, \emph{Freedom and Necessity in Modern Trinitarian Theology}
  (Oxford: Oxford University Press, 2016), 107\textendash 108. Gallaher's judgment
  on Bulgakov's concept of creation is more measured. Gallaher concedes
  that the sophiology of Bulgakov often conflates God and creation in a
  way that is consistent with a heretical framework such as
  pantheism\emph{.} Gallaher, nevertheless, contends that Bulgakov is
  not a pantheist. He is a pan\emph{en}theist. Bulgakov believes,
  Gallaher argues, that "all things exist in God, but have their own
  existence and activity distinct from Him." Brandon Gallaher, "The
  Problem of Pantheism in the Sophiology of Sergii Bulgakov: A
  Panentheistic Solution in the Process Trinitarianism of Joseph A.
  Bracken?" in \emph{Seeking Common Ground: Evaluation and Critique of
  Joseph Bracken's Comprehensive Worldview: A Festschrift for Joseph A.
  Bracken, S. J.}, eds. Gloria Schaab and Marc Pugliese (Milwaukee, WI:
  Marquette University Press, 2012), 147\textendash 167, esp. 147. Gallaher
  juxtaposes pantheism, which rejects the traditional, Christian notion
  that the world is created from nothing (\emph{ex nihilo}), and
  panentheism, which is sufficiently supple for Bulgakov to appropriate
  creation \emph{ex nihilo} so as to preserve much of its integrity.
  Gallaher contends that Bulgakov is too invested in creation \emph{ex
  nihilo} for his sophiology to collapse into pantheism. My article
  supports such an intuition. Gallaher extrapolates creation \emph{ex
  nihilo} through Bulgakov's use of "antinomy" and "apophaticism,"
  whereas I outline Bulgakov's understanding of God's original act of
  creation through the privation theory of evil.} Until recently,
Berdiaev and Bulgakov were seen as complementary role-players for the
same ideas. Berdiaev was cast as the winsome preacher and Bulgakov as
the laborious systematizer.\footnote{"What Berdiaev proclaimed in his
  affirmative style, Bulgakov elaborated in a systematic fashion."
  Michael A. Meerson, "Sergei Bulgakov's Philosophy of Personality,"
  \emph{Russian Religious Thought}, eds. Judith Deutsch Kornblatt and
  Richard F. Gustafson (Madison, WI: University of Wisconsin Press,
  1996), 139.}

The blossoming of interest in Bulgakov has complicated matters. As the
"orthodoxy" of Bulgakov's corpus is reevaluated, often along more
forgiving lines, some have questioned the tendency to overlook his
differences with Berdiaev. The current essay is consistent with this
trend.\footnote{Some scholars, like Gavrilyuk, may overlook such
  differences out of concern for what they regard as a more pressing
  problem: the widespread distrust among Eastern Orthodox theologians
  for anything "Western." It is an anxiety that Florovsky and Lossky's
  criticisms toward figures like Berdiaev and Bulgakov did much to
  inflame. For scholarship that distinguishes Berdiaev and Bulgakov on
  personalism, deification, the apocalypse, and creation, see,
  respectively, Regula M. Zwahlen, "Different Concepts of Personality:
  Nikolaj Berdjaev and Sergej Bulgakov," \emph{Studies in East European
  Thought} 64, no. 3 (November 2012): 183\textendash 204; Ruth Coates,
  \emph{Deification in Russian Religious Thought: Between the
  Revolutions, 1905\textendash 1917} (Oxford: Oxford University Press, 2019),
  110\textendash 173; Cyril O'Regan, \emph{Theology and the Spaces of
  Apocalyptic} (Milwaukee, WI: Marquette University Press, 2009), 52,
  55, 136 nt. 14; Deborah Casewell, "The Authenticity of Creativity:
  The Philosophical and Theological Anthropologies of Nikolai Berdiaev
  and Sergei Bulgakov," in \emph{Building the House of Wisdom}:
  \emph{Sergii Bulgakov and Contemporary Theology, New Approaches and
  Interpretations}, eds. Barbara Hallensleben, Regula M. Zwahlen,
  Aristotle Papanikolaou, and Pantelis Kalaitzidis (Münster:
  Aschendorff, 2024), 123\textendash 126.} Berdiaev and Bulgakov may not have been
as incompatible as water is~with~fire, as Zinaida Gippius~remarked.~But
Gippius was right to notice that there was something forced, even
"wretched," about their social and intellectual overlap.\footnote{Zinaida
  Gippius, \emph{Dnevniki} (Moscow: NPK Intelvak, 1999), 316\textendash 318,
  quoted in Zwahlen, "Different Concepts of Personality," 184.}
Any~common ground or apparent "peace" between Bulgakov and Berdiaev on
freedom and evil~was haunted by a more elementary dispute over the
origins of nothing. And nothing, as I will argue,~is nothing to sneeze
at;~something comes of it.~

\subsection*{The Underground}

Against the fashion of his time for "rational egoism," Dostoevsky
created an irrational and "repugnant" egoist in the Underground
Man.\footnote{James P. Scanlan, \emph{Dostoevsky the Thinker} (Ithaca,
  NY: Cornell University Press, 2002), 63\textendash 64, 76.} This is not to say
that the Underground Man's "consummate" egoism is of a routine sort.
His narcissism is paradoxical and principled. Its provocative claim is
that humans want freedom to choose more than they wish to act in
accordance with their best interests. It is through a defense of freedom
over rational self-interest that the Underground Man delineates a more
radical self-centeredness than his contemporaries thought possible. His
sole desire is to be dependent on nothing. The weight of the paradoxes
that the Underground Man sustains as a result\textemdash self-adoration as
self-loathing, self-affirmation as self-destruction, freedom as
self-imprisonment\textemdash exposes the inadequacy of such a ground. His liberty
is undone at every turn. The Underground Man might succeed in showing
that the will is freer than many at his time would have allowed, and
that one does not have to accept what others say is good for oneself,
but this gain is illusory. For he also establishes that the naked will
offers the stability of a high-velocity vacuum, which is to say, no
sustainable stability at all. The Underground Man exhibits a free will
that is grounded in, and headed for, nothing.

\subsection*{Berdiaev: The Underground and the \emph{Ungrund}}

In 1938, late in life, Berdiaev divulged to Lev Shestov that
"Dostoevsky and Nietzsche played a much larger role in my life than
Schelling and German idealism."\footnote{Nataliia Baranova-Shestova,
  \emph{Zhizn' L'va Shestova v dvukh tomakh} (Paris: La Presse Libre,
  1983), vol. 1: 194, quoted in Edith W. Clowes, "Groundlessness:
  Nietzsche and Russian Concepts of Tragic Philosophy," in
  \emph{Nietzsche and the Rebirth of the Tragic,} ed. Mary Ann Frese
  Witt (Madison, WI: Fairleigh Dickinson University Press, 2007), 131.}
One might suspect that Berdiaev was telling Shestov, a philosopher also
styled in Dostoevsky's mold, what he wished to hear. As it happens, he
was telling the truth.

Berdiaev's first reflection on the Underground Man was published thirty
years prior, in "Tragedy and the Everyday" (1905). The purpose of the
article was, in part, to evaluate Shestov's full-throated endorsement of
the Underground Man and his passion for liberty. Berdiaev tried to be
more measured. He contended that the Underground Man's unbridled pursuit
of freedom might be laudable, but it was also destructive\emph{.} Rather
than fester in fields of underground detritus, Berdiaev summons his
reader to "go further into the mountains to create."\footnote{Nikolai
  Berdiaev, "Tragediia i obydennost'," \emph{Sobranie sochinenii v
  piati tomakh} (Paris: YMCA-Press, 1989), vol. 3: 397.} Berdiaev would
not abandon the Underground Man. The character continued to inspire him
as he worked to construct a more positive philosophy of freedom. Three
decades later, Berdiaev returned in earnest to the Underground Man in
\emph{Dostoevsky} (\emph{Mirosozertsanie Dostoevskogo,} 1934). Ruinous
underground negativity would serve as more than a mere foil for
Berdiaev's larger philosophical ambitions.

\emph{Dostoevsky} begins by summarizing the Underground Man rather
conventionally. He is a figure in "rebellion against the external world
order." He will not accept that "man needs a will directed towards
reason and his own benefit." He is bereft of any desire for "universal
harmony." Outrage soon morphs into "exorbitant self-love," and "he
moves from the surface of the earth to the underworld."\footnote{Nikolai
  Berdiaev, \emph{Mirosozertsanie Dostoevskogo}, in \emph{Sobranie
  sochinenii v piati tomakh} (Paris: YMCA-Press, 1995), vol. 5: 236.} At
last, "the Underground Man appears\textemdash an unattractive, shapeless
person\textemdash and reveals his dialectic": boundless freedom is possible
through the "destruction of human freedom and the decomposition of
personality." This is the dialectic of "irrational
freedom."\footnote{Nicholas Berdiaev, \emph{Dostoievsky: An
  Interpretation}, trans. Donald Attwater (San Rafael, CA: Semantron
  Press, 2009), 55.} It swears by originality. The result is regress
towards nothing.

Berdiaev continues to see Dostoevsky's Underground Man as an uniquely
destructive iteration of freedom. But mutiny in the underground
interrogates more than the life-habits of modern intellectuals. The
target of his revolt is God. The "dialectic" is as follows: humans can
only destroy God by superseding God. If humans are to replace God
through becoming gods they must first extinguish their humanity. This
argument follows Ludwig Feuerbach on the human capacity to emulate and
supersede the divine. "Deification," so conceived, is actually
"apotheosis."\footnote{Feuerbachian apotheosis claims to "exalt
  anthropology into theology, very much as Christianity, while lowering
  God into man, {[}makes{]} man into God." Ludwig Feuerbach, \emph{The
  Essence of Christianity}, trans. George Eliot (Amherst, NY:
  Prometheus, 1989), xviii. However, in the eyes of most Christian
  theologians, apotheosis would constitute a pseudo-divinization that
  leads one away from God (i.e., "apo") and one's humanity.
  Theologians often frame Christian deification or "theosis"
  (literally, "becoming God," or becoming one with God) as the
  fulfillment of human nature according to God's purposes. On
  Feuerbach's influence in Russia, see Irina Paperno, \emph{Suicide as a
  Cultural Institution in Dostoevsky's Russia} (Ithaca, NY: Cornell
  University Press, 1997), 140\textendash 161.} Berdiaev regards such a
Feuerbachian "dialectic" to be, on its own terms, a
dead-end.\footnote{In \emph{The Meaning of} \emph{The Creative Act},
  Berdiaev endorses Feuerbach's focus on theological anthropology yet
  dismisses as demonic his concept of the human being. Feuerbach, for
  Berdiaev, endorses a "religious anthropology turned inside out." See
  Nikolai Berdiaev, \emph{The Meaning of the Creative Act}, trans.
  Donald Lowrie (New York: Collier Books, 1962), 61. It is not clear, in
  my view, whether Berdiaev exorcises himself of the Feuerbachian legacy
  that he wanted most to avoid. Berdiaev continued to be attracted to
  Feuerbach's disciple, Friedrich Nietzsche, specifically his mission to
  free humans from philistine, external constraints on their creativity.
  See Nel Grillaert, \emph{What the God-seekers Found in Nietzsche: The
  Reception of Nietzsche's Ubermensh by the Philosophers of the Russian
  Religious Renaissance} (Amsterdam: Rodopi, 2008), 207\textendash 248; Clowes,
  "Groundlessness," 131\textendash 133.} Nevertheless, in the right dosage, the
Underground Man's example can jolt modern, secular folks from their
spiritual slumber. He shows that "the road to liberty can only end
either in the deification of man or in the discovery of God; in the one
case, he is lost and done for; in the other, he finds
salvation."\footnote{Berdiaev, \emph{Dostoievsky}, 56.} The Underground
Man, for Berdiaev, clarifies the spiritual stakes of existence. Our free
will must be understood as the capacity either to defy God or to turn
toward God, grounding the human will in the divine rather than in
nothing.

Berdiaev regards the Underground Man as a spiritual stage which, however
fraught, is necessary and wisdom-bearing. One of the key developments of
his revolution against "the external order" is the recovery of an
interior "depth" that Berdiaev believes has eluded Western
civilization since the Renaissance. The havoc wrought by the Underground
Man "loosens" or reveals the soul. Onlookers can now peer into the
soul's farthest reaches. Below the surface, "platonic calmness" is
nowhere to be found. One uncovers "hidden tempests," the ferocity of
which reveal a "struggle in man between the God-man and the man-God,
between Christ and the Anti-Christ."\footnote{Berdiaev,
  \emph{Dostoievsky}, 58.} In the underground of the soul, Berdiaev
insists that an apocalyptic "abyss {[}opens{]} and therein God and
Heaven, the Devil and Hell, {[}are{]} revealed anew."\footnote{Berdiaev,
  \emph{Dostoievsky}, 49.} The Underground Man shows that the conflict
between good and evil is not a clash between God and humanity so much as
between the divine and the demonic. The "field of battle" lies
\emph{within} each human creature. God and the Devil take up arms amid
the fluctuations of the underground, within the enigmatic depths of the
"heart"\textemdash or will.

The Underground Man, according to Berdiaev, shows that "evil has a deep spiritual nature"
(\textit{zlo imeet glubinnuiu, dukhovnuiu prirodu})."\footnote{Berdiaev, \emph{Mirosozertsanie Dostoevskogo},
  243.} Moral depravity is nothing short of demonic. Berdiaev's
fundamental point is that the self is too divided between virtue and
vice for the latter to dissipate on its own. Evil must be actively
subdued and defeated. Evil's chameleon nature further complicates
matters. Berdiaev writes that "evil comes forward under an appearance
of good, and one is deceived; the faces of Christ and of
Antichrist \ldots{} become interchangeable."\footnote{Berdiaev,
  \emph{Dostoievsky}, 60.} The soul is torn between grounded liberty and
groundless license. In its potentially limitless freedom, the soul is in
perennial danger of succumbing to wickedness. Evil begins to seem so
powerful, so cunning, so real, so substantive, so intertwined with the
soul, that, as Berdiaev would later state, the "feeling of evil becomes
a metaphysical feeling."\footnote{By "metaphysical feeling," I take
  Berdiaev to mean that Dostoevsky confronts the reader with the
  palpable realization that evil cannot be reduced to a psychological
  response. The implication is that one cannot talk properly about evil
  without considering its metaphysics. See Nikolai Berdiaev, "Unground
  and Freedom," \emph{CrossCurrents} 7, no. 3 (Summer 1957): 247\textendash 262,
  253. Also see Nikolai Berdiaev, "Iz etiudov o Ia. Beme. Etiud I:
  Uchenie ob Ungrunde i svobode." \emph{Put'} (February 1930), 56.}

In presenting the underground as a bottomless abyss, Berdiaev is
elaborating upon his own idiosyncratic interpretation of the mystical
thought of Jacob Boehme. As Berdiaev speculates:

\begin{quote}
If Dostoevsky would have developed to the end his teaching about God,
about the Absolute, then he would have been forced to recognize the
polarity of the divine nature itself, to have found in him also a chasm
of darkness, thus approximating Jacob Boehme\textquotesingle s teaching
of the \emph{Ungrund}. The human heart is, at its most fundamental,
polar, but the human heart is embedded in the abysmal depths of
being.\footnote{Berdiaev, \emph{Mirosozertsanie Dostoevskogo}, 243.}
\end{quote}

\noindent Here, Berdiaev processes Boehme\textemdash and rather heavy-handedly\textemdash as much
through Freud's unconscious and Nietzsche's Dionysius as through
Dostoevsky's underground. A quick review of Berdiaev's reading of Boehme
will help us to understand the metaphysical backdrop to this notion of
the underground\textemdash a reading often indistinguishable from Berdiaev's own
views.

Berdiaev interpreted Boehme as suggesting that God and the universe are
founded in the "\emph{Ungrund}" (in German, "non-ground"). He also
would refer to "\emph{Ungrund}" as \emph{mē on,} Greek for
"nonbeing." Following Boehme, or what he took to be Boehme, Berdiaev
presumed that prior to God and creation was groundlessness, nonbeing, or
"nothingness."\footnote{The \emph{Ungrund}, for Berdiaev, is
  "nothingness, the unfathomable eye of eternity, and at the same time
  a will, a will without bottom, abysmal, indeterminate" (Berdiaev,
  "Unground and Freedom," 253).} Berdiaev does not mean to suggest
\emph{absolute} nothing. For him, "nothing" signifies a vacuum that is
home to formless, chaotic energy and is analogous to a feral "will."
In the beginning, for Berdiaev, is a nothing that is indeterminate
freedom. Berdiaev refers to this \emph{relative} nothing as the "first
divinity" (\emph{Pervo-Bozhestvo}).\footnote{Berdiaev, "Unground and
  Freedom," 254.} A moral, rational, conscious God arrives later. All
of God's creative acts, according to Berdiaev, impose harmony on this
anarchic "nothing" while relying on its vitality and dynamism. Humans,
made as they are in God's image, replicate this dynamic. Berdiaev
assumes that God and humans contain within themselves a "dark source"
(\emph{temnyi istok}) of groundless, Dionysian energy.\footnote{Berdiaev,
  "Unground and Freedom," 257. Nikolai Berdiaev, \emph{Smysl
  tvorchestva}, in Berdiaev, \emph{Filosofiia svobody. Smysl
  tvorchestva. Opyt opravdaniia cheloveka} (Moscow: Akademicheskii
  proekt, 2020), 239.} God and humanity emerge from an ungovernable yet
indispensable "nothing." Both live in tragic self-conflict with the
"dark residue" (\emph{temnyi ostatok}) of nothing until the end of
time.\footnote{Berdiaev, \emph{Smysl tvorchestva}, 375.} Only humans,
however, interact with this "source" in a manner that leads to evil.
The underground is where the ambiguous freedom of primal nothing is
stored in humans after the Fall.

The Underground Man, for Berdiaev, points down toward humanity's
self-division, up toward God's self-conflict, and back to the free and
stormy void that was at the dawn of the cosmos. Berdiaev implies that
the Underground Man is the closest approximation to this original cosmic
absence. All the qualities that Berdiaev ascribes to the
\emph{Ungrund}\textemdash or \emph{mē on}\textemdash he also attributes to the Underground
Man and his habitat. The \emph{Ungrund} is chaotic, arbitrary, free,
formless, irrational, feverish, agonistic, and ambiguous. And so is the
Underground Man. His will is grounded in the "nothing" of the
underground. By making his abode there, he hopes to exist in a primitive
state of freedom that is beyond good and evil. It is as if he wants to
return to nothing to replicate God's own birth from the abyss. The
Underground Man's project is one of \emph{de-creation}. He wants to undo
what God has done, even if his ambitions are frustrated. He yearns to go
back to the beginning and do it all over again his way. To de-create,
then, is to side with the devil, to choose to unmake oneself in order to
become God.

Berdiaev's notion of evil as the freedom to uncreate is not wholly
discordant with accepted Christian opinion. In particular, groundless
freedom oriented toward nothing is close to St. Augustine's conception
of evil as privation. His metaphysical backstory to this notion
\emph{is} a substantial deviation, however. The difficulty is that
Berdiaev correlates seminal "nothing"\textemdash the "first divinity"\textemdash with
an aboriginal freedom that is beyond good and evil. Most Christian
theologians would find it odd to link the divine to this conception.
Thus, Berdiaev's identification of the divine with meonic freedom was
bound to be met with skepticism. Among these skeptics was Fr. Sergei
Bulgakov.

\subsection*{Bulgakov: Underground Heroics and the Soul}

Bulgakov's most significant exploration of the underground occurs in his
early work, \emph{Unfading Light} (1917).\footnote{Sergei Bulgakov,
  \emph{Unfading Light: Contemplations and Speculations}, trans. Thomas
  Allan Smith (Grand Rapids: Eerdmans, 2012).} By this point, Bulgakov
had written three noteworthy pieces on Dostoevsky. He had delivered two
public lectures, "Ivan Karamazov as a Philosophical Type" (1901) and
"A Crown of Thorns: In Memory of F. M. Dostoevsky" (1906). He had also
composed a lengthy essay, "Russian Tragedy: On Dostoevsky's
\emph{Demons}" (1914).\footnote{Sergii Nikolayevich Bulgakov, "Ivan
  Karamazov kak filosofskii tip," \emph{Sergii Bulgakov v dvukh
  tomakh,} ed. Irina Rodnianskaia (Moscow: Nauka, 1993), vol 2: 15\textendash 45;
  Bulgakov, "Venets Ternovyi\textemdash pamiati F. M. Dostoevskogo,"~\emph{Sergii
  Bulgakov v dvukh tomakh}, vol. 2: 222\textendash 239; Bulgakov, "Russkaia
  Tragediia\textemdash o \textquotesingle Besakh\textquotesingle{}
  Dostoevskogo,"~\emph{Sergii Bulgakov v dvukh tomakh}, vol. 2:
  429\textendash 526.} In \emph{Landmarks} (1909), he touched on "underground
psychology" to elucidate the animus of aspiring Russian
revolutionaries.\footnote{Sergei Bulgakov, "Heroism and Asceticism:
  Reflections on the Religious Nature of the Russian Intelligentsia,"
  \emph{Vekhi: Landmarks}, eds. and trans. Marshall Shatz and Judith
  Zimmerman (London: M.E. Sharpe, 1994), 17\textendash 51.} But he had yet to
entertain the underground and its notorious fanatic at length.

The opportunity presented itself in \emph{Unfading Light,} while
discussing matters of creation. Bulgakov identified in the
"underground" a nothing that was analogous to a seminal cosmic void.
His logic is similar to Berdiaev's. It shall come as no surprise, then,
that the problems of freedom and evil are right up front. Bulgakov
begins his foray into the underground with the following passage:

\begin{quote}
The nature of humankind is marked by genius and nothingness. The
underground is the "inside out" {[}\emph{iznanka}{]} of being. Every
creature has an underground, although it is able not to know about this,
and not able to sink into it \ldots{} by sinking into it, each person lives
through the eerie cold and dampness of the grave. To want oneself in
one's own selfhood {[}\emph{khotet' sebia v sobstvennoi samosti}{]}, to
lock oneself in one's creatureliness as in the absolute, means to want
the underground and to be affirmed in it. And therefore, the real hero
of the underground is Satan who fell in love with himself as God, and
who was affirmed in his own selfhood and turned out to be captive to his
own underground.\footnote{Bulgakov, \emph{Unfading Light}, 187. Sergii
  Nikolayevich Bulgakov, \textit{Svet nevechernii: sozertsaniia i
  umozreniia}, in Bulgakov, \emph{Pervoobraz i obraz: sochineniia v dvukh tomakh},
  eds. N. M. Kononov and M. I. Potapenko (Moscow: Iskusstvo, 1993), 169.}
\end{quote}

\noindent The paradoxes here are familiar. Bulgakov, like Berdiaev, interprets the
Underground Man and his type as persons engrossed in a hapless demonic
revolt. These subterranean revolutionaries lust after inexhaustible
freedom only to witness their efforts climax in an excruciating scene of
enslavement to their own egos. The Underground Man is a free will headed
for nothing. And the instability of his groundless ground leads Bulgakov
to a conversation about the metaphysics of the soul.

For all the parallels in their argument, Bulgakov is actually attempting
to distance himself from Berdiaev. To be persuaded of this, it is enough
to read the footnotes. Bulgakov inserts a pregnant citation after he
pronounces Satan as exemplary of the underground. He reflects on
Berdiaev's \emph{The Meaning of The Creative Act} (1916), which had been
published a year earlier.\footnote{Ana Siljak notes how Berdiaev
  returned to \emph{The Meaning of the Creative Act} "repeatedly"
  throughout his life. See Ana Siljak, "The Personalism of Nikolai
  Berdiaev," in \emph{The Oxford Handbook of Russian Religious
  Thought}, eds. Caryl Emerson, George Pattison, Randall A. Poole
  (Oxford: Oxford University Press, 2020), 313. In his autobiography,
  Berdiaev claimed that "all the themes to which I devoted my life and
  work were contained or prefigured in this book." Nikolai Berdiaev,
  \emph{Self-Knowledge: An Essay in Autobiography}, trans. Katharine
  Lampert (Philmont, NY: Semantron, 2009), 100\textendash 101.} Bulgakov was eager
to underscore their differences rather than similarities. He gives the
following appraisal:

\begin{quote}
The dual and contradictory nature of creatureliness, woven out of
divinity and nothingness, does not admit the immanent divinization of
humankind which constitutes the distinctive feature of N. A. Berdiaev's
anthropology. \ldots{} In our view, the creative impulse and the frenzy of
the "underground" merge indistinguishably in the "creative act" as
he proclaimed it.\footnote{Bulgakov, \emph{Unfading Light}, 469, nt. 5.}
\end{quote}

\noindent The satanic context of this notation is crucial. Bulgakov criticizes
Berdiaev for awarding the dubious economy of the underground an unduly
prestigious role in human improvement.\footnote{The reservations of
  Bulgakov towards \emph{The Creative} \emph{Act} were not atypical. As
  Coates puts it, many of his contemporaries feared that there was a
  line which "Berdiaev had crossed, beyond which the God-sanctioned
  high human calling of synergy becomes demonic titanism" (Coates,
  \emph{Deification in Russian Religious Thought}, 133\textendash 134).} The
result is a picture of human nature\textemdash mind, soul, and body\textemdash that is too
diabolical to redeem until the \emph{Endzeit}. Rather Bulgakov assumes,
as do most Orthodox theologians, that the process of redemption can
begin now and that it takes the form of divinization, which, properly
pursued, is not blasphemous but salvific. Christian \emph{theosis}
stands in contrast to Feuerbachian \emph{apotheosis}. The doctrine of
\emph{theosis} outlines a path toward deification in which humans do not
compete with the Creator or deny their nature as contingent creatures.
"God became human so that we might become god," wrote St. Athanasius
of Alexandria in the fourth century.\footnote{"{[}God{]}, indeed,
  assumed humanity so that we might become God." St. Athanasius,
  \emph{On the Incarnation} (Crestwood, NY: St. Vladimir's Press, 1996),
  93.} As Ruth Coates has most recently shown, this point of faith
became mainstream in Russian Orthodoxy.\footnote{Coates,
  \emph{Deification in Russian Religious Thought}, 24\textendash 54, 156. Also see
  Richard F. Gustafson, "Soloviev's Doctrine of Salvation," in
  \emph{Russian Religious Thought}, eds. Judith Deutsch Kornblatt and
  Richard Gustafson (Madison, WI: University of Wisconsin Press, 1996),
  31\textendash 49. Vladimir Solovyov (1853\textendash 1900)\textemdash intellectual predecessor of
  Bulgakov and Berdiaev\textemdash revived the doctrine of \emph{theosis}.}
Christians are called to internalize God's energies so thoroughly that
their moral and physical composition converges with the divine. Bulgakov
assumed that unless deification is conceived as the transcendent
culmination of a process that human beings can undertake now, then
humans and their material environment are deprived of their inherent
goodness and worth. The cosmology of Berdiaev was case in point.

Bulgakov's criticisms seem fair. Berdiaev did believe, as we noted, that
creative acts were a propitious combustion of the soul's volatile and
vaporous elements. He assumed that chaotic potentiality or
"nothingness" was integral to the tragic side of the cosmos, residing
within humans, in the underground of their souls, until the eschaton.
And he would have agreed that the salvific function of creativity
underscored the integration of evil within the natural order, because
the "creative impulse" arose from the uncreated, amoral energies of
the underground. Bulgakov implies, in \emph{Unfading Light,} that
Berdiaev treats evil as if it is ontologically real. Evil appeared so
pervasive in Berdiaev's thought that Bulgakov often seems to doubt
whether his theory of creativity could guarantee the future deification
of humanity, let alone at present. He suspected that Berdiaev was in
possession of a more fundamental metaphysical tool than "creativity"
to ensure humanity reached its ideal destination. Bulgakov thus
criticizes Berdiaev elsewhere in \emph{Unfading Light} for championing a
"creative gesture" (\emph{tvorcheskii zhest}) rather than a bonafide,
creative \emph{act}. Bulgakov discerns in Berdiaev a view of human
nature that is as "powerless" as it is "pretentious," qualities that
Bulgakov associates foremost with the devil and his legions.\footnote{This
  comment builds on a footnote from Part III of \emph{Unfading Light},
  in which Bulgakov explains how humans are created in the image of God.
  The note is worth quoting in full: "This confusion of image and
  Prototype, of ego and Ego, distinguishes the fundamental motifs of
  Fichte's metaphysics, who equates the human I, taken in the greatest
  intensity, with the divine I. The intuition of the transcendence of
  the spirit in relation to all of its determinations or products lies
  at the basis of the philosophy of creativity in N. A.
  Berdiaev \ldots{} but he sees insufficiently the difference between image
  and Prototype, between the unlimited creativity of humankind on the
  basis of sophianicity and the absolute divine creative act. Therefore
  the result is an objectless and for that reason powerless although
  pretentious, creative gesture" (Bulgakov, \emph{Unfading Light}, 487,
  nt. 3).}

\subsection*{\emph{Demons} of the Heart}

Bulgakov uses the concept of the underground to construct an alternative
picture of human interiority. Berdiaev sees in the human soul a capacity
for colossal depth, which he refers to as the underground\textemdash to which
Bulgakov counters in \emph{Unfading Light}, "there are two abysses
{[}\emph{bezdny}{]} in the human soul: dead-end nothing {[}\emph{glukhoe
nichto}{]}, an infernal underground, and God's heaven which has
imprinted the image of the Lord."\footnote{Bulgakov, \emph{Unfading
  Light}, 187.} These abysses are immiscible; they cannot overlap.

Bulgakov is expanding an argument he made two years earlier in "Russian
Tragedy: On Dostoevsky's \emph{Demons}" (1914). He insisted that
Dostoevsky's celebrated line\textemdash that the struggle between God and the
devil occurs on the "battlefield of the heart"\textemdash not be taken to
suggest that evil is tragically but inevitably interlaced with the good.
To make this point, he notes how Stavrogin's character is oddly truant
in \emph{Demons}. He recedes into the background, despite serving as the
center of gravity for wickedness in the novel. The fact that Stavrogin
is "terribly, ominously, infernally not there" confirms, in Bulgakov's
view, his status as the paragon of the "forces of evil."\footnote{Bulgakov,
  "Russkaia tragediia," 503.} Having been "possessed by the spirit of
nonbeing {[}\emph{dukh nebytiia}{]}," his personality is whittled down
to a "psychological skeleton-iron-will." What remains of Stavrogin is
not a distilled essence but layer upon layer of artificiality. He is no
longer a personality but a "mask of masks."\footnote{Bulgakov,
  "Russkaia tragediia," 512.} Bulgakov compares him to "an actor, not
on the stage, not in art, but in real life."\footnote{Bulgakov,
  "Russkaia tragediia," 511.} Stavrogin may wish "to cope with the
disintegration of his personality, to be born to life," but there is
nothing left to exhume and reassemble.\footnote{Bulgakov, "Russkaia
  tragediia," 512.} His example indicates, for Bulgakov, that evil is a
form of "nothingness {[}that swallows{]} up its victim."\footnote{Bulgakov,
  "Russkaia tragediia," 512.} It is quite possible, Stavrogin reveals,
to be "dead before death."

A peculiar tragedy ensues. Bulgakov reads Stavrogin as so disconnected
from his material body that he can no longer act. Like a disincarnate
spirit, he lives vicariously through neighboring bodies. The temptation
of others becomes Stavrogin's only course of action. He serves as a
"provocateur" who stokes fire in others' hearts, but "he himself does
not burn and is obviously incapable of igniting." He is a "medium"
who receives others' hopes for salvation and love only to pervert them
within the toxic vortex of his interiority. He is the inverse of the
angelic light that caught fire but did not incinerate the bush in
Exodus. Stavrogin's demonic luminescence is not ablaze but destroys
nevertheless. He is an "icy reflection of alien fire and
light."\footnote{Bulgakov, "Russkaia tragediia," 512.} He is a
"black hole"\textemdash a "black grace" (\emph{chernaia blagodat'})\textemdash that
becomes visible only as it devours.\footnote{Bulgakov, "Russkaia
  tragediia," 513. A century later, the Archbishop Rowan Williams would
  read \emph{Demons} in a similar vein. See Rowan Williams\emph{,
  Dostoevsky: Language, Faith and Fiction} (Waco, TX: Baylor University
  Press, 2008), 63\textendash 110. Williams frames Stavrogin as a paradigmatic
  example of the demonic: as that which "\,'disincarnates,' dis-tracts
  us from the body and the particular" (Williams\emph{, Dostoevsky},
  83). Stavrogin, he insists, is a "will arbitrarily exercised,"
  because he is nothing \emph{more} than a will\emph{.} He cannot live
  in the real world, much less "{[}discriminate{]} between good and
  evil." He can only draw other bodies into his "self-consuming void"
  (Williams\emph{, Dostoevsky}, 25). Williams projects evil as an
  immaterial absence and an explicitly demonic one at that. For a
  contrasting reading, see Susan McReynolds, \emph{Redemption and the
  Merchant God: Dostoevsky's Economy of Salvation and Antisemitism}
  (Evanston, IL: Northwestern University Press, 2005), 144\textendash 156. Whereas
  Williams and Bulgakov read Stavrogin's subjective lack to signify that
  evil is ontologically non-primary and, thus, impermanent, McReynolds
  sees in Stavrogin a moral and metaphysical challenge to the assumption
  that evil naturally implodes and that malicious actions are
  universally redeemable. His refusal to receive forgiveness for abusing
  the eleven-year-old Matresha suggests, for McReynolds, that his
  project is anything but nihilistic. If anything, he asserts that it is
  the Christian assumption that any sin\textemdash no matter how grotesque\textemdash can
  be forgiven that obliterates any meaningful distinction between good
  and evil. McReynolds concludes that "refusing to accept Christ's
  forgiveness may be the only way {[}for Stavrogin{]} to show respect
  for Matresha's loss." In doing so, he confirms that his actions were
  inexcusable and will forever be considered so (McReynolds,
  \emph{Redemption and the Merchant God}, 150). McReynolds implies that
  one would be wise to presume that ours is a cosmos in which evil and
  the good are co-foundational and co-eternal rather than follow
  Dostoevsky and Augustine's attempt to render the first and final word
  to the good.}

Bulgakov considers the revolutionary Kirillov to be the most tragic of
Stavrogin's victims. His agonizing demise, Bulgakov claims, "discloses
the religious abysses {[}\emph{bezdny}{]} of the human
spirit."\footnote{Bulgakov, "Russkaia tragediia," 513.} The language
of abysses, plural, is significant. Stavrogin's interiority represents a
single nefarious void. Kirillov possesses not one, but two, bottomless
trajectories. One is angelic, the other\textemdash demonic. The tragedy is that
the virtuous, "simple-minded heart" of Kirillov is primary.\footnote{Bulgakov,
  "Russkaia tragediia," 514.} Unbelief is not his failing. He may
possess a child-like disposition, but "naively negating atheism remains
infinitely inferior to Kirillov's 'mystical requests.'\,"\footnote{Bulgakov,
  "Russkaia tragediia," 515.} Stavrogin corrupts Kirillov's pure
"love for Christ" and all creation into an "idol" of "self-will."
Messianic-wrapped dreams become the vessel of "not atheism, but
anti-theism." Kirillov believes in God "but does not want Him."
Kirillov hopes to rescue himself from God by refashioning himself into a
"man-God" (\emph{chelovekobog}). His attempt to plagiarize God and
replace Him through a "rebellion" of "self-will" amounts to an
abstract "caricature" of divine freedom.\footnote{"Just as Satan is a
  caricature of God, so self-will is a caricature of freedom and the
  religious revolt is a parody of might." Bulgakov, "Russkaia
  tragediia," 514.} Kirillov cannot reinvent himself as a
self-sovereign deity.\footnote{Bulgakov distinguishes two forms of
  apotheosis. Feuerbach's apotheosis is communal and humanitarian. It is
  "anthropotheosis," prefiguring Marx and Engels. The apotheosis of
  Max Stirner worships self-will and foregrounds Nietzsche. One suspects
  that Kirillov is between Feuerbach and Stirner. Sergii Bulgakov,
  "Religiia chelovekobozhiia u L. Feierbakha," \emph{Sergii Bulgakov v
  dvukh tomakh,} ed. Irina Rodnianskaia (Moscow: Nauka, 1993), vol. 2:
  162\textendash 221.} He cannot supplant the actual God-man in Christ. He can
only dismantle what God has created. He discovers, devastatingly, that
"outside of God is nothing, nonbeing."\footnote{Bulgakov, "Russkaia
  tragediia," 517.} Kirillov's theological declaration of independence
concludes in the ultimate abstraction: suicide.

For Bulgakov, Stavrogin's subjective vacancy and the gravitational pull
thereof illustrates what it means to suggest that evil is not
ontologically real\emph{.} Evil does not have a legitimate claim to
actuality because its venom is incompatible with \emph{living} flesh and
bone\emph{.} Evil cannot abide a body. To frame evil as a "plagiarist"
of reality is to acknowledge its radical destructivity. Bulgakov does
not mince words. Evil is no facile mimicry of divine freedom and
creativity. It is rather its negation, "a rape of the human spirit, of
the image and likeness of God" in all its materiality.\footnote{Bulgakov,
  "Russkaia tragediia," 504. The metaphor of sexual violence alludes
  to Stavrogin's record of child abuse.} The movements of the
underground are similarly without reality or existence. The underground
is a recapitulation of Stavrogin's interiority, which Bulgakov equates
with the "Gadarene abyss" (\emph{Gadarinskaia bezdna}) from the New
Testament.\footnote{Bulgakov, "Russkaia tragediia," 506. The story of
  the Gerasene or Gadarene demoniac tells of a man possessed by a
  "legion" of demons. These spirits endow him abnormal strength. His
  behavior becomes indistinguishable from an animal. Jesus exorcises the
  demons, but the spirits do not recede peacefully. The demons transfer
  to a pig herd, after which they are driven off a cliff to drown in a
  lake. See Matthew 8:28\textendash 34, Luke 8:26\textendash 39, Mark 5:1\textendash 20. An abstract
  from Luke's account (8:32\textendash 39) serves as the epigraph to
  \emph{Demons.}} It is a place where demonic hordes make their
intentions clear, as they did in the Gospels by propelling a drove of
swine off a lake-side escarpment. The underground makes room only for
corpses.

Bulgakov later insinuates that his Dostoevskian meditations on evil
were, at root, thoroughly Augustinian.\footnote{For a defense of the
  "privation theory of evil" by a contemporary philosopher, see Samuel
  Newlands, "Evils, Privations, and the Early Moderns," in \emph{Evil:
  A History}, ed. Andrew P. Chignell (Oxford: Oxford University Press,
  2019), 273\textendash 305.} In \emph{Unfading Light,} he endorses St.
Augustine's interpretation of evil as "a negation (\emph{negatio}),
corruption (\emph{correptio}), and deprivation (\emph{privatio}) of
good-being."\footnote{Bulgakov, \emph{Unfading Light}, 273.} All manner
of vice, for Augustine and Bulgakov, are privations of virtue. Evil does
not have substance, essence, being, or reality apart from the good. Vice
is not self-sustaining. Neither can freedom that is devoid of the good
claim to possess ontological substance or permanence. Unfettered license
is the crux of evil, because it is grounded in nothing positive. The
notion that wickedness is on a mission to free itself from the good
proves that evil is reliant on goodness. Evil is like a parasite that
perishes after sucking its host dry. "Evil cannot therefore have
independent significance," Bulgakov writes.\footnote{Bulgakov,
  \emph{Unfading Light}, 273.} He masterfully shows how any attempt to
prove otherwise devolves into parody and tragedy. Moral depravity, for
Bulgakov, models a hostile or negative form of dependency; evil cannot
participate directly or harmoniously in the good. Neither can evil
compete in a straightforward or equal manner with goodness, like
heavyweights exchanging blows. Stavrogin demonstrates that evil is not a
subject\emph{,} but subject-lessness. Evil is not a thing but a process
towards nothing\emph{.}

Bulgakov contends that evil, as a "rebellious nothing" of pure
freedom, "does not have the power to splash {[}its{]} dead waves
through the weir of being that God has erected."\footnote{Bulgakov,
  \emph{Unfading Light}, 234.} Similarly, the underground's vortex of
self-laceration cannot penetrate the more basic laceration or division
between its own conceits and the celestial pursuits of the soul's other
abyss\emph{.} Fortunately, the irreconcilable nature of humanity's inner
depths has much to teach the inquiring metaphysician. According to Bulgakov,
one can learn something about the good from the evil
that tries to negate it. The parasitic ravages of Stavrogin's "black
grace," he argues, sketch "how the healing, saving,
regenerating, liberating grace of God works." The charades of
Kirillov's "anti-theism" outline "how {[}genuine{]} deification is
possible."\footnote{Bulgakov, "Russkaia tragediia," 504.} And the
underground illuminates the workings of the soul's heavenly abyss.

One can now more adequately parse Bulgakov's image of a bifurcated soul
and its incongruous "nothings." There is the "rebellious nothing" of
underground. It is a place where the self can assure itself, with much
sound and fury, that it has no limits. The outcome of such
self-perpetuating hubris is the corporeal flotsam of the "Gadarene
abyss." Then there is celestial nothing\textemdash the heavenly abyss. This
space is characterized by profundity rather than artificiality, by
"pregnancy" rather than entropy, and by the divinization rather than
the dissolution of matter. The underground cannot contribute to the
life-giving objectives of the heavenly abyss, for it resides in an
absence that is too far removed from the radiance and reality of God's
being. Evil may corrupt the good, but it cannot compromise goodness. The
heart is split between a heavenly womb and an underground tomb. Never
the two shall meet.\footnote{Some might ask how this dualistic
  conception of the soul aligns with Bulgakov's eschatological optimism.
  Bulgakov believed that all souls as well as all of the soul would be
  saved at the end of time. This article assumes that the answer lies in
  the \emph{ouk on}, an idea explored in ensuing sections. The
  underground of the soul and its consistency as "rebellious nothing"
  are contained within a deeper "nothing" that Bulgakov calls the
  \emph{ouk on}. God, according to Bulgakov, creates the \emph{ouk on}
  to make space for human freedom. The \emph{ouk on} gives humans the
  potential to pervert God's will but is not itself a perversion. We may
  speculate: if the borders of the underground belong to God, then this
  interior tomb can be universally raided and redeemed by Christ.
  Bulgakov seems to assume that if God admits the most profound
  iteration of nothing into His being in the beginning, then God will
  admit all souls into Himself in the end. For an overview of Bulgakov's
  beliefs on universal salvation, see Paul L. Gavrilyuk, "Universal
  Salvation in the Eschatology of Sergius Bulgakov," \emph{The Journal
  of Theological Studies} 57, no. 1 (April 2006): 110\textendash 132.} To be human
is to be "created a son of abysses."\footnote{Bulgakov, \emph{Unfading
  Light}, 188.}

\subsection*{Creation \textit{Ex Ouk On}}

There is more to say of nothing. Bulgakov aims
to explore with greater precision the disembodied gorge towards which
the underground catapults. Bulgakov's Underground Man is not in pursuit
of a dynamic\textemdash that is, a relative\textemdash nothing\emph{.} He seeks a more
robust absence than \emph{mē on.} This is a space deprived of any energy
or potentiality that one could trace to God\emph{.} Bulgakov follows
Plato's terminology. He invokes the most forceful negative article in
Greek\textemdash  "\emph{ou"\textemdash }to accentuate the aridity of this oblivion. The
Underground Man yearns for the nonbeing or nothing of "\emph{ouk on},"
"the limit beyond which lies dead-end bottomless nonbeing, 'the outer
darkness' {[}\emph{kromeshnaia t'ma}{]}."\footnote{Bulgakov,
  \emph{Unfading Light}, 186.} Bulgakov is not equating \emph{ouk on}
with the "outer darkness" of hell, though \emph{ouk on} is adjacent to
the infernal deep. The underground, for Bulgakov, points to an absence
so deaf and desolate that it resembles the traditional Christian account
of "pure, empty nothing, nonbeing before the
world-creation."\footnote{Bulgakov, \emph{Unfading Light}, 234.} Here
is the \emph{ouk on}.

The underground facilitates Bulgakov's attempt to distance himself from
the philosophical school of Neo-Platonism. In \emph{Unfading Light,} he
goes so far as to label this tradition the "hostile competitor" of
Christianity.\footnote{Bulgakov, \emph{Unfading Light}, 164.} According
to Bulgakov, Neo-Platonism presumes that creation is an "involuntary
emanation"\textemdash or emission\textemdash from an abyss that was seminal and internal
to God. He saves his most vehement critiques for a particular vein of
Neo-Platonic thought that runs from Boehme to German Romanticism to
Russians like Berdiaev. For these thinkers, Neo-Platonic emanations were
grounded in a morally flawed, primordial abyss of nonbeing. According to
Bulgakov, they held that "nothing" evolved dialectically to resolve
its deficiencies over time.\footnote{Bulgakov, \emph{Unfading Light},
  152\textendash 155.} The combination of automatic emanations, "evolutionary
dialectics," and primeval voids contained a more ambitious agenda than
the explanation of the genesis of the world. One could also comprehend
the problem of evil, or the disparity between the imperfections of
creation and the goodness of the Creator. The question of evil was
indeed integral to Berdiaev's enthusiasm for the savage liberty of
Boehme's abyss, the \emph{Ungrund}. He reasoned that if freedom and
chaos were prior to God, then "God is not responsible for the evil that
comes from {[}freedom{]}."\footnote{Berdiaev, "Unground and Freedom,"
  258. Berdiaev had reached this conclusion by the time that he wrote
  \emph{The Meaning of the Creative Act}. The introduction states, "For
  the greatest of the mystics, Jakob Boehme, evil was in God\textemdash and it
  was falling-away from God; in God was the source of darkness\textemdash and God
  was not responsible for evil" (Berdiaev, \emph{The Creative Act},
  15).} God may not be liable for evil, but neither is God wholly good.
Berdiaev regarded such concessions as worth the cost. The idea of a
self-upgrading "nothing" awarded him an explanation for evil, an alibi
for God, and an assurance that the cosmos could one day defeat the
"suffering of the \emph{Ungrund}."\footnote{Berdiaev, "Unground and
  Freedom," 253.}

Bulgakov saw a risk in explaining away the mysteries of creation and
evil in this manner. Berdiaev and his predecessors, in Bulgakov's
opinion, allowed evil to become so integrated into the nature of things
that it reflected a more fundamental moral ambiguity within the God who
created the world. One could mask the damage by intoning panegyrics to
freedom, creativity, and final reconciliation. But to naturalize moral
error through the mechanical gears of "evolutionary-dialectics" does
not solve the problem of evil. It makes the problem worse. Evil no
longer appears inexcusable but merely inconvenient. Contemptible acts
seem natural, and the cosmos\textemdash impersonal. Under such circumstances, one
cannot call God good. Original sin does not exist. There is origin-less
sin. Bulgakov was convinced that a "repugnance towards the flesh" was
a fitting response towards such an immensely flawed world.\footnote{Bulgakov,
  \emph{Unfading Light}, 179.} One cannot rejoice over God's creation as
if it were a beneficent gift. And it is difficult to conceive of a
divine gift as such\emph{.} Bulgakov worried that emanationism rendered
the idea of receiving a gift from a transcendent God unintelligible by
blurring the line between God and world, giver and gift. Neo-Platonism,
in his view, tends toward pantheism. Moreover, since pantheism jettisons
belief in a transcendent level of reality, Bulgakov conflated pantheism
with atheism.\footnote{Bulgakov, \emph{Unfading Light}, 50\textendash 51, 189.
  Bulgakov was, of course, aware that the philosopher Johann Gottlieb
  Fichte (1762\textendash 1814) was accused of atheism for his pantheistic
  sympathies. Bulgakov mostly agreed with these criticisms. In some
  ways, he goes a step further. Bulgakov associates emanationism with
  atheism. He claims that if "the world is the \emph{mē on} of God\textemdash we
  arrive at pantheism with either the acosmism or atheism that springs
  from it" (Bulgakov\emph{, Unfading Light}, 189). Bulgakov argues that
  pantheists, like atheists, lack a direction or "object" to which to
  pray. Bulgakov is clear: "where there is no prayer there is no
  religion" (Bulgakov, \emph{Unfading Light}, 25). Bulgakov links
  Berdiaev's cosmology with Fichte's pantheism, and, implicitly, with
  his atheism (Bulgakov, \emph{Unfading Light}, 487).} He did not take
Berdiaev as the fearless critic of atheism for which he is often assumed
and celebrated.\footnote{This section and those that follow
  synthesize many passages from~\emph{Unfading Light}. In Part I,
  Chapter III, Bulgakov outlines the problems\textemdash Neo-Platonism,
  nothingness, and materiality\textemdash which he attempts to solve in Part II.
  Most of our attention focuses on Part II, Chapter I. Here, Bulgakov
  explores the relation between "creaturely nothing" and the underground
  (Bulgakov, \emph{Unfading Light}, 186\textendash 192). A more exhaustive account
  is required for the decision to include Part II, Chapter II, wherein
  Bulgakov introduces the concept of Sophia. Bulgakov aligns Sophia with
  the Orthodox notion of uncreated, divine "energies" (Bulgakov,
  \emph{Unfading Light}, 220).~These energies presume a different
  definition of "emanation" than the Neo-Platonists whom Bulgakov
  critiques. Divine energies proceed~from the uncreated divine essence
  while maintaining their status as uncreated. Unlike the divine
  essence, divine energies radiate through the world; they are
  accessible to creatures. The problem is that Bulgakov's notion of
  Sophia seems to~contradict the version of creation \emph{ex nihilo}
  that he outlined in Chapter I. In Chapter II, he leans towards
  creation \emph{ex deo\textemdash }creation out of God\textemdash rather than creation
  \emph{ex nihilo}. He appears to portray Sophia emanating into
  the~\emph{ouk on} rather than God creating \emph{mē on} from \emph{ouk
  on}. These contradictions are beyond our purview. One might suspect,
  nevertheless, that Sophia is the person who creates~\emph{mē on} out
  of~\emph{ouk on}. This is not to suggest that Sophia is the lowly,
  contingent demiurge of Neo-Platonism. Her uncreated perfection
  intimates that she has been filtered through the Orthodox notion of
  divine energies.~This explains why Bulgakov returns to his prior
  conception of creation \emph{ex nihilo} without noting a
  contradiction~(Bulgakov, \emph{Unfading Light}, 241, 243). This essay
  thus assumes it is justified to relate later passages on
  the~difficulty of conceiving and reaching~\emph{ouk on} to previous
  reflections on the underground (Bulgakov, \emph{Unfading Light},
  239\textendash 243; 266\textendash 270). On divine energy and essence, see David Bradshaw,
  "The Concept of the Divine Energies,"~in \emph{Divine Essence and
  Divine Energies: Ecumenical Reflections on the Presence of God in
  Eastern Orthodoxy}, eds. Constantinos Athanasopoulos and Christoph
  Schneider (Cambridge, UK: James Clarke, 2013), 27\textendash 49.}

Berdiaev invoked the underground to define history as an escalating
series of automatic emissions from a dynamic "nothing." The first
emanation was God. Bulgakov, by contrast, is not concerned with the
relationship between the underground and an abyss that is prior to God.
There cannot be anything before God, as far as he is concerned. Bulgakov
is more narrowly focused on the nothing that existed at the beginning of
the world. He employs the underground to uphold the idea that God
creates from a void so empty that the world must be the result of a
volitionary "fiat" rather than an emanation.\footnote{Bulgakov,
  \emph{Unfading Light}, 212.} His contention, therefore, is that "the
world is created out of nothing in the sense of \emph{ouk
on}."\footnote{Bulgakov, \emph{Unfading Light}, 189.} Following St.
Athanasius, Bulgakov proposes that only a universe fashioned in such
audacious freedom could produce creatures free to "dissolve again into
nonbeing."\footnote{This is a quote from Athanasius. Athanasius was
  also an advocate of divinization, a theory equally important to
  Bulgakov (Bulgakov, \emph{Unfading Light}, 472).} He was convinced
that only a world created out of nothing\textemdash \emph{ex nihilo}\textemdash could be
revered or rejected as a gift. An emanation was too "passive" to be
good.\footnote{Bulgakov contends that if the world is an emanation, one
  must assume that the world "passively diffuses"\textemdash rather than
  "manifests"\textemdash "divine light" (Bulgakov, \emph{Unfading Light},
  184).}

One reason that emanationist frameworks attract adherents is that the
idea of something coming from nothing seems
preposterous\emph{.}\footnote{Bulgakov's allegation of Neo-Platonism as
  a "hostile competitor" to Christianity is hyperbolic. While
  Gnosticism understood matter as too defective to participate in the
  Good, even a version of Neo-Platonism as extreme as Gnosticism
  followed the wisdom of the classical world: the intellect
  "participates" in the Good because it proceeds or pours forth from
  the Good. The "participatory ontology" of Ancient Greece and Rome,
  and its under-riding metaphor of "overflow," justified
  creation-as-emanation in the eyes of many Neo-Platonists. Christian
  theologians affirmed this basic ontology of participation. They simply
  used creation \emph{ex nihilo} to ensure that the soul's sharing in
  the Good\textemdash or in God\textemdash does not mean that the soul is identical with
  the Form of the Good. The world is gift. The emphasis that
  participatory metaphysics places on divine excess continued to put
  pressure on the axiom, in theories of creation \emph{ex nihilo}, of a
  divide between Creator and creature, as evidenced by the emanationist
  language of the Christian mystic Pseudo-Dionysius the Areopagite or a
  medieval theologian like Bonaventure. As Bulgakov's argument unfolds,
  he moves closer to the synthesis of emanation and creation \emph{ex
  nihilo} outlined by Pseudo-Dionysius, "the mysterious author of 'the
  Areopagiticum'\," (Bulgakov, \emph{Unfading} \emph{Light}, 190). See
  also W. J. Sparrow Simpson, "Introduction," in \emph{Dionysius the
  Areopagite on the Divine Names and the Mystical Theology}, trans. C.
  E. Rolt (London: SPCK, 1920), 14. Theologians have recently
  demonstrated an openness to Neo-Platonism. See John Milbank,
  "Christianity and Platonism in East and West," in \emph{Divine
  Essence and Divine Energies}: \emph{Ecumenical Reflections on the
  Presence of God in Eastern Orthodoxy}, eds. Constantinos
  Athanasopoulos and Christoph Schneider (Cambridge, UK: James Clarke,
  2013), 158. This tendency has led some, like Kathryn Tanner, to
  acknowledge that creation \emph{ex nihilo} is a "mixed metaphor" of
  "natural and personalistic images." Kathryn Tanner, "Creation 'Ex
  Nihilo' as Mixed Metaphor," \emph{Modern Theology} 29, no. 2 (April
  2013): 138. Creation out of nothing, some claim, chastens two
  classical notions of cosmic origins: creation-as-emanation and
  creation through pre-existing materials. See Andrew Davison,
  \emph{Participation in God: A Study in Christian Doctrine and
  Metaphysics} (Cambridge: Cambridge University Press, 2019), 68.}
Scientists have unearthed subatomic fields that seem to verge on
nothing.\footnote{The "zero-point energy" of the Higgs field, and its
  elusive particle the Higgs boson, does not signify a wholly empty,
  inviolate expanse in which there is \emph{no} substance, \emph{no}
  energy, and \emph{no} movement. The term, zero-point energy,
  identifies the lowest level of substance, energy, and movement
  possible. Nevertheless, some scientists, like Lawrence Krauss,
  continue to refer to subatomic particles as \emph{nothing.} Lawrence
  Krauss, \emph{A Universe from Nothing: Why There Is Something Rather
  than Nothing} (New York: Free Press, 2012). For philosophical or
  scientific reflections on nothing, see Roy Sorensen, \emph{Nothing: A
  Philosophical History} (Oxford: Oxford University Press, 2022); Andrew
  Davison, "Looking Back toward the Origin: Scientific Cosmology as
  Creation 'ex nihilo' Considered 'from the Inside,'\," in
  \emph{Creation 'ex nihilo': Origins, Development, Contemporary
  Challenges}, eds. Gary Anderson and Markus Bockmuehl (Notre Dame, IN:
  University of Notre Dame Press, 2018), 367\textendash 382.} However, none have
discovered an expanse free of energy or being. Unlike the relative
nothing of \emph{mē on}, total nihil is too negative to occupy a
position within a causal chain. "Nothing can come from nothing,"
Parmenides averred.\footnote{This article invokes the term "being" in
  the Platonic rather than in the Heideggerian sense.} Bulgakov is aware
that his support for the very concept of creation \emph{ex ouk on} (or
\emph{ex nihilo}) is exceedingly difficult. He claims that the "ancient
elder Parmenides" will ever "{[}raise{]} his voice anew," "insisting
only that which is \ldots{} \emph{mē on} \ldots{} exists; there is no
\emph{ouk on}."\footnote{Bulgakov, \emph{Unfading Light}, 190. Plato
  assumed that the world was created from preexisting elements rather
  than from nothing. See "Timaeus," in \emph{Plato: Complete Works},
  ed. John M. Cooper (Indianapolis, IN: Hackett Publishing Company,
  1997), 1238. In Genesis 1:1\textendash 2, God creates from preexisting waters.}
The \emph{ouk on} is a "limit concept" (\emph{predel'noe poniatie}) so
onerous that it will stretch any philosophical thread close to the
breaking point.\footnote{Bulgakov, \emph{Unfading Light}, 234.} Bulgakov
knows that he is in for philosophical turbulence.

\subsection*{Nothing Doubled}

The occasional ambiguity of Bulgakov's musings might be forgiven as the
by-product of logical or linguistic pressure. He is explicit,
nevertheless, that "God himself is the cause of nothing."\footnote{This
  is a quote by St. Maximus the Confessor that Bulgakov takes from S. L.
  Epifanovich, \emph{St. Maximus the Confessor and Byzantine Theology}
  (Kiev, 1915), 260\textendash 61, quoted in \emph{Unfading Light}, 470, nt. 10.}
Some forms of nonbeing remain impossible. Yet, contra Parmenides,
Bulgakov asserts that something can come from a radical iteration of
nothing.\footnote{Bulgakov distinguishes between nonbeing as it relates
  to the concept \emph{and} to the activity of "the absolute." He
  thinks Parmenides underestimates the extent to which the "activity"
  of the absolute can interact with \emph{ouk on} (Bulgakov,
  \emph{Unfading Light}, 188).} What follows is a remarkable formulation
of creation \emph{ex nihilo}. Bulgakov contends that God opens up the
void by "{[}outlining{]} a circle of His intentional inaction as the
realm of creaturely freedom." The cosmos begins when God creates
nothing in a moment of "divine self-exhaustion" (\emph{bozhestvennoe samoistoshchenie}) 
or self-delimitation.\footnote{Bulgakov,
  \emph{Unfading Light}, 209. The theory that God created the world from
  nothing by withdrawing into Godself is not unique to Bulgakov. The
  idea originates in the Kabbalism of the Jewish scholar Chayyim
  Vital~(1543\textendash 1620). For a history of the doctrine of divine "zimzum"
  (or "contraction"), and its influence on modern theological and
  philosophical discourse, see Christoph Schulte, \emph{Zimzum: God and
  the Origin of the World}, trans. Corey Twitchell (Philadelphia:
  University of Pennsylvania Press, 2023).} The "perfect nothing, which
God calls into being," is the \emph{ouk on}.\footnote{Bulgakov,
  \emph{Unfading Light}, 239.} The ensuing step is the "conversion" of
\emph{ouk on} into \emph{mē on.} God creates the conditions for the
world to emanate from meonic nothing. The maxim "nothing comes from
nothing" (\emph{mē on} from \emph{ouk on}), in Bulgakov's hands,
upholds rather than contests creation \emph{ex nihilo}.\footnote{The
  formulation, "nothing from nothing," should not be taken too
  literally. Bulgakov is reluctant to call \emph{mē on} "nothing,"
  even if it is a form of relative nothing or non-being alongside
  \emph{ouk on}. Bulgakov seems to equate \emph{mē on} with the
  potentiality that is associated with energy, a kind of background
  energy that is too fungible to manifest as a "thing." \emph{Ouk on}
  is more paradoxical; it is a form of potentiality that is devoid of
  energy. Bulgakov refers to \emph{ouk on} as nothing (\emph{nichto})
  and to \emph{mē on} as no-thing or something (\emph{nechto}). He does
  not want to stray too far from the standard definition of creation
  \emph{ex nihilo}: the creation of something from nothing. While
  "nothing from nothing" is rhetorically powerful, it is more accurate
  to say that Bulgakov espouses the creation of no-thing from nothing.}

However ingenious his thinking, problems persist. The notion of God
fashioning a chasm of nonbeing as desolate as \emph{ouk on} is
perplexing, though it does have profound meaning.\footnote{Absences,
  voids, abysses are philosophically timely. Slavoj Žižek argues the
  fact that everything comes from and hurdles towards nothing suggests
  that the universe is grounded in a "pre-ontological" nothing. From
  here, Žižek reformulates the paradoxes of the Freudian death drive
  within a cosmic context. He insists that being arises from the
  primordial void when "division divides itself from itself." Slavoj
  Žižek, \emph{Less Than Nothing: Hegel and the Shadow of Dialectical
  Materialism} (London: Verso, 2012), 15. The birth of being and the
  drive towards the void are simultaneous. For Žižek, no thing or
  subject can survive such negativity other than an "obscene, 'partial
  object,'\," sustained by the "Holy Spirit" of negation (Žižek,
  \emph{Less Than Nothing}, 5). Žižek is not entirely serious. He plays
  with religious concepts as would Lacan. He assumes that God is a
  fictional Other that one uses to grasp reality. Mikhail Epstein also
  sets forth a "theology of the vacuum," but with greater sincerity and
  optimism than Žižek. See Mikhail Epstein, \emph{Religia posle ateizma.
  Novye vozmozhnosti teologii} (Moscow: ACT Press, 2013), 323\textendash 327.
  Epstein may concur with Žižek that the doubling of negatives is
  integral to the capacity to create something from nothing. But he is
  more willing to equate two "nots" with a real positive. He insists
  that affirmations which result from negations constitute ideal
  metaphysical environs in which a subject or personality is
  destabilized rather than destroyed. This instability opens life-giving
  possibilities. Nothing, for Epstein, creates a person in a way that it
  does not for Žižek. Epstein's statement that "nothingness itself
  contains the beginning of its own nothingness" should not be
  understood to endorse agnosticism (Epstein, \emph{Religia posle
  ateizma}, 327). God, for Epstein, is a "Suprasubject"
  (\emph{Sverkhsub'\,'ekt}), who, like human subjects, is constituted by
  dynamic nots, including the difference between a subject and that
  which a subject creates, which, in God's case, is a human subject
  (Epstein, \emph{Religia posle ateizma}, 349\textendash 351). Epstein implies
  that God is "born" from nothing but leaves this nothing a
  metaphysical mystery.} The maximal language that Bulgakov associates
with the \emph{ouk on} can make matters worse. 
He frames the ouk on as "perfect nothing" (\textit{sovershennoe nichto}), "pure nothing" (\textit{chistoe nichto}), 
"ultimate emptiness" (\textit{okonchatel’naia pustota}), 
or the "absolute null of being" (\textit{absoliutnyi nul’ bytiia}).\footnote{Bulgakov,
  \emph{Unfading Light}, 207\textendash 208, 191.} The question becomes: how can
the fullness of being, or God, create "the full negation of
being?"\footnote{Bulgakov, \emph{Unfading Light}, 189.} Accusations of
metaphysical absurdity can be assuaged by noting that Bulgakov is
working with a more qualified conception of "pure nothing" than it
sometimes appears. Not once does Bulgakov refer to the \emph{ouk on} as
"absolute Nothing" (\emph{absoliutnoe Nichto}). God as absolute
reality precludes absolute\textemdash capitalized\textemdash Nothing. Bulgakov, as stated,
deemed that pantheism's identification of God with everything was
tantamount to atheism. Conversely, absolute nothing is only possible if
God does not exist. Therefore, the~\emph{ouk~on~}must be a different
type of nothing, one that stands in relation to creation and to the
possibilities of free will and evil. That relation is what Bulgakov is
trying to explain. For him, the~\emph{ouk~on}~is neither located before
nor outside God but at the "edge of Being," the edge of God. Here lies
a simple nothing that is wholly devoid of the potentiality~associated
with the formless energies in a cosmic vacuum. Bulgakov is, however,
willing to concede that~\emph{ouk~on}~retains a causal link to the
source of all potentiality in God, so long as one does not reduce the
mystery implicit within the origins of "pure nothing" to a finite
process of cause and effect. We can risk saying that, for Bulgakov,
the~\emph{ouk~on}~"contains" potentiality, but it is of a more
ephemeral or derivative sort than~the vitality of~\emph{mē~on.}~The
emptiness of~\emph{ouk~on~}provides space, and thus the
possibility,\emph{~}for creatures to accept or reject~\emph{mē~on,~}or
the potentiality that\emph{~}God has gifted.\emph{~}The~\emph{ouk~on~}is
not absolute nothing, but the bleakest echelon of relative nothing.~

Even this more limited understanding of the "dark foundation of the
cosmos" remains more radical than most philosophers would dare
conceive.\footnote{Bulgakov, \emph{Unfading Light}, 267. Smith's
  translation\textemdash "dark, mute foundation"\textemdash emphasizes the "deafness"
  of the \emph{ouk on}. I have rendered the translation more literally,
  as the "dark foundation of the cosmos" (\emph{temnaia osnova
  mirozdaniia}).} Bulgakov must admit that creation \emph{ex nihilo}
"remains a riddle \ldots{} a miracle, a mystery," an enigma.\footnote{Bulgakov,
  \emph{Unfading Light}, 189.} He does not offer an exhaustive or
rational explanation for the origin of the world, for to do so is
impossible and counterproductive. Yet Bulgakov will not settle for
fideism. He insists that "to identify what has been accomplished is for
human consciousness fully in keeping with its powers."\footnote{Bulgakov,
  \emph{Unfading Light}, 189.} Creation is not explicable so much as
intelligible. One has to reason "indirectly," by "illegitimate"
(\emph{nezakonnorozhdennyi}) means.\footnote{Bulgakov, \emph{Unfading
  Light}, 190. Bulgakov notes how Plato, in the \emph{Timaeus,} assumes
  that \emph{ouk on} is accessible to thought via negation but fails to
  insert \emph{ouk on} prior to creation. Bulgakov implies that one
  requires the more experiential or rebellious negation of the
  underground to make such a cosmic induction.} One must travel
underground. It is at this point that Bulgakov refers to the experience
of those who carve out a space in the bastard underbelly of existence.
As he writes:

\begin{quote}
The absolute null of being as its sole pure possibility without any
actualization remains transcendent for the creature, which always
represents the indissoluble alloy of being and nonbeing. But this
"outer darkness," this naked potentiality {[}\emph{golaia
potentsial'nost'}{]}, in the underground of creatureliness, is like some
second center (pseudo-center) of being, competing with the Sun of the
world, the source of its fullness. For the heroes of the underground it
has a unique attraction; it summons in them the irrational, blind will
towards nothing, a dizzying yearning for the abyss. \ldots{} The kingdom of
nihilism, the cult of nothing, hell, exists only at the expense of the
positive forces of being, by an ontological theft.\footnote{Bulgakov,
  \emph{Unfading Light}, 191.}
\end{quote}

\noindent The "heroes of the underground" reverse the order of creation. They
begin by misconstruing \emph{mē on} as a howling vacuum of indeterminate
potentiality. Since "pure possibility" is "not actualizable," or
actualizable as nothing, it lacks the one capacity that matters most:
the ability to sustain and contain material existence\emph{.}
De-creation is synonymous, for Bulgakov, with dematerialization. "Naked
potentiality" breaks the "indispensable alloy" of materiality and
nothing, matter and \emph{mē on}.\footnote{Bulgakov also describes evil
  as the "the actualization of nothing" (Bulgakov, \emph{Unfading
  Light}, 267).} Comprised of "pure"\textemdash brute\textemdash possibility, the
underground is preset for a cosmic rewind. One witnesses "the
metaphysical annihilation of being, the decomposition of the \emph{mē
on} into the \emph{ouk on}, the plunging of the creature into its
original, dark nothingness."\footnote{Bulgakov, \emph{Unfading Light},
  207.}

\subsection*{Facing the Self, Facing the Darkness}

We can now determine how Bulgakov perceives the underground to develop
within its full psychological and cosmic backdrop. The underground is
for those who act as if their creatureliness is absolute. Enthusiasts of
the underground resemble pantheists, from whom they can be distinguished
by their zealous solipsism. Underground-dwellers withhold their
attention from any object other than the bare vitality that resides
within their individual selves. To choose the underground is thus to
affirm "oneself in one's own selfhood."\footnote{Bulgakov claims: "to
  want oneself in one's own selfhood, to lock oneself in one's
  creatureliness as in the absolute, means to want the underground and
  to be affirmed in it" (Bulgakov, \emph{Unfading Light}, 187).
  Bulgakov, in my view, interprets the underground through the
  subject-object relation, the beloved philosophical tool of German
  Idealism. He follows Hegel's assumption that all conscious subjects
  attempt to know objects in the world. Hegel assumed that if a subject
  tries to know an object that is not sufficiently distinct from the
  subject\textemdash or is devoid of content\textemdash then the subject self-dissolves.
  The underground similarly emphasizes how subjective action devoid of
  an object is powerless. Brandon Gallaher argues that Bulgakov's notion
  of creation \emph{ex nihilo\textemdash }as a subject moving towards an utterly
  distinct object of pure nothing\textemdash models his debt to Hegel. See
  Brandon Gallaher, "Bulgakov's Chalcedonian Ontology and the Problem
  of Human Freedom," forthcoming. We might go so far as to say that the
  Underground Man shows how only God can move towards an object as empty
  as \emph{ouk on} and not self-destruct. See Hegel's account of the
  transition from Stoicism to Skepticism in Georg Wilhelm Friedrich
  Hegel, \emph{The Phenomenology of Spirit}, trans. Terry Pinkard
  (Cambridge: Cambridge University Press, 2018), 117\textendash 123.} The heroes
of the underground wall themselves within the pure interiority of their
own consciousness to avoid any feedback, any hint of criticism or
dependency, which activity involving matter and flesh inevitably
produces. They shackle themselves with breathless self-justifications
only to divulge a deeper insecurity, the "pain of impotence and lack of
talent."\footnote{Bulgakov, \emph{Unfading Light}, 187.} The
underground cannot deliver on its promise of self-sufficiency. Any
pretense to the ardor of eremitic cave-dwellers descends into banality.
Putrefaction rather than divinization sets in. One can discern within
the interplay of self-doubt and metaphysical ambition a yearning for an
oblivion from which there is no return. A space shaped by a vendetta
against creatureliness, for Bulgakov, is a void that is more sweeping
than death, more consuming than raw possibility. The "cult of nothing"
hopes for complete estrangement from God, for total de-creation, for a
darkness so outermost that one cannot hear weeping and the gnashing of
teeth. Naked potentiality faces the deafening silence of the \emph{ouk
on}.

What began as a project of self-affirmation, Bulgakov insists,
crescendos in a botched "metaphysical suicide."\footnote{Bulgakov,
  \emph{Unfading Light}, 268.} The underground illustrates a central
point from \emph{Unfading Light}: the bid "to get out of the inflamed
circle of being" discovers that it "can never get to the end" of
being.\footnote{Bulgakov, \emph{Unfading Light}, 268.} Those who flee
underground never reach pure nothing. They wind up "twisting and
turning convulsively" in the cacophonous pseudo-nothing of the
underground.\footnote{Bulgakov, \emph{Unfading Light}, 268.} The
closest comparison, for Bulgakov, is hell. In the underground, as in
hell, one "cannot say with full sincerity: die, for already in the very
act of affirmation towards nonbeing it realizes itself as
being."\footnote{Bulgakov, \emph{Unfading Light}, 268.} The enduring
shadows of the "infernal underground" confirm that even the most
abysmal negations presuppose a "positive {[}expression{]} about
being."\footnote{Bulgakov, \emph{Unfading Light}, 109.} Bulgakov
implies that the underground does not only overestimate what creatures
can achieve. The paragons of the underground are doomed, because the
God-free gorge for which they fantasize does not exist. They are
working, we might add, with a mistaken notion of \emph{ouk on} as
absolute nothing.

Some would insist that the incapacity of the underground to annihilate
being proves that the world is the product of a simple emanation.
Bulgakov, ever the optimist, presumes that the failures of the
underground offer indirect evidence that the world is created \emph{ex
nihilo}. The logic is simple. If the sole desire of the underground is
to undo what God creates, then Bulgakov assumes it is plausible for an
inexplicably stark void to occupy a seminal role at creation. The fact
that mortals cannot act on such a severe gradation of nothing is
expected. Pure nothing is "transcendent for the creature," but not for
God. And yet, Bulgakov contends that the \emph{ouk on} cannot be so
perplexing as to verge on metaphysical nonsense. He assumes that if one
can identify an "affirmation" of being in the hellish underground,
then one can do the same in a yet more cavernous abyss. This lower
circle of nothing\textemdash the \emph{ouk on}\textemdash would have to be so depleted of
potentiality that it could not emanate from God. The covert affirmation
of pure nothing is that God created it through a radical act of free
will.

It is worth recalling that the chief blunder of the underground is its
inability to eradicate what God has made so as to allow for renaissance
and self-divinization on the other side. From this vantage point,
Bulgakov makes bolder assessments on the accessibility of \emph{ouk on}.
A blank metaphysical slate, he reassures us, cannot literally be
"caught by the senses" (\emph{chuvstva}).\footnote{Bulgakov,
  \emph{Unfading Light}, 240.} And yet, Bulgakov implies that the
\emph{ouk on} can be "felt" (\emph{oshchushchaetsia}) or "touched"
(\emph{oshchupat'}) intuitively in the underground.\footnote{He writes:
  "in the cold of death \ldots{} the abyss of \emph{ouk on} is
  felt {[}\emph{oshchushchaetsia}{]}" (Bulgakov, \emph{Unfading Light},
  191). Further, "in order to reach the \emph{ouk on} one has
  to peek behind the coulisses of being so to say, or by remaining on
  its facial surface, feel {[}\emph{oshchupat'}{]} its underside"
  (Bulgakov, \emph{Unfading Light}, 240).} Bulgakov's reading of
\emph{Demons} provides direction on how to frame the "experience" of
pure nothing. In the suicide note left by Stavrogin, which Bulgakov
considered the consummate confession of a derelict personality, he
divulges that "all that poured out of me was denial." Eventually, he
admits that "not even denial came out."\footnote{Bulgakov quotes
  Stavrogin's note directly ("Russkaia tragediia," 512).} He was not
divine but "small and sluggish." Bulgakov intimates that the semblance
of dynamism that de-creation proffers cannot go on indefinitely. The
thrill of ontological larceny succumbs to sloth. Bulgakov does not make
the point explicit, but we might assume that the boundless "vulgarity"
(\emph{poshlost'}) of the underground is undergirded by a similar
expenditure.\footnote{In the segment, "What is Matter," Bulgakov uses
  horizontal images\textemdash "coulisees," facial surfaces, or flat
  drawings\textemdash to tease out the "illegitimate judgement" by which
  creation \emph{ex nihilo} is perceived. The metaphor of flatness, in
  my view, can elucidate prior associations which Bulgakov makes between
  the \emph{ouk on}, the underground, banality, and the "coldness of
  death" (Bulgakov, \emph{Unfading Light}, 240\textendash 241). Dmitry
  Merezhkovsky describes the demonic in Gogol's art as "eternal
  flatness, eternal vulgarity." See Dmitry Merezhkovsky, \emph{Gogol' i
  Chert} (Moscow: Issledovanie, 1906), 2. Bulgakov likens the demonic
  vulgarity of the Underground Man to the hapless opportunist Khelstakov
  from Nikolai Gogol's \emph{The Inspector General} (Bulgakov,
  \emph{Unfading Light,} 187)\emph{.} It does not seem too far-fetched
  to presume that Bulgakov is extrapolating from the spatial-temporal
  paradigm put forth by Merezhkovsky in his reflections on Gogol.
  Bulgakov was an acquaintance of Merezhkovsky and a contributor to his
  journal \textit{Novyi put} {[}New path{]}. Thomas Allan Smith,
  "Translator's Introduction," in \emph{Unfading Light} (Grand Rapids:
  Eerdmans, 2012), xxii.} In the inertia that precedes the "cold of
death," space-time slows, congeals, and flattens.\footnote{Bulgakov,
  \emph{Unfading Light}, 191. Stagnation in the underground literalizes
  and parodies Bulgakov's notion of divine "self-exhaustion" prior to
  the creation of the \emph{ouk on} (Bulgakov, \emph{Unfading Light},
  209).} Creeping stagnation in the underground reveals its environs as
a façade, a bottomless surface, distended over an ultimate
emptiness\emph{.}\footnote{Bulgakov, \emph{Unfading Light}, 291.} Here
"the abyss of the \emph{ouk on} is felt." The failure of the
underground to identify alternative life-resources upholds, in the
strongest sense, nature's "unceasing whisper: you do not have the root
of your being in yourself; you are created."\footnote{Bulgakov,
  \emph{Unfading Light}, 181.} Thus is creaturely life a relief from
despair.

The Underground Man cannot be dismissed as fiction. He is proof, for
Bulgakov, that "everything living {[}experiences{]} the temptation of
metaphysical suicide."\footnote{Bulgakov, \emph{Unfading Light}, 268.}
This "vertiginous urge downwards" accommodates the Serpentine
temptation in Genesis to be "like gods" but does not linger to hear
God's sentence.\footnote{Bulgakov, \emph{Unfading Light}, 268\textendash 269.} One
exiles oneself from the Eden. Neither is it necessary to wait for the
afterlife to perceive that "metaphysical pessimism" descends into
"insincerity and hypocrisy."\footnote{Bulgakov, \emph{Unfading Light},
  268.} Entombment in the underground will suffice. Bulgakov saw the
underground as an experiment from which one can induce human morality,
interiority, and origins. Creation \emph{ex nihilo} remains an enigma.
Yet the underground shows that creation out of nothing\textemdash or the creation
of nothing out of nothing\textemdash is plausible by appealing to
experience.\footnote{Decades later, Bulgakov returns to the relation
  between creation and self-destruction while contemplating the end of
  all things. In "The Problem of the Conditionality of Immortality"
  (1936\textendash 1937), Bulgakov criticizes a new theory of human destiny:
  conditionalism. Conditionalism abandons the idea that the human soul
  is naturally eternal. See Sergei Bulgakov, "The Problem of the
  Conditionality of Immortality," in \emph{The Sophiology of Death:
  Essays on Eschatology: Personal, Political, Universal}, ed. and trans.
  Roberto J. De La Noval (Eugene, OR: Cascade, 2021), 41. Hell, for
  conditionalists, does not exist. What awaits humans is either heaven
  or annihilation. While Bulgakov is no friend to the doctrine of hell,
  he regards conditionalism as more troubling, in some respects, than
  the idea of eternal torment. He is perturbed by the conditionalist
  belief that humans are free to take their own souls out of existence.
  It is as if conditionalists assume that humans can commit
  "metaphysical suicide" (Bulgakov, "The Problem of the
  Conditionality," 67). However, Bulgakov is confident that humans
  cannot obliterate themselves any more than they can act on the
  \emph{nothing} from which they were made. He writes: "surely as man
  cannot create something from nothing, so too he cannot plunge any
  being into non-being, dissolve it into nothingness \ldots{} humanity can
  transform the modes of being and destroy its given forms, and in this
  sense the destructive energy of man is empirically not limited. But
  ontologically this energy remains powerless: the world is upheld by
  God in its being and it cannot be returned by man to the abyss of
  non-being, to the darkness of nothing" (Bulgakov, "The Problem
  of the Conditionality," 67). Bulgakov continues to assume, late in
  his career, that the impossibility of metaphysical suicide implies
  that God alone can create from nothing.}

\widowpenalty=0
\subsection*{The Question of Naked Potentiality}

Bulgakov agreed with Berdiaev that naked potentiality comprised the
underground of the soul.\footnote{Berdiaev prefigures the "tehomic
  theology" of Catherine Keller. Her book, \emph{The Face of the Deep,}
  deconstructs creation \emph{ex nihilo} by focusing on its failure to
  eradicate the "aboriginal potentiality" of the turbulent waters or
  "\emph{tehom"} prior to creation in Genesis. See Catherine Keller,
  \emph{The Face of the Deep: A Theology of Becoming} (London:
  Routledge, 2003), 76. Keller replaces \emph{creatio ex nihilo} with
  \emph{creatio ex profundis} (Keller, \emph{Face of the Deep},
  155\textendash 238). She hopes to re-release the "self-organizing
  potentiality"\textemdash the "feminine" "chaoplexity"\textemdash of the primordial
  "Deep." Keller would criticize Berdiaev's trust that God will
  harmonize the tehomic abyss in the eschaton. He would remain beholden,
  in her view, to the "masculine" fixation\textemdash epitomized in creation
  \emph{ex nihilo\textemdash }with "\emph{chaoskampf}" (Keller, \emph{Face of
  the Deep}, xix). Bulgakov might critique Keller's "Deep" as more
  violent than she assumes. Indeed, military analysts have highlighted
  "chaoplexity" as an effective strategy to annihilate enemies in
  battle. Christopher R. Paparone and George L. Topic Jr., "Dealing
  with Chaoplexity," \emph{Army Sustainment} 45, no. 5 (2013): 6. Much
  rides on the nature of the "\emph{khôra}" or "receptacle" in
  Plato's \emph{Timaeus.} Bulgakov associates \emph{khôra} with the
  emptiness of \emph{ouk on} while Keller and Berdiaev link \emph{khôra}
  with untamed potentiality\emph{.}} He conceded that baptism into this
underground was an ill-fated attempt to be reborn within a primordial
cosmic chasm. Bulgakov departed, however, from the assumption that the
underground constituted authentic, divinely charged depth. The heroes of
the underground exposed naked, raw potentiality as emaciated rather than
purified, because it was grounded in a void wholly lacking in
potentiality. The fact that naked potentiality was the epiphenomenon of
\emph{ouk on}, moreover, confirmed that raw possibility traveled in one
direction. Like a "centrifugal force," Bulgakov assumed that naked
potentiality flung everything it touched far from the reservoir of God's
being.\footnote{Bulgakov, \emph{Unfading Light}, 269.} The \emph{ouk on}
was not at the core of God; it was at the edge.\footnote{In spherical
  space, only the center is deep. Nevertheless, the \emph{ouk on} is
  deep in the sense that it is "the bottom of hell"\textemdash the foundation
  of the underground (Bulgakov, \emph{Unfading Light}, 234).} The
liminality and emptiness of \emph{ouk on} together revealed the
"freedom" of naked potentiality as too insubstantial and feral to be
useful.\footnote{In his late autobiographical
  work,~\emph{Self-Knowledge}, Berdiaev~tackles~the assumption, widely
  held by his contemporaries, that his views of the \emph{Ungrund} are
  identical to those of Boehme. While Berdiaev aligns himself with
  Boehme\textquotesingle s definition of the \emph{Ungrund} as a
  "primitive, pre-ontological freedom" (\emph{pervichnuiu,
  dobytiistvennuiu svobodu}), he insists that his understanding of the
  relation between primordial freedom and God~differs~from Boehme's. He
  writes: "according to Boehme the freedom of the \emph{Ungrund} is in
  God {[}\emph{v Boge}{]}, as His dark beginning, whereas I conceived
  freedom to be outside of God {[}\emph{vne Boga}{]}. More precisely,
  freedom exists outside of 'Gott' {[}or God{]} but not outside of
  'Gottheit' {[}or Godhead{]}; for, about the ineffable Gottheit,
  nothing can be thought." Nikolai Berdiaev, "Samopoznanie: opyt
  filosofskoi avtobiografii,"~\emph{Sobranie sochinenii~}(Paris:
  YMCA-Press, 1949), vol. 1: 113. Of course, it is possible that
  any~difference~between~Boehme~and Berdiaev on the \emph{Ungrund} was
  initially more minimal than Berdiaev~suggests. Regardless of whether
  Berdiaev is telling the truth, Bulgakov would have seen the spatial
  and temporal language that Berdiaev continued to ascribe to the
  \emph{Ungrund} as revealing. He would have balked at the idea that a
  primal freedom can exist outside ("\emph{vne}") and before
  ("\emph{dо"}) God and not serve as an ontological source\textemdash a
  Boehmian "first divinity"\textemdash that remains partly within
  God\emph{.~}There is an argument to be made that the later Berdiaev
  remained a firmer disciple of Boehme than he intended.} A potentiality
so denuded or cheapened, in his view, merely "{[}begat{]} towards
isolation" and loneliness.\footnote{Bulgakov, \emph{Unfading Light},
  208.} Bulgakov could not justify the limitless liberty of the
underground and any resulting evil as a tragic necessity. He could only
conceive of the underground as "tragic" in order to dismiss the idea
that its ruinous energies could be repurposed for good.\footnote{According to Joshua 
Heath, Bulgakov alternates between two definitions of tragedy. First,
Bulgakov understands tragedy as the "opposition of good and evil," or the
"confrontation of being with the 'nothing' 'from which' the world was made."
Second, he invokes tragedy as "diremption," or "the separation of what ought
to be joined in love." See Joshua Heath, "The Eternal Sacrifice," \textit{Modern
Theology} 41, no. 1 (January 2025): 143–166, esp. 144. Heath explores how
Bulgakov's conception of tragedy as diremption sublimates and transforms
the other definition of tragedy. Heath reads Bulgakov to suggest that Christ's
cry of dereliction does not imply tragic separation within the Trinity; rather,
this is the moment in which the Trinity overcomes any dualism between good and
evil, a dualism that would, moreover, jeopardize the absolute nature of the good.
On the cross, evil is not reproduced; it is transformed. Instead of punishing
or forsaking the Son, as if He were a typical sacrifice, the persons of the Trinity
co-mourn or "co-die" with the Son (Heath, "Eternal Sacrifice," 153). The Trinity
thus responds to the threat of diremption in a manner that reaffirms the perfection
of its unity. God admits that which God is not (i.e., evil) without taking on
an external imperfection or conceding one internally. As Heath writes,
Bulgakov assumes "only that identity is absolute, which is constituted by the admission
of what it is not" (Heath, "Eternal Sacrifice," 153). I would add that the principle of extravagant—or absolute—admission applies to God’s dealings with primordial nothing. My article also attempts to explain why Bulgakov cast the tragic conflict between good and evil in terms of the struggle of being with nothing. A battle so defined, it would seem, is not endless. If evil is based in nothing, then it is too compromised to exist indefinitely.} The underground was evidence
that evil lacked a natural abode in the heart, in the cosmos, and in
God.

Berdiaev and Bulgakov defined \emph{mē on} as the invisible fuel of
creativity and life, but they diverged on specifics. Berdiaev equated
\emph{mē on} with naked potentiality. Bulgakov saw naked potentiality as
closer to \emph{ouk on} than to \emph{mē on}.\footnote{Bulgakov,
  \emph{Unfading Light}, 189.} Meonic nonbeing, in his view, was
"something" (\emph{nechto}) with ontological reality, in contrast to
the empty nothing (\emph{nichto}) of \emph{ouk on} and its "rebellious
nothing" of pure possibility.\footnote{Bulgakov, \emph{Unfading Light},
  188\textendash 189.} The potentiality of \emph{mē on} was not unreal so much as
"\emph{some}thing" unknown. Bulgakov, nevertheless, \emph{was} certain
that \emph{mē on} did not presume a vision of time and space, like the
underground, which plagiarized creation \emph{ex nihilo}. As he wrote,
\emph{mē on} was "not creativity out of nothing but creativity
in nothing out of divine something."\footnote{Bulgakov, \emph{Unfading
  Light}, 208.} Berdiaev assumed that \emph{mē on} propagated itself
because it was uncreated. For Bulgakov, \emph{mē on} was creative
because it accepted its status as created.\footnote{Bulgakov writes:
  "the \emph{mē on} is pregnancy, the \emph{ouk on} is sterility"
  (Bulgakov, \emph{Unfading Light}, 189).} Meonic freedom radiated from
the point at which God fashioned "something" out of nothing as a
radical display of love.\footnote{Bulgakov refers to \emph{mē on} as
  "the essence of creatureliness" (Bulgakov, \emph{Unfading Light},
  191).} The primary qualities of \emph{mē on}\textemdash "nonmanifestation and
nondefinition"\textemdash thus pointed to a surplus of perfection rather than an
excess of deficiency.\footnote{Bulgakov, \emph{Unfading Light}, 189.}
Meonic nonbeing, whatever it was precisely, could be no less than a
celestial spring that nourished the "whole wealth and fullness of
being" in all its materiality.\footnote{Bulgakov, \emph{Unfading
  Light}, 189.} The sublimity of the created \emph{mē on} also
accentuated that the \emph{ouk on}, for Bulgakov, was not evil. Only the
attempt to return and create from the \emph{ouk on} was evil. By itself,
the \emph{ouk on} signaled that the "world was founded by a cross" of
measureless humility. The Absolute provided space for creatures to make
moral decisions rather than assert its absoluteness over
creation.\footnote{Bulgakov, \emph{Unfading Light}, 185.} Creation
\emph{ex nihilo}, in all its layers, comprised an unrepeatable act of
grace.\footnote{Brandon Gallaher's explanation of Sophia as a "divine
  energy field" underscores Bulgakov's preoccupation with energy
  (Gallaher, "The Problem of Pantheism," 154\textendash 155, 163). I suggested
  previously that Bulgakov invokes Sophia as a divine "person" who
  creates the \emph{mē on}. It is also plausible that Sophia accentuates
  the uncreated aspect of the background energy in the universe.
  \emph{Mē on}, for Bulgakov, emphasizes its created nature.}

Berdiaev saw little distinction between divine and human creativity.
Each drew from an internal expanse brimming with uncreated potentiality.
He suspected that any teaching which insisted otherwise might have
originated in the "desire to humiliate humans."\footnote{Nicolas
  Berdyaev, \emph{The Destiny of Man} (New York: Harper \& Brothers,
  1960), 26.} On such points, Bulgakov thought that Berdiaev could not
have been further from the truth.\footnote{Georges Florovsky~labels
  Berdiaev's\emph{~Creative Act}~as a "new phase of utopianism." He
  criticizes Berdiaev's emphasis on interiority for paying scant
  attention to asceticism.~See Georges Florovsky,~\emph{Ways of Russian
  Theology} (Part Two),~\emph{Collected Works of Georges Florovsky}, ed.
  Richard S. Haugh, trans. Robert\emph{~}Nichols\emph{~}(Vaduz,
  Liechtenstein: Büchervertriebsanstalt, 1987), vol. 6,\emph{~}275.
  Bulgakov is preoccupied with the Hegelian aspects of Berdiaev's
  utopianism. At the same time, he~illustrates how Berdiaev's
  fascination with the \emph{Ungrund} may be guilty of the same
  metaphysical pessimism that Berdiaev famously attributed to Gogol.
  Berdiaev wrote, "the tragedy of Gogol lay in the fact that he never
  could see and depict the human, the image of God in man." Nikolai
  Berdiaev,~\emph{The Russian Idea~}(New York: The Macmillan Company,
  1948), 8.} The inability of humans to reduce themselves to nothing
illustrated, for Bulgakov, that God alone contained the freedom to
create from nothing. Bulgakov regarded this limitation as empowering.
The "gulf" between humanity and God\textemdash between "image and Prototype
(\emph{Pervoobraza})\textemdash was a foundation necessary to affirm the dignity
and trajectory of the human creature. Bulgakov assumed that humans could
only "become god," as Athanasius had envisioned, so long as God was
starkly different from them.\footnote{Bulgakov intimates that the
  Athanasian aphorism, "God became human so that we might become god,"
  does not suggest that God is no longer Absolute. Indeed, God must
  remain God for the formula to work.} There had to be a real Prototype,
a truly transcendent object, toward which people could aspire. According
to Bulgakov, the distinction between God and the world functioned as a
receptacle for evil, without which evil was redistributed within divine
and human hearts. It is true that Berdiaev tried to address these
depths\textemdash but Bulgakov remained skeptical of those efforts. For him, if
Berdiaev seemed ambiguous as to whether it was freedom or
determinism\textemdash creativity or History\textemdash that saved, it was because
Berdiaev was more cynical about the capacity of humans to manage their
depths than he let on.

Berdiaev was intent on giving humanity a promotion. But, in Bulgakov's
view, this would be an artificial or ineffectual advance.\footnote{Bulgakov,~\emph{Unfading
  Light}, 487, nt. 3.} Berdiaev made creatures seem better by making the
Creator seem worse. God was no longer Absolute. And the demotion of
Divinity came at the cost of the human capacity to achieve divine
perfection. So it was that Bulgakov, in the end, disparaged Berdiaev's
creative act as a parody of creative freedom.

\widowpenalty=10000
\clubpenalty=10000
\subsection*{Conclusion}

Ancient Greece did not have a number for zero.\footnote{Robert Kaplan,
  \emph{The Nothing that Is: A Natural History of Zero} (Oxford: Oxford
  University Press, 1999), 17.} Hellenic ambiguity towards numerical
naught was consistent with a larger metaphysical misgiving towards
radical absence. Berdiaev and Bulgakov, as Orthodox thinkers immersed in
the Platonic tradition, inherited the pervasive Greek perplexity at
nothing. Each saw the Underground Man's failure to reduce himself to a
cipher in the recesses of his mind as proof that absolute nothing was
metaphysical nonsense. The underground became an opportunity to analyze
more moderate gradations of nothing within human interiority, and,
ultimately, within the primordial void out of which God created the
world. In this way, their inquiry into "deep psychology" matured
into "deep theology."\footnote{Yuri Corrigan,
  "Dostoevsky's Depth Theology," Forthcoming in \emph{The Oxford
  Handbook of the Russian Novel}. Natural theology is the use of the
  natural world or experience to reason inductively about God.}

{\setlength{\skip\footins}{8pt minus 6pt}
Bulgakov did not deem the deep theology of Berdiaev a success. He
criticized Berdiaev for extolling freedom and creativity as a panacea
while evincing a more basic trust in automized processes to usher God,
humanity, and their unruly interiors to the appropriate terminus.
Berdiaev fetishized the verticality, mystery, and vitality of the
Deep.\footnote{\emph{The Meaning of the Creative Act} begins with the
  lines: "The human spirit is in prison. \ldots{} The true way is not a
  movement to right or left in the plane of 'the world,' but rather
  movement upward and downward on lines of the ultra-worldly, movement
  in spirit and not in 'the world'\," (Berdiaev, \emph{The Creative
  Act}, 2).} He believed in creativity, not in creation.\footnote{For
  this distinction I am indebted to Rowan Williams. Williams intuited
  the profundity of Bulgakov's thought long ago, juxtaposing it to the
  "emptiness" of Berdiaev's rhetoric. See Todd Breyfogle and Rowan
  Williams, "Time and Transformation: A Conversation with Rowan
  Williams," \emph{CrossCurrents} 45, no. 3 (Fall 1995): 293\textendash 311.} As
a result, his project was thwarted by naivete, contradiction, and, most
troubling to Bulgakov, a strained relation with the horizontal nature of
material existence. Berdiaev may have rejected, in theory, the
assumption that any stratum of being or nonbeing could be independent of
divinity. But in Bulgakov's view, Berdiaev continued to be unduly
captivated by the prospect of unmediated access to potentiality, and
this fascination or fantasy prevented him from entertaining more
paradoxical shades of nonbeing like the \emph{ouk on}. Once again,
Bulgakov pointed out how Berdiaev paid a heavy price for such
imaginings: if God was not free, neither were humans. Evil, moreover,
permeated farther into reality than Berdiaev cared to admit.

Bulgakov, in response, developed a vision of interiority that was not
juxtaposed to the texture of creaturely life. He contended that the
material world, despite the challenges and seductions it presents, was
an unfathomable gift along with the nothing on which it rests. This gift
could be celebrated in the heavenly abyss of one's soul. One could, as
with any gift, decline. But there was a catch. One refused to believe in
the inexplicable event of creation from nothing at the risk of
entertaining a more literal irrationality in the underground. Bulgakov
presupposed that the underground emerged from the illusion that one
could exist at all, acting as if what one owned were not a blessing from
God. The underground was foremost not a failure to create but a failure
to receive the creation as a gift and its creator as the giver. Bulgakov
was adamant that this was a giver who was worthy of being accepted. For
him, there could be no dark source, no tragedy, in God.\footnote{Bulgakov
  insists: "[I]t is impious and absurd to speak about 'tragedy in God'. [\ldots{}] 
  God is at liberty to introduce himself into the tragic
  process of world history, while remaining in himself and for himself
  free from it. This is why in the Absolute itself there is no place for
  tragedy, which is rooted in the antagonism of the shattered forces of
  relative being" (Bulgakov, \emph{Unfading Light}, 186). Later in his
  career, Bulgakov entertains the idea that there exists "tragedy in
  God." However, he invokes divine tragedy in a very different sense
  than does Berdiaev (see footnote 119).} Evil remained a mystery, like
creation \emph{ex nihilo}. Not knowing everything was no excuse for
failures in gratitude and humility. One could be thankful that evil did
not have "a place in this universe" because its energies moved away
rather than towards the center of reality.\footnote{Rowan Williams,
\emph{On Augustine} (London: Bloomsbury, 2016), 101.} One could also
hope that if history began with the admission of nothing, then it would end with a more astonishing reception: the salvation of all.

\newpage

\begin{center}
  \includegraphics[width=0.75cm]{articlend.png}
\end{center}
\vspace{-1.75em}
\biobox{\textbf{Bradley Underwood}, an ordained Baptist minister, is a doctoral candidate in the Department of Slavic Languages and Literatures at Northwestern University and Associate Director of the Northwestern University Research Initiative in Russian Philosophy, Literature, and Religious Thought. He is currently writing a dissertation on evil in Russian thought and literature, focusing on Nikolai Gogol and Leo Tolstoy.}

\label{sec:underwood}

% Reset footnote counter
\section{Lightsey - Le style c'est l'homme}

% Centered grey box
\abstractbox{\textit{Le style c'est l'homme}}{Sergii Bulgakov, Vladimir Nabokov, \& the Artist's \textit{Podvig}}{Daniel Adam Lightsey}{What affinities might Russian Orthodox theologian Sergii Bulgakov and poet, lepidopterist, and master prose stylist Vladimir Nabokov possess? I argue that both writers, for all their practical, biographical, and conceptual differences, help to develop what I term "the artist’s \textit{podvig}": the creation of poetic exploits in loving, awe-full response to and further discovery of the mysterious depths of reality. By analyzing aspects of Bulgakov's sophiological aesthetics as well as close readings of Nabokov's fiction and class lectures, we see how both writers describe the exploits of the artist, arguing for the preeminent value of personal creativity in humanity’s response to the many layers of reality. Furthermore, both writers goad us to sense the world and all its mysterious depths and heights with keener awareness to its poetic achievement, perhaps especially in those moments of Nabokovian timeless singularity.}{Sergii Bulgakov, Vladimir Nabokov, \textit{podvig}, exploit, art, aesthetics, theological senses} % Placeholder for abstract text

\setcounter{footnote}{0}

\fancypagestyle{chaptertitlepage}{
  \fancyhf{} % Clear all header and footer fields
  \fancyhead[L]{\begin{minipage}[t]{0.7\textwidth}\publisher\end{minipage}}
  \fancyhead[R]{\begin{minipage}[t]{\textwidth}\raggedleft \datefont\fontsize{10}{11}\selectfont Volume 1 (2024): \thepage\textendash\pageref{sec:lightsey} \\ \doi{10.71521/khzf-8x41} \end{minipage}}
  \renewcommand{\headrulewidth}{0pt} % No header rule on title pages
  \fancyfoot[RE]{\thepage}
  \fancyfoot[LO]{\thepage}
}
% First article
\fancypagestyle{chaptercontentpage}{
  \fancyhf{} % Clear all header and footer fields
  \fancyhead[CE]{%
    \fontsize{11}{11}\leftmarkfont%
    \addfontfeature{LetterSpace=10.0}%
    \textit{\MakeUppercase{\leftmark}}%
  }
  \fancyhead[CO]{\authorheadfont\addfontfeature{LetterSpace=10.0}\fontsize{11}{11}\selectfont\textbf{{\uppercase{Daniel Adam Lightsey}}}}

  \fancyfoot[RE]{\thepage}
  \fancyfoot[LO]{\thepage}
  \renewcommand{\headrulewidth}{0pt} % No header rule on content pages
}
\newpage
\chaptertitle{\textit{Le style c'est l'homme}}{Sergii Bulgakov, Vladimir Nabokov, \& the Artist's \textit{Podvig}}{Daniel Adam Lightsey}

\addcontentsline{toc}{chapter}{Le style c'est l'homme:\\Sergii Bulgakov, Vladimir Nabokov, \& the Artist's Podvig \\ \textit{by} Daniel Adam Lightsey}

\setcounter{footnote}{0}
\seriffont
\fontsize{12}{18}\selectfont
\epigraph{}{It is difficult to express the joy of the practice in words, he then says, if there is a rehearsal \ldots{} then he breaks free from all indirectness and he is absolutely active, immersed, totally identified with what he is doing; \ldots{} in a word if he is rehearsing, as he has been just now with Seiobo, or if he is continuing the Seiobo rehearsal, \ldots{} then he feels in the deepest of depths that there is a soul.\footnote{László Krasznahorkai, \emph{Seiobo There Below,} trans. Ottilie Mulzet (New York: New Directions, 2013), 236.}}
\epigraph{}{Everything, finally, was a source of wonder, not to say love.\footnote{Patrick White, \emph{Riders in the Chariot} (New York: New York Review Books, 2002), 596.}}
\vspace{2em}

\noindent What affinities might Russian Orthodox dogmatician Sergii Bulgakov and poet, lepidopterist, and master prose stylist Vladimir Nabokov possess? I argue that both writers, for all their practical, biographical, and conceptual differences, help to develop what I term "the artist's \emph{podvig}": the creation of poetic exploits in loving, awe-full response to and further discovery of the mysterious depths of reality. In keeping within the ambit of "the nature of reality and how it can be known" as well as reflecting upon "the transcendent in modernity,"\footnote{The two quotations were key thematics from the inaugural Northwestern University Research Initiative in Russian Philosophy, Literature, and Religious Thought conference held at Northwestern University in April 2023, at which this essay was originally presented in a condensed form.} this essay is an extended \emph{marginalia} of sorts to Charles Taylor's judgment that one keen form of resistance to the concept of a buffered-self entrapped within a metaphysically mechanistic "Immanent Frame"\footnote{Charles Taylor, \emph{A Secular Age} (Cambridge: Harvard University Press, 2007), esp.~25\textendash 89, 539\textendash 776.} comes by way of the arts. Furthermore, this form of resistance can, though not exhaustively or \emph{only} or perhaps even primarily, aid late moderns in how to perceive that "the world contains living depths in which its being is seen to spring from something that lies beyond it."\footnote{Semyon Frank, \emph{Reality and Man,} trans. Natalie Duddington (New York: Taplinger Publishing Company, 1966), 212.}

At first glance, the linking together of the Marxist economist later turned Russian Orthodox theologian and arguably the greatest novelist of the 20\textsuperscript{th} century might seem rather hasty or forced. After all, one might declare, Bulgakov surely does not need a Nabokovian "enhancement."\footnote{The obvious answer is of course not. One motivation for this kind of essay, though, is to work towards a greater understanding of how Bulgakov's use of and encounters with the arts shaped his own philosophical-theological imagination. Bulgakov's theologically inflected literary criticism (especially his "period' of thought from approximately 1900\textendash 1920) has yet to receive an exhaustive treatment. Throughout his works and life, Bulgakov interacts with and extensively calls upon poets, prose-writers, painters, sculptors, composers, etc. to explore \emph{theological} and philosophical problematics, even after his disclosure in a 1926 letter that he had moved away from such topics, see \emph{Тихие думы} (Moscow: Республика, 1996), 460. For example, Bulgakov, often from memory seemingly, and ofttimes without direct attribution, calls upon the works of (or knew personally), to name only a few, Alexander Pushkin, William Shakespeare, (Jean-Baptiste Poquelin) Molière, Johann Wolfgang von Goethe, Wolfgang Amadeus Mozart, Antonio Salieri, Hans Holbein the Younger, Matthias~Grünewald, Heinrich Heine, Giacomo Leopardi, Lord Byron, Victor Hugo, Charles Dickens, Friedrich Schiller, Fyodor Tyutchev, Ivan Turgenev, Charles Baudelaire, Nikolai Gogol, Fyodor Dostoevsky, Lev Tolstoy, Pablo Picasso, Anton Chekhov, Guy de Maupassant, Afanasii Fet, Richard Wagner, A. S. Golubkina, Wassily Kandinsky, Julia Nikolayevna Reitlinger, Pyotr Tchaikovsky, Andrei Bely, Alexander Blok, Aleksei Tolstoy, Mikhail Lermontov, Angelus Silesius, Alexander Bogdanov.} And, conversely, Nabokov betrays no trace of Bulgakov, does he?\footnote{Or some might object to such a project as this after the fashion of Georges Florovsky, who asks, "Can artistic intuition penetrate the spiritual world?" Ultimately, Florovsky says no, "artistic vision cannot replace faith. Neither meditation nor rapture may be substituted for religious experience." See Florovsky, "From the Ascetic Mystics of Soloviev to the Mystical Romance of Blok," in \emph{Theology and Literature}, trans. Robert Nichols (Vaduz: Buchervertriebsanstalt, 1989), 148. In this brief piece, Florovsky is specifically responding to certain Silver Age artists and poets who, he believes, possess a flimsily crafted approach to the spiritual life. Regardless, faith and artistic vision are not rival powers in competition for a person's soul. To perceive them in such a combative fashion is to misunderstand, perhaps completely, their very natures.} The ostensible danger here is that one replaces Nabokov's "magic" with "maggots"\footnote{Vladimir Nabokov, \emph{Strong Opinions} (New York: McGraw-Hill Book Company, 1973), 305.} by situating him in a sophianic straitjacket and thus reading his texts too "illustratively." Furthermore, did not Nabokov himself critique those who mistook his development of "Byzantine imagery" in his earlier poetry for "an interest in 'religion,' which, beyond literary stylization," he later in life asserts, "never meant anything to me."\footnote{Vladimir Nabokov, \emph{Poems and Problems} (New York: McGraw-Hill Company, 1970), 13\textendash 14.} This and other explicit repudiations for possessing any interest in "organized mysticism, [or in] religion, [or in] the church\textemdash any church"\footnote{Nabokov, \emph{Strong Opinions,} 39.} have led some to conclude that Nabokov was a kind of immanentized aesthete, not concerned with any sense of transcendence, and thus his fiction can be read as endorsing, tacitly or not, a charming "nihilistic flux {[}in{]} all things."\footnote{Nicholas Adams, George Pattison, and Graham Ward, \emph{The Oxford Handbook of Theology and Modern European Thought} (Oxford: Oxford University Press, 2013), 11. For an especially apt critique of this sentiment, see Erik Eklund, "\,'A green lane in Paradise': Eschatology and Theurgy in~\emph{Lolita},"~\emph{Nabokov Studies}~17 (2020): 35\textendash 60. Philosophical theologians are not the only ones to misunderstand the nature of theological, metaphysical, and spiritual aspects in Nabokov's work, however. Another misleading opinion, if not greatly qualified, is that Nabokov "always believed that fiction was neither moral, social, nor psychological but a sensuous exercise in style," Morris Dickstein, \emph{Leopards in the Temple: The Transformation of American Fiction 1945\textendash 1970} (Cambridge: Harvard University Press, 2002), 124. Also, the eminent Nabokov scholar Brian Boyd asserts that as regards "Cлово," one of Nabokov's early short stories, Nabokov "sets the human and the transcendent too starkly together \ldots{} {[}leading{]} to a blind corner," \emph{Vladimir Nabokov: The Russian Years} (Princeton: Princeton University Press, 1990)\emph{,} 203. This is a failure to understand that the truly transcendent cannot, in fact, be set over and against the immanent. Concerning how "the transcendent" or "infinite" is (mis)classified, this error can be assuaged by returning to the basic apprehension of many ancients, perhaps nowhere better crystalized than Augustine of Hippo's description of the divine life as \emph{interior intimo meo et superior}~\emph{summo}~\emph{meo} (\emph{Conf.} 3.6.11). For a magisterial contemporary take on this issue, see Rowan Williams, \emph{Christ the Heart of Creation} (London: Bloomsbury Continuum, 2018).} However, this judgment is mistaken, one might even say \emph{willfully} so as regards some commentators, as has been demonstrated by a host of Nabokovian scholars. To take but four examples, demonstrated by Johnson, Alexandrov, Barabtarlo, Link, Eklund, and many others, Nabokov possesses a kind of personal form of metaphysics and/or spirituality, showed great attention to the notion of life after death, consistently played with scriptural imagery/thematics, and, according to his wife possessed as his "main theme" the notion of "потусторонность" (variously translated as "otherworld," "the hereafter," "the beyond").\footnote{Véra Nabokov, "Предисловие," in \emph{Стихи}, by Vladimir Nabokov (New York: Ardis, 1979), 3. Cf. Vladimir Alexandrov, who points out that "otherworld" is not a wholly satisfactory translation of потусторонность because it is a "noun derived from an adjective denoting a quality or state that pertains to the 'other side' of a boundary separating life and death," \emph{Nabokov's Otherworld} (Princeton: Princeton University Press, 1991), 3. Perhaps only worthy of a passing aside: Bulgakov too employs the word in a 1937 essay during his concluding description of Nabokov's beloved Pushkin: "Кончина Пушкина озарена потусторонним светом," "Жребий Пушкина," in \emph{Тихие думы,} 269. For but a small sampling of the vast secondary literature on these issues, see D. Barton Johnson, \emph{Worlds in Regression: Some Novels of Vladimir Nabokov} (Ann Arbor: Ardis, 1985), esp.~155\textendash 223; Gennady Barabtarlo, "Nabokov's Trinity: On the Movement of Nabokov's Themes," in \emph{Nabokov and his Fiction: New Perspectives,} ed.~Julian W. Connolly (Cambridge: Cambridge University Press, 1999), 109\textendash 138; Christopher A. Link, "Recourse to Eden: Tracing the Roots of Nabokov's Adamic Themes," \emph{Nabokov Studies} Volume 12 (2009): 63\textendash 127; Eklund, "A green lane in Paradise"; \emph{idem}., "The Gist of Masks: Notes on Kinbote's Christianity and Nabokov's Authorial Kenosis," \emph{Nabokov Online Journal} Volume XV (2021): 1\textendash 29; \emph{idem.}, "Rereading the World: A Theological Appraisal of Vladimir Nabokov's Metaliterary Eschatology," \emph{Religion and Literature} Volume 57, Number 1 (2025): forthcoming. As far as plucky originality, precision of close reading, and sheer output, Eklund is arguably \emph{the} leading contemporary scholar with regards to demonstrating Nabokov's importance for explicitly \emph{theological} discussions, not only or merely \emph{spiritual, mystical,} or \emph{metaphysical} ones\textemdash and all without doing violence to Nabokov's seemingly \emph{non-}theological self-stance.}

\emph{In nuce}, Nabokov's self-confessed disinterest in "organized mysticism," "religion," or "chu\-rch" certainly does not equate to a denial of interest in the mysterious depths of the world that consistently, creatively disclose themselves to those with eyes to see and ears to hear, whether it be, for instance, in mind, consciousness, memory, mimetic peculiarities found in nature\textemdash all of which greatly marked Nabokov and all of which he persistently explores in his prose, poetry, classroom lectures, and scientific exploits. After all, to promiscuously prooftext \emph{Look at the Harlequins}, "the brook and the boughs and the beauty of the Beyond all began with the initial of Being."\footnote{Vladimir Nabokov, \emph{Look at the Harlequins,} in \emph{Novels 1969\textendash 1974}, ed.~Brian Boyd (New York: Library of America, 1996), 577.} It is perhaps not so curious, then, that at the conclusion of a January 1964 published interview, when asked if he believed in God, Nabokov responded\textemdash with his penchant for titillation\textemdash "I know more than I can express in words, and the little I can express would not have been expressed, had I not known more."\footnote{Nabokov, \emph{Strong Opinions,} 45.}

\subsection*{Подвиг (\emph{Podvig})}

It is generally remarked that \emph{podvig} enjoys no one-for-one English equivalent, although those that come nearest the mark are \emph{exploit,} \emph{extraordinary deed}, \emph{gallant feat}, all the while possessing an adventurous quality constitutive to the endeavors of an intrepid hero of renown. The term itself is well known in Russian poetry and literature. One literary example that Bulgakov directly cites\footnote{For two examples, see Sergii Bulgakov, "\foreignlanguage{russian}{Природа в философии Вл. Соловьева}," in \emph{Сочинения в двух mомах}, ed.~S.S. Khoruzhii (Moscow: Наука, 1993), 1:40; Sergii Bulgakov, \emph{Философия хозяйства}, in \emph{Сочинения в двух томах}, 1:146.} (and Nabokov had almost certainly encountered) is Vladimir Solovyov's 1882 poem "Три Подвига" ("Three Exploits").\footnote{Vladimir Solovyov, \emph{Стихотворения и шуточные пьесы} (Munich: Wilhelm Fink Verlag, 1968), 77\textendash 78. All translations are mine unless otherwise noted. My gratitude to Vladimir Marchenkov for pointing me toward this poem.} This poem progresses through three feats from the classical Greco-Roman mythological imagination, from Pygmalion to Perseus to Orpheus. With each movement, the reader senses a victory of sorts is near, yet the poem warns one not to stop short, not to anticipate too early that the exploit is fully accomplished. To do so will only culminate in one's jubilee quickly shifting to mourning. And yet, in the final progression of the poem, Solovyov exhorts the intrepid reader to "rise up," to "call death to fight to the death!" And in so doing, a victory is near at hand.\footnote{Although he later cooled in his intellectual affections for Solovyov, Bulgakov nevertheless designated him a great modern Russian "поэт-философ" {[}"poet-philosopher"{]}, see Sergii Bulgakov, "Без Плана. \foreignlanguage{russian}{Несколько замечаний по поводу статьи Г.И. Чулкова о поэзии Вл. Соловьева}," in \emph{Тихие думы}, 216\textendash 233; and "Стихотвоpения Владимира Соловьева," in \emph{Тихие думы}, 51\textendash 55. What's more, Bulgakov declares, "I regard Solovyov as having been my philosophical 'guide to Christ' at the time of a change in my own world outlook," \emph{Sophia: The Wisdom of God: An Outline of Sophiology,} trans. Patrick Thompson, O. Fielding Clarke, and Xenia Braikevitc (Hudson: Lindisfarne Press, 1993), 10. Moreover, while there does not seem to be any direct textual evidence of Nabokov citing Solovyov, it is frankly unthinkable to imagine he did not read Solovyov, the latter being indispensable for much of the Russian Silver Age imaginative harvest. Thus, while Nabokov does not directly cite "Три Подвига" like Bulgakov does on multiple occasions, it is certainly not unfitting to imagine he came across it, perhaps even during those heady summer days "when the numb fury of verse-making first came over" him. See Vladimir Nabokov, \emph{Speak, Memory,} in \emph{Novels and Memoirs 1941\textendash 1951}, ed.~Brian Boyd (New York: Library of America, 1996)\emph{,} 542. Cf. Dana Dragunoiu,~who suggests that Nabokov\textquotesingle{} s "encounter with Kant\textquotesingle s third critique \ldots{} may have been mediated by Vladimir Solovyov," \emph{Vladimir Nabokov and the Arts of Moral Acts}~(Evanston: Northwestern University Press, 2021), 8\textendash 9. Many thanks to both Joshua Heath for kindly directing me to Bulgakov's "Без Плана" and Erik Eklund for our exchanges concerning Nabokov and Solovyov.}

Bulgakov employs the term throughout his literary career, though it often possesses a fugitive presence. One place \emph{podvig} indispensably appears is in his famous 1909 \emph{Vekhi} essay. Bulgakov positions the term, as well as others from its word family\textemdash \emph{podvizhnik} and \emph{podvizhnichestvo} (both can be used to express a kind of ascetic zealousness towards a certain selfless project)\textemdash as a corrective to the perceived revolutionary, maximalist heroism of the so-called intelligentsia, the latter of which Bulgakov judged to sacrifice tomorrow for the present moment, inevitably devolving into atomistic endeavors of Luciferian arrogance.\footnote{Sergii Bulgakov, "Героизм и подвижничество," in \emph{Сочинения в двух томах}, ed.~I. B. Rodnianskaia (Moscow: Наука, 1993), 2:302\textendash 342.} The \emph{podvizhnik,} on the other hand, is one who humbly attends to love of God, neighbor, environment, history, etc. in the \emph{present moment}, keeping her eyes on her "immediate work," shunning the "pretensions" of the intelligentsia's hero.\footnote{Bulgakov, "Героизм и подвижничество," 2:324\textendash 25.}

Make no mistake, Bulgakov's \emph{podvizhnik} is no shy subservient, but an active, dynamic, and creative shaper of a faithful life of love, humility, and "self-mastery,"\footnote{Bulgakov, "Героизм и подвижничество," 2:331.} challenging any consequentialist ethic that utilizes past and present sufferings as "manure for someone's future harmony."\footnote{Fyodor Dostoevsky, \emph{The Brothers Karamazov,} trans. Richard Pevear and Larissa Volokhonsky (New York: Farrar, Straus, and Giroux, 2002), 244.} One sees this quite clearly, for example, in his 1933 oration on St.~Seraphim of Sarov. Bulgakov uses the term in relation to the life of the revered saint to spur listeners on to receive God's grace for "exploits of faith and love, by efforts of heart and will."\footnote{Sergii Bulgakov, "Уголь пламенеющий," in \emph{Церковная Радость (проповеди)} {[}Paris, 1938{]}, 30, in \emph{Автобиографические статьи: заметки, статьи} (Orel: Орловской государственной телерадиовеща, 1998), 329\textendash 330.} What's more, Bulgakov pursues the notion of "ascetic exploit"\footnote{Sergii Bulgakov, \emph{Утешитель: О богочеловечестве:} Часть II (Moscow: Общедоступный православный университет, 2003), 345. Cf. 348 where Bulgakov again returns to developing his notion of the \textit{podvizhnik}.} in a more developed Pneumatological vein in his 1936 \emph{The Comforter}, and in so doing, he widens his description of the \emph{podvizhnik} to include the "creative self-determination and audacity" of the individual person pursuing a life in the Spirit.\footnote{Sergii Bulgakov, \emph{The Comforter,} trans. Boris Jakim (Grand Rapids: Eerdmans, 2004), 306.} Enlivened by this sense of "renewed theological creativity,"\footnote{Sergii Bulgakov, "Dogma and Dogmatic Theology," in \emph{Tradition Alive: On the Church and the Christian Life in Our Time,} ed.~Michael Plekon (Lanham, MD: Rowman and Littlefield Publishers, 2003), 74.} Bulgakov even extends these notions of expedition, humble audacity, and creativity to the development of Christian theology itself, arguing that\textemdash in vehement contrast to any theory of dogma as a static deposit of fixed propositions\textemdash "the very foundation of dogma assumes," among other characteristics, "a combination of effort {[}and{]} \emph{creative intuition}," and, furthermore, that at the "heart of dogmatic theology lies dogmatic \emph{quest.}"\footnote{Bulgakov, "Dogma," 68, emphasis mine.}

Nabokov too knew \emph{podvig} well\textemdash being, of course, the title of his fifth Russian novel, published in 1932\emph{.} In \emph{Podvig,} the reader encounters Martin Edelweiss, a youthful émigré forced to flee the Crimea during the time of Russian revolutionary tumult. Alighting upon an understanding that "human life flowed in zigzags,"\footnote{Vladimir Nabokov, \emph{Glory,} trans. Dmitri Nabokov and Vladimir Nabokov (New York: Vintage International, 1991), \emph{Glory}, 8\textendash 9. As a structural schematic for the novel, tracing the initial "zigzags" through to the novel's concluding "picturesque and mysterious windings" (205) is worth some stress, as is the enchanting similarity to how Bulgakov describes the journey of a free spiritual being: "This curved, zigzag ascent is a search for oneself, an effort to disclose the unique fact of individuality in creation as well as its place in the whole, in the pleroma, in the sophianic proto-image of being. Even when it is submerged in the ocean of universal being, no streamlet of life loses its identity," Sergii Bulgakov, \emph{Bride of the Lamb}, trans. Boris Jakim (Grand Rapids: Eerdmans, 2002), 143\textendash 144.} Martin is consistently "stirred \ldots{} deeply" by "immemorial and tender banalities," enchanted with "glory, love, tenderness for the soil, a thousand rather mysterious feelings," and ofttimes, as when he was a child, experiences an "unbearable intensification of all his senses, a magical and demanding impulse, the presence of something for which alone it was worth living."\footnote{Nabokov, \emph{Glory,} 82, 156, and 20, respectively.} More and more, this sensual intensification begins to mark, and haunt, Martin with a specific pull towards \emph{home}: as he becomes "familiar with that \emph{special smell}, the \emph{smell} of prison libraries, which \emph{emanated} from Soviet literature," or his mother's use of the unique "\emph{sounds} used in Russia," or being caught off guard by "{[}a{]}nother trifle, but somehow that stick seemed \emph{to smell} of Russia."\footnote{Nabokov, \emph{Glory,} 140, 173, respectively, emphases mine.}

The \emph{podvig} that Martin shapes\textemdash and is thus shaped by\textemdash is ostensibly simple enough: cross illegally into Russia from Latvia and then walk back after only the space of a day. The entire novel acts as a \emph{dénouement} of sorts to the \emph{podvig} in question, as the work abruptly ends soon after Martin heads off for the border, leaving the reader in a state of uncertainty as regards his fate, although one suspects his demise. As the novel unfolds, the reader experiences how Martin becomes increasingly captivated with this "secret exploit"\footnote{Nabokov, \emph{Glory,} 172.} of a twenty-four-hour jaunt, but the so-called \emph{purpose} of it escapes those closest to Martin, nowhere better witnessed than Martin's last exchange with his good friend from Cambridge, Darwin.

\begin{quote}
{[}Darwin{]} "Only I do not quite see what's the purpose of it."

\emph{{[}Martin{]} "Give it a little thought, and you will."}

"Some plot against the good old Soviets? Want to see someone? Deliver a secret message, rig up something? I confess that as a boy I rather fancied those gloomy bearded chaps who threw bombs at the troika of the ruthless governor."

\emph{Morosely, Martin shook his head.}

"And if you want to visit the land of your fathers\textemdash although your father was half-Swiss, wasn't he?\textemdash still, if you want to see it so badly, would it not be simpler to obtain a regular soviet visa and cross the border by train? Don't want to? Perhaps, after the assassination in that Swiss café, you think you won't be given a visa? All right, I'll get you a British passport."

\emph{"What you're imagining is all wrong," said Martin, "I expected you'd understand everything at once."}

[\ldots{}]

"If, finally, what you are after is just pure risk, there's no need to travel so far. Let us invent something unusual, something that can be executed right now, right here, without overstepping the windowsill."

\emph{Martin remained silent, and his face looked sad.}

"This is absurd," reflected Darwin, "absurd and rather peculiar. Stayed quietly in Cambridge while they had their civil war, and now craves a bullet in the head for spying. Is he trying to mystify me? What an idiot conversation."\footnote{Nabokov, \emph{Glory,} 199\textendash 200, italics added.}
\end{quote}

Darwin's disbelief and, in the end, irritation with Martin's decision to risk so much for seemingly so little is in keeping with the nature of the \emph{podvig} itself, as Nabokov understands it\emph{.} In his foreword to the 1971 English edition of \emph{Podvig,} tweaked in translation to \emph{Glory,} Nabokov elaborates upon why the modified title: "if you once perceive in 'exploit' the verb 'utilize,' gone is the \emph{podvig,} the inutile deed of renown. The author," he continues, "chose therefore the oblique 'glory,' \ldots{} the glory of high adventure and disinterested achievement; the glory of this earth and its patchy paradise; the glory of personal pluck; the glory of a radiant martyr."\footnote{Nabokov, \emph{Glory,} xii-xiii.} Thus, while mystifying to one such as Darwin (and perhaps some readers as well), it is "the glorious exploits of disinterested curiosity" that drive Martin's desire for a "distant perilous journey,"\footnote{Nabokov, \emph{Glory,} 126, 185, respectively.} something that eludes the designs of utility, practicality, or a purely mechanistically materialistic calculus.

Darwin's bewilderment\textemdash shared more or less by others throughout the novel who, for example, think a person with such plans should "stay home and find something \emph{constructive} to do" or "simply refuse to believe that a young man, pretty much removed from Russian political problems and more of a foreign cut I'd say, could prove capable of\textemdash well, of a \emph{high deed}, if you like"\footnote{Nabokov, \emph{Glory,} 178, 204, respectively, emphasis added.}\textemdash is understandable, especially according to any sort of logic of utility.\footnote{Also, as is noted in Nora Buhks, \emph{Эшафот в хрустальном дворце: О русских романах В. Набокова} (Moscow: Новое литературное обозрение, 1998), 65, Darwin's misunderstanding could be due in part to his own youthful experiences, as Martin learns that Darwin's college studies were interrupted by "{[}t{]}hree years in the trenches, France and Mesopotamia, the Victoria Cross \ldots{}," all of which began when he was but eighteen, \emph{Glory,} 58\textendash 59.} If Martin was crossing the border for some relatively reasonable purpose ("Some plot against the good old Soviets? \ldots{}") then all would at least \emph{make sense,} despite the many dangers he would undoubtedly face. But why play coy about the purpose? Darwin seems to ponder. Moreover, why not simply utilize the traditional conventions of international crossings, either by way of a Soviet or British passport? Or, barring these, why not simply contrive an uncommon deed to achieve in the relative safety of Berlin, something that may possess the sensation of risk "without overstepping the windowsill," as it were? These are all, of course, well-meaning queries proffered by a dear friend who does not wish to see his companion go through with a seemingly suicidal feat. But they also reveal, consciously or not, a kind of consequentialist paradigm that the novel consistently attempts to problematize, "suggesting that the expected futility of Martin's exploit is the very ground of its virtue."\footnote{Dragunoiu, \emph{Vladimir Nabokov and the Arts of Moral Acts,} 40.}

To be clear, however, the inutility of a \emph{podvig} is certainly not a celebration of some kind of inane nihilistic purposelessness, at least not according to Nabokov and Bulgakov. Disclosed in his 1971 English foreword, Nabokov reveals that \emph{Glory} was, in fact, his \emph{only} novel written with a "purpose," namely, "in stressing the thrill and the glamour that my young expatriate finds in \emph{the most ordinary pleasures} as well as in \emph{the seemingly meaningless adventures of a lonely life.}"\footnote{Nabokov, \emph{Glory,} x, emphasis added.} This commentary resonates with Bulgakov's charge to approach the mundanity of \emph{today} with humble creative audacity. While neither Bulgakov nor Nabokov combines their usage of \emph{podvig} with an explicit theory of the artist and her poetic exploits,\footnote{Although Bulgakov comes the closest in at least using some of these thought forms in the same essay, see "Труп красоты," in \emph{Сочинения в двух томах}, 2:537.} we find ample evidence from their respective literary corpuses to suggest this kind of imaginative speculation is not untoward.

\subsection*{Bulgakov: Sophiological Aesthetics}

To build a theological grounding for the artist's \emph{podvig,} we first turn to Bulgakov's theological anthropology and sophiological aesthetics. Bulgakov argues that humanity should be classified primarily as a "существо пиитическое" {[}"poetical creature"{]},\footnote{Sergii Bulgakov, "Догматическое обоснование культуры," in \emph{Сочинения в двух томах}, 2:637.} an "artistic being,"\footnote{Sergii Bulgakov, \emph{Icons and the Name of God,} trans. Boris Jakim (Grand Rapids: Eerdmans, 2012)\emph{,} 43.} a "ζώον ποιητικόν" {[}"poetic animal"{]}.\footnote{Sergii Bulgakov, "Religion and Art," in \emph{The Church of God: An Anglo-Russian Symposium,} ed.~E. L. Mascall (London: S.P.C.K. 1934), 181.} As such, human persons are free, dynamic creators in the world\textemdash "gods by grace,"\footnote{Bulgakov, \emph{Bride of the Lamb}, 87.} in fact\textemdash actively mirroring the Divine's transcendent act of creation in-and-beyond the cosmos. Germane to these claims is, of course, Bulgakov's Sophiology, the knotty minutiae of which need not detain us too long although a digest is instructive, at least as regards Sophia's relation to beauty, the artist, and the making of art.\footnote{Bulgakov developed his Sophiological "system" over many years. For some of his major texts on his Sophiology, see Bulgakov's \emph{Philosophy of Economy: The World as Household}, trans. and ed.~Catherine Evtuhov (New Haven: Yale University Press, 2000)\emph{,} esp.~123\textendash 156; \emph{Unfading Light: Contemplations and Speculations,} trans., Thomas Allan Smith (Grand Rapids: Eerdmans, 2012), esp.~181\textendash 283; \emph{The Lamb of God}, trans. Boris Jakim (Grand Rapids: Eerdmans, 2008)\emph{,} esp.~89\textendash 211; \emph{The Comforter,} esp.~177\textendash 218; \emph{Sophia: The Wisdom of God}; and \emph{The Bride of the Lamb,} esp.~3\textendash 250.} Why this \emph{recherché} system matters for the argument at hand is that Bulgakov himself believed that it could give "strength for new inspiration, for new creativity, for the overcoming of the mechanization of life and of human beings."\footnote{Bulgakov, \emph{Sophia: The Wisdom of God,} 21} (Nabokov himself was inspired by a similar intuition when writing \emph{Podvig,} which he previously thought of entitling \emph{Романтический Век} {[}\emph{Romantic Times}{]}, due in part to hearing too often that "Western Journalists call our era 'materialistic,' 'practical,' utilitarian,' etc."\footnote{Nabokov, \emph{Glory,} x.}). Furthermore, as we will see, beauty and artistic \emph{poiesis} are not \emph{only} relevant for delighting the senses, though they certainly do in many instances; when they are situated within Bulgakov's Sophiological musings, a kind of "aesthetic apocalypse" occurs, by which "Sophia\textquotesingle s manifestation \emph{is} the experience of participating in \emph{beauty}"\textemdash allowing for artist and participant observer to "share a common connection with Sophia, the divine as revealed in the order of creation."\footnote{J. R.~Seiling, "From Antinomy to Sophiology: Modern Russian Religious Consciousness and Sergei Bulgakov's Critical Appropriation of German Idealism," PhD Dissertation (University of Toronto, 2008), 263, emphases mine. For a brief selection of the growing body of secondary literature on Bulgakov and beauty, see Katharina Breckner, "Beauty and Art in Solovjev (1850\textendash 1903) and in Bulgakov (1874\textendash 1948): Does Beauty Save the World?" \emph{Logos} \emph{i Ethos} 1 32 (2012): 7\textendash 17; Bruce Foltz, \emph{The Noetics of Nature: Environmental Philosophy and the Holy Beauty of the Visible} (New York: Fordham University Press, 2014), esp.~chapter 5; Jennifer Newsome Martin, \emph{Hans Urs von Balthasar and the Critical Appropriation of Russian Religious Thought} (Notre Dame: University of Notre Dame Press, 2015), esp.~chapter 2; Aidan Nichols, \emph{Redeeming Beauty: Soundings in Sacral Aesthetics} (London: Routledge, 2021), esp.~chapter 4; Brandon Gallaher, "All Things Shining: Sergii Bulgakov's Theology of Beauty," \emph{The Wheel} 26/27 (2021): 42\textendash 49; Daniel Adam Lightsey, "\,'The Human Thirst to See Heavenly Azure': Sergii Bulgakov, 'Holy \emph{Anamnesis,}' and Beauty," in \emph{Moral Conversion in Scripture, Self, and Society,} eds.~Krijn Pansters and Anton ten Klooster (Berlin: De Gruyter, 2024), 225\textendash 237.} However, all of this requires some unpacking.

While Bulgakov's Sophiology is multifaceted, at its heart is a pursuit of a Christian cosmology, specifically as regards the Creator-creation relation. To craft a faithful Christian cosmology, Bulgakov is resolute one must avoid two pitfalls. On the one hand, one must take care to evade the Scylla of pantheism\textemdash a plastic term which can rear many heads but, at root, identifies the world with God or the Divine as the world \emph{tout court}, ofttimes devolving into a thinly guised atheism. On the other hand, one should give a wide berth to the Charybdis of diverse dualisms, all of which invariably fall prey to the various whirlpools of ontologizing evil and violence, or advancing another (sub)divine entity alongside Creator-Divinity (usually as either diametric oppositions of "good" versus "evil" or as a subordinate demiurge who creates, however remarkably, with haphazard adroitness), or positing the world to be \emph{outside} of God, which, at its worst, suggests the Creator as a rival power to creatures who must, in the end, 'kill' the former in order to make a secure place for themselves in the void. Other cosmologies, of course, exist, but be that as it may, Bulgakov argues that these sundry forms of pantheism and dualism are, inevitably, too crude to understand adequately the grandeur of the doctrine of \emph{creatio ex nihilo.}

Creation out of nothing, for Bulgakov, is \emph{creatio ex Deo} whereby God, eternally possessing the divine world, or Divine Sophia, as His own nature, \emph{releases} it from the depths of hypostatic being\emph{,} making it the cosmos in the true sense, \emph{creating} the world 'out of nothing,'\," that is, out of divine content, formally and finally grounded in God's \emph{love}. God's mode of relation in the act of creation is yet a further manifestation of this ontology of love, whereby the trihypostatic Person relinquishes possession of the divine world, giving it room, as it were, to have its own self-being, releasing it as a kenotic act of love to possess "divinely extra-divine and even non-divine being,"\footnote{Bulgakov, \emph{Bride}, 50.} namely, creaturely Sophia. To be sure, ontologically there is no separation between God and any\emph{thing} (Bulgakov consistently cites Acts 17:28 in this regard); thus, Divine Sophia is the "very \emph{foundation} of the world, and the divine world {[}Divine Sophia{]} is the essence of the creaturely world {[}creaturely Sophia{]}."\footnote{Bulgakov, \emph{Bride}, 50, emphasis original.} However, lest the apparition of pantheism manifest itself, Bulgakov makes clear that just as one must understand the sense of \emph{identity} regarding Divine and creaturely Sophia, one must recognize Divine and creaturely Sophia's \emph{difference} as the latter is to be understood in the light of its \emph{createdness.} Creation \emph{is}, therefore, a "supra-eternal act of God's self-determination," founded upon the kenotic love of the Triune Person who ecstatically donates "creaturely, non-divine being, given to itself"\footnote{Bulgakov, \emph{Bride,} 51.} in all its genuinely diverse multiplicities, thematic possibilities, ontic potentialities, and concrete connections between all the individual members of being\textemdash all the while united, in its creaturely \emph{limitedness}, to the all-unity of integral wisdom.

As such, Bulgakov continues, the creaturely world contains no "ontological novelties" in relation to God because, channeling St.~Irenaeus of Lyons, "revealed in this world are the same words of the supra-eternal Word that make up the ideal content of the Divine Sophia, the life of God. \ldots{} And this is the same life-giving power of the Holy Spirit that clothes the words of the Word with life and beauty."\footnote{Bulgakov, \emph{Bride}, 50.} And yet, while no ontological novelties appear on the so-called transcendental horizon, "\emph{Art} itself, its element, is deeper, more general, more original than all particular arts, \ldots{} through the inspiration of Beauty and participation in it all humanity is called to art."\footnote{Bulgakov, \emph{Unfading Light,} 402.} Thus, humanity's practices of \emph{poiesis} and artistic exploits command considerable theological attention from Bulgakov.

\subsection*{Bulgakov on Nature, Beauty, and Art}

As we see clearly from several illuminating passages from his 1917 treatise on the philosophy of religious consciousness, \emph{Unfading Light,} Bulgakov argues that art is a "sensing of the ultimate depth of the world and that trembling which it arouses in the soul," leading to the potentiality of an "eros of creativity," an "erotic encounter of matter and form, their enamored confluence."\footnote{Bulgakov, \emph{Unfading Light,} 395. While most of what follows by way of an analysis of Bulgakov's views regarding nature, beauty, and art stems from his "earlier" texts\textemdash from the 1910s and early 1920s\textemdash nearly all of these thought forms concerning the nature of art and the place of the artist are recapitulated in one of his later essays from 1937, occasioned by the centennial of Alexander Pushkin's death in a duel, see "Жребий Пушкина," in \emph{Тихие думы,} 251\textendash 269.} As such, art can be thought of as "life in beauty."\footnote{Bulgakov, \emph{Unfading Light,} 261, respectively. To anticipate, this is strikingly similar to Nabokov's description, see below.} According to Bulgakov, the artistic endeavor, including but not limited to the traditional art forms themselves, is part and parcel of being human. Through the inspiration of Beauty as such, humanity is compelled to join this creative dance.

Furthermore, in describing nature vis-à-vis human artistry and creativity, Bulgakov argues that "Nature is a great and wonderful artist"\footnote{Bulgakov, \emph{Unfading Light,} 261} who "reveals its 'secrets' only to those who know where to look for them and would remain impenetrable to {[}humanity{]} if {[}one{]} did not possess a certain intuition in {[}one's{]} search."\footnote{Bulgakov, \emph{Philosophy of Economy,} 144.} Bulgakov's notion of a "certain intuition" in the human search is crucial for our notion of the artist's \emph{podvig}. He asserts, "An idea senses itself in beauty {[}and is{]}, attracted to itself with an erotic attraction, in a certain cosmic amorousness \ldots{} {[}where{]} nature {[}is{]} in love with its own loveliness. \ldots{} Only poets and artists see and know this cosmic Aphrodite, \ldots{} {[}where{]} nature {[}is{]} in love with its own idea, {[}and{]} creation {[}is{]} in love with its form."\footnote{Bulgakov, \emph{Unfading Light,} 260.} Bulgakov relates this to the image of the Shulamite and Bridegroom from the \emph{Song of Songs.} "Does not all of nature in its erotic fatigue and its amorous rapture whisper these passionate confessions of the one enamored," he asks. "And does not the poet overhear these sighs and murmuring, does not an artist see these embraces opening out" in a kind of "eros of creativity" in the world below and the "erotic interpenetration of form and matter" in heaven: "Such is the pan-eroticism of nature,"\footnote{Bulgakov, \emph{Unfading Light}, 261.} he declares.

The poet and artist do not \emph{only} sense nature as in love with itself but also participate in this pan-eroticism by way of personal poiesis. This is plainly witnessed in Bulgakov's philosophical treatise concerning the nature of the word\emph{.} Written sporadically from roughly 1917 to 1921, \emph{Philosophy of the Name} was in part a reaction to the Имяславие ("Name-glorifiers") controversy that came to a boil in the early 20\textsuperscript{th} century.\footnote{For an insightful overview, see Scott Kenworthy, "The Name-Glorifiers (Imiaslavie) Controversy," in \emph{The Oxford Handbook of Russian Religious Thought,} eds.~Caryl Emerson, George Pattison, and Randall Poole (Oxford: Oxford University Press, 2020), 327\textendash 342.} Well into the work, Bulgakov begins to contemplate the "beauty of a word," especially those "primordial words" that constitute the language and root things, their power and meaning\textemdash of which, with "childlike, deep wisdom, the poet" knows.\footnote{Sergii Bulgakov, \emph{Philosophy of the Name}, trans. Thomas Allan Smith (DeKalb: Northern Illinois University Press, 2022), 152, 143, respectively.} In considering the art of the word to be poetry, Bulgakov rhapsodizes concerning how the poet's exploit of images can

\begin{quote}
take possession and subjugate us, and independently of the direct meaning, with its descants and accompanying sounds, in them, or rather with them, the voices of the universe sound for us, the sounding of the cosmos is audible. In poetry a word ceases to be only a sign that it uses for signaling meaning, 'concepts'; here it appears as itself, i.e., as a symbol, and waves ripple away from it as a cosmic surge. It seems that one more moment and the lyre of Orpheus will tame wild animals and move mountains\textemdash the word will receive its efficacy, for it touches the root of being. Poetry immediately borders on the magic of a word; it is to a certain degree already magic in the sense that every powerful word is magical.\footnote{Bulgakov, \emph{Philosophy of the Name,} 161. Much has been made of Bulgakov's interaction with 18\textsuperscript{th}-19\textsuperscript{th} century German Idealists and Romantics such as Kant and Hegel and especially Schelling and Fichte. One figure who has yet to receive a thorough comparative analysis in relation to Bulgakov on these issues is the 18\textsuperscript{th} century genius, Johann Georg Hamann, who considers poetry "the mother-tongue of the human race," while rhapsodizing that God is the "poet at the beginning of days" who reveals himself "to creatures through creation." From the essay \emph{Aesthetica in Nuce} (1762) in Hamann, \emph{Writings on Philosophy and Language,} trans. Kenneth Haynes (Cambridge: Cambridge University Press, 2007), 63, 75, 78, respectively.}
\end{quote}

Thus, in part, the vocation of the poet (as well as the prose writer\footnote{Bulgakov, \emph{Philosophy of the Name,} 160.}) is to make oneself open to "the power to know the language of {[}the{]} 'flame of things,'\," namely, the soul and life of the creaturely world, creaturely Sophia, which was originally gifted a "life-giving principle" by the "Giver of Life" who does not create death.\footnote{Bulgakov, \emph{Bride of the Lamb,} 81. Bulgakov quotes here from the apocryphal Book of Wisdom 1:13.}

From this perspective, and to tie it all together, the artist's \emph{podvig} is therefore animated by an erotic striving of creativity where art aids humanity in a "path towards the discovery of beauty."\footnote{Bulgakov, \emph{Unfading Light,} 383.} Bulgakov characterizes this path of discovery by utilizing Viacheslav Ivanov's pithy phrase: "\emph{a realibus ad realiora}" ("from the real to the more real"), by which Ivanov meant for Symbolist art to remain connected to material reality even while aspiring towards the disclosure of a more real reality.\footnote{Bulgakov, \emph{Unfading Light,} 383. See Viacheslav Ivanov, \emph{Selected Essays,} trans. Robert Bird, ed. and intro. Michael Wachtel (Evanston: Northwestern University Press, 2001), 28, 35, 56. Bulgakov too seems to suggest this: "it must remain art {[}doing the work of \emph{this} world's creativity: writing novels, making visual art, composing music, etc.{]}, for only by \emph{being itself} is it news of the empyrean world, the promise of Beauty," \emph{Unfading Light,} 404, emphasis mine.} Bulgakov explores this idea within the context of art's relationship to beauty as such, where the former gives prophetic witness to the latter, where, for example, a poet is "obedient to the commands of the muse, forget{[}s{]} about himself, by surrendering to inspiration, and strive{[}s{]} to cross beyond the barrier of personal limitation."\footnote{Bulgakov, \emph{Philosophy of the Name,} 159.}

To be sure, Bulgakov was no naïf. He knew the potential for corruption, distortion, and the propagandizing of art in the pursuit of counterfeit beauties; in fact, he sharply critiques those he judges to moralize art or commit other utilitarian violences that often lead to a kind of ossification.\footnote{For example, Bulgakov insists that art "exists only in an atmosphere of freedom and disinterestedness. It must be free also from religion (of course this does not mean from God), and from morality (although not from the Good). Art is autocratic, and by its premeditated subordination it would only show that it does not believe in itself and is afraid of itself. But what is fainthearted art capable of? Then instead of creative quests the convention of stylization makes a home for itself, and instead of inspiration, correctness of canon," \emph{Unfading Light,} 394.} All the while, Bulgakov maintains that art, though fraught, has the potential to acquire for itself "an unfathomable depth" as Beauty is further revealed. Art is, argues Bulgakov, "the key that opens this depth \ldots{} {[}calling all{]} to a life in beauty and giv{[}ing{]} prophetic witness about it."\footnote{Bulgakov\emph{, Unfading Light,} 383.} As a "regal vocation,"\footnote{Bulgakov\emph{, Unfading Light,} 383.} art is concerned first with erotic longing. This "being in love," he continues, "this eros of being, gives broth to {[}the artist's{]} inspiration," to her "creative fire,"\footnote{Bulgakov\emph{, Unfading Light,} 382\textendash 383} to which the "spiritual limitedness of positivism remains alien."\footnote{Bulgakov\emph{, Unfading Light,} 395.} In other words, the arts in part help to continually, progressively reveal the authentic nature of creation and the many spheres of reality.

\subsection*{Nabokov: Art as Creation}

In his lectures on Dostoevsky, Nabokov contends, "Art is a divine game because this is the element in which man comes nearest to God through becoming a true creator in his own right."\footnote{Vladimir Nabokov, \emph{Lectures on Russian Literature,} ed.~Fredson Bowers (Boston: Mariner Books, 2002), 106.} Furthermore, he elaborates in another address that this sort of striving towards a poetic activity is akin to a creator deity fashioning a cosmos, arguing, "the real writer {[}is{]} the fellow who sends planets spinning and models a man asleep and eagerly tampers with the sleeper's rib."\footnote{Vladimir Nabokov, \emph{Lectures on Literature}, ed.~Fredson Bowers (Boston: Mariner Books. 2002), 2.} What's more, the adventurous quality of the artist's \emph{podvig} is witnessed in the artist's poetic deed and the reader's search: "{[}T{]}he glory of God is to hide a thing, and the glory of man is to find it,"\footnote{Vladimir Nabokov, \emph{Bend Sinister} (New York: Vintage, 1990), 106.} per \emph{Bend Sinister,} which is certainly echoing Proverbs 25:2. Mirroring this deity\textemdash and Nature, who possesses a "marvelous system of spells and wiles" to which "the writer of fiction" is but a follower of its lead\footnote{Nabokov, \emph{Lectures on Literature}, 5. Cf. Bulgakov's affinity with these thoughts regarding his speculations on Nature as "a great and wonderful artist" (\emph{Unfading Light}, 261) who "reveals its 'secrets' only to those who know where to look for them and would remain impenetrable to {[}humanity{]} if {[}one{]} did not possess a certain intuition in {[}one's{]} search," (\emph{Philosophy of Economy,} 144).}\textemdash the artist, in a "fit of lucid madness," poised to make individuated feats of renown, possesses "the zest of a deity building a live world from the most unlikely ingredients."\footnote{Nabokov, \emph{Lectures on Literature,} 290\textendash 291. It is again worth some stress how pervasively Nabokov's use of scriptural, liturgical, and theological stylizations is exploited. On this note, see Link's brilliant point, "Even as a sincere disavowal of any so-called 'interest in religion' (which may be fully granted), {[}Nabokov's{]} statement is, importantly, not a denial of the genuine role such 'Byzantine' materials self-evidently play in his literary inventions, especially in the 'private curatorship' of his formative years; to the contrary, it is a clarifying admission indicating, at the very least, an extensive aesthetic and thematic assimilation of such materials, obviously viewed by the author as worthy of literary treatment, ripe for continual reworking\textemdash though always on fresh canvases, as it were, under the distinctive strokes of his own brushes, in vibrant hues inimitably mixed on his own palette," "Recourse to Eden," 115.}

One must, of course, note how Nabokov spurned any instrumentalization of the artist's role in society\textemdash such as being held responsible for reporting "on the topics of the day," or serving as "a social commentator, {[}or{]} a class-war correspondent." In short, the exploit of the artist, for Nabokov, cannot be reduced to exploring the "general ideas" of "national, folklore, class, masonic, religious, or any other communal aura" that in all likelihood distract the \emph{good} reader from tasting "the nectar of possible talent" a writer might possess.\footnote{Nabokov, \emph{Strong Opinions,} 112\textendash 113. Bulgakov too was concerned about the petrification of words when they lose their "taste, smell, color" to the "verbal husk" of a utilitarian theory of language: "The magical use of a word, of course, is other than semantic or logical because the guiding aim here is not to express a thought but to unleash energy, to manifest the nocturnal, underground, concealed energy of a word," \emph{Philosophy of the Name,} 162, 164, respectively. Even given their differences here, both writers are concerned with artists and words \emph{not} being reduced to generalized, perfunctorily communicatory ends.} To use a Nabokovism, "true art deals not with the genus, and not even with the species, but with an aberrant individual of the species."\footnote{Nabokov, \emph{Strong Opinions,} 155. Or, as he never tired of telling his students, one must attend to "the supremacy of the detail over the general, of the part that is more alive than the whole, of the little thing which a man observes and greets with a friendly nod of the spirit while the crowd around is being driven by some common impulse to some common goal," Nabokov, \emph{Lectures on Literature,} 373.} Of course, this kind of statement could be seen to encourage an array of potential interpretations, including a lethargic misapplication of \emph{l'art pour l'art,} arguably further linking the artist to an atomistic, and truly nihilistic, \emph{incurvatus in se}. But this is not the only, nor even the best understanding; better would be simply to read this as another declaration (very similar to Bulgakov, in fact) of the impossibility of diminishing the artist and her work(s) to an easily definable classificatory system that can, in the wrong hands, reduce the particular genius of an artist to labels such as American Southern Gothic or French \emph{Nouveau Roman}.

Importantly, as we see from Nabokov's creature Fyodor Godunov-Cherdyntsev's "joyous energy \ldots{} {[}in{]} looking for the creation of something new, something still unknown, genuine, corresponding fully to the gift which {[}is{]} felt like a burden,"\footnote{Vladimir Nabokov, \emph{The Gift,} trans. Michael Scamell and Vladimir Nabokov (New York: Vintage International, 1991), 94.} the artist's \emph{podvig} begins with the artist's joyous burden to create, but it is not a solipsistic endeavor; instead, it entails participation by others who, in the case of the fiction writer, read and, more importantly, \emph{re-read}.\footnote{See Tom Whelan, "\,'And So~the~Password Is\textemdash ?': Nabokov and~the~Ethics of~Rereading," in \emph{Nabokov and the Question of Morality: Aesthetics, Metaphysics, and the Ethics of Fiction,} eds.~Michael Rodgers and Susan Elizabeth Sweeney (London: Palgrave Macmillan, 2016), 21\textendash 32. Whelan argues that Nabokov's insistence upon good readers being re-readers is attributable to "Nabokov's generosity" and "sympathy" as a writer (24, 28, respectively).} Nabokov describes the dance of artist and participant, as seen in his lecture "Good Readers and Good Writers," as like "the master artist" trekking "up a trackless slope \ldots{} and at the top, on a windy ridge, whom do you think he meets? The panting and happy reader, and there they spontaneously embrace and are linked forever."\footnote{Nabokov, \emph{Lectures on Literature,} 2.} From this linkage, the artist's \emph{podvig} takes on a further dimension than encountered previously, namely, the potentiality of a jubilant dilation of artist and "re-reader," embracing in their quest for the "aesthetic bliss" of a realm "where art"\textemdash which Nabokov describes as, "curiosity, tenderness, kindness, ecstasy" as well as more succinctly: "Beauty plus pity"\footnote{Nabokov, \emph{Lectures on Literature,} 251.}\textemdash is the norm."\footnote{Vladimir Nabokov, \emph{The Annotated Lolita}, ed.~Alfred Appel, Jr.~(New York: Penguin, 1991), 315. Cf. Eklund's argument, "true creativity requires not only an intentional orientation toward participation in transcendence and goodness \ldots{} but an intentional denial of immanentized or solipsistic beauty," "Eschatology and Theurgy in~\emph{Lolita}," 52. Eklund's analysis of Nabokov and Berdyaev, especially as regards \emph{theurgy}, is compelling. One future point worth considering between Nabokov vis-à-vis Berdyaev and Nabokov vis-à-vis Bulgakov would be Bulgakov's critique of theurgy (as developed by Solovyov and Berdyaev) as in need of a good washing within a sophiurgic "metacritical tub."}

Thus, as described above, in Nabokov's judgment, poetic making is firstly about the individual artist herself\textemdash and certainly not about contributing general ideas to a group or to the "public." And yet, Nabokov demonstrates a fierce attentiveness to the infinite mysteries of otherness throughout his works, a kind of ethose grounded in the cultivation of a profound willingness to compassionately see and sense the depths of exterior and interior reality (or \emph{realities}). Thinking reality to be "a very subjective affair," Nabokov imagined \emph{reality} as something to which one "can get nearer and nearer"; but, as he says in one interview, "you never get near enough because reality is an infinite succession of steps, levels of perception, false bottoms, and hence unquenchable, unattainable."\footnote{Nabokov, \emph{Strong Opinions,} 11.} Going further, in another interview, Nabokov challenges the use of 'reality' by the interviewer, arguing that, "To be sure, there is an average reality, perceived by us all, but that is not true reality. \ldots{} Paradoxically, the only real, authentic worlds are, of course, those that seem unusual. \ldots{} Average reality begins to rot and stink as soon as the act of individual creation ceases to animate a subjectively perceived texture."\footnote{Nabokov, \emph{Strong Opinions,} 118. Cf. 154. Also see Michael Wood's brilliant meditation on the nature of "the real" in Nabokov's fiction, \emph{The Magician's Doubts: Nabokov and the Risks of Fiction} (Princeton: Princeton University Press, 1994), chapter. 2: "The real is a banality and a longing. It is what we hold in our hands, often without knowing it, and what always escapes us; what is there and what can't be there; what we would miss if we lost it, what we miss and dream we have lost," 30.} One could speculate that this kind of logic is pertinent to Ivanov's phrase "\emph{a realibus ad realiora}" that Bulgakov so often quotes: an understanding that compels re-readers to progress beyond the "average" \emph{reality} of the so-called first look at the text, fostering a careful consideration to the ever-more \emph{real} particularities of work of art (and, more grandly, \emph{other life} before them). At any rate, the seeming uncircumscribable variability of \emph{reality} does not lead Nabokov the fiction writer away from mundane particularities but further up and in, so to speak. Examples are abundant. For instance, as the narrator's half-brother in \emph{The Real Life of Sebastian Knight} declares,

\begin{quote}
It has always distressed me that people in restaurants never notice the animated mysteries {[}of other persons{]}, \ldots{} I once reminded a businessman with whom I had lunched a few weeks before, that the woman who had handed us our hats had had cotton wool in her ears. He looked puzzled and said he hadn't been aware of there having been any woman \ldots{} {[}and later, he continues{]} a person who fails to notice a taxi-driver's hare-lip because he is in a hurry to get somewhere, is to me a monomaniac.\footnote{Vladimir Nabokov, \emph{The Real Life of Sebastian Knight} (New York: Vintage International, 1992), 106\textendash 107.}
\end{quote}

Mayhap there is no distinguishable utility in deliberately looking for and remembering those countless idiosyncrasies that make up the "animated mysteries" in one's midst, but there\-in lies a \emph{podvig} of sorts, one that is founded upon an "ethics of inutility"\footnote{Dragunoiu, \emph{Vladimir Nabokov and the Arts of Moral Acts,} 22: "As a moral impulse that has no recognizable utility but has discovered a graceful means of expression, Nabokovian courtesy is an art of the moral virtues."} that senses otherness as entirely singular and therefore unconditionally valuable. Take, for example, "The Vane Sisters," the narrator begins his story with an enthusiastic description of seemingly normal icicles "drip-dripping," and the experience of "{[}t{]}his twinned twinkle was delightful but not completely satisfying; or rather it sharpened {[}the narrator's{]} appetite for other tidbits of light and shade, and {[}he walked{]} on in a state of raw awareness that seemed to transform the whole of {[}his{]} being into one big eyeball rolling in the world's socket."\footnote{Vladimir Nabokov, "The Vane Sisters," in \emph{The Stories of Vladimir Nabokov} (New York: Vintage, 1996)\emph{,} 619.} This intimate attentiveness to the apparently mundane helps to discipline the careful re-reader to the hidden layers present in the work itself (e.g., the famous ghostly acrostic closing out the story), making something as seemingly whimsical as beholding the thawing of icicles a worthwhile activity with which to be enamored and formed. As such, the true creator, the individual artist\textemdash and the participative re-reader\textemdash gropes toward surprise, awe, kindness, astonishment, discovery, and compassion, all in response to the gift of intimately strange, "unquenchable, unattainable" otherness.

To conclude, both writers describe the exploits of the artist in Romantic terms, arguing for the preeminent value of personal, individual creativity in humanity's response to the many layers of reality. Furthermore, both writers goad us to sense the world and all its mysterious depths and heights with keener awareness to its poetic achievement, perhaps especially in those moments of Nabokovian timeless singularity. As he dreamily muses in \emph{Speak, Memory,} in these moments of "timelessness" a kind of "ecstasy" occurs, and "behind the ecstasy is something else \ldots{} like a momentary vacuum into which rushes all that I love. A sense of oneness with sun and stone. A thrill of gratitude to whom it may concern\textemdash to the contrapuntal genius of human fate or to tender ghosts humoring a lucky mortal."\footnote{Nabokov, \emph{Speak, Memory,} 479. Vivian Bloodmark, a philosophical "friend" of Nabokov's (who, just so, happens to have an anagrammatic name of Vladimir Nabokov), evidently helped him to develop a sense of "cosmic synchronization," especially as regards the far-reaching, far-seeing of the poet: "the poet feels everything that happens in one point of time. \ldots{} {[}where{]} trillions of \ldots{} trifles occur\textemdash all forming an instantaneous and transparent organism of events, of which the poet \ldots{} is the nucleus," Nabokov, \emph{Speak, Memory,} 544.} With an almost uncanny affinity Bulgakov too describes these moments of overwhelming unity in his Prague spiritual diary, whereby one's loving attention to the detail of the present becomes ever more finely attuned. Recording his impressions from a June morning in 1924, after waking to streets "made fresh after the night," Bulgakov muses how one can easily "attend to the song of the world borne to you from all sides, \ldots{} the immovable \emph{here} and \emph{now \ldots{}} an unceasing \emph{now.} And what a sin against oneself and against the world," he continues, "what cowardice you expose in yourself when you escape to the \emph{then.}"\footnote{Sergii Bulgakov, \emph{Spiritual Diary,} trans. Roberto De La Noval and Mark Roosien (Brooklyn: Angelico Press, 2022), 59. One can hear in the distance Nabokov's narrator of "The Fight," who concludes the short-story by ruminating, "perhaps what matters is \ldots{} the play of shadow and light on a live body, the harmony of trifles assembled on this particular day, at this particular moment, in a unique and inimitable way," in \emph{The Stories of Vladimir Nabokov,} 146.} This revealing passage subtly recapitulates the argument of his \emph{Vekhi} essay, where Bulgakov, on the one hand, critiques the intelligentsia revolutionary hero for sacrificing today for the sake of a better world tomorrow and, on the other, hails the \emph{podvizhnik} who is "freed from heroic postures and pretensions," instead, fondly attending to her "immediate work."\footnote{Bulgakov, "Героизм и подвижничество," 2:324\textendash 325.} Bulgakov reflects in his diary, "How to feel this in a single moment, the beating of the world's heart and myself as a drop of the warm blood of the world making through the world's body?"\footnote{Bulgakov, \emph{Spiritual Diary,} 60.} This feeling to which Bulgakov points presumes a kind of porosity of being, a sort of Nabokovian oneness with sun and stone, which can overwhelm one, as the narrator of "Beneficence" relays it, with a sense of "the blissful bond between me and all of creation,"\footnote{Vladimir Nabokov, "Beneficence," in \emph{The Stories of Vladimir Nabokov,} 77.} a sense that is perhaps never truly lost, even if one finds oneself entangled within an immanentized, physicalist metaphysic. Both writers aid us in imagining, as the narrator of \emph{Bend Sinister} intimates, "a rent" in our world "leading to another world of tenderness, brightness, and beauty\emph{,}"\footnote{Vladimir Nabokov, \emph{Bend Sinister,} in \emph{Novels and Memoirs 1941\textendash 1951,} ed.~Brian Boyd (New York: Library of America, 1996), 165\textendash 66.} and in purposely making these realities all the more real in the here and now.\footnote{Especial thanks to Celeste Jean, Danny Sebastian, Steve Long, and the two anonymous reviewers for making this labor more generative; any errors of substance, style, or syntax, of course, remain my own.}

\vspace{0.5em}
\begin{center}
  \includegraphics[width=0.75cm]{articlend.png}
\end{center}

\biobox{\textbf{Daniel Adam Lightsey} is a doctoral candidate in religious studies at Southern Methodist University. His current work has to do mostly with Sergii Bulgakov's theology of beauty in relation to his metaphysics of personhood, all with a wider view to how creativity and artistic \textit{poiesis} were understood within twentieth-century Russian religious-philosophical-literary milieus. Some of his other recent work can be found in \textit{Moral Conversion in Scripture, Self, and Society} (De Gruyter, 2024) and \textit{Art, Desire, and God: Phenomenological Perspectives} (Bloomsbury Academic, 2023).}

\label{sec:lightsey}

\section{Poole - Kinddom of spirits}

\fancypagestyle{chaptercontentpage}{
  \fancyhf{} % Clear all header and footer fields
  \fancyhead[CE]{%
    \fontsize{11}{11}\leftmarkfont%
    \addfontfeature{LetterSpace=10.0}%
    \textit{\MakeUppercase{\leftmark}}%
  }
  \fancyhead[CO]{\authorheadfont\addfontfeature{LetterSpace=10.0}\fontsize{11}{11}\selectfont\textbf{{\uppercase{Randall A. Poole}}}}

  \fancyfoot[RE]{\thepage}
  \fancyfoot[LO]{\thepage}
  \renewcommand{\headrulewidth}{0pt} % No header rule on content pages
}
\setcounter{footnote}{0}
\abstractbox{"The Kingdom of Spirits"}{Semyon Frank and Russian Religious Personalism}{Randall A. Poole}{Personalism holds that persons are rational, moral, creative, and spiritual beings who bear an intrinsic worth or dignity and who are the very center of reality: its ontological center (persons are the highest form of reality), its axiological center (persons are the supreme value in reality), and its epistemological center (through persons reality is intelligible). This essay deals with Russian and Russo-French personalism, spanning the one hundred years from Ivan Kireevsky to Semyon Frank, whom V. V. Zenkovsky famously regarded as Russia’s greatest philosopher and in whose works Russian religious personalism arguably achieved its highest degree of development. The essay gives attention to Vladimir Soloviev’s importance in Russian personalism, to interwar Russo-French personalism in the figures of Nikolai Berdiaev and Jacques Maritain, to the personalist rebirth of human rights in the twentieth century, and to the similarities between John Zizioulas’s neo-patristic personalism and Frank’s personalist ontology of absolute realism.}{Nikolai Berdiaev, Semyon Frank, Jacques Maritain, Vladimir Soloviev, John Zizioulas, personalism, personhood, human dignity, human rights, Christian humanism}

\fancypagestyle{chaptertitlepage}{
  \fancyhf{} % Clear all header and footer fields
  \fancyhead[L]{\begin{minipage}[t]{0.7\textwidth}\publisher\end{minipage}}
  \fancyhead[R]{\begin{minipage}[t]{\textwidth}\raggedleft \datefont\fontsize{10}{11}\selectfont Volume 1 (2024): \thepage\textendash\pageref{sec:poole1} \\ \doi{10.71521/2bre-nm65} \end{minipage}}
  \renewcommand{\headrulewidth}{0pt} % No header rule on title pages
  \fancyfoot[RE]{\thepage}
  \fancyfoot[LO]{\thepage}
}
\chaptertitle{"The Kingdom of Spirits"}{Semyon Frank and Russian Religious Personalism}{Randall A. Poole}

\addcontentsline{toc}{chapter}{"The Kingdom of Spirits":\\ Semyon Frank and Russian Religious Personalism\\\emph{by} Randall A. Poole}

\seriffont
\fontsize{12}{18}\selectfont

\subsection*{Introduction: Russo-French Personalism and Human Rights}

The term "personalism" (\emph{"der Personalismus"}) seems to have entered modern philosophical discourse in 1799, when Friedrich Schleiermacher (1768\textendash 1834) introduced it in his book \emph{Über die Religion: Reden an die Gebildeten unter ihren Verächtern} (\emph{On Religion: Speeches to Its Cultured Despisers}).\footnote{Thomas D. Williams and Jan Olof Bengtsson, "Personalism," \textit{The Stanford Encyclopedia of Philosophy} (Summer 2022 online edition); \url{https://plato.stanford.edu/archives/sum2022/entries/personalism/}  Williams is also the author of the valuable study, \emph{Who Is My Neighbor? Personalism and the Foundation of Human Rights} (Washington, D.C.: The Catholic University Press of America, 2005).} The immediate predecessor of personalism as an international philosophical movement was Rudolph Hermann Lotze (1817\textendash 1881), especially through his three-volume work on philosophical anthropology, \emph{Mikrokosmus} (1856\textendash 1864).\footnote{See Johan De Tavernier, "The Historical Roots of Personalism," \emph{Ethical Perspectives}, vol.~16. no. 3 (2009): 363. See also David Sullivan, "Hermann Lotze," \emph{Stanford Encyclopedia of Philosophy} (Winter 2024 online edition); \url{https://plato.stanford.edu/archives/win2024/entries/hermann-lotze/}. The Sullivan article has a section on personalism as one of Lotze's three legacies. Lotze was a strong influence on the Russian philosopher Lev M. Lopatin (1855\textendash 1920), who was a main representative of neo-Leibnizianism in Russian thought (together with Aleksei A. Kozlov, Sergei A. Askol'dov, and Nikolai O. Lossky). In their classic histories of Russian philosophy, V. V. Zenkovsky and N. O. Lossky classify Lopatin and the others as personalists (in the Leibnizian sense). See V. V. Zenkovsky, \emph{A History of Russian Philosophy}, trans. George L. Kline, 2 vols. (New York: Columbia University Press, 1953), vol.~2: 630\textendash 676; and Nicholas O. Lossky\emph{, History of Russian Philosophy} (New York: International Universities Press, 1951), 158\textendash 162. For revisions of these traditional classifications, see James P. Scanlan, "Russian Panpsychism: Kozlov, Lopatin, Losskii," in \emph{A History of Russian Philosophy, 1830\textendash 1930: Faith, Reason, and the Defense of Human Dignity}, ed.~G. M. Hamburg and Randall A. Poole (Cambridge: Cambridge University Press, 2010), 150\textendash 168. See also the end of the section below, "Russian Personalism before Berdiaev."} In France, the philosopher Charles Renouvier (1815\textendash 1903) published a book under the title \emph{Le Personnalisme} in 1903. By then a robust American school of idealist personalism was emerging in the figures of Borden Parker Bowne (1847\textendash 1910), George H. Howison (1834\textendash 1916), Ralph T. Flewelling (1871\textendash 1960), Albert C. Knudson (1873\textendash 1953), and others.\footnote{See in particular Albert C. Knudson, \emph{The Philosophy of Personalism: A Study in the Metaphysics of Religion} (New York and Cincinnati: The Abingdon Press, 1927).} Soon thereafter personalism became a philosophical force in interwar France, where the exiled Russian philosopher Nikolai Berdiaev (1874\textendash 1948) was one of its spokesmen, together with Emmanuel Mounier (1905\textendash 1950) and Jacques Maritain (1882\textendash 1973). So far as I know, the first book in which Berdiaev used the term "personalism" is \emph{The Destiny of Man}, published in Russian in 1931 and in French in 1935.\footnote{Nikolai Berdiaev, \emph{O naznachenii cheloveka: Opyt paradoksal'noi etiki} (Paris: "Sovremennye zapiski," YMCA Press, 1931). "Personalism" is used in the heading of the second section of chapter 3 (pp.~60\textendash 67). In the English edition, see Nicolas Berdyaev, \emph{The Destiny of Man}, trans. Natalie Duddington (New York: Harper Torchbooks, 1960), 54\textendash 61.} In any event it is one of the first uses of the term in the interwar Russo-French religious-philosophical milieu.\footnote{Another Russian living in Paris, Alexandre Marc, also used the term in 1931. Marc (whose real name was Alexandre Markovich Lipiansky) was a Russian Jew of socialist-revolutionary tendencies who had converted to Catholicism. He founded and edited the rightist review \emph{Ordre Nouveau}. See De Tavernier, "The Historical Roots of Personalism," 367; and Antoine Arjakovsky, \emph{The Way: Religious Thinkers of the Russian Emigration in Paris and Their Journal, 1925\textendash 1940}, trans. Jerry Ryan, ed.~John A. Jillions and Michael Plekon (Notre Dame, IN: University of Notre Dame Press, 2013), 329.} Some months earlier, on December 8, 1930, Mounier's review \emph{Esprit} and the personalist movement associated with it were conceived at a meeting at Berdiaev's home in Clamart. The actual founding meeting took place the following day at another location.\footnote{Arjakovsky, \emph{The Way}, 192. \emph{Esprit} began publication in October 1932 and continues to this day. Berdiaev wrote, "I was present at the meeting at which \emph{Esprit} was founded. This took place at the home of I., a left-wing Roman Catholic, subsequently a Deputy and a member of the Socialist Party." Nicolas Berdyaev, \emph{Dream and Reality: An Essay in Autobiography}, trans. Katharine Lampert (New York: Macmillan, 1951), 274.} In his memoirs Berdyaev wrote that the movement's combination of socialism and personalism was expressed in the newly coined term \emph{personalisme communautaire}.\footnote{Berdyaev, \emph{Dream and Reality}, 274.} Maritain claims that he himself coined the term, which then became associated especially with Mounier.\footnote{Jacques Maritain, \emph{The Peasant of the Garonne: An Old Layman Questions Himself about the Present Time}, trans. Michael Cuddihy and Elizabeth Hughes (New York: Holt, Rinehart, and Winston, 1968), 51, as cited in Juan Manuel Burgos, \emph{An Introduction to Personalism}, trans. R. T. Allen (Washington, D.C.: Catholic University Press, 2018), 53.} Berdiaev adopted a similar term, "personalist socialism," to designate his own mature social philosophy, which he expounded in \emph{Slavery and Freedom} (1936).\footnote{Nikolai Berdyaev, \emph{Slavery and Freedom}, trans. R. M. French (Philmont, NY: Semantron Press, 2009).}

Today personalism is recognized as a whole philosophical worldview with a long and rich history.\footnote{See Williams and Bengtsson, "Personalism," \emph{Stanford Encyclopedia of Philosophy}; Burgos, \emph{An Introduction to Personalism}; and Rufus Burrow, \emph{Personalism: A Critical Introduction} (St.~Louis, MO: Chalice Press, 1999).} It has become popular to be a personalist. On June 15, 2018, David Brooks published an op-ed in the \emph{New York Times} titled, "Personalism: The Philosophy We Need." Personalism holds that persons are rational, moral, creative, and spiritual beings who bear an intrinsic worth or dignity and who are the very center of reality: its ontological center (persons are the highest form of reality), its axiological center (persons are the supreme value in reality), and its epistemological center (through persons reality is intelligible). Most forms of personalism are broadly theistic, with personalism itself plausibly tracing its origins to the Christian doctrine of the Holy Trinity: three persons in one God. One of the most influential personalists of the twentieth century was Pope John Paul II.\footnote{He is the center of attention in Williams, \emph{Who Is My Neighbor? Personalism and the Foundation of Human Rights}.} Another was Dr.~Martin Luther King, who built the American Civil Rights Movement on personalist foundations.\footnote{See Rufus Burrow, \emph{God and Human Dignity: The Personalism, Theology, and Ethics of Martin Luther King, Jr.} (Notre Dame: University of Notre Dame Press, 2006). There is a section on Berdiaev in King's essay, "Contemporary Continental Theology," which he likely wrote in 1951\textendash 1952 as a doctoral student at Boston University's School of Theology. The essay is available at https://kinginstitute.stanford.edu/king-papers/about-papers-project.}

Personalism is attracting renewed scholarly attention because of its connection with the twentieth-century history of human rights.\footnote{See Samuel Moyn, \emph{Christian Human Rights} (Philadelphia, PA: University of Pennsylvania Press, 2015).} The main figure in this intellectual history is Jacques Maritain, whose 1942 book \emph{The Rights of Man and Natural Law} set him on the path to become, in Moyn's estimation, "the premier postwar philosopher of human rights."\footnote{See Samuel Moyn, "Personalism, Community, and the Origin of Human Rights," in \emph{Human Rights in the Twentieth Century,} ed.~S.-L. Hoffman (Cambridge, UK: Cambridge University Press, 2011), 87, 90.} As chair of the 1947 UNESCO Committee on the Philosophical Principles of the Rights of Man, he was one of the intellectual architects of the Universal Declaration of Human Rights.\footnote{Soon after its founding, the UN's Commission on Human Rights, chaired by Eleanor Roosevelt, asked UNESCO to help with its work on an international bill of rights. UNESCO invited 150 leading intellectuals from around the world to submit papers on philosophical issues raised by human rights, especially the problem of how to ground them. The Committee on the Philosophical Principles of the Rights of Man convened to discuss the papers, which are available on UNESCO's website and which were also published: UNESCO ed., \emph{Human Rights: Comments and Interpretations}, intro. Jacques Maritain (New York: Columbia University Press, 1949). Maritain also contributed a chapter, "On the Philosophy of Human Rights."} His 1936 treatise \emph{Integral Humanism} is a profound explication of personalism, yet offers no explicit endorsement of human rights.\footnote{Jacques Maritain, \emph{Integral Humanism, Freedom in the Modern World, and a Letter on Independence}, ed.~Otto Bird, trans. Otto Bird, Joseph Evans, and Richard O'Sullivan, K.C. (Notre Dame: University of Notre Dame Press, 1996).} That changed with \emph{Scholasticism and Politics}, based on nine lectures that Maritain delivered at the University of Chicago in the autumn of 1938. What he calls the personalist conception of democracy "is first of all determined by the idea of man as God's image, and by the idea of the common good, of human rights and of concrete liberty; and it is based on Christian humanism."\footnote{Jacques Maritain, \emph{Scholasticism and Politics}, ed.~Mortimer J. Adler (Garden City, NY: Image Books, 1960), 87.} He refers repeatedly to the "primordial rights of the person," which are the very basis of governmental authority. It is clear that the direct cause of Maritain's shift between 1936 and 1938 was Pope Pius XI. In March 1937 the pope issued two encyclicals, \emph{Mit brennender Sorge} and \emph{Divini Redemptoris}, both of which forcefully deployed the idea of human rights against their respective targets (Nazism and Communism).\footnote{Maritain quotes from \emph{Divini Redemptoris} in \emph{Scholasticism and Politics} (110).} This indicated that the Church was moving further along the path laid by Leo XIII, which progress authorized Maritain's new position on human rights.

Berdiaev's personalism, by contrast, had all along been an explicit defense of human rights. In \emph{The Destiny of Man} he wrote categorically: "The only political principle which is connected with absolute truth is the principle of the subjective rights of the human personality, of the freedom of spirit, of conscience, of thought and speech."\footnote{Berdyaev, \emph{The Destiny of Man}, 198.} For him, everything in politics is relative except human rights. Even more remarkable, pre-revolutionary Russian religious idealism, long before Berdiaev brought it to the West and christened it "personalism," forcefully defended the idea of human dignity and the rights which issue from it. The two greatest metaphysical idealists of nineteenth-century Russia, Boris Chicherin (1828\textendash 1904) and Vladimir Soloviev (1853\textendash 1900), were resolute champions of human rights.\footnote{Boris Chicherin's \emph{Filosofiia prava} (\emph{Philosophy of Right}) (1900) is a profound idealist defense of human rights and the rule of law. Human dignity and human rights, including the "new" right to a dignified or worthy human existence, are central to Soloviev's \emph{Justification of the Good} (1897), often regarded as the most important Russian work of moral theory. See Vladimir Solovyov, \emph{The Justification of the Good: An Essay on Moral Philosophy}, trans. Natalie A. Duddington, edited and annotated Boris Jakim (Grand Rapids, MI: William B. Eerdmans Publishing Company, 2005), especially Part 3, ch.~4 ("The Moral Norm of Social Life"). For a classic account of these philosophers (and their followers), emphasizing the priority they gave to human rights, see Andrzej Walicki, \emph{Legal Philosophies of Russian Liberalism} (Oxford: Oxford University Press, 1987).} Their legacy inspired the 1902 collection \emph{Problems of Idealism}. In his contribution to the volume, Berdiaev wrote: "The person in his 'natural' rights is sovereign. \ldots{} Ethically, nothing can justify violation of the natural rights of man, for there is no end in the world, in the name of which the sacred strivings of the human spirit could be infringed upon or in the name of which the principle of the human person as an end in itself could be betrayed."\footnote{N. A. Berdiaev, "The Ethical Problem in the Light of Philosophical Idealism," in \emph{Problems of Idealism: Essays in Russian Social Philosophy}, ed., trans., and intro. Randall A. Poole (New Haven, CT: Yale University Press, 2003), 179.} Thus, as paradoxical as it may seem in the face of centuries of Russian autocracy, the "old Russian tradition of religious personalism," as Moyn refers to it,\footnote{Moyn, \emph{Christian Human Rights}, 68\textendash 69; Moyn, "Personalism, Community, and the Origin of Human Rights," 86\textendash 87.} was a robust theory of human rights decades before the Catholic Church and Maritain made their own very similar personalist discoveries and transformed international understanding of human rights.

\subsection*{Russian Personalism before Berdiaev}

Though the term "personalism" was not commonly used by Russian philosophers before Berdiaev, it has been appropriated by historians of Russian thought to describe a deep feature of the Russian religious-philosophical tradition. The most recent and impressive example is Sergei Polovinkin's book \emph{Russkii personalizm}, which appeared in 2020, two years after the author's death.\footnote{S. M. Polovinkin, \emph{Russkii personalizm} (Moscow: Sinaksis, 2020).} It runs to more than 1100 pages. The defense of human dignity and of personhood (\emph{lichnost'}) was arguably the most important preoccupation of nineteenth and twentieth-century Russian thought.\footnote{See \emph{A History of Russian Philosophy, 1830\textendash 1930: Faith, Reason, and the Defense of Human Dignity}, ed.~Hamburg and Poole. See also Nikolaj Plotnikov, "\,'The Person is a Monad with Windows': Sketch of a Conceptual History of 'Person' in Russia," \emph{Studies in East European Thought}, vol.~64 (2012): 269\textendash 299; and \emph{Personal'nost': Iazyk filosofii v russko-nemetskom dialoge}, ed.~N. S. Plotnikov and A. Haardt (Moscow: Modest Kolerov, 2007).} The \emph{lichnost'} theme was a very broad one. The term can mean person, personhood, personality, individual, individuality, and even self. It did not necessarily carry metaphysical or theistic connotations. In the nineteenth century, a whole range of Russian thinkers including the Westernizers, Alexander Herzen, the "subjective sociologists" Peter Lavrov and Nikolai Mikhailovsky, and liberals of all stripes used \emph{lichnost'} to stress that the individual was an autonomous, active moral agent capable of introducing his or her own values into history and of striving for progress. It was the subject of Ivanov-Razumnik's classic work, \emph{Istoriia russkoi obshchestvennoi mysli} (\emph{History of Russian Social Thought}) (1907). Ivanov-Razumnik was a neo-populist thinker, a Socialist Revolutionary, who extolled the "primacy of the person" and asserted that "the good of the real human person should serve as the criterion of our acts and our worldview."\footnote{Quoted by Richard S. Wortman, \emph{The Power of Language and Rhetoric in Russian Political History: Charismatic Words from the 18\textsuperscript{th} to the 21\textsuperscript{st} Centuries} (London and New York: Bloomsbury, 2018), 75. \emph{Lichnost'} is one of the "charismatic words" which form the subject of Professor Wortman's book.} He celebrated "ethical individualism," contrasting it to "philistine egoism."

By the time Ivanov-Razumnik's book appeared, a different group of Russian thinkers had achieved prominence. They were the metaphysical idealists and religious philosophers of the early twentieth-century Russian religious renaissance, which drew its inspiration from Vladimir Soloviev. They gave \emph{lichnost'} the philosophically exalted meaning of \emph{personhood}, which emphasized the person's absolute worth, dignity, and rights. \emph{Problems of Idealism} was an important milestone in this development. As Pavel Novgorodtsev (1866\textendash 1924) wrote in his foreword to the book, Russian idealism attached "primary importance to the principle of the absolute significance of personhood."\footnote{P. I. Novgorodtsev, "Foreword to the Russian Edition," \emph{Problems of Idealism}, 83.} Though he and the other contributors did not use the term "personalism," certainly they advanced the concept.

The Russian tradition of religious-philosophical personalism began with the Slavophiles, specifically with Ivan Kireevsky's concepts of "integral personhood" (\emph{tsel'naia lichnost'}) and "believing reason" and with Aleksei Khomiakov's concept of \emph{sobornost}, which refers to the qualities of an ideal community (for him the Church) through which people can most fully realize their personhood.\footnote{Further on these concepts, see my essay, "Slavophilism and the Origins of Russian Religious Philosophy," in \emph{The Oxford Handbook of Russian Religious Thought}, ed.~Caryl Emerson, George Pattison, and Randall A. Poole (Oxford: Oxford University Press, 2020), 133\textendash 151.} In one place Kireevsky (1806\textendash 1856) wrote that "only a reasoning and free personality is what is essential in the world. It alone has a distinctive significance. Everything else has only a relative significance."\footnote{Ivan Kireevsky, "Fragments," in \emph{On Spiritual Unity: A Slavophile Reader}, trans. and ed.~Robert Bird and Boris Jakim (Hudson, NY: Lindisfarne Books, 1998), 284.} The Russian personalist tradition continued with Boris Chicherin, though he referred to it as "individualism."\footnote{{B. N. Chicherin, \emph{Filosofiia prava} (Moscow: I. N. Kushnerev, 1900), 228. Following Chicherin, Pavel Novgorodtsev also used the term "individualism" in \emph{Problems of Idealism} and in his 1909 book, \emph{Krizis sovremennogo pravosoznaniia} (\emph{The Crisis of Modern Legal Consciousness}). In his chapter in \emph{Problems of Idealism} he referred to the "subtle and penetrating thinker" Charles Renouvier, who within a year published his book, \emph{Le Personnalisme}. Novgorodtsev associates him with the "revival of individualism," by which he means the ethical and metaphysical defense of personhood. See P. I. Novgorodtsev, "Ethical Idealism in the Philosophy of Law (On the Question of the Revival of Natural Law)," \emph{Problems of Idealism}, 307. In his book \emph{The New Middle Ages} (1924)\textemdash translated into English under the title \emph{The End of Our Time}\textemdash Berdiaev made the fundamental personalist distinction between the individual, a biological and sociological concept, and the person, a spiritual and religious concept. See Nicolas Berdyaev, \emph{The End of Our Time}, trans. Donald Attwater (New York: Sheed and Ward, 1933), 35, 85\textendash 86. Maritain made this very distinction in his book, \emph{Three Reformers}, published in 1925. In it he wrote, famously, "As individuals, we are subject to the stars. As persons, we rule them." See Jacques Maritain, \emph{Three Reformers: Luther, Descartes, Rousseau} (New York: Charles Scribner's Sons, 1929), 21.}} Chicherin always thought of himself as a Hegelian, yet by 1880 he had adopted a liberal, Kantian interpretation of Hegel that stressed the intrinsic, absolute worth of human personhood. He embraced Kant's idealist conception of human nature, which was based on the dual power of reason to posit or recognize ideals and to determine the will according to them. Chicherin directly drew metaphysical conclusions (theism) from this quintessential human capacity for ideal self-determination.\footnote{For a fuller account see my essay, "The Liberalism of Russian Religious Idealism," in \emph{The Oxford Handbook of Russian Religious Thought}, 255\textendash 276, esp.~259\textendash 263.} Rational autonomy or self-determination was also the distinctively human principle in Vladimir Soloviev's tripartite conception of human nature, between the absolute or divine principle and the material principle. Together the human and divine principles form \emph{Bogochelovechestvo} (Godmanhood or divine humanity), Soloviev's central concept. It is the free human realization of the divine principle in ourselves and in the world\textemdash the process of \emph{theosis} or deification.\footnote{See Poole, "Vladimir Solov'ëv's Philosophical Anthropology: Autonomy, Dignity, and Perfectibility," in \emph{A History of Russian Philosophy,} 1830\textendash 1930, ed.~Hamburg and Poole, 131\textendash 149.} Other currents in Russian religious-philosophical personalism include Lev Lopatin's "concrete spiritualism" and Sergei Trubetskoi's "concrete idealism"\textemdash "concrete" designating the personalist focus;\footnote{On Lopatin, see my essay, "William James in the Moscow Psychological Society: Pragmatism, Pluralism, Personalism," in \emph{William James in Russian Culture}, ed.~Joan Delaney Grossman and Ruth Rischin (Lanham, MD: Lexington Books, 2003), 131\textendash 158, esp.~138\textendash 143. On Trubetskoi, see Martha Bohachevsky-Chomiak, \emph{Sergei N. Trubetskoi: An Intellectual Among the Intelligentsia in Prerevolutionary Russia} (Belmont, MA: Nordland, 1976); O. T. Ermishin\emph{, Kniaz' S. N. Trubetskoi: Zhizn' i filosofiia} (Moscow: Sinaksis, 2011); and P. P. Gaidenko, \emph{Vladimir Solov'ev i filosofiia Serebrianogo veka} (Moscow: Progress-Traditsiia, 2001), 121\textendash 161.} Nikolai Berdiaev's philosophy of freedom and creativity;\footnote{See Ana Siljak, "The Personalism of Nikolai Berdiaev," in \emph{The Oxford Handbook of Russian Religious Thought}, 309\textendash 326; and Regula M. Zwahlen, "Different Concepts of Personality: Nikolaj Berdjaev and Sergej Bulgakov," \emph{Studies in East European Thought}, vol.~64 (2012): 183\textendash 204.} Sergei Bulgakov's Trinitarian theology of personhood and personalistic metaphysics of all-unity;\footnote{Some of the seminal texts are from the 1920s, such as \emph{The Tragedy of Philosophy} (1927), written in Crimea before Bulgakov's expulsion from the Soviet Union in 1922, and \emph{Glavy o Troichnosti} (\emph{Chapters on Trinitarity}) (1928\textendash 1930). See Sergii Bulgakov, \emph{Tragedy of Philosophy}, trans. Stephen Churchyard (New York: Angelico Press, 2020). On these works, see Joshua Heath, "On Sergii Bulgakov's \emph{The Tragedy of Philosophy}," \emph{Modern Theology} 37/3 (July 2021), 805\textendash 823; Heath, "Sergii Bulgakov's Linguistic Trinity," \emph{Modern Theology} 37/4 (October 2021): 888\textendash 912; and K. M. Antonov, "Problema lichnosti v myshlenii o. S. Bulgakova i problematika bogoslovskogo personalizma v XX veke," \emph{Rozhdenie personalizma iz dukha Novogo vremeni: Sbornik statei po genealogii bogoslovskogo personalizma v Rossii}, ed.~V. N. Boldareva (Moscow: Izdatel'stvo PSTGU, 2018), 176\textendash 185.} Nikolai Lossky's panpsychic "hierarchical personalism";\footnote{See James P. Scanlan, "Russian Panpsychism: Kozlov, Lopatin, Losskii," in \emph{A History of Russian Philosophy, 1830\textendash 1930}, ed.~Hamburg and Poole, 150\textendash 168; and Gaidenko, \emph{Vladimir Solov'ev i filosofiia Serebrianogo veka}, 211\textendash 241.} Lev Karsavin's "symphonic" idea of the person within his metaphysics of all-unity;\footnote{See Martin Beisswenger, "Eurasianism: Affirming the Person in an 'Era of Faith,'\," in \emph{A History of Russian Philosophy, 1830\textendash 1930}, ed.~Hamburg and Poole, 363\textendash 380.} and Semyon Frank's personalist ontology of "absolute realism"\textemdash which will occupy us below.

\subsection*{Neo-Patristic Personalism (John Zizioulas)}

Eastern Orthodox theology was an important source of the development of Russian personalism. In the Christian patristic era the concept of the person received powerful new development.\footnote{John M. Rist, \emph{What is a Person? Realities, Constructs, Illusions} (Cambridge: Cambridge University Press, 2020), 22\textendash 55.} It figured centrally in the thought of the Greek Church Fathers, in the great Trinitarian and Christological debates. On these historical and dogmatic foundations, twentieth-century Orthodox theology centralized the concept of the person. Orthodox personalism was a main feature of the neo-patristic revival, associated first of all with Georges Florovsky and Vladimir Lossky.\footnote{See \emph{Georges Florovsky: Russian Intellectual and Orthodox Churchman}, ed Andrew Blane (Crestwood, NY: St.~Vladimir's Seminary Press, 1993). See also Paul L. Gavrilyuk, \emph{Georges Florovsky and the Russian Religious Renaissance} (Oxford: Oxford University Press, 2014) and my review of this volume in \emph{Modern Greek Studies Yearbook: A Publication of Mediterranean, Slavic, and Eastern Orthodox Studies} (University of Minnesota), vols. 30/31 (2014/2015): 514\textendash 517. For a recent edition of some of his works, see \emph{The Patristic Witness of Georges Florovsky: Essential Theological Writings}, ed.~Brandon Gallaher and Paul Ladouceur (London: T\&T Clark, 2019). For Lossky's theology of the person, see Vladimir Lossky, \emph{In the Image and Likeness of God}, ed.~John H. Erikson and Thomas E. Bird (Crestwood, NY: St.~Vladimir's Seminary Press, 2001).} Before returning to Russian religious-philosophical personalism proper, let me take one prominent example of Orthodox or neo-patristic personalism: John Zizioulas (1931\textendash 2023), who was Metropolitan of Pergamon in the Ecumenical Patriarchate of Constantinople and one of the most influential Orthodox Christian theologians of the past century. His book, \emph{Being as Communion} (1985), helps us to understand the patristic foundations of personalism, including of course Russian personalism.\footnote{John D. Zizioulas, \emph{Being as Communion: Studies in Personhood and the Church} (Crestwood, NY: St.~Vladimir's Seminary Press, 1985). Subsequent references to this book cited parenthetically in the text. See also Aristotle Papanikolaou, \emph{Being with God: Trinity, Apophaticism, and Divine-Human Communion} (Notre Dame, IN: University of Notre Dame Press, 2006), a study of Zizioulas and Lossky.}

According to Zizioulas, "The person both as concept and as a living reality is purely the product of patristic thought" (27). He had in mind the Greek Fathers in particular: "With a rare creativity worthy of the Greek spirit they gave history the concept of the person with an absoluteness which still moves modern man even though he has fundamentally abandoned their spirit" (35). Before the Church Fathers, the Greek term for person, \emph{prosōpon}, and the Latin term \emph{persona}, meant something very different ("mask" or "role," respectively). In these earlier meanings, "person" designated something superficial that was distinguished from man's essential being, which essential being belonged to a greater whole, for the Greeks the cosmos and for the Romans the state. Zizioulas stresses that in its pre-patristic meanings "person" lacked specific ontological content or depth. Being was above the human individual and bound him in various ways. The individual was a product of this higher ontological necessity. What the Church Fathers accomplished, according to Zizioulas, was to invest the concept of the person with essential being, to "ontologize" it, by identifying it with the term "hypostasis," which already was closely linked with the term "substance" (\emph{ousia}). As a result of this identification, "from an adjunct to a being (a kind of mask) the person becomes the being itself and is simultaneously\textemdash a most significant point\textemdash the \emph{constitutive element} (the 'principle' or 'cause' ) of beings" (39).

This transformation in the meaning of "person" was achieved in the course of the theological elaboration of the doctrine of the Holy Trinity, the final formula of which is "one substance (\emph{ousia}), three persons (\emph{hypostaseis})." In the West, the formula was understood to mean that the ontological principle of God is the divine substance rather than the person. Zizioulas regards this as a misinterpretation: It is the person of the Father, not the common substance, that is the principle or cause of God's being (40). The "monarchy of the Father," of his personhood, means that God's being is not an ontological necessity issuing from the divine substance but is rather the result of God's free will to exist. God's existence is personal, and therefore it is also free (persons possess free will) and trinitarian (persons exist in community). The divine substance is personal by its very nature: it "never exists in a 'naked' state, that is, without hypostasis, without 'a mode of existence.' \ldots{} Outside the Trinity there is no God, that is, no divine substance" (41). Trinitarian theology reveals that the ontological principle of God is not a pre-existing substance but a person, the Father, whose existence is an act of free will and whose love begets the Son and brings forth the Holy Spirit, forming the Trinity or community of persons in which God subsists.

Zizioulas's personalism extends to human persons, of course. Human persons are limited because they are not the free cause of their own existence (as God is); rather they are created. Authentic personhood is uncreated and enjoys "absolute ontological freedom." It is the condition of personhood itself: "If God does not exist, the person does not exist" (43). For Zizioulas, divine personhood is both the ground and goal of human personhood. The goal is "that man might become a person in the sense that God is one, that is, an authentic person" (44). Zizioulas specifies that this is the very content of salvation as theosis. It is the full realization of our personhood in God, the fulfillment of the hope "that the personal life which is realized in God should also be realized" in us (50).

Just how is the personal life realized in God and how are we to strive to realize it in ourselves? Here too Zizioulas is reasonably clear. He quotes 1 John 4:16, "God is love," and explains that love constitutes God's very being or substance. It is not a qualifying property of being but "\emph{the supreme ontological predicate}" and the basis of God's ontological freedom (it abolishes the ontological necessity of substance). In a striking formulation Zizioulas writes, "Love as God's mode of existence 'hypostasizes' God" (46). It makes his "mode of existence" a personal one, a Trinitarian one. From the love that is God and is his ontological freedom, the Father begets the Son and brings forth the Holy Spirit. In this connection Zizioulas refers to God's ecstatic character, to "the fact that His being is identical with an act of communion" (44). As he otherwise puts it, "\emph{To be} and \emph{to be in relation} becomes identical" (88). At the human level, it is clear that while love cannot yet (at our level of being) become the freely-willed cause of our very existence (as it is for God), we should strive, through love, to "hypostasize" ourselves ever more deeply and to immortalize ourselves in the process, because "life for the person means the survival of the uniqueness of its hypostasis, which is affirmed and maintained by love" (49).

\subsection*{Russian Religious Personalism: From Soloviev to Frank}

In Russia, the patristic heritage described by Zizioulas was recovered by the country's four theological academies, which undertook a massive, multigenerational effort (from approximately 1821 to 1918) to publish the works of Church Fathers in Russian translation, accompanied by extensive scholarly analysis.\footnote{Cyprien Kern, \emph{Les traductions russes des texts patristiques: Guide bibliographique} (Paris: Éditions de Chevetogne, 1957). See also Jeremy Pilch, \emph{'Breathing the Spirit with Both Lungs': Deification in the Work of Vladimir Solov'ev} (Leuven: Peeters, 2018), 19\textendash 21. Further: Patrick Lally Michelson, \emph{"The First and Most Sacred Right": Religious Freedom and the Liberation of the Russian Nation, 1825\textendash 1905} (Ph.D.~dissertation, University of Wisconsin, Madison, 2007). See especially the first chapter, "In the Image and Likeness of God: The Patristic Tradition of Human Dignity and Freedom in Nineteenth-Century Russia." See also Michelson, "Russian Orthodox Thought in the Church's Clerical Academies," \emph{The Oxford Handbook of Russian Religious Thought}, 102\textendash 105.} Among those featured were the Cappadocian Fathers (Basil of Caesarea, Gregory of Nazianzus, and Gregory of Nyssa) and Maximus the Confessor, whose "similitude anthropology" was a profound theology of personhood that emphasized human agency and self-determination in the salvific process.\footnote{See Andrew Louth, \emph{Maximus the Confessor} (London: Routledge, 1996). According to Louth, "For Maximus, what is distinctive to being human is self\textemdash determination (\emph{autexousios kinesis}: movement that is within one's own power). \ldots{} This self-determination is not, however, absolute: human beings are created in God's image, and it is in their self-determination that they reflect God's image" (60).} Ivan Kireevsky contributed significantly to this Russian \emph{ressourcement}. He, his wife Natal'ia, the Optina spiritual elder (\emph{starets}) Makarii (Ivanov) (1788\textendash 1860), and the Archpriest F. A. Golubinskii (1797\textendash 1854), professor of philosophy at the Moscow Theological Academy, all collaborated in translating and publishing the works of Simeon the New Theologian, Maximus the Confessor, Isaac the Syrian, and other Eastern Church Fathers.\footnote{Patrick Lally Michelson, \emph{Beyond the Monastery Walls: The Ascetic Revolution in Russian Orthodox Thought, 1814\textendash 1914} (Madison, WI: University of Wisconsin Press, 2017), 54. See also Jeremy Pilch, \emph{'Breathing the Spirit with Both Lungs'}, 59\textendash 61.} Their publication program resulted in sixteen volumes by 1860.

One remarkable result of the Russian recovery of the writings of the Church Fathers was Vladimir Soloviev's concept of \emph{Bogochelovechestvo}, which combined the patristic sources with modern philosophical ones such as Kant and Schelling. His main theological sources were the patristic doctrine of theosis and Chalcedonian Christology, which confirms that the two natures of Christ, divine and human, are united in his person in perfect harmony, without "division or confusion"\textemdash Christ being the integral "Godman." Soloviev was "profoundly knowledgeable" of patristics, as has most recently been demonstrated by Jeremy Pilch.\footnote{Pilch, \emph{'Breathing the Spirit with Both Lungs'}, 56.} This knowledge informed his seminal work, \emph{Lectures on Divine Humanity} (completed in 1881),\footnote{Pilch, \emph{'Breathing the Spirit with Both Lungs'}, 57\textendash 111.} which deserves to be seen as a powerful theory of personalism.

Recall that according to Soloviev's tripartite conception of human nature, human beings combine in themselves three principles: the absolute or divine principle, the material principle, and (between them) the distinctively human principle (self-determination).\footnote{Vladimir Solovyov, \emph{Lectures on Divine Humanity}, trans. Peter Zouboff, revised and edited by Boris Jakim (Hudson, NY: Lindisfarne Press, 1995), 158. Subsequent page references cited parenthetically in the text.} Together the divine and human principles form "divine humanity"; together they are also the source of human personhood, dignity, and rights. (It is highly significant that Soloviev begins \emph{Lectures on Divine Humanity} with an astute critique of the modern effort to ground human dignity and rights in secular humanism.) As the next step in his personalist theory, he posits that every person is absolutely unique with an individual character that reflects "the idea that determines the essential significance of this being in everything" (52). Such ideas are not abstract products of thought but rather metaphysical entities or "foundations of being" (60). The bearer of an idea is a person. Person and idea are correlative as subject and object, and their inner unity is necessary to achieve real, full being (64). Furthermore, "all ideas are inwardly interconnected, equally partaking of the one all-embracing idea of absolute love," which contains everything else within itself as \emph{all-unity} (63). The person bearing the all-one idea of absolute love is God (64). As a result of the Fall, all-unity becomes a project, which is to be realized through divine humanity and "the deification (\emph{theosis}) of all that exists" (137).

Soloviev stipulated that divine humanity was to come about through the "kingdom of \linebreak ends"\textemdash Kant's ideal of a moral community of persons who respect each other as ends-in-them\-selves, whose highest end is nothing other than the kingdom of God, and who are self-governed by freely and inwardly accepted laws of virtue. Soloviev's social ideal of "free theocracy" is clearly modeled after the kingdom of ends, which is personalistic through-and-through.\footnote{See Randall A. Poole, "Kant and the Kingdom of Ends in Russian Religious Thought (Vladimir Solov'ev)," in \emph{Thinking Orthodox in Modern Russia: Culture, History, Context}, ed.~Patrick Lally Michelson and Judith Deutsch Kornblatt (Madison, WI: University of Wisconsin Press, 2014), 215\textendash 234.} Whether his metaphysics of all-unity preserved the personalism of its preceding stages is a matter of dispute.\footnote{See Zenkovsky, \textit{A History of Russian Philosophy}, vol. 2: 512.} The resolute personalistic development of all-unity belongs to his successors such as Sergei Bulgakov and Semyon Frank. In what follows I shall focus on Frank (1877\textendash 1950).

In Zenkovsky's estimation, Frank was the greatest Russian philosopher.\footnote{Zenkovsky, \emph{A History of Russian Philosophy}, vol.~2: 853, 872.} Frank himself gave that distinction to Soloviev, in an influential English collection of Soloviev's writings that he edited at the end of his life.\footnote{\emph{A Solovyov Anthology}, ed.~S. L. Frank, trans. Natalie Duddington (London: SCM Press, 1950). In his introduction to this volume Frank wrote that "Solovyov is unquestionably the greatest of Russian philosophers and systematic religious thinkers" (9).} The metaphysics of all-unity forms the overall framework of \emph{The Unknowable} (1938), widely regarded as his most important work.\footnote{S. L. Frank, \emph{The Unknowable: An Ontological Introduction to the Philosophy of Religion}, trans. Boris Jakim (Athens, OH: Ohio University Press, 1983). Page references cited parenthetically in the text ("U").} In it Frank wrote, "the total-unity of being is a kingdom of spirits" (136). The "kingdom of spirits" is a metaphysical concept. In contrast to the kingdom of ends (free theocracy) in Soloviev's system, it describes all-unity itself. With that deft move, Frank "personalized" his whole system. He himself characterized \emph{The Unknowable} as a work of "personalist ontology."\footnote{Philip Boobbyer, \emph{S. L. Frank: The Life and Work of a Russian Philosopher, 1877\textendash 1950} (Athens, OH: Ohio University Press, 1995), 166.} His last major work, \emph{Reality and Man: An Essay in the Metaphysics of Human Nature} (1956), is a more accessible statement, with some revisions and new material.\footnote{S. L. Frank, \emph{Reality and Man: An Essay in the Metaphysics of Human Nature}, trans. Natalie Duddington (New York: Taplinger Publishing Co., 1966). Page references cited parenthetically in the text ("RM").} In these works, Russian religious personalism arguably achieved its highest degree of development.\footnote{See Evgenii Zinkovskii, \emph{Poniatie lichnosti v antropologii Semena Franka v perspective klassicheskogo opredeleniia "persona est naturae rationalis individua substantia"} (Karaganda: Rimsko-Katolicheskaia Tserkov, "Credo," 2018). See also Peter Ehlen, \emph{Zur Ontologie und Anthropologie Simon L. Franks}, in Simon L. Frank, \emph{Die Realität und der Mensch: Ein Metaphysik des menschlichen Sein} (Freiburg and Munich: Verlag Karl Alber, 2004). There is a Russian translation: Peter Elen, \emph{Ontologiia i antropoligiia S. L. Franka}, trans. A. S. Tsygankova (Moscow: Institut filosofii RAN, 2017).}

Frank is perhaps not the most obvious Russian philosopher to consider with regard to the theme of the influence of Eastern Orthodox theology on the development of Russian personalism. Greek patristic references in his works certainly are no more common than other religious sources, including from traditions other than Christianity. The religious-philosophical tradition with which he most explicitly identified was Neoplatonism; he once called Nicholas of Cusa his only teacher in philosophy (U xi). But in the last two decades of his life he did make fairly frequent references to the patristic doctrine of deification, identifying it with \emph{Bogochelovechestvo}, which concept he embraced as his own. He may well have learned about it from his personal friend Myrrha Lot-Borodine (1882\textendash 1957), who wrote a classic account of it, \emph{La déification de l'homme selon la doctrine des Pères grecs} (1970), originally published in 1932\textendash 1933 as a series of articles.\footnote{Philip J. Swoboda, "\,'Spiritual Life' versus Life in Christ: S. L. Frank and the Patristic Doctrine of Deification," \emph{Russian Religious Thought}, ed.~Judith Deutsch Kornblatt and Richard F. Gustafson (Madison: University of Wisconsin Press, 1996), 242. On her studies of the patristic doctrine of deification and her close relations with Frank, see Teresa Obolevitch, \emph{Myrrha Lot-Borodine: The Woman Face of Orthodox Theology} (St.~Paul, MN: IOTA Publications, 2024), 167\textendash 187, 219\textendash 232.} But in the 1940s Frank left no doubt of his Christian universalism. Orthodoxy was one source among many of his religious philosophy.\footnote{Boobbyer, \emph{S. L. Frank}, 193\textendash 194.}

There is a striking similarity between Frank's personalism and Zizioulas's. Brandon Gallaher points to it in his "Postscriptum" to Fr.~Robert Slesinski's book, \emph{The Philosophy of Semyon Frank}.\footnote{Brandon Gallaher, "Postscriptum," in Robert F. Slesinski, \emph{The Philosophy of Semyon Frank: Human Meaning in the Godhead} (Fairfax, VA: Eastern Christian Publications, 2020), 220.} On the basis of the similarity, Gallaher writes that "Frank is a profoundly Orthodox thinker where Being as Communion, the idea of Godmanhood expressed concretely in the God-Man, Jesus Christ, and deification are at the heart of his philosophy of Being."\footnote{Gallaher, "Postscriptum," 225.} Of course Zizioulas could not have influenced Frank. On the other hand Zizioulas might very well have read Frank since he worked with Georges Florovsky and studied Russian religious thought (including Berdiaev) apart from émigré theologians such as Vladimir Lossky.\footnote{Specifically with regard to Berdiaev's personalism, he wrote: "Neither does the personalism of a Berdyaev, with his mysticism of a cosmic spiritualism and his gnosiocentric anthropology, bear any relationship to the concept of person I have put forward." See John Zizioulas,~\emph{The One and the Many: Studies on God, Man, the Church, and the World Today}, edited by Fr.~Gregory Edwards (Alhambra: Sebastian Press, 2010), 20\textendash 21. However, in his article, "Elder Sophrony's Teaching on the Person in Relationship to Contemporary Theological Currents," he reflects on Berdiaev in a more positive light. Here Zizioulas says: "An attempt to remove the meaning of the person from the narrowness of individualism is made by N. Berdiaev. He distinguishes the individual from the person, emphasizing that the individual is a~\emph{quantitative~}concept, subject to addition, composition, and use for higher purposes, while the person is a~\emph{qualitative~}notion, which cannot be the means to any end. This is an important step towards an Orthodox personalism. {[}...{]} It is the first step of personalism in the direction of love. However, even this thinker, profoundly influenced by German idealism, does not fully liberate the person from the dominion of cogitation." Zizioulas's article appeared in the following conference volume: ΓΕΡΟΝΤΑΣ ΣΩΦΡΟΝΙΟΣ, Ο ΘΕΟΛΟΓΟΣ ΤΟΥ ΑΚΤΙΣΤΟΥ ΦΩΤΟΣ ΠΡΑΚΤΙΚΑ ΔΙΟΡΘΟΔΟΞΟΥ ΕΠΙΣΤΗΜΟΝΙΚΟΥ ΣΥΝΕΔΡΙΟΥ (Athens, 19\textendash 21 October 2007). I am grateful to Dr.~Raul-Ovidiu Bodea for the information and references in this note and for his translation of the above passage (from a Romanian edition of the Greek volume). In 2024 Bodea completed his doctoral studies in theology at KU Leuven, Belgium, defending his thesis, \emph{Existentialism as a Methodological Paradigm for Orthodox Theology: Nikolai Berdyaev and John Zizioulas}.}

\subsection*{Personhood and Being: Frank's Personalist Ontology}

Perhaps the most obvious thing that could be said about Frank's philosophy is that it is based on his conviction that there is far more to reality than the external, natural world that confronts the senses. For him, the empirical, objective world, the world of facts, does not exhaust reality, and "conventional empiricism," as he calls it, is only one way of knowing reality (or rather part of it). He believed that there are other ways of knowing reality and that they disclose more of it, not just its surface layer but reality as whole, reality as such, in its (spiritual) depths. These ways come from inner experience. Frank's essential claim is that reality reveals itself in our inner experience \emph{as persons}, which is (or can be) reliable and truthful in its testimony. In \emph{Reality and Man}, he writes, "A philosophy adequate to the task of obtaining true knowledge of reality is therefore always based upon living inner experience, in some sense akin to the experience that is called 'mystical'\," (17). His approach can be traced to St.~Augustine, to whom Frank ascribes fundamental importance and whose words he commends, "Go not outward but inward; the truth is within man" (U 198). Reality is revealed in various types of inner experience, including the basic intuition of being ("immediate self-being" or "I am"), the experience of communion (the "I-thou" relation), aesthetic experience, moral experience, metaphysical experience, and religious experience.

In a more fundamental way, our inner experience is truthful to reality because human beings as persons share a basic ontological affinity with the spiritual core of reality. Persons are spiritual beings: according to Frank, human beings are persons because they are capable of "spiritual life," through which they seek to ground themselves in spiritual being (RM 173\textendash 74). Because persons co-belong to spiritual being and are permeated by it, their inner experience can express its truth. Personhood or personal being, Frank writes, "belongs to the inmost core of reality and must be recognized as its centre and primary source" (RM 104). This is what is meant by "personalist ontology": being consists in personhood and in communion among persons, and it reveals itself to persons.

Frank calls the reality that reveals itself in our inner experience "the unknowable." The term is not unproblematic, since his view is that inner experience does convey knowledge of reality. For example, the second chapter of \emph{Reality and Man} bears the title, "Reality and Knowledge of It." Only its external aspects can be known rationally, scientifically, theoretically, logically. In the external, phenomenal realm, being is objectified and therefore subject to objective knowledge. The deeper layers of reality are noumenal. (Frank does not explicitly use this Kantian distinction, but it expresses his thought perfectly well. He does use Rudolf Otto's term "numinous.") These layers evade rational thought but can be experienced from within. In this realm, Frank says, we are dealing with experiential consciousness, not cognitive or theoretical consciousness (U xx). He refers to the "metalogical" and "transrational" nature of being. Being is "concrete" and cannot be captured by the abstract concepts of rational thought. A similar contrast was made before him by many Russian religious philosophers, beginning with Kireevsky and Khomiakov. They criticized post-Kantian German idealism for reducing being to consciousness, and contrasted it to their own "concrete idealism," which they extolled for its distinctive "ontologism." Frank dislikes the term "idealism" because of its phenomenalistic and rationalistic associations, but his philosophy, in its reliance on inner experience and tracing its origins to Plotinus, St.~Augustine, and Nicholas of Cusa, is a type of ontological idealism. He called it "absolute realism" to emphasize the primacy of being over thought or consciousness (U 66).

There are other reasons why Frank thinks that reality is "unknowable," all relating to its nature as all-unity. What follows is one of his lucid descriptions of the concept: All-unity as a true unity is

\begin{quote}
the unity of unity \emph{and} diversity: a unity which not only embraces all its own parts and points but also inwardly permeates them in such a way that it is also contained as a \emph{whole} in each part and point. Thus, each point of being, though it has all else outside of itself, nonetheless in its place and in its way is the whole itself, total unity itself. Being something singular \emph{together} with all else, all existents are constituted by their \emph{separateness}. But having all in themselves and also being connected to all else, all existents have \emph{all} immanently in this double sense: in themselves and for themselves. All existents are permeated by all and permeate all (U 114).
\end{quote}

\noindent The \emph{unity of separateness and of mutual penetration} is an essential concept for Frank; it fundamentally informs his personalist ontology, as we will have several occasions to see. Defying rational or logical explanation, it is one good reason why Frank describes reality as "unknowable." Another is that reality is an absolutely indivisible whole, which means we cannot conceptually grasp it; its parts are analytically separable only at the expense of the whole. Reality is a "metalogical unity"; it is "something \emph{greater and other} than all we can know about it" (U 33).

Together with the wholeness of all-unity, Frank stresses that everything in it is singular and unique (individuality). What is singular and individual "forms the genuine essence of being in its concreteness," but is ungraspable in general concepts and "is thus a clear index of the unknowable in the essence of reality" (U 35, 36). Another such index is that reality is not finished. Frank describes being as potentiality, existent potency." Being as a whole is not frozen and static; \emph{it is not what it already is}. On the contrary, it is \emph{plastic}: it not only is, it is becoming; it is in the process of self-creation. It is growing, changing, being formed" (U 47). From the last sentence in particular it will come as no surprise that Frank refers to true reality as "life," in this respect (but not in others) following Tolstoy (RM 76 ff.). Finally, reality is unknowable because it, in the supreme form of absolute mind, the \emph{ens realissimum}, or the "primordial ground," is the very condition of truth: "\emph{We} cannot speak of ultimate truth and express it in our concepts, but this is only because ultimate truth \emph{itself} speaks, wordlessly, for itself and of itself, expresses and reveals itself. And we have neither the right nor the possibility to fully express through our thinking this self-revelation of ultimate truth. We must be silent before the magnificence of truth itself" (U 96).

The highest expression of individuality in all-unity is the person. Persons are the culmination of the mysterious and unknowable unity of separateness and of mutual penetration that characterizes all-unity as a whole. They "are a kind of \emph{miracle} that surpasses our understanding" because, through their free will, they are able to combine in themselves the transcendent and immanent that forms their spiritual center and being (U 177). Frank writes:

\begin{quote}
The mystery of the human person as an individuality \ldots{} consists precisely in the fact that what is \emph{universally valid} is expressed in the deepest \emph{singularity} that defines the essence of the person. This universal validity is the all-embracing infinitude of transcendent spiritual being \ldots{} so that precisely this uniqueness, this singularity, is the form that is permeated by the transcendent that is common to all people (U178).
\end{quote}

\noindent Indeed only persons, with their uniqueness and absolute singularity, can express the infinite and universal with maximal adequacy. "The essence of total unity as spirit, as the reality of intrinsically valuable and intrinsically valid being, acquires ultimate \emph{definiteness} only in concrete individuality, in contrast to the definiteness of objective being {[}the external world{]}, which is always abstract and general" (U 178).

Frank embraced the mystery of personhood and of being with his approach of "wise, knowing ignorance" (\emph{docta ignorantia}). This approach, which is similar to apophatic or negative theology, reveals the unity of separation and of undivided wholeness (or mutual penetration). Such unity is an "antinomian monodualism" in which two opposites (logically separate according to the law of mutual negation) are inwardly united and mutually permeating, the result being: "the one is \emph{not} the other but it also \emph{is} the other; and only with, in, and through the other is it what it genuinely is in its ultimate depth and fullness" (U 97). This antinomian monodualism, Frank writes, can be understood as a triadism or trinity (with the third element being the unity or synthesis of the two opposites). He adds: "This contains the most profound and general reason why human thought constantly arrives\textemdash in its most diverse religious and philosophical expressions\textemdash at the idea of trinity as the expression of the ultimate mystery of being" (U 98). He may well have had in mind the patristic concept of \emph{perichoresis}, used to describe the communion or interpenetration of the three persons of the Trinity and, on occasion, the relationship between the divine and human natures of Christ.

\subsection*{The Kingdom of Spirits}

For all Frank's strictures against the (rational) knowableness of reality, \emph{The Unknowable} and \emph{Reality and Man} remain works of fundamental ontology and philosophical anthropology, offering a positive, powerful vision of being and personhood\textemdash of the "kingdom of spirits." We know Frank believed that reality reveals itself in the inner experience of persons. His point of entry is self-consciousness, in which we first encounter reality as such, rather than in its external, closed form as objective being. We encounter inner being, self-revealing reality, which Frank calls "immediate self-being": being revealing itself to itself (and to us, who are co-participants) as being (U, ch.~5). We experience it in the recognition, "I am," through which we are also aware of being itself, in our simultaneous recognition, "it is" (or the world is). According to Frank, "\emph{Immediate self-being is the am-form (Bin-form) of being}" (U 108). It is the first condition of all-unity, since only self-transparent being can bring external material being (opaque to itself) into unity with itself (at least ideally).

Immediate self-being is not simply another term for our self-consciousness or even for the "transcendental unity of apperception" (though it is closer to that); it is rather the source of them. Frank defines it as one of various modal forms of being, "different means or degrees of the self-revelation and self-realization of being" (U 110). It is a mode of absolute reality (or simply the absolute), which is clearest in one of its aspects, immediate self-identical being. But it has another, opposing aspect, selfhood, in which it is not self-identical with the absolute. "Here we have our first concrete example of antinomian monodualism," Frank notes (U 110). Immediate self-being is a dual-unity of being as pure immediacy (self-identical with the absolute) and selfhood (at some remove from it). Interestingly, Frank says this dual-unity is most clearly expressed in the Upanishads, "in which Brahman, the absolute, is identified with Atman, the deepest ground of the soul" (U 111). Indeed this is a powerful (and ecumenical) image of his conception of personhood striving to ground itself in spiritual being. Until such grounding is achieved selfhood is prone to subjectivity (spurious self-grounding instead of self-transcending), as Frank also indicates.

In the course of his analysis of immediate self-being in its dual-unity as pure immediacy and selfhood, Frank introduces for the first time his concept of the kingdom of spirits. The kingdom is all-unity (pure immediacy of being), realized in a multitude of separate particular selves (U 114\textendash 115). Frank's designation of them as spirits indicates their trajectory from selfhood to fully realized personhood as they increasing transcend themselves, ground themselves in spiritual being, and ultimately achieve deification or union with God. He devotes much of \emph{The Unknowable} and \emph{Reality and Man} to this process. Immediate self-being, especially in its aspect of selfhood, is the pivotal point in the process because its nature is to transcend itself. It is "being in the form of becoming, potency, striving, and realization" (U 115). The self's capacity for transcendence is rooted in freedom. In \emph{The Unknowable} Frank writes of the "higher freedom which emanates from our selfhood" and which can direct it beyond itself, toward others and toward spiritual being (116). In \emph{Reality and Man} he calls this higher freedom "self-determination" (167). In both books, he shows that self-transcendence can be directed outward or inward. Outward transcending is oriented toward other selves in the form of the "I-thou" relation. Inward transcending is oriented toward spiritual being and beyond that to the "primordial ground" (holiness or divinity) (U 112\textendash 123). He deals most explicitly with the nature of personhood when he discusses inward transcending, but of course it is also relevant to the "I-thou" relation.

Frank writes eloquently about the mysterious process by which another being reveals himself to me as "thou," by which I recognize him as "thou," and by which I myself, in this encounter, first fully recognize myself as "I." For Frank this process of mutual recognition is miraculous. According to him, "the 'I-thou' relation is a 'communion,'\," in the sense of \emph{the mutual penetration of the separate} (U 142). It is genuine inner unity, a dual-unity, represented by the concept of "we," which Frank calls "a wholly special, miraculous mode or form of being" (U 149). It reveals the inner structure of reality as such, "the real, inner, existent-for-itself mutual interwovenness and mutual permeatedness of the 'one' and the 'other.'\," Indeed, "we can say that \emph{in the 'I-thou' relation genuine concrete total unity in its transrational, unknowable essence is revealed for the first time precisely as living being.}" This leads Frank to a striking conclusion: "\emph{Being is the kingdom of spirits}, and the kingdom of spirits consists precisely in the fact that \emph{the one} always exists \emph{for the other}, that, transcending itself, the one affirms itself only by abandoning itself for the other" (U 144). In \emph{Reality and Man}, Frank has a chapter section titled, "Reality in the Experience of Communion." There he writes: "In communion or actual apprehension of another person not through the cognitive gaze but through vital contact, we come into touch with the mysterious depths of living reality, no longer merely in our inner life, but outside us" (62). In these mysterious depths, we find the "inner mutual interpenetrability of the kingdom of spirits," in which every "I" and "thou" are in perfect communion in the all-unity of "we" (66\textendash 67). This vision of perfect communion was inspired by the concept of sobornost' and, most fundamentally, by the Christian Trinity.

Members of the kingdom of spirits also pursue self-transcendence inwardly, grounding themselves in spiritual being. The concept of the ground is important for Frank; according to him, subjectivity and potentiality are defining qualities of selfhood, which therefore requires a ground in the objectivity and actuality of spiritual being, and beyond that in the "primordial ground" (holiness, divinity, ultimately the absolute). We become increasingly aware of the ground in moral, metaphysical, and religious experience, which together form what Frank calls "spiritual life." Through free will (self-determination) we are capable of inward transcendence toward the ground. Like many Russian (ontological) idealists of his time, Frank thought that the ideals of spiritual life (truth, the good, beauty)\textemdash by which the will is self-determining\textemdash entailed a transcendent ground, in which intrinsic value and being are one. In particular he linked the ideal of truth, the very concept of truth, with the primordial ground of reality.

The capacity for self-transcendence is the basis of Frank's conception of personhood. This capacity enables man "to separate himself from himself" and to arrive at a "higher, spiritual I." Here is Frank's definition: "\emph{It is this higher, spiritual selfhood which constitutes what we call the person. The person is selfhood as it stands before higher, spiritual, objectively valid forces and is permeated by and represents these forces}" (U 174). From his overall account in \emph{The Unknowable} and \emph{Reality and Man}, it is clear that he defines personhood more generally as the process of inward self-transcendence, of self-determination by the ideals of spiritual life, and of grounding in spiritual being. In a succinct statement of it: "The 'person' is the mode of man's being in its necessary transcending inward, into the depths, into the deep layer of reality that surpasses man's being" (U 239). It is worth noting that in \emph{Reality and Man}, he simply uses Kant's formulation: Autonomy or self-determination constitutes man as a person (157). (True, he makes clear that autonomy is actually theonomy.) \emph{The Unknowable} includes a chapter section, "The Mystery of the Person." In it Frank writes that "man as a person is always and essentially something \emph{greater and other} than all we can perceive in him as a finished determination constituting his being. That is to say, he is a kind of infinitude, so that he has an inner bond to the infinitude of the \emph{spiritual kingdom}" (176).

\subsection*{"The Argument from Personhood."\\Godmanhood and Christian Humanism}

The last three chapters of \emph{The Unknowable} and three of the last four chapters of \emph{Reality and Man} are devoted to the reality of God and to his relation with man and the world, as revealed in our metaphysical and religious experience. Having established the reality of personhood, Frank now turns more directly to the grounds for the reality of God. In his last book he wrote, "The only completely adequate 'proof of the existence of God' is the existence of the human person taken in all its depth and significance as an entity that transcends itself" (RM 104). This type of "argument from personhood," which he also made in \emph{The Unknowable} (200), was utterly convincing to Frank: "If the human being is aware of himself as a person, i.e., as a being generically distinct from all external objective existence and transcending it in depth, primacy and significance, if he feels like an exile having no true home in this world\textemdash that means that he \emph{has} a home in another sphere of being," the sphere, that is, of ultimate reality. "The apprehension of the reality of God is, thus, immanently given in the apprehension of my own being as a person" (RM 104, 106).\footnote{Soloviev pioneered this type of "argument from personhood." In \emph{Lectures on Divine Humanity}, it took the specific form of a Kantian "argument from human perfectibility." He wrote that the human capacity for "infinite development" presupposes an ultimate end toward which it is directed, which he called the positive absolute of all-unity or perfect "fullness of being." Infinite human striving toward the absolute ideal convinced Soloviev of the ontological reality of the absolute. He formulated this in striking terms: "Thus, belief in oneself, belief in the human person, is at the same time belief in God." See Solovyov, \emph{Lectures on Divine Humanity}, 17, 23.}

On the basis of his argument from personhood, Frank embraced Soloviev's idea of Godmanhood, giving it much more attention in \emph{Reality and Man} than in \emph{The Unknowable}. It is not just that in ourselves as persons we recognize the reality of God, but that the divine is immanently present in us in some form (Soloviev speaks of a divine principle). Naturally Frank thinks that the inner unity of God and man (Godmanhood) can only be apprehended as the mutual penetration of separate elements, through antinomian monodualism (U 245, 257). The Chalcedonian dogma of the divine and human natures ("without division or confusion") properly applies, Frank acknowledges, to Christ, but, he asks, "does this imply that there can be \emph{no other form} of combining these two principles in the human person?" And his view is that "something 'divinely-human' is inherent in man's being as such" (RM 140). This is the theological foundation of Christian humanism, which, Frank notes, found its highest philosophical expression in Nicholas of Cusa (RM 124). \emph{Reality and Man} includes an account of Christian humanism and its historical fate, something new compared to \emph{The Unknowable}. We have seen that Frank admired St.~Augustine in certain respects, but he recognized that his emphasis on human depravity was at odds with Christian humanism. Ever seeking the coincidence of opposites, he suggests that the true approach to understanding divine grace and human freedom, and more generally the relation between God and man, can be found in a synthesis of Augustinianism and Pelagianism. He says that St.~Thomas Aquinas expressed such a synthesis in simple words: "We must pray as though everything depended upon God; we must act as though everything depended upon ourselves."\footnote{These words have also been attributed to St.~Ignatius Loyola.}

\subsection*{Frank and Human Rights}

In the introduction I noted that personalism, a new form of Christian humanism, has been rediscovered because of its role in the intellectual history of human rights in the twentieth century.\footnote{See also Ana Siljak, "A New Christian Humanism: Nikolai Berdyaev and Jacques Maritain," in Bernard Hubert, \emph{An Exceptional Dialogue, 1925\textendash 1948: Nikolai Berdyaev and Jacques Maritain}, edited in English and with an introduction by Ana Siljak, trans. C. Jon Delogu (Montreal and Kingston: McGill-Queen's University Press, 2025), 3\textendash 36.} How does Frank fit into this history? He was committed to human rights from his early years as a Russian liberal. This commitment owed much to his close collaboration and friendship with Peter Struve, whom he joined in leading the Russian Liberation Movement that would culminate in the Revolution of 1905. In January 1906 Frank prepared a draft "Constituent Law on the Eternal and Inalienable Rights of Russian Citizens," a type of declaration of rights for the then hoped-for constitutional monarchy.\footnote{Gennadii Aliaiev, "The Concept of Democracy in Simon Frank's Philosophy of Liberal Conservatism," \emph{Studia z Historii Filozofii}, vol.~14, no. 3 (2023): 97\textendash 119, here 103.} The experience of the Bolshevik Revolution and Russian Civil War confirmed his skepticism of any theory of human rights based on a materialistic rather than spiritual conception of human nature. He was convinced that a secular humanistic grounding of human rights was untenable; the foundation had to be "religious humanism," as he put it as early as 1909, in his famous essay in \emph{Vekhi}.\footnote{S. L. Frank, "The Ethic of Nihilism: A Characterization of the Russian Intelligentsia's Moral Outlook," \emph{Vekhi\slash Landmarks: A Collection of Articles about the Russian Intelligentsia}, ed.~and trans. Marshall S. Shatz and Judith E. Zimmerman (Armonk, NY: M. E. Sharpe, 1994), 155.}

Frank's first major work of social philosophy, \emph{The Spiritual Foundations of Society}, was published in 1930, just before the interwar elaboration of personalism and its theory of human rights by Berdiaev and Maritain. In this work his approach to human rights is reserved, because he associates them with the modern democratic idea of "the sovereignty of individual and collective human will."\footnote{S. L. Frank, \emph{The Spiritual Foundations of Society: An Introduction to Social Philosophy}, trans. Boris Jakim (Athens, OH: Ohio University Press, 1987), 127. Subsequent pages references cited parenthetically in the text.} But true human life, he counters, consists not in the self-assertion of the will, but rather in a principle he calls \emph{service}: service to truth and to the realization of God's will (126\textendash 127). As a spiritual being, man "realizes his freedom, his self-determination, only insofar as he serves the higher, Divine \emph{truth}" (128). Human rights must be grounded not in "the will of people" but in the one "innate" right to serve the divine truth and God's will (129). "The supreme principle of \emph{service} determines the entire structure of rights and obligations that make up the social order," Frank writes (129). He was highly critical of the spiritually ungrounded "liberal" conception of human rights (130). For him, human rights are not absolute in themselves; their value comes from service to the Absolute. At the same time, he did recognize freedom of conscience and religious belief as "a kind of genuinely primordial right," since it is the condition of the recognition of absolute values and of authentic service to them (138).

Ten years later, in 1940, as Maritain was realizing that personalism provided a powerful theory of human rights, Frank began working on a new book, published later in 1949 as \emph{The Light Shineth in Darkness}.\footnote{S. L. Frank, \emph{The Light Shineth in Darkness: An Essay in Christian Ethics and Social Philosophy}, trans. Boris Jakim (Athens, OH: Ohio University Press, 1989). Pages references cited parenthetically in the text.} In it he offers a more positive assessment of human rights than in its 1930 predecessor. First he diagnoses the "crisis of humanism" in a way that is very similar to Berdiaev's classic account in \emph{The New Middles Ages}: modern humanism is anti-religious and thus also anti-human, since without a spiritual element there is no distinctive concept of the human; what is left is "bestialism," as Nietzsche makes clear (21\textendash 28, 26).\footnote{Berdiaev defines the theme of modern history in the following words: "It is an unfolding of ideas and events wherein we see Humanism destroying itself by its own dialectic, for the putting up of man without God and against God, the denial of the divine image and likeness in himself, lead to his own negation and destruction," See Berdyaev, \emph{The End of Our Time}, 29 (italics removed).} The "good news" is that Christianity revealed Godmanhood (Frank's Solovievian framework is obvious), the divine-human ground of human existence, and we moderns can recover that ground (64). Christianity was, for Frank, the greatest spiritual revolution in history because with it human beings discovered their personhood and their dignity, or at least acquired a "wholly new consciousness" of them that surpassed earlier presentiments (65). Though he does not use the term here, he contends that "personalism" came with Christianity and consists in the recognition that human beings are persons, invested with absolute dignity, because they are rooted in divine-human being. Next he makes the same connection among Godmanhood, theosis, and human rights that Soloviev made before him: "All the 'eternal rights of man' that were proclaimed later originate from the 'powers' granted by Christ to people, from the 'power to become sons of God' (John 1:12)" (65\textendash 66). This power, the power of human self-determination according to the image and likeness of God, is the true essence of humanism and it is more than a claim or "right" of man. "It is the holy obligation of man to defend his dignity, to remain true to his high origin" (66).

In \emph{The Light Shineth in Darkness}, Frank's position on human rights has evolved to such an extent that they (together with natural law) now constitute his criteria for a rational and just social order. He defines justice according to the principle of \emph{suum cuique} (to each his own), which entails that society should guarantee everyone's "natural rights," i.e., "those needs and claims of the individual which emanate from his nature as the creatural bearer of God's image" (172). Such rights are both individual and social: they protect individual freedom but at the same time must be compatible with the existence of society, since the latter is the very condition of human progress or the development of the higher intellectual and spiritual potential of persons. The task is to establish "maximal equilibrium and harmony" between the individual and society, between freedom and solidarity. If Frank gave more weight to social solidarity than does "liberal individualism," it was because he recognized (again explicitly following Soloviev) that society is necessary for whatever degree of human progress is possible given human sinfulness.\footnote{At the end of this chapter section (for its title see the next footnote) of his book, Frank writes: "Vladimir Solovyov says that the task of the state can never be to establish heaven on earth; it has another task, not less essential: \emph{to prevent the appearance of hell on earth}" (180). Frank does not provide a citation, but he is clearly paraphrasing \emph{The Justification of the Good}: "The purpose of legal justice is not to transform the world which lies in evil into the Kingdom of God, but only to prevent it from changing \emph{too soon} into hell" (324).} He used the term "Christian realism" to describe his mature social philosophy and made clear that it was firmly oriented toward his personalist ontology of the "kingdom of spirits" (173).\footnote{The title of section 5 (pp.~171\textendash 181) of the book's fifth chapter is: "The General Character and Fundamental Content of 'Natural Law.' The Meaning of Christian Realism." See also Philip Boobbyer, "A Russian Version of Christian Realism: Spiritual Wisdom and Politics in the Thought of S. L. Frank (1877\textendash 1950)," \emph{The International History Review}, vol.~38, no. 1 (2016): 45\textendash 65.}

{\centering *** \par}

\noindent Personalism is one of the great philosophical movements of the past two centuries. If persons are not reducible to or explicable by naturalistic processes, if their distinctive capabilities refute naturalism, then personhood must be woven into the deepest layers of reality. The theistic metaphysical implications of personhood were central, as we have seen, to Russian and to Russo-French personalism. While Frank is not as closely identified with Russo-French personalism as Berdiaev and Maritain, his philosophy of "absolute realism" is quite similar to Maritain's Thomistic realist personalism. I do not know if the two philosophers ever met,\footnote{In a March 1938 letter to Maritain, Berdiaev makes clear that they had spoken of Frank but in a way that suggests Frank and Maritain had not met. The letter indicates that Frank was hoping Maritain might write a letter of recommendation to support his application for a French research stipend; Berdiaev was conveying the request to Maritain. See Hubert, \emph{An Exceptional Dialogue, 1925\textendash 1948: Nikolai Berdyaev and Jacques Maritain}, 209\textendash 210. In January 1940 Maritain left France for North America, which means that if he and Frank met, presumably it was in the period between March 1938 and January 1940. In \emph{Reality and Man} (193), Frank refers to Maritain's \emph{Réflexions sur l'intelligence et sur sa view propre} (1924).} but there is a beautiful passage in Maritain's~\emph{The Rights of Man and Natural Law}~that captures their shared spiritual wisdom: "The worth of the person, his liberty, his rights arise from the order of naturally sacred things which bear upon them the imprint of the Father of Being and which have in him the goal of their movement. A person possesses absolute dignity because he is in direct relationship with the Absolute, in which alone he can find his complete fulfillment." In short, "the person is a spiritual whole made for the Absolute."\footnote{Jacques Maritain, \emph{The Rights of Man and Natural Law}, in Maritain, \emph{Christianity and Democracy. The Rights of Man and Natural Law}, trans. D. C. Anson, intro. D. A. Gallagher (San Francisco, CA: Ignatius Press, 2011), 67, 112.}

\begin{center}
  \includegraphics[width=0.75cm]{articlend.png}
\end{center}

\biobox{\textbf{Randall A. Poole} is Professor of Intellectual History at the College of St.~Scholastica, a senior fellow of the Center for the Study of Law and Religion at Emory University School of Law, and co-director of the Northwestern University Research Initiative in Russian Philosophy, Literature, and Religious Thought. He is the translator and editor of \emph{Problems of Idealism: Essays in Russian Social Philosophy} (2003) and co-editor of five other volumes: \emph{A History of Russian Philosophy, 1830\textendash 1930: Faith, Reason, and} \emph{the Defense of Human Dignity} (2010, 2013), \emph{Religious Freedom in Modern Russia} (2018), \emph{The} \emph{Oxford Handbook of Russian Religious Thought} (2020), \emph{Evgenii Trubetskoi: Icon and Philosophy} (2021), and \emph{Law and the Christian Tradition in Modern Russia} (2022). He is also the author of many articles and book chapters on Russian intellectual history, philosophy, and religion.}

\label{sec:poole1}

\setcounter{footnote}{0}

% First article
\fancypagestyle{chaptercontentpage}{
  \fancyhf{} % Clear all header and footer fields
  \fancyhead[CE]{%
    \fontsize{11}{11}\leftmarkfont%
    \addfontfeature{LetterSpace=10.0}%
    \textit{\MakeUppercase{\leftmark}}%
  }
  \fancyhead[CO]{\authorheadfont\addfontfeature{LetterSpace=10.0}\fontsize{11}{11}\selectfont\textbf{{\uppercase{Caryl Emerson}}}}

  \fancyfoot[RE]{\thepage}
  \fancyfoot[LO]{\thepage}
  \renewcommand{\headrulewidth}{0pt} % No header rule on content pages
}

\fancypagestyle{chaptertitlepage}{
  \fancyhf{} % Clear all header and footer fields
  \fancyhead[L]{\begin{minipage}[t]{0.7\textwidth}\publisher\end{minipage}}
  \fancyhead[R]{\begin{minipage}[t]{\textwidth}\raggedleft \datefont\fontsize{10}{11}\selectfont Volume 1 (2024): \thepage\textendash\pageref{sec:emerson}  \\ \doi{10.71521/7c5g-3903} \end{minipage}}
  \renewcommand{\headrulewidth}{0pt} % No header rule on title pages
  \fancyfoot[RE]{\thepage}
  \fancyfoot[LO]{\thepage}
}
\newpage

\chaptertitle{Afterword}{}{Caryl Emerson}

\addcontentsline{toc}{chapter}{Afterword\\\emph{by} Caryl Emerson}
\setcounter{footnote}{0}
\seriffont
\fontsize{12}{18}\selectfont

\noindent It is seemly to end the journal's inaugural issue with an essay devoted to Russian personalism, peaking on the philosopher Semyon Frank (1877\textendash 1950). He did not have the Romantic dazzle and visionary flair of Nikolai Berdyaev, nor Sergii Bulgakov's theological precision laced with lyricism and pathos. Multi-ethnic, inclusive, ecumenical, averse to millenarianism and utopia, Frank was a transnational thinker. As long as faith was the ground, he preferred intermediate or middle spaces filled with "both-and" rather than exclusive dogmatic binaries.\footnote{See Philip Boobbyer, "Semyon Frank," Chapter 29 of \emph{The Oxford Handbook of Russian Religious Thought}, eds.~Caryl Emerson, George Pattison, and Randall A. Poole (Oxford: Oxford University Press, 2020): 496\textendash 509, here 498.} The bigger and more unknowable these mediating spaces, the more mysterious will be the energy that connects us and the more imperative the presence of an Absolute. And also, he concluded, the more crucially individualized all "I-Thou" interactions on this site will become.

Frank's mature personalism was attentive above all to the concrete encounter. Early in his career he had praised Nietzsche for distinguishing between "love for one's neighbor" (the reflex of kinship) and a more abstract or altruistic "love of the distant," which bypassed persons in favor of a love of "things and phantoms."\footnote{See S. L. Frank, "Friedrich Nietzsche and the Ethics of 'Love of the Distant'\," {[}1902{]} in \emph{Problems of Idealism. Essays in Russian Social Philosophy}, trans., ed.~and introduced by Randall A. Poole (New Haven: Yale University Press, 2003): 198\textendash 241; and Semyon Frank, "The Ethic of Nihilism" {[}1909{]}, in \emph{Vekhi / Landmarks}, trans. and ed.~by Marshall S. Shatz and Judith E. Zimmerman (Armonk, NY: M. E. Sharpe, 1994): 131\textendash 155.} This latter love had every right and reason to exist; it prods us toward the ideal. But Frank came to acknowledge, by the time of his intricate dissection of the I-Thou relationship in his 1938 masterwork \emph{The Unknowable}, the dangerously simplifying temptation of such distant vision.\footnote{See S. L. Frank, \emph{The Unknowable. An Ontological Introduction to the Philosophy of Religion} {[}1938{]}, trans. Boris Jakim (New York: Angelico Press, 1982/2020), chapter 6, "Outward Transcending: the 'I-Thou' Relation," 124\textendash 155.} Because compassion happens only in the present, not in the past or future, a fully-realized I-Thou or "love for one's neighbor" is differentiated, time-consuming, and difficult. Frank's insight here recalls Mikhail Bakhtin's early comments on the ethically binding force of our singularity or uniqueness {[}\emph{edinstvennost'}{]}; the 'I', Bakhtin insists, has no "alibi in Being," no exit out of answering for who, at any moment, I am. Over a decade ago, Mikhail Epstein, one of the distinguished contributors to this issue, developed both these Bakhtinian ideas into what he calls the "diamond rule."\footnote{"Differential ethics: from the golden rule to the diamond rule," Ch. 15 in Mikhail Epstein, \emph{The Transformative Humanities, A Manifesto,} trans. and ed.~by Igor Klyukanov (New York/London: Bloomsbury, 2012): 217\textendash 224.} Unlike the Golden Rule or the categorical imperative, which presume similarity (do unto others as you would have them do unto you\textemdash and this deed is both reciprocal and repeatable), the multi-faceted Diamond Rule is predicated on an ethics of radical particularity. Cleanse yourself of the fantasy that others are merely a mirror of you; as light falls on the subject, each facet creates its own depths. Even though the other is unknowable in the large, however, we can discipline ourselves to access needy parts of others (and expose needy parts of our own erring and incomplete selves) in ways that are more apophatic than duplicative. As Christian humanists, Bakhtin (and Semyon Frank too) would probably have signed on to the Diamond Rule, which Epstein summarizes as: "do unto others what the other needs done and what only you can do right now, from your own time and place."

Diamond-rule optics is a stunningly attractive exemplar of personalism, but not one that lends itself easily to a politics. Political thinking in the modern state tends to depersonalize. It aggregates, organizes parts externally, enters into combat, strives toward social justice and a legal definition of rights. It can be reconciled with determinism and positivism. The radical personalist would insist that a concrete encounter, to be worthy of the eye-to-eye relation, must begin elsewhere, with one's own concrete act of inner spiritual healing. Any reality worth the name begins there for the human subject. Or as Berdyaev put the matter in 1934: "There can be no worse aberration than to identify the \emph{object} with reality. To know and to objectify or to abstract are currently regarded as synonyms. But the very opposite is true: effective knowledge involves familiarity."\footnote{Nicolas Berdyaev, \emph{Solitude and Society}, trans. by George Reavey (London: Geoffrey Bles: The Centenary Press, 1936), 51, emphasis added. An accurate (and more informative) literal translation of the title of Berdyaev's 1934 Russian original is: \emph{The 'I' in the World of Objects. An Essay on the Philosophy of Aloneness} {[}одиночество{]} \emph{and Communion / Communication} {[}общение{]}.}

The options seemed to be: the objectified distancing of politics versus the familiarity of persons. Russian émigré circles in the 1930s and '40s mercilessly debated these models and the tactics that each required.\footnote{For an even-handed survey of these debates (with Struve and Ilyin championing the "political" option and Berdiaev, Frank and Bulgakov arguing for the innerly redemptive), see Stuart Finkel, "Nikolai Berdiaev and the Philosophical Tasks of the Emigration," Ch. 17 in G. M. Hamburg and Randall A. Poole, \emph{A History of Russian Philosophy, 183\textendash 930. Faith, Reason, and the Defense of Human Dignity} (Cambridge: Cambridge University Press, 2010): 346\textendash 362, especially 356\textendash 361.} What made the debates so excruciating is that these exiled Russian idealists, aristocrats and egalitarians alike, were against \emph{all} reigning ideological systems\textemdash communism, fascism, free-market capitalism with its bourgeois complacency and devotion to material profit\textemdash and yet they found nihilism, as a politics, abhorrent. As Ana Siljak argues in her editor's Introduction to the English translation of the Berdyaev-Maritain correspondence, for these philosophers an economically unmonitored, spiritually unmoored individualism would invariably drive people to "succumb to a tyranny of one sort or another."\footnote{Ana Siljak, "A New Christian Humanism: Nikolai Berdyaev and Jacques Maritain," introductory essay to Bernard Hubert, \emph{Nikolai Berdiaev and Jacques Maritain: An Exceptional Dialogue (1925\textendash 1948}), edited by Ana Siljak, trans. Christopher Jon Delogu (Montreal: McGill Queen's University Press, 2024), here 22.} The dignity of human beings lies not with their individuality but with their personality {[}\emph{lichnost'}{]}, which is relational and must be grounded in an Absolute. In the interwar period, however, no serious poet or philosopher had the luxury to ignore politics.\footnote{For an eloquent discussion of key intellectual and literary players in French-Anglophone circles that complement the Russian diaspora, see Alan Jacobs, \emph{The Year of our Lord 1943. Christian Humanism in an Age of Crisis} (Oxford: Oxford University Press, 2018).}

This maiden issue of the journal is abundantly graced with Christian humanisms from the Parisian Orthodox diaspora. Rowan Williams considers Sergii Bulgakov's quest for a "soul" in socialism and how the Church should respond. Bradley Underwood takes on Berdyaev and Bulgakov as analysts of the metaphysical Underground, that generative site of evil and playground of Nothings. Daniel Adam Lightsey links Vladimir Nabokov, supreme aesthete, with Bulgakov's sophiological hymn to the creative artist. But theologians do not define the agenda. Four contributors discuss Dostoevsky's novels, and the other three interrogate -isms constructed outside the religious realm (Darwinism, Empiricism, Marxism). The Dostoevsky essays, diverse as they are, each challenge us to rethink a received wisdom. Working with the cast of \emph{The Brothers Karamazov,} Gary Saul Morson asks us to reconsider what it means to have\textemdash or to adopt\textemdash a belief. The novel's narrator assures us that realists are not unnerved by genuine miracles. But why is Alyosha's faith tested by the awfulness of a "reverse miracle," by his Elder's unnaturally rapid bodily decay? (Morson's answer: faith must be freely given, a choice; certainty is the province of the Grand Inquisitor.) Amy Singleton Adams, countering Dostoevsky's reputation for extremity and urban scandal, celebrates his moments of smallness, tenderness, the theophany of contemplative landscapes and icons. Likewise within an Orthodox perspective but with darker implication, the Dostoevsky of Denis Zhernokleyev insists on the impossibility of an autonomous ethics that is graspable by human experience, whether in the bosom of nature or in the nineteenth-century novel. And Peter Winsky cautions us not to dismiss as mere caricature or satire the figure of Father Ferapont, earnest hesychast fanatic and sworn foe of the Elder Zosima\textemdash for among Dostoevsky's goals is to make "\emph{finding the good} more difficult for his hero and readers."

The more secular entries and -isms continue this mission of making the good harder to find. Since\textemdash depending on the hermeneutics of the thinker\textemdash "the good" can mean both the objectively true and the morally defensible, each essay has a fascinating seam where Russian apologists for raw, mechanistic matter come up against the transcendent. Jillian Pignataro is concerned to set right the organicist Strakhov's critique of Darwin: the enemy wasn't Darwin (whom Strakhov respected) as much as Social Darwinism, and in Strakhov's view an "internal teleology" was compatible with natural selection. Julia Berest takes on another vigorous import into Russia of the 1860s, the empiricism and utilitarianism of John Stuart Mill, showing how Mill's recondite \emph{System of Logic,} while alienating Orthodox conservatives, galvanized Russian academic philosophy (in this process, Strakhov's organic holism and Chicherin's synthesism play a predominant role). Finally, Daniela Steila gives us a rich, loosened-up and humanized picture of "critical" and "alternative" Marxisms at the century's turn. Is the human subject energized or imprisoned by a consciousness of necessity? Can the Marxist critique, which continued to inspire those who rejected its political expression in Bolshevik policy, be scientific without being fatalistic and blind to persons in the present? If Mikhail Epstein ends his essay on Russian-Jewish identities with the cautious hope that each of those overlapped peoples could now return to a more modest existence, then Steila ends hers on a larger anxiety, the modernist conceit that the "human subject is the mighty conqueror of nature and the ultimate ruler of the universe." The hubris of such a position is as inherent in Marxism as it is in the Book of Genesis or in Berdyaev's numerous attempts at an anthropodicy. Ultimately, what brings the secular and non-secular essays in this issue together, perhaps unexpectedly, is the contested legacy of the European Enlightenment. Reassessing this legacy became an obsession among twentieth-century Russian religious philosophers, and their case has been handsomely continued by American literary humanists with a profound interest in theology, such as Duke University's Thomas Pfau.\footnote{See, for example, Thomas Pfau, "The Failure of Charity and the Loss of Personhood: Beyond the Enlightenment Impasse," \emph{Tradition and Discovery: The Journal of the Polanyi Society}, vol.~43.2 (July 2017): 4\textendash 20, and at more length in his \emph{Minding the Modern} (University of Notre Dame Press, 2016).}

In closing, a few words about the origins of Northwestern's Research Initiative RPLRT\textemdash "ripple art," as one of our research scholars, Michael Ossorgin, dubbed it. With its well-curated forum (of essays, interviews, posts), international conferences, and now annual journal, this acronym is gaining some traction in the battered world of Russian Studies. The Initiative is appropriately situated in a university whose Press hosts the strongest Slavic book series in the country (SLRT: Studies in Russian Literature and Theory) and whose Slavic Department houses the professor who for decades has taught the biggest in-person classes on Dostoevsky and Tolstoy in the northern hemisphere (Gary Saul Morson). The embryonic stage of this research community was a zoomed reading group spearheaded in July 2021 by Susan McReynolds, Chair of Northwestern's Slavic Department. Several of the contributors to this first issue of the journal\textemdash graduate students, junior scholars and senior academic mentors\textemdash were pulled in at that time. Susan, Bradley Underwood (an ordained Baptist minister now in the Slavic PhD program at Northwestern), and Paul Contino (of Pepperdine University) resolved on a topic: the human person. According to the collective memory of this original inner circle, Rowan Williams\textemdash already targeted as a highly desirable participant for the coalescing group\textemdash had mentioned two names as indispensable for understanding contemporary personalism: Robert Spaemann and the Eastern Orthodox theologian Christos Yannaros. We began with Spaemann's 1996 book, \emph{Persons: The Difference between 'Someone' and 'Something.'}

Am I treating you as someone, or as a thing? As a neighbor or as a phantom? The topic proved apt for the first hard-lockdown year of the pandemic, which saw each of us reified, isolated, but also infinitely more intimately available. As talking squares on a screen, more of us could interact visibly as persons than would ever have been possible to manage geographically or corporeally. From Spaemann (a German Catholic) the group moved on, or better moved outward and back in search of historical ground, to the Eastern Orthodox theologian Christos Yannaras and then to classic thinkers in the Russian religious tradition (Bulgakov, Solovyov, Dostoevsky, Pavel Florensky, with detours into Levinas), all the while gathering members from around the globe. When the world made its "transition to in-person"\textemdash a shocking phrase that no one would have dreamed of before 2020\textemdash McReynolds gave the group a more stable institutional identity by founding the NU Research Initiative in Russian Philosophy and Religious Thought (Literature was tucked in later). She invited Randall Poole, an intellectual historian at the College of St.~Scholastica, to be its co-director, and together they welcomed Brad (an indefatigable facilitator) as associate director. The abomination of Putin's war against Ukraine and the co-option of the Moscow-based Russian Orthodox hierarchy into this re-imperializing mission made the task of the Initiative both trickier, and more necessary. But consider the comment by Rowan Williams in his essay for this issue. Socialism\textemdash and every group ideology\textemdash has both an anthropology and a soul. Sergii Bulgakov (along with his fellow Russians Berdyaev and Frank, who also began their careers in Marxist economic thought and who also transcended it) insisted on beginning any authentic human economy with the movements of the soul. Define that as you like, but where you'll end up is never with mere things, phantoms, or dead matter. You'll end up with a person in need of an Other. This is probably as close to a compact mission statement as the Initiative will ever come.

\vspace{2em}
\begin{center}
  \includegraphics[width=0.75cm]{articlend.png}
\end{center}

\biobox{\textbf{Caryl Emerson} is A. Watson Armour III University Professor Emerita of Slavic Languages and Literatures at Princeton University.  Her scholarship has focused on the Russian classics
(Pushkin, Tolstoy, Dostoevsky), Mikhail Bakhtin, and Russian music, opera and theater.  Recent
projects include the French Neo-Thomist philosopher Jacques Maritain and the interwar
Russian diaspora (philosophers and creative artists), the Russian modernist prose writer
Sigizmund Krzhizhanovsky (1887\textendash 1950), the allegorical-historical novelist Vladimir Sharov (1952-
2018), and the co-editing, with George Pattison and Randall A. Poole, of \emph{The Oxford Handbook
of Russian Religious Thought} (2020).}

\label{sec:emerson}

\fancypagestyle{chaptercontentpage}{
  \fancyhf{} % Clear all header and footer fields
  \fancyhead[CE]{%
    \fontsize{11}{11}\leftmarkfont%
    \addfontfeature{LetterSpace=10.0}%
    \textit{\MakeUppercase{\leftmark}}%
  }
  \fancyhead[CO]{\authorheadfont\addfontfeature{LetterSpace=10.0}\fontsize{11}{11}\selectfont\textbf{{\uppercase{Bradley Underwood}}}}

  \fancyfoot[RE]{\thepage}
  \fancyfoot[LO]{\thepage}
  \renewcommand{\headrulewidth}{0pt} % No header rule on content pages
}

% First article
\fancypagestyle{chaptercontentpage}{
  \fancyhf{} % Clear all header and footer fields
  \fancyhead[CE]{%
      \fontsize{11}{11}\leftmarkfont%
      \addfontfeature{LetterSpace=10.0}%
      \textit{\MakeUppercase{\leftmark}}%
    }
  \fancyhead[CO]{\IBMPlexSansSemiBold\textbf{Jillian Pignataro}} % Centered header on odd pages with bold author name
  \renewcommand{\headrulewidth}{0pt} % No header rule on content pages
  \fancyfoot[RE]{\thepage}
  \fancyfoot[LO]{\thepage}
}

% Set line spacing for serif font
 % Adjust the value as needed

\part{Reviews}

\fancypagestyle{chaptercontentpage}{
  \fancyhf{} % Clear all header and footer fields
  \fancyhead[CE]{%
    \fontsize{11}{11}\leftmarkfont%
    \addfontfeature{LetterSpace=10.0}%
    \textit{\MakeUppercase{\leftmark}}%
  }
  \fancyhead[CO]{\authorheadfont\addfontfeature{LetterSpace=10.0}\fontsize{11}{11}\selectfont\textbf{{\uppercase{Octavian Gabor}}}}

  \fancyfoot[RE]{\thepage}
  \fancyfoot[LO]{\thepage}
  \renewcommand{\headrulewidth}{0pt} % No header rule on content pages
}

\fancypagestyle{chaptertitlepage}{
  \fancyhf{} % Clear all header and footer fields
  \fancyhead[L]{\begin{minipage}[t]{0.7\textwidth}\publisher\end{minipage}}
  \fancyhead[R]{\begin{minipage}[t]{\textwidth}\raggedleft \datefont\fontsize{10}{11}\selectfont Volume 1 (2024): \thepage\textendash\pageref{sec:gabor} \\ \doi{10.71521/vpfz-ax37} \end{minipage}}
  \renewcommand{\headrulewidth}{0pt} % No header rule on title pages
  \fancyfoot[RE]{\thepage}
  \fancyfoot[LO]{\thepage}
}
\section{Gabor - Review of Coates}
\setcounter{footnote}{0}
\chaptertitle{Becoming Like God}{The Russian Ideal of Deification at the Beginning of the Twentieth Century}{Octavian Gabor}
\addcontentsline{toc}{chapter}{Becoming Like God:\\The Russian Ideal of Deification at the Beginning of the Twentieth Century\\\emph{by} Octavian Gabor}

\seriffont
\fontsize{12}{18}\selectfont

\noindent Ruth Coates, \emph{Deification in Russian Religious Thought: Between the Revolutions, 1905\textendash 1917} (Oxford: Oxford University Press, 2019), 232 pp.

\vspace{1em}

\noindent One approach in describing the history of humanity is to trace the various and intricate ways in which humans have approached the theme of deification. Questions about the special connection between humanity and divinity are raised even prior to Christian thought. At the beginning, the approach had an epistemic flavor, focusing on the \emph{source}, in the sense of cause (the Greek \emph{aitia}) of knowledge. In Parmenides, for example, the youth searching for knowledge is led by the goddess Dike, who says in fragment B2, "Come now, and I will tell you, and you, hearing, preserve the story,/ the only routes of inquiry there are for thinking."\footnote{Patricia Curd's translation, quoted in Patricia Curd, \emph{The Legacy of Parmenides: Eleatic Monism and Later Presocratic Thought} (Las Vegas: Parmenides Publishing, 2004), 24.} Parmenides seems to suggest that the ultimate source for knowledge is the goddess. Humans can also have this knowledge if they pursue divinity and accept the logos that the goddess imparts. If indeed this theme has always been in the background of the problem of understanding who we are, there is no surprise that Russian religious thinkers focused on it during a time of turmoil: the time between the revolutions of 1905 and 1917. This is the period under scrutiny in Ruth Coates' \emph{Deification in Russian Religious Thought}.

Coates believes that the main purpose of the Russian writers during this period of time was to respond to the question about overcoming death: "how to transform death into everlasting life."\footnote{Ruth Coates, \emph{Deification in Russian Religious Thought: Between the Revolutions, 1905\textendash 1917} (Oxford: Oxford University Press, 2019), 1.} In some sense, the interest of intellectuals in religious problems, or their appeal to religion in solving intellectual problems, led to the Russian religious renaissance, beginning around 1900 and continuing to 1922.

Coates recalls Karkkainen's claim that Eastern theology doesn't focus so much on guilt as on mortality as the main problem of humanity.\footnote{Ibid., 27.} I would add that even this mortality needs qualification: in the East, mortality is the manifestation of people's separation from God, and it is expressed in sickness. The cosmos, as we experience it, is for the Eastern Orthodox ethos corrupted, as a sick body that needs recovery. The only recovery that is genuinely available for this is deification. Anything else, regardless of what that may be, is still an experience of a diseased reality. The four Russian writers Coates studies (Dmitry Merezhkovsky, Nikolai Berdiaev, Sergei Bulgakov, and Pavel Florensky) are embedded in the cultural and theological atmosphere created by this model. It is a perspective that avoids the juridical approach in Western Christianity, where Christ's sacrifice is a payment for sins. The doctrine of deification recalibrates this payment in different terms: Christ takes on human nature so that humans can also, as much as this is possible, take on divine nature.

Coates points out that the first formal definition of deification was given by Dionysius the Areopagite, in his \emph{Ecclesiastical Hierarchy}: "the attaining of likeness to God and union with him so far as is possible."\footnote{Cf. ibid., 24.} Certainly, the understanding of human salvation as deification did not begin with the Areopagite. Two centuries before him, Athanasius of Alexandria wrote the famous dictum, "the Logos became man so that man can become God."\footnote{See c.~54 of St.~Athanasius, \emph{On the Incarnation} (Yonkers, NY: St.~Vladimir's Seminary Press, 2011), 167.}

Coates points out that the idea of deification pre-dates Christianity. Of course, this does not mean that deification, as understood by Christian thought, is what the Greeks and the Romans thought of when they approached the notion of a deified emperor. In fact, one may say that what the Romans described as an emperor-god was the antithesis of the god-man who participates in divine nature through the grace of God, and never due to his own power. Still, Coates is correct in suggesting that the notion did not appear out of nothing. One may even consider the Aristotelian account of divine life in the \emph{Nicomachean Ethics}, where he says that humans are called, \emph{as far as this is possible} to live the life of the divine. This expression, "as far as this is possible," is parallel to Dionysius' definition of deification mentioned above, even if the Greek words are different.

Coates proposes that there is an implicit claim about human nature within the context of deification. She writes that the doctrine presupposes a "dynamic anthropology."\footnote{Coates, \emph{Deification}, 31.} The claim seems right at first sight, but it still depends on how we may define this notion. In her account, "dynamic anthropology entails an understanding of human nature as fluid and undetermined, and of the human will as essentially free, either to reject the approach of God or to accept it and cooperate with God in realizing His purpose for the person who is approached, and through that person, for the world."\footnote{Ibid., 31.} This dynamicity may be better understood, I believe, if we apply the Aristotelian model of potentiality/actuality. Human nature is complete; but its completeness presupposes the possibility of further becoming like God, not through its own power, but of being acted upon by a different nature, the divine one. Otherwise, if we mean by dynamic a nature that needs something else to be completed as human, we open the door to differences in quality between human beings. Instead, Orthodox Christian thought emphasizes the intrinsic value of each human being because of their belonging to this human nature, which has the ability of receiving and partaking of divinity.

Coates organizes her volume in six chapters. She begins with a study of deification in Patristic thought. Starting from Dionysius the Areopagite's definition mentioned above, "Deification is the attaining of likeness to God and union with him so far as is possible," the author develops the notion of deification working primarily from Norman Russell's \emph{The Doctrine of Deification in the Greek Patristic Tradition}.\footnote{Norman Russell, \emph{The Doctrine of Deification in the Greek Patristic Tradition} (Oxford: Oxford University Press, 2004).}

The second chapter engages the nineteenth century, which created the basis from which the idea of deification flourished in the Russian renaissance. The chapter is divided into two parts. The first describes the monastic culture of nineteenth-century Russia, with a focus on spiritual eldership and hesychasm. The second delves into what may be the author whose work is most centered on the question of what a human being par excellence is, Fyodor Dostoevsky. I will refer more to Dostoevsky while engaging the other chapters.

In the third chapter, Coates analyzes \emph{Tsar and Revolution}, a volume published by the Mer\-ezh\-kov\-skys, as she calls them, the group formed by Dmitry Merezhkovsky, his wife Zinaida Gippius, and their friend Dmitry Filosof. The volume approaches the theme of deification from a political standpoint. Taken together, Coates says, "the essays represent a powerful and informed treatment of the political dimension of the deification theme, the age-old apotheosis of the emperor, and its significance for Russian cultural, political, and social story."\footnote{Coates, \emph{Deification}, 84.} \emph{Tsar and Revolution}'s thesis has two parts, according to Coates. The first is that the alliance of state and church is an experiment in theocracy. The second claims that this experiment is false, primarily because it replaces the man-God Christ with the tsar-god, and the kingdom of heaven with a kingdom of the earth. The solution of the Merezhkovskys was to have a revolution that leads to the absence of any temporal power. The autarchy in Russia stemmed from the combination of two powers in the hands of one man, the Tsar, which took place with Peter the Great. When he became the head of the Russian Orthodox Church, being at the same time a temporal and a spiritual leader, he replicated, according to the Merezhkovskys, the dual nature of Christ, human-divine. Of course, this is one of the ways in which the notion of deification can be corrupted in human thought. One of the best portrayals of this corruption appears in Dostoevsky's \emph{Crime and Punishment}, where Raskolnikov's attempt to become a super-human, or a Napoleon, as the text has it, is parallel to the Tsar's approach to leading both the church and the state. Such status places him above humanity, for it gives him the power to make or lose life.

Coates points out that there are "two distinct conceptions of royal power" co-existing in Russia: "a religious conception of the tsar as the image of Christ as ruler and high priest, as Christ's deputy on earth; and a secular conception of the emperor as 'containing within himself all power and the source of all power,' a conception that evokes the pagan perception of Caesar as deified man, or earthly god." Of course, Merezhkovskys reveal both of them as "blasphemous distortion of Christian truth."\footnote{Ibid., 96.}

The writings in \emph{Tsar and Revolution} show a dichotomy between the true Christ, the two natures living in one person, and the imitation of Christ, which results in false god-men. Anyone who proclaims himself as the anointed one, the Christ, the god-man par excellence, is a false Christ. It is in this context that Michael Cherniavsky's proposal arrives: "the myth of the sacred ruler was counterposed by an equally significant and culture-shaping myth of land and people: the myth of 'Holy Russia.'\,"\footnote{Ibid., 107.}

Chapter 4 moves away from the connection between politics and deification to a more spiritualized account. Coates discusses one of Berdiaev's works, \emph{Meaning of Creativity}. She believes that Berdiaev's view is influenced by the seventeenth-century German theosophist Jakob Boehme. Thus, the first part of the chapter is dedicated to Boehme's work, to show that Berdiaev's discussion of deification is done in Boehme's terms. While it is true that Boehme and Berdiaev "insist on a divine element in humans that for both is connected to their special status as children of God who share with him the quality of eternity,"\footnote{Ibid., 120.} Berdiaev's work is more connected with patristic intellectual tradition.

In my mind, this chapter reveals the difficulties that the notion of deification brings forward. For example, one may ask this question: is there something internal to a human being that makes him able to be divine? In Berdiaev, according to Coates, the indwelling of Christ is the agent of deification, but it remains to be seen whether Christ is an agent in the sense that he accomplishes deification or in the sense that he "empowers us to express our natural divinity."\footnote{Ibid., 124.} Perhaps the very precise cutting between these two options is problematic, for it begins in separation. The fact that Christ and the human being cooperate is clear. It is also clear that the source for deification is divinity. The problem remains in whether this divinity was manifested or not prior to Christ's Incarnation. If it was prior, then the Athanasius' formula is no longer correct. If it was posterior to the incarnation, then we return to the classical understanding, that deification became possible only due to Christ's becoming human.

Coates notices that Berdiaev is mistaken when he suggests that there is a lack of anthropology in the writings of the church. He is not wrong, however, that Marxism replaced Christian anthropology with a corrupted anthropology, in which the new god is the proletariat. One may even say that the notion of deification is so embedded into human thinking that we must always refer to a false god whenever we no longer have connection with the real one.

In Chapter 5, Coates introduces Sergei Bulgakov in his own words, as a Christian materialist with the ambition "to translate {[}the teaching of the Fathers{]} into the language of contemporary philosophical thought."\footnote{Ibid., 143.} While \emph{Philosophy of Economy} may be the work in which Bulgakov introduces himself as such, it is still surprising that Coates focuses on it to explain deification. Perhaps this is due to the fact that Bulgakov opposed Marxism, and the development of the early twentieth-century Russian philosophy out of and against Marxism is a focus of Coates' volume.

Bulgakov, Coates notices, begins his volume with death. Coates associates this with the idea that "Orthodox spirituality is suffused with the pathos of death."\footnote{Ibid., 152.} Deification itself is an answer to death, Coates proposes. But economic activity is as well, since humans work primarily to struggle against death, to avoid it by providing resources for life.

Of course, there may be an equivocation in the use of the notion of death, and it can be debated whether this equivocation results from Coates's or Bulgakov's writings. For Orthodoxy does struggle against death, but the writings of the fathers, primarily those concerned with deification, bring forward a death of the soul. Economic struggle, on the other hand, is against physical death.

Coates shows that \emph{Philosophy of Economy} is not primarily a book about deification, but one that uses the deification metaphor to explain other aspects of political and social life. I wonder, however, how much the metaphor helps in explaining processes that belong to this plan of existence. In deification, incarnation is essential: the movement from this plan of existence toward the spiritual one cannot be accomplished in the absence of incarnation. Any account of deification that does not include the first movement, that of the immortal taking on mortality, as in Athanasius' formulation, corrupts deification, even to the point of changing humans into different kind of beings. After all, Dostoevsky's approach to this in, for example, \emph{Crime and Punishment} demonstrates what happens when man wants to become a super-human based on his own capacities.

This is also connected to how the notion of death is interpreted. Thus, even if all economic activity is directed toward the overcoming of death, no outcome of economic activity will ensure deification, not even immortality, since immortality is not by nature a property of humans. Instead, part of the idea of deification is precisely the fact that the one who is created and mortal by nature takes on attributes of the uncreated and immortal by nature. If humans can achieve the overcoming of death through economic activities, this can only mean that the incarnation itself is no longer needed.

Coates analyzes Bulgakov's claim that human activity is the means by which God brings the material cosmos into the divine life. She believes that his approach derives from his reliance on Schelling's philosophy, and this is actually the reason why the notion of deification at work is a corrupted one. Coates rightly points out that Bulgakov's focus on economics downplays the significance of incarnation, because if indeed humans are the one who, through their work, reconcile the created order with God, then there is no need for an incarnate savior. If Bulgakov's philosophy is understood in the description Coates provides, then she is perfectly right that the Orthodoxy at work in his writings is flawed. I wonder, though, whether this is due to evaluating a work of economics from the perspective of theology. If Bulgakov's purpose in his \emph{Philosophy of Economy} is to explain the human drives in participating in work, then he is no longer called to show that the source of deification is grace: he needs to focus on economic principles only. Similarly, one may find different accounts of Aristotle metaphysical principles depending on whether these principles are spelled out in Aristotle's metaphysics or, let's say, in his works of moral philosophy. Different fields require different explanations and different focuses.

In fact, in \emph{Unfading Light}, published in 1917, Bulgakov writes unequivocally about humans' lack of power in bringing about their own immortality. He writes,

\begin{quote}
Therefore with their own powers, no matter how great they might be, human beings cannot pull themselves out of the gulf of sin and render their nature healthy, but are doomed all the more to be stuck in the swamp of sin, drowning in the clutches of greedy nothing. It is a shortsighted error to think that simply in virtue of "evolution"\textemdash of time and "progress"\textemdash the good will be strengthened in humanity at the expense of evil, and thus humanity becomes all the more perfect by force of things, as if automatically. In reality only evil is accumulated in that way, while good is realized in the world only by free spiritual struggle. Therefore the divinization of humanity can by no means be achieved through the path of evolution.\footnote{Sergius Bulgakov, \emph{Unfading Light}, trans. Thomas Allan Smith (Grand Rapids: William B. Eerdmans Publishing Company, 2012), 350\textendash 351.}
\end{quote}

\noindent In the same book, Bulgakov emphasizes that economic approaches refer only to this plan of existence: "Everything economic in its coarse or fine sense is utilitarian; it pursues a practical goal that is limited by the interests of terrestrial being. All economic tasks, no matter how broad they might be, belong to the surface of \emph{this} world, the current eon."\footnote{Ibid., 365.}

With chapter six, we return to deification proper: the theology of Pavel Florensky, especially as it appears in \emph{The Pillar and Ground of the Truth}. If Bulgakov's view seemed to suggest that it is possible for humans to defy immortality through their own work, Florensky's book, in Coates' perspective, comes very close "to judging his contemporaries' religious aspirations as a demonic exercise in 'self-deification.'\,"\footnote{Coates, \emph{Deification}, 205.} Indeed, Ruth Coates' main thesis is that Florensky understands deification in the context of the tradition of Christian mysticism, based on the Holy Fathers of the Church, and I take this to be a true assessment. But this raises the question: is any other concept of deification appropriate? Can we, for example, list among the concepts of deification that which stems from Bulgakov's philosophy of economy? In my mind, the answer is negative. This is not in the sense that we cannot use the term "deification" to describe the idea that humans may have access to something that does not belong to them by nature through their own work. Of course, a horse from a painting is a horse, but only in name, as Aristotle would say. Similarly, an eye that is not connected to a living body is an eye only in name, but it is not properly an eye because it does not do what an eye does. Corrupted notions of deification\textemdash primarily based in Marxist definitions\textemdash are also deification only in name, but they are not genuinely so because they do not do what deification does. Their action does not lead to humans' taking on another nature by grace, through the work of the Holy Spirit. Instead, the action of making oneself immortal though your own means is the opposite of deification, even if, on the surface, remains the same. This is because in deification, as it is understood in Orthodox theology, the agent of deification is essential.

This may suggest that writers who use deification differently than the Holy Fathers have a wrong understanding of it. While this may certainly be the case, this conclusion does not follow necessarily. It may be, as I think in Bulgakov's case, that these writers use metaphors based on deification to explain aspects of human existence that are in agreement with it. Human cooperation with God toward immortality is indeed understood within the context of deification, but this does not mean that their actions alone are deification. Thus, if we were to refer to economic activity as a means to obtain immortality, emphasizing that such work has the same purpose as deification, this is not intrinsically unorthodox unless it excludes the divine work that is a precondition of deification.

I think Fyodor Dostoevsky's work emphasizes precisely this aspect. His departure from Marxism\textemdash just like the departure of the majority of the writers about whom Coates writes\textemdash can be described, after all, as an acknowledgement of the brokenness of the deification account and a return to deification proper. This does not mean, though, that remnants of Marxist ideology are not to appear in their approaches, and thus Bulgakov's writings may be such a case.

Returning to Florensky, in his writings we recover the traditional orthodox approaches in which truth is a person. Thus, knowledge of truth becomes a question of partaking of the divine person in us, and so deification starts in the work of accepting the incarnation of the second person of the Trinity, and a knowledge of that Person by partaking of his person in love. Thus, the epigraph of \emph{Pillar and Ground}, as Coates notes, is the phrase of Gregory of Nyssa: "knowledge becomes love."

Coates' conclusion subsumes these approaches under what can be called, "human, all too human." She writes, "The millenarian hopes of the Merezhkovskys, Berdiaev, and Florensky were not realized. The new order swept in by the Revolution proved to be of the human, all-too human kind. Christ did not come to reign on earth; the age of the Holy Spirit did not dawn."\footnote{Ibid., 208.}

\vspace{2em}
\begin{center}
  \includegraphics[width=0.75cm]{articlend.png}
\end{center}

\biobox{\textbf{Octavian Gabor}~is Professor of Philosophy at Methodist College. He works on Greek philosophy, Dostoevsky, and the notion of personhood. Recent journal articles and book chapters include: "Dostoevsky in Romanian Culture: At the Crossroads between East and West" (2024), "Responses to Divine Communication: Oedipus and Socrates" (2020), "Taming the Beast: Constantin Noica and Doing Philosophy in Critical Political Contexts" (2019), "Justice between Mercy and Revenge in Sophocles'~\emph{Antigone}~and Plato's~\emph{Crito}" (2019), and "Constantin Noica's Becoming Within Being and Meno's Paradox" (2017). He has translated from French to Romanian and Romanian to English. His most recent translations are Andre Scrima's~\emph{Apophatic Anthropology}~(2016) and Constantin Noica's~\emph{The Romanian Sentiment of Being}~(2022, with Elena Gabor) and~\emph{Pray for Brother Alexander}~(2018).}

\label{sec:gabor}

\fancypagestyle{chaptercontentpage}{
  \fancyhf{} % Clear all header and footer fields
  \fancyhead[CE]{%
    \fontsize{11}{11}\leftmarkfont%
    \addfontfeature{LetterSpace=10.0}%
    \textit{\MakeUppercase{Russian Liberalism in Theory and Practice}}%
  }
  \fancyhead[CO]{\authorheadfont\addfontfeature{LetterSpace=10.0}\fontsize{11}{11}\selectfont\textbf{{\uppercase{Randall A. Poole}}}}

  \fancyfoot[RE]{\thepage}
  \fancyfoot[LO]{\thepage}
  \renewcommand{\headrulewidth}{0pt} % No header rule on content pages
}

\fancypagestyle{chaptertitlepage}{
  \fancyhf{} % Clear all header and footer fields
  \fancyhead[L]{\begin{minipage}[t]{0.7\textwidth}\publisher\end{minipage}}
  \fancyhead[R]{\begin{minipage}[t]{\textwidth}\raggedleft \datefont\fontsize{10}{11}\selectfont Volume 1 (2024): \thepage\textendash\pageref{sec:poole2} \\ \doi{10.71521/tdqg-ns71} \end{minipage}}
  \renewcommand{\headrulewidth}{0pt} % No header rule on title pages
  \fancyfoot[RE]{\thepage}
  \fancyfoot[LO]{\thepage}
}

\section{Poole Review}

\chaptertitle{Russian Liberalism \\ in Theory and Practice}{}{Randall A. Poole}

\addcontentsline{toc}{chapter}{Russian Liberalism in Theory and Practice\\\emph{by} Randall A. Poole}

\noindent Vanessa Rampton, \emph{Liberal Ideas in Tsarist Russia: From Catherine the Great to the Russian Revolution} (Cambridge: Cambridge University Press, 2020), ix, 229 pp.

\vspace{1em}

\noindent Paul Robinson, \emph{Russian Liberalism} (Ithaca, NY: Northern Illinois University Press, an imprint of Cornell University Press, 2023), x, 289 pp.
\vspace{1em}

\noindent The history of liberalism is always a critically important topic. But this is true especially now, when liberal values and institutions are in retreat in countries (such as the United States) where they once seemed relatively secure, and when prospects for liberal development in countries such as Russia and China seem as remote as ever. These two books focus on Russia, but both take a comparative approach that emphasizes the normative or universal claims of liberalism as political theory. What are these claims? What makes liberalism normative, and not just one ideology among others?

The essential value that constitutes liberalism and distinguishes it from its competitors is the human person and his or her liberty. The supreme liberal value is human dignity, the intrinsic worth of every person. If we accept this proposition, then there are no higher values (such as class, nation, society, or humanity) to which individual persons could be rightly sacrificed (though of course they can choose to put the good of others over themselves). The good of society consists in the good of every person in it. In Kant's famous formulation, the person is an end-in-itself, never merely a means for other ends.

Human dignity is the source of human equality. If the person has intrinsic and infinite worth, then all persons are equal in value, though obviously human beings are not equal in their individual strengths and weaknesses, in their life circumstances, in their conduct, and in a myriad of other ways. (The sources of human dignity are a matter of philosophical controversy, but one viable general source is the human potential to do good.) Human dignity is also widely regarded as the source of natural or human rights, guaranteed and enforced by the rule of law. The state and higher international institutions, as the instruments of the rule of law, ought to embody it and their officials ought to subordinate themselves to it. In the end the rule of law rests upon civil society: citizens who have a keen consciousness (ultimately a moral one) of human dignity and rights and who engage in various forms of community and civic activity to defend them, from voting and governmental participation to (in cases of state violation of legal norms) civil disobedience and collective coercive conduct.

In sum, liberalism can be defined as a normative political philosophy of human dignity, equality, and rights, upheld through the rule of law and civil society (at local, national, and global levels). "Liberal individualism" is something of a shibboleth that distorts authentic liberalism, since the latter recognizes that persons can develop and realize their potential\textemdash that human well-being, progress, and flourishing are possible\textemdash only in community and society. But if there is broad consensus about the fundamental liberal values, there is much less agreement about how societies and economies should be organized to best serve human dignity and equality and about how best to promote human flourishing. Still, at the most general level, it can be said that liberalism is about essential human values and the best ways to build societies worthy of them. Thus understood, the importance of liberalism as a political and social philosophy is obvious.

{\centering *** \par}

\noindent So, too, is the specific topic of Russian liberalism, for at least three reasons. First, liberalism is an important part of Russian history, especially its intellectual history. Since Ivan the Terrible, autocracy (the conceptual opposite of the rule of law) has been the dominant feature and structure of Russia's political history. The country's oppressive political reality meant that Russian liberals had to defend their ideals and values, their hopes and dreams, with even more emphatically persuasive force and clarity. They succeeded brilliantly: The intellectual history of Russian liberalism is very rich. Thwarted in practice, Russian liberalism developed theoretically (the only way it could) and reached high levels of philosophical sophistication. Some of the best liberal theorists globally (e.g., Boris Chicherin and Pavel Novgorodtsev) are Russian. We can learn a lot from them.

Second, the political failure of Russian liberalism is a human tragedy. For centuries Russians have deepened our common humanity with their thought and culture, and for centuries they have suffered mightily at the hands of their illiberal governments (tsarist, Soviet, post-Soviet). And not only Russians have suffered, as the current war in Ukraine makes clear. This leads to the third reason why Russian liberalism matters. Russia is a nuclear power. An illiberal Russia will always pose a grave danger to international security. (Paul Robinson is a military historian and a security expert.) None of this is to deny that societies that have prided themselves on being liberal have also often miserably failed to live up to their own standards, at home and abroad, with great human costs.

\newpage
{\centering *** \par}

\noindent By the very nature of the topic, both of these books are largely intellectual histories of Russian liberalism. This is explicitly so in the case of Vanessa Rampton's study. Russian liberalism had two types of philosophical defenders: positivists and idealists (or neo-idealists). Her book is a study of their ideas, the historical contexts in which they worked, and their attempt to put liberal ideas into practice at the time of the 1905 revolution and the ensuing "Duma Monarchy." Russia's most famous liberal, the historian Pavel Miliukov (1859\textendash 1943), was a positivist, and he is one of the book's subjects. Another is the sociologist Maksim Kovalevskii (1851\textendash 1916). Miliukov has been extensively studied but Kovalevskii has not, and this book sheds welcome new light on him. But the neo-idealists are at the center of Rampton's attention. She regards them as the most consistently liberal, not only in their defense of liberalism's core values, but also in their recognition that while human dignity must always remain an inviolable principle, liberalism involves compromise, accommodation to complex historical realities that often resist ideals, and appreciation of the inevitable tension among competing values, especially between "negative" and "positive" liberty. This type of open-ended, "empirical," pluralistic liberalism was famously championed by Isaiah Berlin, whose perspective Rampton adopts. Yet Berlin always took a skeptical approach to Russia's neo-idealists, let alone to its religious philosophers, while she believes they were truer to his pluralistic liberalism than were the positivists. That's interesting.

Her book is a compact, inviting, and accessible study. It is synthetic and interpretive, based both on existing scholarship and primary sources, with the balance tipping more toward synthesis than original research. There is large and complex literature on Russian liberalism (not to mention liberalism more generally), the philosophical aspects of which are challenging, so interpretive synthesis is appropriate and valuable. (In the interests of fair disclosure I should note that Rampton values my own work, draws upon it, and presents it well.)

The introduction establishes the book's comparative perspective by presenting the main concepts and figures in western liberalism, with astute indications of their relevance to Russia. The discussion focuses on selfhood, freedom, liberal practices, and the tension between freedom and justice. Referring to John Stuart Mill, Alexis de Tocqueville, and Benjamin Constant, Rampton writes: "These theorists are of particular interest to us here because they eschewed abstract notions of liberalism, and articulated views of the relationship between the individual and society that justified an approach to freedom as a permanent recalibration between different values and goods, dependent on the particularities of time and place" (24). She notes that these liberals were widely read in Russia.

Chapter 1 is an expert, succinct survey of Russian intellectual history from the Enlightenment to 1900, focusing on how Russian thinkers (properly liberal or not) understood the key liberal ideas of human dignity, equality, freedom, rights, law, and human progress. Rampton identifies Aleksandr Radishchev's early importance in the intellectual history of Russian liberalism, noting that he (1749\textendash 1802) "openly questioned the legitimacy of autocracy in the name of inalienable individual rights and humanitarian values" (43). The chapter features nineteenth-century Russia's two greatest philosophers, Boris Chicherin (1828\textendash 1904) and Vla\-di\-mir So\-lo\-vi\-ev (1853\textendash 1900), both metaphysical idealists. Rampton appreciates that Russian socialism (Alexander Herzen, Peter Lavrov, Nikolai Mikhailovskii), in its defense of individual moral autonomy against supposed necessary laws of historical development, contributed to the distinctive ethical thrust of Russian liberalism.

Three of the book's six chapters emphasize neo-idealist liberalism: chapters 2, 4, and 5. Chapter 2 is the most general and deals with the "revolt against positivism" in various spheres of culture and thought in fin-de-siècle Europe and Russia. Neo-idealism was an important element in this critique. According to Rampton, "The fundamental premise of idealism is that the mind and its ideals are not merely epiphenomena of the brain. Ideals have their own causal power, which for philosophical idealism indicates that there is more to reality than the physical world" (68). The chapter then turns in detail to the Russian neo-idealist defense of liberalism, with particular attention to its institutional center, the Moscow Psychological Society, and to the large collective volume \emph{Problems of Idealism} (1902).

Chapter 4 takes up another famous volume, \emph{Vekhi} (\emph{Landmarks}) (1909). Although the liberalism of its contributors was inconsistent, Rampton argues that as a whole they understood (and tried to convince their readers) that liberalism depended on the complex interplay of several factors: a free and dynamic spiritual life (and through it the inner recognition of human dignity), culture and education, civic and legal consciousness, civil society, and civic, legal, and political institutions. Chapter 5 is devoted to two major neo-idealist liberals: Bogdan Kistiakovskii (1868\textendash 1920 and Pavel Novgorodtsev (1866\textendash 1924). Kistiakovskii was a Russian neo-Kantian (one type of neo-idealist) who believed that "lawful socialism" (i.e., socialism that respects the liberal principle of the rule of law) was the best way to realize ideals of positive liberty such as self-realization and the right to a dignified existence. Novgorodtsev was arguably the most important Russian liberal theorist after Chicherin. He championed the revival of natural law. While Kistiakovskii resisted drawing metaphysical or theistic conclusions from idealism, Novgorodtsev embraced them.

Other parts of the book deal with positivist liberalism: specifically the last section of chapter 2 and chapter 6 (Miliukov and Kovalevskii). Rampton's analysis presents the neo-idealists as both better philosophers and better liberals. The positivists tended to pin their hopes for a liberal Russia on their (non-empirical) belief in progress as a necessary historical law and on their environmentalist approach to human nature, which conceived "the transformation of the individual personality as the by-product of institutional and social change" (81). Chapter 3 is the most historical in the book. It concerns Russia's main liberal party, the Kadets ("Kadet" was short for "Constitutional Democrat"), in the run-up to and aftermath of the 1905 Revolution. The chapter is informed by the author's sympathetic understanding of the dilemmas faced by a liberal party in a thoroughly illiberal polity.

Rampton is most interested in how Russian liberals understood their values (which, in the absence of any reigning examples from their native land, they took to be universal human ones) in particular historical circumstances. She attends to their attempts to apply these values to improve their оwn society (especially in the period 1900\textendash 1914), and to how their efforts altered their (and our) understanding of liberalism. She deems the neo-idealists to have been the better liberals both in theory and practice, not only because they understood the permanent tension between liberal ideals and historical realities, but because they were more likely to embrace this tension as dynamic, creative, and indeed truly progressive. The positivists, by contrast, were more likely to resist it, preferring to collapse it under one or another scheme of historical inevitability or necessity. The neo-idealists were not utopians. They conceived progress as a moral task to be accomplished by human beings driven by ideals and working to realize them \emph{as much as possible} in specific historical circumstances. The positivists, by contrast, generally conceived progress as a historical law somehow unfolding of its own accord and leading inevitably to a perfect human society.

In the conclusion to her book, Vanessa Rampton returns to the comparative, even global perspective with which she began: How can liberalism, with its universal claims to cherishing human dignity, defending human rights, and promoting human flourishing, at the same time be specific and relevant to local human communities and cultures? Her book leaves little doubt that Russian liberals, especially the neo-idealists among them, thought deeply about that question and can offer even our own bewildered age some valuable approaches to it.

{\centering *** \par}

\noindent The main question raised in Paul Robinson's book is the inevitable one, "Why has liberalism failed to take root in Russia?" He does not take a deterministic view of the failure of Russian liberalism: there were periods (Catherine II and Alexander I, the Great Reforms, the "Duma Monarchy," perestroika) when Russian liberalism might have become, if not the dominant principle, at least an important political factor in the country's development. Explaining why that never happened is the author's task, and he carries it out well. The short answer is that despite the long and rich history of liberal ideas among Russian intellectuals, Russian autocracy has always stifled the development of a strong civil society\textemdash the ultimate social condition of liberalism. The short answer is made long by the 260 years from the beginning of Catherine II's reign to today.

While there are many studies of Russian liberalism, Robinson's is the only one which examines Russian liberalism as a whole from its origins in the late eighteenth century through the Soviet era to post-soviet Russia, including also the inter-war Russian emigration. That alone commends the book. Most of the existing literature on Russian liberalism focuses on the tsarist period and on the Russian Revolution and Civil War. Chapters 2\textendash 6 of the book cover this long period. These chapters are based mainly on existing scholarship and offer a good introduction. Chapters 7\textendash 11 cover Russian liberalism in emigration, in the Soviet period, during perestroika, under Yeltsin, and under Putin. These five chapters are the best in the book and the most valuable, since much less work has been done on Russian liberalism in the period from 1922 to 2022. Here Robinson draws extensively and effectively on sources published in Russian since about 1990. His account of Russian liberalism over the past century is an important contribution to the literature.

Chapter 1 takes up some of the main general problems of Russian liberalism, including its origins, periodization, and national peculiarities\textemdash what was specifically Russian about it? Robinson begins the history of Russian liberalism with the reign of Catherine II (1762\textendash 1796), though many historians have dated its origins much later, from the mid-nineteenth century or even further in the future. Robinson's approach is defensible, based mainly on the intellectual origins of Russian liberalism during the late Enlightenment. He introduces a theme that runs throughout the book: whether liberalism in Russia is mainly a Western import or artifice or whether there is an authentically Russian liberalism, a Russian national tradition of liberalism. This is an important issue that transcends Russia: Can universal values (liberal values such as human rights) take national forms and be strengthened in the process? Opponents of liberalism (in Russia and elsewhere) attack it as "western" or "foreign." This has become a basic and crude tactic of Putinism ("liberalism is gay").

The book's subsequent chapters follow the same structure. They begin with a basic historical overview of the period in question and then consider the development of Russian liberalism across three categories: cultural liberalism, political liberalism, and socio-economic liberalism. This tripartite structure has its organizational virtues, but it doesn't work equally well in each of the chapters and the distinctions among the three types can at times seem artificial. Logically, cultural liberalism deals with liberal intellectuals (cultural elites) and their ideas; political liberalism with the realization or implementation of these ideas, with its pragmatics and with liberal political movements and parties; and socio-economic liberalism with the social-economic conditions and results of liberalism. But there is some overlap among the three categories, and at times they all operate at the level of ideas: cultural ideas, political ideas, and socio-economic ideas.

Across the chapters, "cultural liberalism" presents the leading Russian liberal thinkers and their ideas, from Alexander Radishchev to Andrei Sakharov (1921\textendash 1989) and Sergei Kovalyov (1930\textendash 2021), together with variations of the two main philosophical theories of Russian liberalism (positivism and idealism). Robinson mentions (even if briefly) virtually every significant Russian liberal thinker and gives attention to some of their major writings. Under the category of political liberalism, the book spans Catherine II's \emph{Instruction} to the Imperial Legislative Commission, Alexander I's constitutional plans, the Great Reforms, the history of the liberation movement that led to the Revolution of 1905, the Duma Monarchy, the history of the Kadet party from 1905 to the emigration, the human rights movement in the Soviet period, perestroika, "shock therapy" under Yeltsin, and the extinguishing of liberalism under Putin. The chapter sections on socio-economic liberalism consider such topics as the Russian peasant commune and the Stolypin agrarian reforms, capitalism and industrialization, the "right to a dignified human existence" championed by Vladimir Soloviev, and Soviet socialism and post-Soviet privatization, emphasizing throughout the overall weak socio-economic foundations of liberalism in Russia.

As mentioned above, the best part of the book is on Russian liberalism over the past century. In 1922 Lenin deported scores of Russian philosophers; many other Russian liberals also fled the country. In Chapter 7 Robinson demonstrates that the history of émigré liberalism is an essential part of the history of Russian liberalism as a whole. He explores both the fate of the Kadet party in exile and the intellectual legacy of émigré philosophers and economists. That legacy includes a Christian conception of human dignity and personhood that philosophers such as Nikolai Berdiaev advanced against Western liberalism, which was taken to be agnostic or atheistic and therefore destructive of spiritual freedom. This type of religious critique of Western liberalism (and of human rights) has gained traction today, not only in the Russian Orthodox Church but also among some Western Christian thinkers (e.g., John Millbank, William Cavanaugh, and Vigen Guroian). It is distorted, neglects the religious origins of human rights, and is in general fraught with risks.

The Soviet project was inimical to liberalism. "Miraculously, though, pockets of liberalism survived," Robinson writes (117). In Chapter 8 he presents a very good account of Soviet liberalism, generally following Mikhail Epstein's view that, "Liberalism was the major intellectual force of the entire dissident movement" (120). Dissident discourse focused on human rights and the rule of law, as Robinson shows in detail. In many respects the liberal ideas of the dissident movement triumphed with perestroika (Chapter 9). In 1988 Mikhail Gorbachev resolved that the Soviet Union should be a state under the rule of law (\emph{pravovoe gosudarstvo}). This liberal ideal has remained elusive in the period since the collapse of the Soviet Union in 1991. Indeed, Putinism has all but destroyed it. For good reason does Robinson both begin and end his study with the grim observation that today Russian liberalism is in an extremely "parlous state" (2, 203).

\vspace{2em}
\begin{center}
  \includegraphics[width=0.75cm]{articlend.png}
\end{center}

\biobox{\textbf{Randall A. Poole} is Professor of Intellectual History at the College of St.~Scholastica, a senior fellow of the Center for the Study of Law and Religion at Emory University School of Law, and co-director of the Northwestern University Research Initiative in Russian Philosophy, Literature, and Religious Thought. He is the translator and editor of \emph{Problems of Idealism: Essays in Russian Social Philosophy} (2003) and co-editor of five other volumes: \emph{A History of Russian Philosophy, 1830\textendash 1930: Faith, Reason, and} \emph{the Defense of Human Dignity} (2010, 2013), \emph{Religious Freedom in Modern Russia} (2018), \emph{The} \emph{Oxford Handbook of Russian Religious Thought} (2020), \emph{Evgenii Trubetskoi: Icon and Philosophy} (2021), and \emph{Law and the Christian Tradition in Modern Russia} (2022). He is also the author of many articles and book chapters on Russian intellectual history, philosophy, and religion.}

\label{sec:poole2}

\end{document}
